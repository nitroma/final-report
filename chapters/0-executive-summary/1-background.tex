\section{Background}

Substituted aromatic amines are essential intermediates involved in the synthesis of many pharmaceutical compounds, agrochemicals, and dyes \cite{vogt_amines_2000}. Since the direct addition of the amino group via electrophilic aromatic substitution is not possible, it is introduced via nitration and subsequent reduction []. 

Following multiple deadly industrial accidents in chemical plants performing batch nitration, especially the 2019 Xiangshui chemical plant explosion, Chinese authorities are actively attempting to strengthen the control and management of nitration manufacturing \cite{el_diario_china_2019}. Furthermore, global chemical firms, for whom most chemical plants in China produce intermediates, are urged to source their raw materials more responsibly [].

%In this context, Nitroma, a foreign-invested enterprise, seeks to develop an inherently safer nitration process for the production of substituted aromatics amines intermediates. , which fulfils the new safety and environmental standards imposed by Chinese authorities, while offering a socially responsible alternative to incumbent suppliers for multinational firms eager to improve their environmental portfolio. 

In this context, Nitroma, a foreign-invested enterprise, seeks to develop an inherently safer nitration process for the production of substituted aromatics amines intermediates. While fulfilling the new safety and environmental standards imposed by Chinese authorities, Nitroma seizes the opportunity of offering a socially responsible alternative to current suppliers for multinational firms eager to improve their environmental portfolio. 

Recent developments in flow chemistry and process intensification enable the inherently safer design of critical industrial processes such as nitration. Nitroma will exploit the opportunity to transition from batch to continuous nitration to safely manufacture important intermediates on a large-scale \cite{di_miceli_raimondi_safety_2015}. Using multi-criteria decision methods, the optimum combination of amino derivatives of toluene to be produced by Nitroma has been determined: 4-aminobenzaldehyde, a precursor of pharmaceuticals and vanillin, \ortho-toluidine, an intermediate in the manufacture of herbicides, and 4-aminobenzoic acid, an essential building block for the production of folic acid/vitamin B9, will be produced by Nitroma \cite{bowers_toluidines_2000,bruhne_benzaldehyde_2011,maki_benzoic_2000}.

The productivity of the Nitroma's process is flexible and can be adjusted to market demand. Modularity is indeed in-built in the process which can control the partial oxidation of p-nitrotoluene to its benzaldehyde or benzoic acid derivatives. To meet the current customer demand, the plant must operate a 240-day 4-aminobenzaldehyde campaign and a 35-day 4-aminobenzoic acid campain, thus yielding 652 tonnes/year of 99\% pure o-toluidine, 538 tonnes/year of 99.3\% pure 4-aminobenzaldehyde and 100 tonnes/year of pure 4-aminobenzoic acid.

Nitroma's demonstration plant is located in the Nanjing Jiangbei New Material Science Park in Jiangsu (China). In addition to secured feedstock supply from China’s largest toluene manufacturer, Sinopec Yangzi Petrochemical, the Industry Park hosts all the necessary utilities and is well connected to main transportation hubs, thus guaranteeing that products can easily be shipped to its international buyers []. Moreover, Nitroma's continuous process minimises the risk of s




% thus guaranteeing that Nitroma's products can easily be shipped to its international buyers [].


This report details the initial design of Nitroma's plant, obtained following Process Intensification and Green Chemistry principles to develop an inherently safer and environmentally friendly continuous process.


%,

