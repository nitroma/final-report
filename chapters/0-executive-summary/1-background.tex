\section*{Background}

Substituted aromatic amines are essential intermediates involved in the synthesis of many pharmaceutical compounds, agrochemicals, and dyes \cite{vogt_amines_2000}. However, producing them directly through electrophilic aromatic substitution of the amino group is not possible, hence it is introduced via nitration and subsequent reduction \cite{dugal_nitrobenzene_2005}. Nitration poses major safety risks, and recent deadly industrial accidents on batch nitration plants such as the 2019 Xiangshui plant explosion have prompted Chinese authorities to actively strengthen the control and management of nitration manufacturing \cite{el_diario_china_2019}. Furthermore, global chemical firms, for whom most chemical plants in China produce intermediates, are being urged to source their raw materials more responsibly \cite{stanway_global_2019}.

In this context, Nitroma, a foreign-invested enterprise, seeks to develop an inherently safer nitration process to produce substituted aromatics amines in China's growing market. Nitroma plans to seize the opportunity of offering a socially responsible alternative to current suppliers for multinational firms eager to improve their environmental portfolio. 

Recent developments in flow chemistry and process intensification enable the inherently safer design of industrial-scale nitration. Nitroma will exploit the opportunity of transitioning from batch to continuous nitration to safely manufacture important intermediates on a large-scale \cite{di_miceli_raimondi_safety_2015}. Using multi-criteria decision methods, the optimum combination of amino derivatives of toluene to be produced by Nitroma has been determined: 4-aminobenzaldehyde, a precursor of pharmaceuticals and vanillin, \ortho-toluidine, an intermediate in the manufacture of herbicides, and 4-aminobenzoic acid, an essential building block for the production of vitamin B9 \cite{bowers_toluidines_2000,bruhne_benzaldehyde_2011,maki_benzoic_2000}.

Nitroma's production capacity is flexible and adjustable to market demand. Modularity is indeed in-built in the process which can control the partial oxidation of p-nitrotoluene to its benzaldehyde or benzoic acid derivatives. Based on current customer demand, this initial plant design will operate a 240-day 4-aminobenzaldehyde campaign and a 35-day 4-aminobenzoic acid campain, capable of yielding 652 tonnes/year of 99\% pure o-toluidine, 538 tonnes/year of 99.3\% pure 4-aminobenzaldehyde and 100 tonnes/year of pure 4-aminobenzoic acid.

Nitroma's demonstration plant is located in the Nanjing Jiangbei New Material Science Park in Jiangsu (China). In addition to secured feedstock supply from China’s largest toluene manufacturer, Sinopec Yangzi Petrochemical, the Industry Park hosts all the necessary utilities and is well connected to main transportation hubs, thus guaranteeing that products can easily be shipped internationally []. 

%Moreover, Nitroma can ensure reliable supply to this customer thanks to its continuous process which minimises the risk of safety hazards and thus supply chain interruptions.

This report details the initial design of Nitroma's plant, obtained following Process Intensification and Green Chemistry principles to develop an inherently safer and environmentally friendly continuous process.



