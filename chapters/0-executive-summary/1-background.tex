\section{Background}

Substituted aromatic amines are some of the most important intermediates in chemical manufacturing. There are indeed involved in the synthesis of many essential pharmaceutical compounds, agrochemicals, and dyes \cite{vogt_amines_2000}. Since the direct addition of the amino group via electrophilic aromatic substitution is not possible, it is introduced via nitration and subsequent reduction. Nitration is a very hazardeous 

Following multiple deadly industrial accidents in chemical plants performing batch nitration, especially the 2019 Xiangshui chemical plant explosion, Chinese authorities are actively attempting to strengthen the control and management of nitration manufacturing \cite{el_diario_china_2019}.

%Recent developments in flow chemistry are enabling process intensification of critical industrial processes to make them inherently safer. The transition from batch to continuous nitration opens up the possibility to safely manufacture important intermediates on a large-scale \cite{di_miceli_raimondi_safety_2015}.

%In this context, Nitroma seeks to develop an inherently safe, multi-purpose, continuous liquid-phase nitration process to produce o-toluidine, 4-aminobenzaldehyde and 4-aminobenzoic acid. This report details the preliminary design for Nitroma’s demonstration plant, to be located in the Nanjing Chemical Industry Park (China). Synthesis pathways for nitration, oxidation and hydrogenation reactions were selected and modelled in innovative packed-bed microreactors or trickle-bed reactors. Novel downstream processing units, including continuous falling-film melt crystallisers, enable Nitroma to reach the target product purity. Finally, a techno-economic feasibility study was conducted, taking into account health, safety and environmental considerations.
