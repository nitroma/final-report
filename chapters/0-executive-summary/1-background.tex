\section{Background}

Substituted aromatic amines are essential intermediate involved in the synthesis of many pharmaceutical compounds, agrochemicals, and dyes \cite{vogt_amines_2000}. Since the direct addition of the amino group via electrophilic aromatic substitution is not possible, it is introduced via nitration and subsequent reduction. 

Following multiple deadly industrial accidents in chemical plants performing batch nitration, especially the 2019 Xiangshui chemical plant explosion, Chinese authorities are actively attempting to strengthen the control and management of nitration manufacturing \cite{el_diario_china_2019}. Furthermore, global chemical firms, for whom most chemical plants in China produce intermediates, are urged to source their raw materials more responsibly [].

In this context, Nitroma, a foreign-invested enterprise, seeks to develop an inherently safer nitration process for the production of substituted aromatics amines intermediates, which fulfils the new safety and environmental standards imposed by Chinese authorities, while offering a socially responsible alternative to incumbent suppliers for multinational firms eager to green their image. 

Recent developments in flow chemistry and process intensification enable the inherently safer design of critical industrial processes. Nitroma will exploit the opportunity to transition from batch to continuous nitration to safely manufacture important intermediates on a large-scale \cite{di_miceli_raimondi_safety_2015}. Using multi-criteria decision methods, the optimum combination of amino derivatives of toluene to be produced by Nitroma is composed of: 4-aminobenzaldehyde, a precursor of pharmaceuticals and vanilin, \ortho-toluidine, an intermediate in the manufacture of herbicides, and 4-aminobenzoic acid, an essential building block for the production of folic acid/vitamin B9, will be produced by Nitroma \cite{bowers_toluidines_2000,bruhne_benzaldehyde_2011,maki_benzoic_2000}.

The productivity of the Nitroma's process is flexible and can be adjusted to market demand. Modularity is indeed in-built in the process which can control of the partial oxidation of p-nitrotoluene to its benzaldehyde or benzoic acid derivatives. To meet the current customer demand, the plant must operate 280 days in a 4-aminobenzaldehyde focused scenario and 10 days in its 4-aminobenzoic acid counterpart, thus yielding 626 tonnes/year of 99\% pure o-toluidine, 500 tonnes/year of 99.3\% pure 4-aminobenzaldehyde and 107 tonnes/year of pure 4-aminobenzoic acid.


Nitroma's demonstration plant is located in the Nanjing Jiangbei New Material Science Park in Jiangsu (China). The plant is ideally situated for secured feedstock supply since China’s largest toluene manufacturer is less than 10km away from the site. Additionally, the Industry Park hosts all the necessary utilities and is well connected to main transportation hubs, thus guaranteeing that Nitroma's products can easily be shipped to its international buyers []. 








%Synthesis pathways for nitration, oxidation and hydrogenation reactions were selected and modelled in innovative packed-bed microreactors or trickle-bed reactors. Novel downstream processing units, including continuous falling-film melt crystallisers, enable Nitroma to reach the target product purity. Finally, a techno-economic feasibility study was conducted, taking into account health, safety and environmental considerations.
