\section*{Separation system}

The separation of PNT from a liquid mixture of predominantly PNT and MNT with trace amounts of ONT has been selected for detailed design. The choice has been because PNT is an essential precursor for production of both aminobenzaldehyde and aminobenzoic acid and needs to be at least 90\% purity. Due to the negligible difference in boiling points yet sizeable difference in melting points of PNT and MNT, crystallisation has been chosen as the separation methodology. A cooling melt crystalliser has been designed to be implemented downstream of distillation column S103, where solid PNT would crystallise out from the liquid mixture. A subsequent hydraulic wash column has been designed immediately downstream of the crystalliser to recover pure solid PNT from the slurry. 


The crystalliser has been modelled as operating under mixed suspension mixed product removal (MSMPR) mode, where perfect mixing is assumed and the crystalliser operates at outlet conditions. The liquid mixture to be separated has been modelled as (PNT + MNT) binary system because the presence of ONT is negligible. This mixture exhibits a eutectic-forming solid-liquid phase behaviour, where PNT is able to crystallise from the mixture liquid as pure solid. Kinetics for crystal nucleation and growth and thermodynamic data were obtained from the literature and used for design calculations. An agitator has been implemented to achieve MSMPR operation and uniform temperature, and its geometry aims at minimising fouling in the vessel. To provide cooling for the crystalliser, a half pipe cooling coil surrounds the vessel through which cooling medium of 10\% ethylene glycol aqueous solution flows at \SI{7}{L/s} with temperature of \SI{274}{K}. Figure \ref{fig:crystalliser schematic executive} provides a 3D overview of the crystalliser design.


In the hydraulic wash column, solid crystals from the crystalliser would form a moving packed bed as it sediments in the column. The crystals are melted at the bottom of the column using a melter. Filter tubes are imp where MNT and ONT filtrates are separated from the solid PNT. Some molten PNT are recycled back into the bottom of the wash column known as the washing liquid. Counter current washing of the solid crystals also occurs near the bottom of the column. This method is a continuous process used for ultrapurification and proven to achieve 99.9\% separation therefore this recovery was assumed for PNT in the final product outlet. The diameter and steer flow rate were varied to observe the effect on the pressure drop across the length of the column. From this a suitable diameter of 0.17 m and steer flow rate  of $2.65 \times 10^{-5} m^{3}/s$ was chosen. Finally, a sensitivity analysis was conducted on the Kozeny constant to see how this impacts the pressure drop across the column. Figure \ref{fig:wash column schematic executive} is a 3D overview of the hydraulic wash column design.


\begin{figure}[h]
    \centering
    \begin{subfigure}[h]{0.5\textwidth}
    \centering
    \includesvg[scale=0.28,inkscapelatex=false]{chapters/3-separation/figures/Crystalliser_schematic.svg}
    \caption{Schematics for the designed crystalliser vessel: (i) perspective view; (ii) section view.}
    \label{fig:crystalliser schematic executive}
    \end{subfigure}%
    ~ 
    \begin{subfigure}[h]{0.5\textwidth}
    \centering
    \includesvg[scale=0.3,inkscapelatex=false]{chapters/3-separation/figures/Wash_column_schematic_executive.svg}
    \caption{Schematics for the hydraulic wash column: (i) perspective (ii) right-side section (iii) front section.}
    \label{fig:wash column schematic executive}
    \end{subfigure}
\end{figure}
