\section*{Separation system}

Producing high quality (\SI{> 98}{\percent}) 4-ABH and 4-ABA requires careful separation of the PNT and MNT precursors. A cooling melt crystalliser was successfully designed to exploit the significantly higher melting point of PNT than MNT. A state-of-the-art continuous hydraulic wash column then extracts and melts pure solid PNT for ease of further reaction. 

%The separation of PNT from a liquid mixture of predominantly PNT and MNT with trace amounts of ONT has been selected for detailed design. This was selected because PNT is an essential precursor for production of both aminobenzaldehyde and aminobenzoic acid and needs to be at least \SI{90}{\percent} purity. Due to the negligible difference in boiling points yet sizeable difference in melting points of PNT and MNT, crystallisation has been chosen as the separation methodology. A cooling melt crystalliser has been designed to be implemented downstream of distillation column S103, where solid PNT would crystallise out from the liquid mixture. A subsequent hydraulic wash column has been designed immediately downstream of the crystalliser to recover pure solid PNT from the slurry. 

The crystalliser has been modelled under mixed-suspension mixed-product removal (MSMPR) mode. Crystal growth and phase behaviour data from literature have been employed for design computations in MATLAB. A suitable agitator ensures MSMPR operation, where perfect mixing, uniform temperature, and minimal fouling on the vessel walls are achieved. A state-of-the-art cooling system ensures high-performance heat transfer;it consists of a half pipe cooling coil surrounding the vessel through which cooling medium of \SI{10}{\percent} ethylene glycol aqueous solution flows at \SI{7}{L/s} with a temperature of \SI{274}{K}. \Cref{fig:crystalliser schematic executive} provides an overview to a 3D SolidWorks CAD of the crystalliser design.

%The crystalliser has been modelled as operating under mixed-suspension mixed-product removal (MSMPR) mode, where perfect mixing is assumed and the crystalliser operates at outlet conditions. The liquid mixture to be separated has been modelled as (PNT + MNT) binary system because the presence of ONT is negligible. This mixture exhibits a eutectic-forming solid-liquid phase behaviour, where PNT is able to crystallise from the mixture liquid as pure solid; the remaining liquid after this process is known as the mother liquor. Kinetics for crystal nucleation and growth and thermodynamic data were obtained from the literature and used for design calculations. An agitator has been implemented to achieve MSMPR operation and uniform temperature, and its geometry aims at minimising fouling in the vessel. To provide cooling for the crystalliser, a half pipe cooling coil surrounds the vessel through which cooling medium of \SI{10}{\percent} ethylene glycol aqueous solution flows at \SI{7}{L/s} with temperature of \SI{274}{K}. \Cref{fig:crystalliser schematic executive} provides a 3D overview of the crystalliser design.

The hydraulic wash column is a critical unit that separates out the PNT crystals from the mother liquor and concurrently melts them to create a pure liquid product. The column employs hydraulic pressure to transport solid crystals as a moving packed bed down the column. At the bottom, a set of rotating knives and a melter purifies the PNT into a liquid stream. Filter tubes are implemented in the column where MNT and ONT filtrates are separated from the solid PNT. Molten PNT ascends from the bottom of the wash column providing counter current washing of the packed bed. This continuous ultra-purification process allows high recoveries of pure PNT that exceed \SI{99.9}{\percent}. A sensitivity analysis of the diameter,  steer flow rate and Kozeny constant on the pressure drop informed the final optimal design. Herefrom, a diameter of \SI{0.17}{\m}, height of \SI{0.8}{\m}, and steer flow rate of 2.65 $\times$ 10$^{-5}$ m$^{3}$/s have been deemed suitable. \Cref{fig:wash column schematic executive} is a 3D overview of the hydraulic wash column design.

%and is known to achieve \SI{99.9}{\percent} separation of the crystallised component from the remaining; therefore,

\begin{figure}[h]
    \centering
    \begin{minipage}[t]{0.4\textwidth}
        \includesvg[scale=0.28,inkscapelatex=false]{chapters/3-separation/figures/Crystalliser_schematic.svg}
        \caption{Crystalliser vessel schematics: (i) perspective view; (ii) section view.}
        \label{fig:crystalliser schematic executive}
    \end{minipage}\hfill
    \begin{minipage}[t]{0.55\textwidth}
        \includesvg[scale=0.3,inkscapelatex=false]{chapters/3-separation/figures/Wash_column_schematic_executive.svg}
        \caption{Hydraulic wash column schematics: (i) perspective (ii) right-side section (iii) front section.}
        \label{fig:wash column schematic executive}
    \end{minipage}
\end{figure}
