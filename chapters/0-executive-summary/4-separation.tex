\section*{Separation system}

The separation of p-nitrotoluene from liquid mixture of p-nitrotoluene and m-nitrotoluene with trace amounts of o-nitrotoluene has been selected for detailed design. The choice has been because p-nitrotoluene is an essential precursor for production of both aminobenzaldehyde and aminobenzoic acid and needs to be at least 90\% purity. Due to the negligible difference in boiling points yet sizeable difference in melting points of p-nitrotoluene and m-nitrotoluene, crystallisation has been chosen as the separation methodology. A cooling melt crystalliser has been designed to be implemented downstream of distillation column S103, where solid p-nitrotoluene would crystallise out from the liquid mixture. A subsequent hydraulic wash column has been designed immediately downstream of the crystalliser to recover pure solid p-nitrotoluene from the liquid. 


An MSMPR crystallisre with an agitator was the choice of crystalliser to implement due to its 
%results achieved 


The hydraulic wash col

\begin{figure}[h]
    \centering
    \includesvg[scale=0.3,inkscapelatex=false]{chapters/3-separation/figures/Crystalliser_schematic.svg}
    \caption{Schematics for the designed crystalliser vessel: (a) perspective view; (b) section view.}
    \label{fig:crystalliser schematic executive}
\end{figure}

\begin{figure}[h]
    \centering
    \includesvg[scale=0.3,inkscapelatex=false]{chapters/3-separation/figures/Wash_column_schematic_executive.svg}
    \caption{Schematics for the hydraulic wash column: (a) perspective (b) right-side section (c) front section (d) section for detailed view of knives (e) section for detailed view of filter tubes crossing.}
    \label{fig:wash column schematic executive}
\end{figure}