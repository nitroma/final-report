\section*{A really good Separation system}

The separation of p-nitrotoluene from a liquid mixture of predominantly p-nitrotoluene and m-nitrotoluene with trace amounts of o-nitrotoluene has been selected for detailed design. The choice has been because p-nitrotoluene is an essential precursor for production of both aminobenzaldehyde and aminobenzoic acid and needs to be at least 90\% purity. Due to the negligible difference in boiling points yet sizeable difference in melting points of p-nitrotoluene and m-nitrotoluene, crystallisation has been chosen as the separation methodology. A cooling melt crystalliser has been designed to be implemented downstream of distillation column S103, where solid p-nitrotoluene would crystallise out from the liquid mixture. A subsequent hydraulic wash column has been designed immediately downstream of the crystalliser to recover pure solid p-nitrotoluene from the slurry. 


The crystalliser has been modelled as operating under mixed suspension mixed product removal (MSMPR) mode, where perfect mixing is assumed and the crystalliser operates at outlet conditions. The liquid mixture to be separated has been modelled as (p-nitrotoluene and m-nitrotoluene) binary system because the presence of o-nitrotoluene is negligible. This mixture exhibits a eutectic-forming solid-liquid phase behaviour, where p-nitrotoluene is able to crystallise from the mixture liquid as pure solid. Kinetics for crystal nucleation and growth and thermodynamic data were obtained from the literature and used for design calculations. An agitator has been implemented to achieve MSMPR operation and uniform temperature, and its geometry aims at minimising fouling in the vessel. To provide cooling for the crystalliser, a half pipe cooling coil surrounds the vessel through which cooling medium of 10\% ethylene glycol aqueous solution flows at \SI{7}{L/s} with temperature of \SI{274}{K}. Figure \ref{fig:crystalliser schematic executive} provides a 3D overview of the crystalliser design.


The hydraulic wash column design includes filter tubes where filtrate (m-Nt and o-NT) are separated from the solid p-NT. The solid crystals form a moving packed bed that is melted at the bottom of the column using a melter. Counter current washing of the solid crystals also occurs near the bottom of the column. Hydraulic wash columns used for ultrapurification is a continuous process proven to achieve 99.9\% separation of solid organics from liquid. For the design, an overall 99.9 \% recovery of p-nitrotoluene in the final product outlet has been assumed. The diameter and steer flow rate were varied across the length of the column to observe the effect on the pressure drop . The pressure drop at various locations of the column were observed as key variables (the diameter and steer flow rate) were varied. From this a suitable diameter of 0.17 m and steer flow rate  of $2.65 \times 10^{-5} m^{3}/s$ was chosen. Finally, a sensitivity analysis was conducted on the Kozeny constant to see how this impacts the pressure drop across the column. Figure \ref{fig:wash column schematic executive} is a 3D overview of the hydraulic wash column design.


\begin{figure}[h]
    \centering
    \includesvg[scale=0.3,inkscapelatex=false]{chapters/3-separation/figures/Crystalliser_schematic.svg}
    \caption{Schematics for the designed crystalliser vessel: (i) perspective view; (ii) section view.}
    \label{fig:crystalliser schematic executive}
\end{figure}

\begin{figure}[h]
    \centering
    \includesvg[scale=0.3,inkscapelatex=false]{chapters/3-separation/figures/Wash_column_schematic_executive.svg}
    \caption{Schematics for the hydraulic wash column: (i) perspective (ii) right-side section (iii) front section.}
    \label{fig:wash column schematic executive}
\end{figure}

\begin{figure}[h]
    \centering
    \begin{subfigure}[h]{0.5\textwidth}
    \centering
    \includesvg[scale=0.3,inkscapelatex=false]{chapters/3-separation/figures/Crystalliser_schematic.svg}
    \caption{Schematics for the designed crystalliser vessel: (i) perspective view; (ii) section view.}
    \label{fig:crystalliser schematic executive}
    \end{subfigure}%
    ~ 
    \begin{subfigure}[h]{0.5\textwidth}
    \centering
    \includesvg[scale=0.3,inkscapelatex=false]{chapters/3-separation/figures/Wash_column_schematic_executive.svg}
    \caption{Schematics for the hydraulic wash column: (i) perspective (ii) right-side section (iii) front section.}
    \label{fig:wash column schematic executive}
    \end{subfigure}
\end{figure}
