\section*{Safety and environmental considerations}

Safety and environmental considerations are paramount for Nitroma, with various measures taken to eliminate or minimise as far as possible, major process hazards including fires, explosions and the release of hazardous chemicals. Compliance with all relevant laws, regulations and standards regarding health and safety of both people and the environment will be ensured.  During the initial design stage, inherent safety was implemented to reduce risk as low as reasonably possible. Furthermore, materials for process units were carefully selected to withstand conditions they may be subject to. A plant-wide risk assessment was conducted to identify hazards and associated risk to the people, plant and environment. Whilst through this, controls were put in place to manage the hazards identified, a hazard and operability study (HAZOP) was conducted on the plant section with the highest potential for damage, to evaluate the effectiveness of safeguards of the numerous deviations that could arise. Several recommendations from this study were then implemented, minimising the likelihood of the various risks from occurring as far as possible. For further reassurance, a layers of protection analysis (LOPA) was completed on one of the major consequences identified from the HAZOP – an explosion of the reaction vessel due to gaseous overpressure.  From this, it was decided to implement a safety instrumented function of safety integrity level three, to cover the four initiating causes related to this consequence, allowing the cumulative event likelihood to fall below the target of 1x10-6, thus lowering the risk of multiple fatalities during the operation of the plant for the duration of its lifetime. 

Nitroma waste prevention and minimisation involved choosing an alternative reducing agent and substituting sulphuric acid, which is known to be costly to dispose and environmentally unfriendly, with H-mordenite. This also avoided the formation of nitrogen oxides during nitration. The best techniques available were selected to treat waste streams. Liquid waste streams with nitric acid and high organic content were firstly neutralised with lime to treat the nitric acid and solid deposits were retained through a filter. The overall COD of the liquid waste effluent was reduced below the maximum limit of 250 mg/L by adsorption using activated carbon followed by anaerobic membrane bioreactor. Gaseous waste streams with volatile organic compounds were incinerated, with a low NOx burner used to reduce NOx emissions. The NOx emissions were further reduced by passing the flue gas through a wet scrubber, removing 99\% of the NOx present. 

Finally, the plant layout was designed following a risk-based approach. The principle of flow was used for the equipment layout to avoid long pipelines and ensure efficient material flow. Dow’s fire and explosion index were identified for major process units handling flammable and unstable materials to determine the radius of exposure.  