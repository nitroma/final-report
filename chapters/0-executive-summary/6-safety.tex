\section*{Safety and environmental considerations}

Nitroma takes pride in achieving the highest standard of health and safety whilst producing top quality amine aromatics in our innovative, multi-purpose plant. One of the company's primary goals is to become an leading example in industry for the safe and efficient nitration of aromatic compounds. By designing a continuous liquid-phase process, Nitroma stands out against the majority of competitors whom would typically employ a batch process which, historically is a more dangerous option. Compliance with laws and regulations, including COMAH, COSHH and DSEAR, provided a strong foundation for establishing safe working environment for our personnel and the surrounding area. An inherently safe design was produced through the consideration of the synthesis pathways, the chemical substances utilised and ensuring the number of process units was minimised. The major hazards associated to Nitroma's plant include fires, explosions and the release of hazardous chemicals. Through a plant-wide risk assessment, the initiating causes to these hazards were identified with controls subsequently put in place. A hazard and operability study (HAZOP) was conducted on the o-toluidine production area of the plant (as this region holds the highest potential for damage according to the Dow's fire and explosion index). Following this study, additional safety structures were implemented to minimise critical process risks. A layers of protection analysis (LOPA) was then completed on one of the major consequences identified from the HAZOP – an explosion of the reaction vessel due to overpressure. A control system was implemented in response to this and thus Nitroma can guarantee that the specified target mitigated event likelihood of \num{1e-6} has been achieved.

Nitroma strives to become minimise any adverse affcets our operations may have on the environment and biodiversity. Thus every effort was made to prevent waste, via the synthesis pathways selected and use of recovery systems for the main solvents and reactants. Incineration of liquid streams was avoided through treatment by adsorption and followed by anaerobic membrane bioreactor ().  By reducing the overall chemical oxygen limit below maximum limit of \SI{250}{\mg\per\litre}, Nitroma hopes to minimise effect of the discharge on the  aquatic ecosystems. Extensive research and calculations demonstrated that construction of gaseous waste treatment methods such as cryo-condensation would result in more carbon dioxide being emitted that would be saved. Thus, 
all gaseous waste streams were incinerated in a low-NOx burner unit with the flue gas passing through a wet scrubber, minimising the emission of nitrogen oxides. Finally, the plant layout was designed following a risk-based approach. The principle of flow was used for the equipment layout to avoid long pipelines and ensure efficient material flow. 