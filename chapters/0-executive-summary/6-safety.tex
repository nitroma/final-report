\section*{Safety and environmental considerations}

 One of the Nitroma's primary goals is to become an leading example in industry for the safe and efficient nitration of aromatic compounds. By designing a continuous liquid-phase process, Nitroma stands out against the majority of competitors whom would typically employ a batch process which is historically more dangerous. Compliance with laws and regulations, including COMAH, COSHH and DSEAR, provided a strong foundation for establishing safe working environment for our personnel and the surrounding area. An inherently safe design was produced through the consideration of the synthesis pathways, the chemical substances utilised and minimising process units. A plant-wide risk assessment identified the main hazards which would result in a major hazards associated to Nitroma's plant include fires, explosions and the release of hazardous chemicals
 
 the initiating causes to these hazards were identified with controls subsequently put in place. A hazard and operability study was conducted on the o-toluidine production area of the plant (as this region holds the highest potential for damage according to the Dow's fire and explosion index). Following this study, additional safety structures were implemented to minimise critical process risks. A layers of protection analysis was then completed on one of the major consequence – an explosion of the reactor vessel. A control system was implemented in response to this and thus Nitroma can guarantee that the specified target mitigated event likelihood of \num{1e-6} has been achieved.

Nitroma strives to minimise any adverse effect our operations may have on the environment and biodiversity. Thus every effort was made to prevent waste, via the synthesis pathways selected and use of recovery systems for the main solvents and reactants. All liquid streams pass through an adsorption column followed by an anaerobic membrane bioreactor (AnMBR). This will reduce the overall chemical oxygen limit below maximum limit of \SI{250}{\mg\per\litre} and minimise the effect on aquatic ecosystems. The stream with nitric acid is first neutralised before this treatment, to maintain neutral pH in the discharge. The AnMBR produces biogas, which Nitroma hopes to integrate into the process, harvesting this energy to become more sustainable.  Extensive research and calculations demonstrated that construction of gaseous waste treatment methods would result in more carbon dioxide being emitted than would be saved. Thus, all gaseous waste streams were incinerated in a low-NOx burner unit with the flue gas passing through a wet scrubber, minimising the emission of nitrogen oxides. Finally, the plant layout was designed following a risk-based approach and the principle of flow for the equipment layout to avoid long pipelines and ensure efficient material flow. 