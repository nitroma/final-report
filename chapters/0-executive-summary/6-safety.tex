\section*{Safety and environmental considerations}

Nitroma takes pride in achieving the highest standard of health and safety whilst producing top quality amine aromatics in our innovative, multi-purpose plant. One of the company's primary goals is to become an leading example in industry for the safe and efficient nitration of aromatic compounds. By designing a continuous liquid-phase process, Nitroma stands out against the majority of competitors whom would typically employ a batch process which, historically is a more dangerous option. An inherently safe design was produced through the consideration of the synthesis pathways, the chemical substances utilised and ensuring only a minimal



 


 Compliance with all relevant EHS laws, regulations and standards has been ensured during the design process. 
 
 A plant-wide risk assessment was conducted to identify hazards and associated risk to the people, plant and environment that informed subsequent control systems. A hazard and operability study (HAZOP) was conducted on the plant section with the highest potential for damage to evaluate the effectiveness of existing safeguards. Following this, additional safety structures were implemented to minimise critical process risks. A layers of protection analysis (LOPA) was completed on one of the major consequences identified from the HAZOP – an explosion of the reaction vessel due to overpressure. In response measures were implemented to meet the target mitigated event likelihood of 1x10$^{-6}$.  

Nitroma's waste prevention and minimisation involved utilising an alternative reducing agent and substituting sulphuric acid, which is known to be costly to dispose and environmentally hazardous, with H-mordenite. This also avoided the formation of nitrogen oxides during nitration. The best available techniques  were selected to treat waste streams. Liquid waste streams with nitric acid and high organic content were firstly neutralised with lime to treat the nitric acid and solid deposits were retained through a filter. The overall COD of the liquid waste effluent was reduced below the maximum limit of \SI{250}{\mg\per\litre} by adsorption followed by anaerobic membrane bioreactor. All gaseous waste streams were incinerated in a low NOx burner with the flue gas passing through a wet scrubber, minimising the NOx released. Finally, the plant layout was designed following a risk-based approach. The principle of flow was used for the equipment layout to avoid long pipelines and ensure efficient material flow. 