\section*{Process control and design}

Operational safety and efficiency are at the core of Nitroma's business. To ensure production and quality targets are consistently met while mitigating safety risks, a robust and sophisticated control strategy, with sufficient protective structures, is critical.

A plant-wide control approach was adopted to design a control system capable of minimising disturbance propagation throughout the plant.  The throughput control of the nitric acid and toluene feed to the nitration reactor allows effective control of the plant capacity, since the nitration reaction produces the precursors for the three end-products in Nitroma's portfolio.   

%Forty four process control structures were found to provide adequate disturbances management to achieve the design objectives. Control challenges and key maintenance issues along with associated mitigation strategies were discussed.

A detailed control architecture design was designed around the plant section with highest potential damage 2-nitrotoluene hydrogenation reactor and the distillation column that separates the methanol solvent. Robust ratio and temperature controllers were designed using feedback, feedforward and cascade control arrangements to ensure the system can maintain optimum and safe operating conditions and adjust to disturbances. Inferential temperature control was also cascaded onto the slow composition controller in the distillation column, to ensure tight quality control of the separation. 
Regular and executive alarms, linked to automated interlock actions, were integrated to the control structure as an additional layer of protection to respectively alert to operators of abnormal operation, and to bring the plant into a safe fail state. Following LOPA, a SIL-3 rated pressure regulating system was implemented to further mitigate the risk of overpressure.

%compensate the slow measurement time and



