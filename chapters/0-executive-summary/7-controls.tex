\section*{Process control and design}

Operational safety and efficiency are at the core of Nitroma's business. To ensure production and quality targets are consistently met while mitigating safety risks, a robust and sophisticated control strategy, with clear objective and sufficient protective structures, is critical.

A plant-wide control approach was adopted to design a control system capable of minimising disturbance propagation throughout the plant. Forty four process control structures were found to provide adequate disturbances management to achieve the design objectives. Control challenges and key maintenance issues along with associated mitigation strategies were discussed. The appropriate process throughput manipulator was identified as the speed of the pump feeding nitric acid and toluene to the nitration reactor. Since the nitration reaction produces the precursors for the three end-products in Nitroma's portfolio, this choice ensures that production demands are fulfilled.    

The o-toluidine production section was selected for the detailed control architecture design around the 2-nitrotoluene hydrogenation reactor and the methanol solvent distillation column. Robust ratio and temperature controllers were designed using feedback, feedforward and cascade control arrangements to ensure the system can maintain optimum and safe operating conditions and adjust to disturbances. Another key challenge was the composition control of the bottoms stream from the column reboiler due to composition analysers having long lag times. Inferential temperature control was used to compensate the slow composition controller. Regular and executive alarms, linked to automated interlock actions, were integrated to the control structure as an additional layer of protection to respectively alert to operators of abnormal operation, and bring the plant into a safe fail state if necessary. Following LOPA, a SIL-3 rated pressure regulating system was implemented to further mitigate the risk of overpressure.




