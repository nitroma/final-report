\section*{Process control and design}


%An effective strategy for control and operation of the plant underpins the safe and consistent production of high-quality H2 and graphite. To enable this, a plantwide survey was carried out to select controlled and manipulated variable pairs for each major process unit. These would be used in key control loops to maintain inventories of heat and mass, ensure safe operation within operating envelopes, and guarantee on-spec production. Challenges to control and key disturbances along with their respective solutions and mitigations were also identified.
%Throughput of the plant is controlled via the fraction purged from the membrane permeate stream, which goes on to feed the PSA unit. This H2 purge (81 vol\%) supplies the furnaces responsible for heating the molten salt circulation loops and maintaining the temperature of the reactors. By selecting the intermediate purity H2 stream – higher than the 55% membrane feed but lower than the 99.97vol% PSA outlet – a balance is struck between reducing emissions from the furnace and excessive loading of the PSA columns.
%Detailed design of the reactor modules was conducted based on a piping and instrumentation diagram containing all the control loops, actuators, alarms, and trips, along with consideration of start-up and shutdown procedures for the section. The furnace has been designed with an innovative multi-layer control scheme. Heat flow measurements of the H2 and natural gas feed streams maintain the correct total heat supplied, and ratio control of the fuel-air mixture is combined with cross-limiting to ensure air-rich combustion conditions upon setpoint changes.

%A plantwide control and operations survey identified potential control schemes to mitigate build-up of impurities. A throughput manipulator placed after the process bottleneck (the microwave reactor) ensures a constant flowrate from the atorvastatin nucleus and superstatin chain synthesis routes, and hence production rate.

%From a plantwide control perspective, the process throughput manipulator (TPM) was identified and selected to be the product stream from the last separation train. This ensures that the entire process is suitably controlled to meet production demands.