\section*{Business justification}

% Mustafa's paragraph for motivation of Nitroma

Nitroma’s product line consists of speciality chemicals that are commonly used as intermediates for the manufacture of dyes, herbicides, and pain relief drugs. In recent times, these sectors have seen a boom in demand in China. The growth in the Chinese dye market is driven by the country’s rapidly growing packaging and printing industries (XX). This is mainly due to increased demand for consumer goods coming from a surge in e-commerce. The administration of pain relief drugs has also grown as modern medicine techniques are adopted by the rising Chinese middle class (XX). Similarly, herbicide usage has increased amongst traditional farmers as China shifts toward contemporary farming to sustain a stable food supply for its 1.4 billion citizens (XX). The coronavirus outbreak has had little to no effect on the demand for Nitroma’s products. In fact, the demand for pain relief drugs has risen 1000-fold following the pandemic (XX). In specific, the Chinese markets for PABA, PABH and OTOL have a respective CAGR of 3.33\% (2021-26), 5.04\% (2020-24) and 6.5\% (2018-25) based on analyst reports (XX). Having a diverse product portfolio protects Nitroma from any shift in market trends.

Nitroma’s first year annual production will capture 1.4\%, 1.9\% and 0.2\% of the PABA, PABH and OTOL markets, respectively. The threat from domestic competition in China is strong, with 70-80\% of global OTOL and PABH manufacturers located in the country (XX). However, the industry is highly fragmented and the lack of a dominant company creates a low barrier to entry for Nitroma. Nitroma intends to leverage the financial benefit of employing a continuous nitration process to undercut the average market price of its competitors’ products by 12.5%. This is motivated by the high price sensitivity and low brand loyalty of Nitroma’s customers. The penetrative pricing strategy will also be supported by the reduced corporate tax rate of 15% granted to Nitroma for operating as a foreign-invested enterprise in an industry encouraged by the Ministry of Commerce. Nitroma also seeks the reward of fulfilling unmet demand as many of its competitors are removed from the market due to unsafe practices. Here, Nitroma promises its buyers a stable product supply due its inherently safer continuous nitration process. To retain a stable market position over its lifetime, Nitroma will expand its operating capacity by 10 days per year every 5 year until decommissioning in 2043. To meet its $22.5 million CAPEX, Nitroma will sell a 45% stake to private equity investors for $10.3m and receive a $12.2m loan from local commercial banks at a 5.35% interest rate.2.3m loan from local commercial banks at a 5.35\% interest rate.

Key performance indicators to assess the economic feasibility of the project are reported in Table XX. The overall project was found to be economically feasible, with an NPV of \$84.4m and IRR of 31\% – 4x greater than the hurdle rate of 8.28\%. Nitroma’s healthy financial figures are due to the careful selection of cheap commodity chemicals as feedstock for a synthetic route yielding 3 high-value speciality chemicals. A unique continuous nitration process also allows Nitroma to utilise smaller equipment, reducing the company’s capital and operating expenditure.

A sensitivity analysis was conducted on 5 key variables to investigate the effect of changing macro- and micro-economic conditions. Product price was identified as the most sensitive factor, with a 30\% decrease in product price resulting in a 95\% decrease in NPV. However, Nitroma undercuts the market price by 12.5\% so the effect of this factor is mitigated. Different scenarios were also modelled to test the robustness of business plan. One particular scenario was related to events where COVID-19 resurfaced in Nanjing, China, and strong restriction measures were in place up till 2026, resulting in delay in production and reduction in operational days. Across all tested scenarios, the minimum ending cash balance was \$23.1 million and the longest payback period was 10.3 years. These results show the robustness of this project and should enhance the confidence of the shareholders.