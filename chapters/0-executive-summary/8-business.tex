\section*{Business justification}

% Mustafa's paragraph for motivation of Nitroma

The total capital expenditure of Nitroma is calculated to be \$22.5 million, which will be financed by local banks with \$12.3 million and shareholders with \$10.3 million. After construction and commissioning for two years, the plant will be fully operating and generating a profit of \$9 million in year 2023. Based on the corporate tax of 15\% and interest rate of 5.35\%, the expected payback period is 5.29 years and shareholders are recovering equity investment within 4.03 years. Net present value (NPV) at the end of project, after 20 years of depreciation, is a high value of \$85.8 million and internal rate of return of 41\%, beyond the hurdle rate of 8.28\%. The healthy financial figures are mainly due to a careful selection of cheap raw materials and synthetic route. Beyond that, Nitroma is employing state of the art technologies, for example continuous nitration process, to outtake major competitors in the market, where industry average is \%. Other critical KPIs include average gross profit margin at 54\% and average net profit margin of 33\%. 
Sensitivity analysis has been conducted on a few variables to investigate the effect of plant finance. Product price has been identified as the major factor as a 30\% increase in product price leads to 1.8 times of current NPV. Different scenarios have been modelled to test the robustness of business plan. One particular scenario is related to current events where COVID-19 in Nanjing, China is bounced back, and strong restriction measures is in place for another five years, resulting in delay of production, reduction in interest rate and operation days. All scenarios considered have a minimum positive ending cash balance of \$23.1 million, despite the worst case has a long payback period of 10.3 years. This result shows the profitability of this project and should enhance the confidence of the shareholders.
