\section*{Reactor design}


Transitioning from batch to continuous process requires a complete redesign of the reactors that are commonly used in the industry today. Nitroma developed a novel approach of using a Heat Exchanger Reactor (HEX Reactor) for the nitration of toluene.

-steps taken to model (kinetics, Brinkman) 
- SiC immobilise
%modelling approach [andreas]
The model was implemented and optimised on COMSOL 5.6. 

%optimisation
To ensure the reactor operates in optimal conditions, the following parameters were optimised: Concentric inner cooling pipe diameter and flowrate, direction of cooling water flow, \ch{HNO3} : Toluene inlet ratio and the total number of tubes.

A further sensitivity analysis was performed on the cooling water inlet temperature sin the temperature inside the reactor needs to be well-controlled at all times. Cooling water temperature was simulated to vary by \mypm $\Delta$ 5 K and total length of reactor required to achieve 98\% conversion while keeping temperature below the safe limit of 363K. Although the cooling water is not expected to fluctuate by this amount since it is reused and mixed together from multiple cooling water stream, taking a conservative approach increases the overall safety of the plant in the worst case scenario. The optimal length was determined to be 4.2m. 

The length of the reactor was determined 
The cooling water is not expected to fluctuate this much 
The cooling water was sourced from to enhance energy recovery in the plant while

Traditional methods of using 
rigorous modelling


\begin{figure}[h]
    \centering
    \includegraphics[width=0.7\linewidth]{chapters/0-executive-summary/figures/FYD executive sum.PNG}
    \caption{Mechanical design of nitration reactor}
    \label{fig:executivesummaryreactor}
\end{figure}