\section*{Reactor design}
\begin{figure}[h]
    \centering
    % \includegraphics[width=0.49\linewidth]{}
    \caption{Mechanical design of nitration reactor}
    \label{fig:comsol-S4-CW-X-T}
\end{figure}

Transitioning from batch to continuous process requires a complete redesign of the reactors that are commonly used in the industry today. Nitroma developed a novel approach of using a Heat Exchanger Reactor (HEX Reactor) for the nitration of toluene.

-steps taken to model (kinetics, Brinkman) 
- SiC immobilise

The model was implemented and optimised on COMSOL 5.6. To ensure the reactor operates in optimal conditions, the following parameters were optimised: Concentric inner cooling pipe diameter and flowrate, direction of cooling water flow, \ch{HNO3} : Toluene inlet ratio and the total number of tubes.

A further sensitivity analysis was performed on the cooling water inlet temperature because the temperature inside the reactor needs to be well-controlled. The cooling water was sourced from to enhance energy recovery in the plant while 

The length of the reactor was determined \mypm $\Delta$ 5 K

Traditional methods of using 
rigorous modelling

