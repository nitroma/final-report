% !TeX root = ../../main.tex
\section{Market analyses: PABA, PABH, OTOL}
\label{sec:market-analysis}
China is the world’s largest market for speciality chemicals, growing at \SI{5}{\percent} pa compared to the world average of \SI{3}{\percent} \cite{blad_game_nodate}. China’s continued urbanisation and the rise of its middle class has created long-term demand for speciality chemicals, despite setbacks from international trade friction and a slowdown in the country’s economic growth.

\paragraph{\para-aminobenzaldehyde (PABH)}
PABH is a speciality chemical that is mostly used as a precursor in the manufacture of dyes. China is the second fastest-growing dye market in the world with a CAGR of \SI{5.04}{\percent} (2020-24). The growth in the Chinese dye market is driven by the country’s rapidly growing packaging and printing industry, coming from the rise of e-commerce \cite{reportlinker_synthetic_2021}. The recent coronavirus pandemic has spurred this trend, with Chinese online retailers like Meituan Dianping and Alibaba seeing online sales increase by \SI{152}{\percent} and \SI{34}{\percent} year-on-year respectively in 2020 \cite{kharpal_chinas_2020}. As markets recover from the effects of coronavirus, Asia is expected to continue dominating the dye market across the world with the largest consumption from China \cite{. However, increased awareness surrounding the adverse health effects of chemical dyes is leading to a shift towards the production of natural dyes (ReportLinker 2021). The global market of PABH was found to be \$1.5 billion/year (ReportExpress 2020). China dominates the global production of PABH, with an estimated \SI{76}{\percent} of all PABH manufacturers located in the country (Molbase, Panjiva, ChemExper). Therefore, it is approximated that \SI{76}{\percent} of all PABH demand is also located in China. This results in a local demand of 28,\SI{000}{\tonne\per\year} using the average market price of PABH in China, detailed in Appendix \ref{app:market-share-calc}.

\paragraph{\para-aminobenzoic acid (PABA)}
PABA is a speciality chemical that is most commonly used as a pharmaceutical intermediate in the manufacture of pain relievers (Pubchem 2021). More broadly, a study in Current Medicinal Chemistry found that \SI{1.5}{\percent} of all drugs contain PABA (Kluczyk 2002). This information, along with the market size of pharmaceutical drugs in China, was used to estimate the domestic demand of PABA. The demand was predicted to be 7,\SI{400}{\tonne\per\year}, with detailed calculations found in Appendix \ref{app:market-share-calc}. The global CAGR of the PABA market between 2021-26 was found to be \SI{2.00}{\percent} (WBOC 2021). This was scaled up to a CAGR of \SI{3.33}{\percent} in China to reflect the country’s faster growth in the chemicals sector. As public health reforms take place in China, China’s share of the global pharmaceuticals market is expected to grow to \SI{30}{\percent} in 2030, second only to the United States (Allison 2021). This makes PABA a promising product for Nitroma. Moreover, the coronavirus pandemic has led to a 1000-fold increase in the demand for pain relievers around the world (Beroe 2021).

\paragraph{o-toluidine (OTOL)}
OTOL is used as a precursor in the manufacture of agrochemicals, specifically the herbicides metolachlor and acetochlor (NCBI 2021). Asia-Pacific, with China as its largest consumer, is the fastest growing herbicides market in the world with an expected CAGR of \SI{6.5}{\percent} between 2018 and 2025 (Sumant 2019). This CAGR can be extrapolated onto the OTOL market in China. The growth of the agrochemicals sector in China is driven by the country’s need to sustain a stable food supply for its 1.4 billion citizens. A lack of acceptance of modern farming techniques in China has hindered the adoption of herbicides in the past, but the trend is changing due to an increase in awareness campaigns (Mordor 2019). Meanwhile, the coronavirus pandemic has had little effect on farmers in China, as demand from grocery stores has remained strong (Mordor 2019).

Having a diversified product portfolio protects Nitroma from the consequences of an industry downturn affecting any one of its products. This means there is a reduction in the risk of financial loss if any of Nitroma's products perform poorer than anticipated.

\begin{table}[H]
\centering
\caption{Summary of the Chinese market for PABA, PABH and OTOL
}
\label{tab:market-summary}
\begin{tabular}{@{}ccclcc@{}}
\toprule
\multirow{2}{*}{\textbf{Product}} & \textbf{Market size}   & \multirow{2}{*}{\textbf{CAGR (\%)}}                         & \multicolumn{1}{c}{\multirow{2}{*}{\textbf{Top 3 customers}}}                                                                                                            & \textbf{Number of} & \textbf{Average}      \\
                                  & \textbf{(tonnes/year)} &                                                             & \multicolumn{1}{c}{}                                                                                                                                                     & \textbf{suppliers} & \textbf{market share} \\\midrule
PABA                              & 7,400.00               & \begin{tabular}[c]{@{}c@{}}3.33\\  (2021-26)\end{tabular}   & \begin{tabular}[c]{@{}l@{}}1. Yangzhou Chemical Co.\\ 2. Orichem International Ltd.\\ 3. Nanjing Jinhao Pharmaceutical\\     \;\;\;\;Science and Technology Co.\end{tabular}     & 72                 & \SI{1.4}{\percent}                 \\\midrule
PABH                              & 28,000.00              & \begin{tabular}[c]{@{}c@{}}5.04 \\ (2020-2024)\end{tabular} & \begin{tabular}[c]{@{}l@{}}1. Yangzhou Chemical Co.\\ 2. Orichem International Ltd.\\ 3. Zhejiang Rawa Group Co.\end{tabular}                                            & 57                 & \SI{1.8}{\percent}                 \\\midrule
OTOL                              & 380,000.00             & \begin{tabular}[c]{@{}c@{}}6.50 \\ (2018-2025)\end{tabular} & \begin{tabular}[c]{@{}l@{}}1. Jiangsu Alpha Hawk\\     \;\;\;\;Chemical Industries Ltd.\\ 2. Trustchem Co. Ltd.\\ 3. Jiangsu Changqing \\     \;\;\;\;Agrochemical Co. Ltd.\end{tabular} & 441                & \SI{0.2}{\percent}           \\\bottomrule     
\end{tabular}
\end{table}