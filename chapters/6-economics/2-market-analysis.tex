% !TeX root = ../../main.tex
\section{Market analyses: PABA, PABH, OTOL}
China is the world’s largest market for chemicals, growing at 5\% compared to the world average of 3\% (XX). In specific, the speciality chemicals market in China was worth \$162.8 billion in 2016 (XX) with a compound annual growth rate (CAGR) of 6\% between 2017-2022 (XX). China’s continued urbanisation and the rise of its middle class has created long-term demand for specialty chemicals, despite setbacks from international trade friction and a slowdown in the country’s economic growth.

\paragraph{p-aminobenzaldehyde (PABH)}
PABH is a speciality chemical that is mostly used as a precursor in the manufacture of dyes. China is the second fastest-growing dye market in the world with a CAGR of 5.04\% (2020-24). The growth in the Chinese dye market is driven by the country’s rapidly growing packaging and printing industry. This is mainly due to increased demand for toiletries, food, beverages, and other consumer goods stemming from the rise of e-commerce (XX). The recent coronavirus pandemic has spurred this trend, with Chinese online retailers like Meituan Dianping and Alibaba seeing online sales increase by 152\% and 34\% year-on-year respectively in 2020 (XX). As markets recover from the effects of coronavirus, Asia is expected to continue dominating the dye market across the world with the largest consumption from China and India (XX). However, the increased awareness surrounding the adverse health effects of chemical dyes is leading to a shift in manufacturers towards the production of natural dyes (XX). The global market of PABH was found to be \$1.5 billion/year (XX). China dominates the global production of PABH, with an estimated 76\% of all PABH manufacturers located in China (XX). Therefore, it is approximated that 76\% of all PABH demand is also located in China. This results in a local demand of 28,000 tonnes/year using the average market price of PABH in China.

\paragraph{p-aminobenzoic acid (PABA)}
PABA is a speciality chemical that is most commonly used as a pharmaceutical intermediate in the manufacture of topical pain relievers (XX). More broadly, a study in Current Medicinal Chemistry found that 1.5\% of all drugs contain PABA (XX). This information, along with the market size of pharmaceutical drugs in China, was used to estimate the demand of PABA in China. The demand was predicted to be 7,400 tonnes/year, with detailed calculations found in Appendix XX. The global CAGR of the PABA market between 2021-26 was found to be 2.00\% (XX). This was scaled up to a CAGR of 3.33\% in China to reflect the country’s faster growth in the chemicals sector. As public health reforms take place in China, China’s share of the global pharmaceuticals market is expected to grow to 30\% in 2030, secondly only to the United States (XX). This makes PABA a promising product for Nitroma. Moreover, the coronavirus pandemic has led to the increased usage of pain relievers around the world and in China (XX).

\paragraph{o-toluidine (OTOL)}