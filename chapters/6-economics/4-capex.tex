% !TeX root = ../../main.tex
\section{Detailed operating cost breakdown}
\subsection{Equipment cost}
Purchase cost of equipment is obtained from different resources. Towler\cite{sinnott_chemical_2020} has proposed the following correlation to calculate the buying price in year 2006.
\begin{equation}
    C_{e}=\ a+bS^n
\end{equation}
Where, a, b and n are constants depending on the equipment type and S is the key costing parameter of the equipment. Matche\cite{noauthor_matches_nodate} gives the price in year 2014 and Douglas\cite{douglas_conceptual_1988} provides the method for calculating cost in mid-1968. After applying a location factor and cost escalation factor, an average price from the three sources is taken to be the purchased price. However, as Nitroma employed new technologies such as suspension melt crystallisation, price for a few units cannot be extrapolated from cost correlations. In this case, similar unit price from different suppliers is compared and approximations are made where necessary. For the purpose of reducing corrosion, 316 stainless steel has been identified as the material for majority of equipment.

\begin{table}[H]
\centering
\caption{Equipment purchase cost}
\label{tab:equipment purchase}
\begin{tabular}{lll}
\toprule
\textbf{Equipment Type}    & \textbf{Number of Equipments} & \textbf{Purchase Cost (\$)} \\\midrule
Heat Exchangers            & 20                            & 143,784.99                  \\
Electric Heaters           & 13                            & 640,728.64                  \\
Reactors                   & 6                             & 271,879.49                  \\
Packed Columns             & 10                            & 99,443.17                   \\
Flash Vessels              & 3                             & 2,305.55                    \\
Gravitational Decanters    & 1                             & 4,493.06                    \\
Crystallizer               & 2                             & 42,503.76                   \\
Wash column                & 1                             & 16,668.24                   \\
Refregiration system       & 1                             & 37,935.77                   \\
Storage (Buffer) Tanks     & 12                            & 879,434.57                  \\
Mixers                     & 5                             & 82,320.88                   \\
Pumps                      & 10                            & 47,430.51                   \\
Fan                        & 4                             & 33,157.33                   \\
Granualtor                 & 2                             & 49,241.20                   \\
Waste Treatment Equipments & 3                             & 50,978.91                   \\ \hline
\textbf{Total}             & 93                            & \textbf{2,402,306.05}       \\ \bottomrule
\end{tabular}
\end{table}

Table \ref{tab:equipment purchase} summarized all major units purchased by Nitroma. Due to the variety of equipment types, detail breakdown of individual equipment cost is not discussed here but can be found in Supplementary Information.

The inside battery limits (ISBL) investment is the cost of purchasing and installing all equipment. Lang has proposed ISBL is a function of the installation factor and the purchase cost calculated above.
\begin{equation}
    ISBL=F\sigma C_{e}    
\end{equation}
As both solids and fluids are processed in Nitroma's plant, the installation factor is 3.63, leading to a ISBL of 8.72 million.
ISBL is calculated using various installation factors from Towler\cite{sinnott_chemical_2020}: 
\begin{equation}
    ISBL=\sigma C_{e}*\left(\left(1+f_{p}\right)\frac{f_{er}+f_{i}+f_{c}+f_{s}+f_{l}}{f_{m}}\right)
\end{equation}
\subsection{Additional capital costs}
Values of each installation factor can be found in Appendix

\subsection{CAPEX summary}