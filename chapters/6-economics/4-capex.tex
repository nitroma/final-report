% !TeX root = ../../main.tex
\subsection{Equipment cost}
Purchase cost of equipment is obtained from different resources. Towler\cite{sinnott_chemical_2020} has proposed the following correlation to calculate the buying price in year 2006.
\begin{equation}
    C_{e}=\ a+bS^n
\end{equation}
Where, a, b and n are constants depending on the equipment type and S is the key costing parameter of the equipment. Matche.com gives the price in year 2014 and Douglas provides the method for calculating cost in mid-1968. After applying a location factor and cost escalation factor, an average price from the three sources is taken to be the purchased price. However, as Nitroma employed new technologies such as suspension melt crystallisation, price for a few units cannot be extrapolated from correlations. In this case, similar unit price from different suppliers is compared and approximations are made where necessary. For the purpose of reducing corrosion, 316 stainless steel has been identified as the material for majority of equipment.




\subsection{Additional capital costs}
\subsection{CAPEX summary}