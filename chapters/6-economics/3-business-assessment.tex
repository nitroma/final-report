% !TeX root = ../../main.tex
\section{Corporate overview}
\subsection{Plant location}
AHP and TOPSIS methodologies were used to determine a suitable country for Nitroma’s chemical plant. The key factors taken into consideration are summarised in Table XX . China was identified as the optimal country for Nitroma's plant, owing to its strong local supply of toluene and low business operating costs. China is also the world’s fastest growing herbicides market and the second fastest growing dye and pharmaceutical market, resulting in a favourable demand of Nitroma’s products. To identify a suitable city within China, the spread of toluene suppliers, access to distribution channels, local market demand and business policies across the provinces of China were studied. The Nanjing Jiangbei New Material Science Park in Nanjing, Jiangsu was selected as the location of Nitroma’s plant. China’s largest toluene manufacturer, operated by Sinopec Yangzi Petrochemical, is situated within 10 km of Nitroma’s selected site in Jiangsu. The site is also located at a 40 km distance from the main Nanjing city, as per XX safety requirements. Moreover, the chosen location will allow Nitroma easy access to fresh water from the Jiajiang tributary of the Yangtze River less than 10 km away. Overall, the Jiangsu province hosts 20\% of China’s inland waterways and 4710 km of highways, creating reliable access to buyers within the province and large markets in Zhejiang and Shanghai. Moreover, Jiangsu operates a favourable policy towards foreign-owned and private businesses. A sensitivity analysis on key decision factors (detailed in Appendix XX) did not downgrade China as the optimal location.

\begin{table}[H]
\centering
    \caption{AHP/TOPSIS results for plant location selection}
    \label{tab:location-selection}\footnotesize
\adjustbox{max width=\textwidth}{
\begin{tabular}{l|ll|lll|ll|lll|l|l|llc}
%S[table-format=2.2]S[table-format=1.2]S[table-format=1.3]
\toprule
                                          & \multicolumn{2}{c|}{\splitcell{Local supply chain\\ (\SI{34}{\percent})}}                               & \multicolumn{3}{c|}{\splitcell{Country economics\\ (\SI{17}{\percent})}} & \multicolumn{2}{c|}{\splitcell{Trade\\ (\SI{4}{\percent})}}     & \multicolumn{3}{c|}{Operating costs (\SI{28}{\percent})}   & \splitcell{Competitive\\ Landscape (\SI{9}{\percent})} & \splitcell{Political\\ stability (\SI{7}{\percent})}  &     &                      \\ \cmidrule{2-13}
                                          
                                      & {\rcell{Toluene production (\si{\kilo\tonne\per\year})}} & {\rcell{Market size of products (\si[prefixes-as-symbols=false]{\giga\USD})}} & \rtext{Interest rate (\%)}  & \rcell{Corporate tax rate (\%)} & \rtext{Inflation (\%)} & \rtext{Import duties (\%)} & \rtext{Export duties (\%)} & {\rcell{Electricity cost (\si{\USD\per\kWh})}} & {\rcell{Minimum wage (\si{\USD\per\hour})}} & {\rcell{Cooling water cost (\si{\USD\per\l})}} &  {\rcell{Number competitors}} & {\rcell{Corruption perception index}} & AHP & TOPSIS & Rank \\ \midrule
India & 923       & 20 & 4.00     & 30       &  7.35          &     4.9      & 7.50 & 0.08 & 1.00 & 0.20 & 2                 & 41                & 0.193 & 0.536 &    3              \\ 
China & 4314          & 95 & 3.85  & 15     &       1.60     & 3.4           & 6.50 & 0.08 & 1.68 & 0.33 & 48                 & 41                 & 0.284 & 0.922 & \cellcolor{green}1  \\ 
USA     & 4992        & 501  & 0.25      & 21      &     1.20        & 1.6    & 5.25 & 0.15 & 7.30 & 1.53       & 157               & 69                 & 0.218 & 0.659 & 2 \\ 
Germany      & 930            & 47 & 0.00   & 30      & -0.30            & 1.7      & 6.50 & 0.38 & 10.97 & 2.46     & 33               & 80                   & 0.182         & 0.485 &  4          \\ 
Poland    & 285           & 7.9 & 0.10      &   19   & 12.60            & 2.5 & 6.50 & 0.19 & 3.35 & 0.93         & 0                 & 39                  &    0.123         & 0.290 &     5      \\ 
\bottomrule
\end{tabular}
} % adjustbox
\end{table}

\subsection{Mode of operation: Foreign-invested enterprise}
Nitroma will establish itself as a foreign-invested enterprise (FIE) in China, in line with the Foreign Investment Law 2020 (XX). As such, Nitroma will be incorporated under Chinese laws, with its equity investment coming from foreign investors and debt investment coming from local commercial banks (see Section XX). As an FIE, Nitroma will have independent management of its processes, making it easier to negotiate contracts with buyers under its CMO-styled operation and coordinate logistics with its suppliers. Being the innovator of a novel continuous liquid-phase nitration process, Nitroma will enjoy the benefit of owning intellectual property under its own name without sharing trade secrets with a local partner. Nitroma will not need to share profits either. 

In recent times, China has adopted an increasingly open stance towards foreign business as part of an effort to gain allies in its trade war against the US and globalize its domestic industries (XX). The 2020?? deal between INEOS and China’s state-run Sinopec to build a phenol/acetone?? plant in Nanjing, China is evidence of government interest in harnessing foreign capabilities for the enhancement of local industry (XX). Opportunities for favourable tax treatments also exist under new government policies in China, with the chemical raw material and products manufacturing industry being listed in the Ministry of Commerce’s Catalogue of Encouraged Foreign Investment Industries (XX). 

The risks associated with the decision to operate as an FIE include cultural obstacles, language barriers and lengthy wait times for governmental approvals. Purchasing bias towards wholly local suppliers also exists within the Chinese market. To tackle these risks, Nitroma will hire an entirely local labour force and implement an XX marketing strategy to build consumer trust.


\subsection{Product line capacity}
Nitroma’s plant capacity has been designed based on the local product demand and competitive landscape in China. In its first 5 years of operation, Nitroma aims to achieve the average industry market shares for P-ABA, P-ABH and O-TOL respectively. Based on the estimated 2020 market, this is equivalent to XX, XX and XX tonnes yr-1 of P-ABA, P-ABH and O-TOL respectively. The plant will operate for 300 days per year for XX years.

As an entrant into the Chinese chemicals industry where most buyers are locked into existing contracts with suppliers, Nitroma recognises the challenge of acquiring market share. Previous instances of failed market entries in China, like eBay’s expansion into the saturated Chinese electronics market, have made Nitroma wary of a large-scale launch in a highly competitive industry (XX). However, Nitroma believes it can capture the average industry market share in its first 5 years of operation due to its unique competitive advantage and pricing strategy (discussed in Section XX). Nitroma has chosen to initiate with a smaller capacity to minimise the risk of overproduction and consequent loss of profitability. As a foreign-invested enterprise, Nitroma also acknowledges the importance of utilising its first 5 years to develop rapport and build a local distribution network before expanding to a larger manufacturing scale. Details of plant expansion are discussed in Section XX.

The annual plant operating time was chosen as 300 days because……

\subsection{Pricing strategy}
Nitroma’s customer base is made up of price-sensitive dye, pharmaceutical, and agrochemical manufacturers who look for the cheapest supplier of raw material. Therefore, Nitroma will adopt a penetrative pricing strategy to enter the chemicals market in China. This is a strategy designed to quickly attract customers to Nitroma by offering lower prices and secure market share (XX). Being a high CAPEX business, a penetrative pricing strategy will also ensure that Nitroma is able to timely meet cash flow targets in order to make loan repayments and achieve its projected payback period. Furthermore, this strategy is expected to create goodwill between Nitroma and its buyers, who are likely to develop brand loyalty after discovering a high-value bargain (XX). Brand image or perceived brand quality are not important buying criteria for these customers, so a low price point will not damage Nitroma’s sales. Moreover, the strategy will be effective because there is no product feature differentiation that may lure buyers to competitors – price is the only point of differentiation. 

To obtain the average market price of P-ABA, P-ABH and O-TOL for order quantities above 100 tonnes/year, quotes from various Chinese suppliers were averaged (detailed in Appendix XX).  Nitroma will price its three products 12.5\% lower than their respective average market prices, equivalent to XX for XX, XX for XX and XX for XX. This price cut was chosen because it is the largest value that will not change Nitroma’s MCDM-based product selection (Section XX) and will still allow Nitroma to be priced above the cheapest supplier, preventing the likelihood of a price war. Moreover, a preliminary EP1 analysis showed that Nitroma is able to generate a healthy profit margin of XX\% with this price cut – a positive indicator with reference to the gross margins of incumbents like Sinopec (16.1\%). 

However, it must be noted that Nitroma faces the risk of market backlash and reduced sales volumes in the face of price hikes as buyers may begin to expect permanently low prices. However, this is unlikely as Nitroma will only increase prices in-line with increases in the Chinese Consumer Price Index (CPI) – a measure of the average change in the prices paid by consumers for a basket of consumer goods. This is not profit-motivated. Further income escalation details are discussed in Section XX.
