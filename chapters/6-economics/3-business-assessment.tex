% !TeX root = ../../main.tex
\section{Corporate overview}
\subsection{Plant location}
AHP and TOPSIS methodologies were used to determine a suitable country for Nitroma’s chemical plant. The key factors taken into consideration are summarised in Table XX . China was identified as the optimal country for Nitroma's plant, owing to its strong local supply of toluene and low business operating costs. China is also the world’s fastest growing herbicides market and the second fastest growing dye and pharmaceutical market, resulting in a favourable demand of Nitroma’s products. To identify a suitable city within China, the spread of toluene suppliers, access to distribution channels, local market demand and business policies across the provinces of China were studied. The Nanjing Jiangbei New Material Science Park in Nanjing, Jiangsu was selected as the location of Nitroma’s plant. China’s largest toluene manufacturer, operated by Sinopec Yangzi Petrochemical, is situated within 10 km of Nitroma’s selected site in Jiangsu. The site is also located at a 40 km distance from the main Nanjing city, as per XX safety requirements. Moreover, the chosen location will allow Nitroma easy access to fresh water from the Jiajiang tributary of the Yangtze River less than 10 km away. Overall, the Jiangsu province hosts 20\% of China’s inland waterways and 4710 km of highways, creating reliable access to buyers within the province and large markets in Zhejiang and Shanghai. Moreover, Jiangsu operates a favourable policy towards foreign-owned and private businesses. A sensitivity analysis on key decision factors (detailed in Appendix XX) did not downgrade China as the optimal location.

\begin{table}[H]
\centering
    \caption{AHP/TOPSIS results for plant location selection}
    \label{tab:location-selection}\footnotesize
\adjustbox{max width=\textwidth}{
\begin{tabular}{l|ll|lll|ll|lll|l|l|llc}
%S[table-format=2.2]S[table-format=1.2]S[table-format=1.3]
\toprule
                                          & \multicolumn{2}{c|}{\splitcell{Local supply chain\\ (\SI{34}{\percent})}}                               & \multicolumn{3}{c|}{\splitcell{Country economics\\ (\SI{17}{\percent})}} & \multicolumn{2}{c|}{\splitcell{Trade\\ (\SI{4}{\percent})}}     & \multicolumn{3}{c|}{Operating costs (\SI{28}{\percent})}   & \splitcell{Competitive\\ Landscape (\SI{9}{\percent})} & \splitcell{Political\\ stability (\SI{7}{\percent})}  &     &                      \\ \cmidrule{2-13}
                                          
                                      & {\rcell{Toluene production (\si{\kilo\tonne\per\year})}} & {\rcell{Market size of products (\si[prefixes-as-symbols=false]{\giga\USD})}} & \rtext{Interest rate (\%)}  & \rcell{Corporate tax rate (\%)} & \rtext{Inflation (\%)} & \rtext{Import duties (\%)} & \rtext{Export duties (\%)} & {\rcell{Electricity cost (\si{\USD\per\kWh})}} & {\rcell{Minimum wage (\si{\USD\per\hour})}} & {\rcell{Cooling water cost (\si{\USD\per\l})}} &  {\rcell{Number competitors}} & {\rcell{Corruption perception index}} & AHP & TOPSIS & Rank \\ \midrule
India & 923       & 20 & 4.00     & 30       &  7.35          &     4.9      & 7.50 & 0.08 & 1.00 & 0.20 & 2                 & 41                & 0.193 & 0.536 &    3              \\ 
China & 4314          & 95 & 3.85  & 15     &       1.60     & 3.4           & 6.50 & 0.08 & 1.68 & 0.33 & 48                 & 41                 & 0.284 & 0.922 & \cellcolor{green}1  \\ 
USA     & 4992        & 501  & 0.25      & 21      &     1.20        & 1.6    & 5.25 & 0.15 & 7.30 & 1.53       & 157               & 69                 & 0.218 & 0.659 & 2 \\ 
Germany      & 930            & 47 & 0.00   & 30      & -0.30            & 1.7      & 6.50 & 0.38 & 10.97 & 2.46     & 33               & 80                   & 0.182         & 0.485 &  4          \\ 
Poland    & 285           & 7.9 & 0.10      &   19   & 12.60            & 2.5 & 6.50 & 0.19 & 3.35 & 0.93         & 0                 & 39                  &    0.123         & 0.290 &     5      \\ 
\bottomrule
\end{tabular}
} % adjustbox
\end{table}

\subsection{Mode of operation: Foreign-invested enterprise}
Nitroma will establish itself as a foreign-invested enterprise (FIE) in China, in line with the Foreign Investment Law 2020 (XX). As such, Nitroma will be incorporated under Chinese laws, with its equity investment coming from foreign investors and debt investment coming from local commercial banks (see Section XX). As an FIE, Nitroma will have independent management of its processes, making it easier to negotiate contracts with buyers under its CMO-styled operation and coordinate logistics with its suppliers. Being the innovator of a novel continuous liquid-phase nitration process, Nitroma will enjoy the benefit of owning intellectual property under its own name without sharing trade secrets with a local partner. Nitroma will not need to share profits either. 

In recent times, China has adopted an increasingly open stance towards foreign business as part of an effort to gain allies in its trade war against the US and globalize its domestic industries (XX). The 2020?? deal between INEOS and China’s state-run Sinopec to build a phenol/acetone?? plant in Nanjing, China is evidence of government interest in harnessing foreign capabilities for the enhancement of local industry (XX). Opportunities for favourable tax treatments also exist under new government policies in China, with the chemical raw material and products manufacturing industry being listed in the Ministry of Commerce’s Catalogue of Encouraged Foreign Investment Industries (XX). 

The risks associated with the decision to operate as an FIE include cultural obstacles, language barriers and lengthy wait times for governmental approvals. Purchasing bias towards wholly local suppliers also exists within the Chinese market. To tackle these risks, Nitroma will hire an entirely local labour force and implement an XX marketing strategy to build consumer trust.


\subsection{Product line capacity}
Nitroma’s plant capacity has been designed based on the local product demand and competitive landscape in China. In its first 5 years of operation, Nitroma aims to achieve the average industry market shares for P-ABA, P-ABH and O-TOL respectively. Based on the estimated 2020 market, this is equivalent to XX, XX and XX tonnes yr-1 of P-ABA, P-ABH and O-TOL respectively. In its first 5 years, the plant will operate for 300 days per year.

As an entrant into the Chinese chemicals industry where most buyers are locked into existing contracts with suppliers, Nitroma recognises the challenge of acquiring market share. Previous instances of failed market entries in China, like eBay’s expansion into the saturated Chinese electronics market, have made Nitroma wary of a large-scale launch into a highly competitive industry (XX). However, Nitroma believes it can capture the average industry market share in its first 5 years of operation due to its unique competitive advantage and pricing strategy (discussed in Sections XX and XX). Nitroma has chosen to initiate with a smaller capacity to minimise the risk of overproduction and consequent loss of profitability. Being a foreign-invested enterprise, Nitroma also acknowledges the importance of utilising its first 5 years to develop rapport and build a local distribution network before expanding to a larger manufacturing scale. Details of plant expansion in line with the market growth of P-ABA, P-ABH and O-TOL are discussed in Section XX.

Out of its 300 operational days in Year 1, Nitroma will have 275 effective days of production. This number has been carefully chosen to prevent overproduction based on market demand and Aspen modelling. The in-built plant modularity will allow Nitroma to switch between the production of P-ABA (35 days/year) and P-ABH (240 days/year), whilst O-TOL will always be produced (275 days/year). As suggested by Coulson & Richardson, an additional 25 days per year are dedicated to planned shut-downs for maintenance, inspection, and equipment cleaning (XX). 

\subsection{Pricing strategy}
Nitroma’s customer base is made up of price-sensitive dye, pharmaceutical, and agrochemical manufacturers who look for the cheapest supplier of raw material. Therefore, Nitroma will adopt a penetrative pricing strategy to enter the chemicals market in China. This strategy is designed to quickly attract customers to Nitroma and secure market share by offering lower prices (XX). Being a high CAPEX business, a penetrative pricing strategy will also ensure that Nitroma is able to rapidly generate cash flow in order to meet its loan repayment targets and achieve its projected payback period. Furthermore, this strategy is expected to create goodwill between Nitroma and its buyers, who are likely to develop brand loyalty after discovering a high-value bargain (XX). Brand image or perceived brand quality are not important buying criteria for Nitroma’s customers, so a low price point will not damage Nitroma’s sales. Moreover, the strategy is expected to be effective because there is no differentiation in product features that may lure buyers to competitors.

To obtain the average market price of P-ABA, P-ABH and O-TOL for order quantities above 100 tonnes/year, quotes from various Chinese suppliers were averaged (detailed in Appendix XX).  Nitroma will price its three products 12.5\% lower than their respective average market prices, equivalent to XX for XX, XX for XX and XX for XX. This price cut was chosen because it is the largest value that will not change Nitroma’s MCDM-based product selection (Section XX) and will still allow Nitroma to be priced above the cheapest supplier, reducing the likelihood of a price war. Moreover, a preliminary EP1 analysis showed that Nitroma is able to generate a healthy profit margin of XX\% with this price cut – a positive indicator with reference to the gross margins of incumbents like Sinopec (16.1\%). 

It must be noted that Nitroma faces the risk of market backlash in the face of price hikes as buyers may begin to expect permanently low prices. However, this is unlikely as Nitroma will only increase prices in-line with increases in the Chinese Consumer Price Index (CPI) – a measure of the average change in the prices paid by consumers for a basket of consumer goods. This move would not be profit-motivated. Further income escalation details are discussed in Section XX.

\subsection{Unique selling points}
\paragraph{Cheap product}
Due to the continuous nature of Nitroma’s nitration process, the company will be able to utilise smaller equipment than traditional batch nitration operators that are prevalent in China (XX). Nitroma will also require lesser utilities to support the nitration process as plant downtime and process inefficiencies will be reduced due to the continuous nature of the process (XX). Therefore, Nitroma will have a lower CAPEX and OPEX than its competitors. Furthermore, Nitroma will enjoy the benefit of a reduced corporate tax rate of 15\% because of operating as a foreign-invested company in a government-encouraged industry. As a result, Nitroma will be able to undercut its competitors on product price as discussed in Section XX.
\paragraph{Reliable supply}
Following recent disasters like the 2019 deadly explosion in Chenjiagang Industrial Park, growing concerns around plant safety have led Chinese authorities to introduce upgraded safety legislation (XX). For example, one policy has updated a 19-year-old manufacturing standard with new requirements for operating conditions on chemical sites (XX). Incumbents and newcomers in the Chinese chemicals market must adapt fast to survive the challenging phase of increased regulatory control. In 2019, Jiangsu shut down 9 chemical parks (equivalent to 244 factories) due to safety risks (XX). Nitroma seeks the reward of fulfilling unmet demand as many of its competitors are removed from the market. Nitroma’s inherently safe design means it is not threatened by closures, making it a more reliable supplier for buyers to enter long-term purchasing contracts with.

\subsection{Business timeline}
\subsection{External industry analysis: Porter's 5 Forces}
Intro....
\paragraph{Threat of new entrants}
The average capital cost of a nitration plant is XX, creating a steep barrier to entry (XX). In order for chemical manufacturers to benefit from economies of scale, it is necessary for the process to be large scale. For example, the total annual production of Nitroma’s product line is over 1200 tonnes/year. This makes it difficult for entrants to challenge an incumbent’s market position. It is also difficult for entrants to make a differentiable chemical product. The nitration process itself, however, can be uniquely designed to enhance efficiency, safety, or cost savings. A high-level of technical expertise and industry know-how also prevents entrants from achieving an economical plant design. 

\begin{table}[H]
\centering
    \caption{Force 1: Threat of new entrants}
    \label{tab:force1}\footnotesize
\adjustbox{max width=\textwidth}
\begin{tabular}{lllll}
\hline
\textbf{Factor} & Upfront investment    & Economies of scale    & Product differentiation & Technical knowledge   \\
\textbf{Score}  & \multicolumn{1}{c}{1} & \multicolumn{1}{c}{2} & \multicolumn{1}{c}{3}   & \multicolumn{1}{c}{2} \\ \hline
\end{tabular}
\end{table}

\paragraph{Threat of substitute products}
Substitutes to Nitroma’s product line are limited because customers search for exact chemicals for their process designs. For example, o-toluidine is a specified essential essential raw material in the production of the herbicide metolachlor (XX). Changing raw material involves intensive R&D along with a redesign of the process synthesis route, reactor units and downstream separation. This creates significant switching costs for the customer. Alternate chemicals perform well in producing alternate dyes, agrochemicals, and pharmaceuticals, such as XX could be used to produce XX. Therefore, the threat of substitute products would only become a strong force in the situation where customers switch their end product also. The average market price of these substitutes falls in the same range as the average market price of Nitroma’s product line, such as m-toluidine which costs \$8.90/kg (XX). However, Nitroma will undercut the market will lower prices, as discussed in Section XX, eliminating the threat of cheaper substitutes. 

\paragraph{Bargaining power of buyers}
The customers of Nitroma’s product line consist of XX firms within the textile, pharmaceutical and agrochemical industries in China, representing a total annual market of XX (XX). Purchasing volume is spread evenly across firms within these target industries, preventing one big buyer from dictating the terms of purchase. Therefore, these customers carry little bargaining power. Nitroma’s three products are not differentiated from its competitors, meaning that customer loyalty is low beyond the scope of a contract and customers are sensitive to price. Moreover, customers are well-researched about the product because it serves as a raw material in their respective industry – this makes them better equipped to challenge unfair prices. Finally, buyers face low switching costs as there are numerous competitors producing the same products in China, as detailed in Table XX.

\subsection{Business risks: PESTEL}
\subsection{SWOT analysis}