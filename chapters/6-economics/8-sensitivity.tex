% !TeX root = ../../main.tex
\section{Sensitivity analysis of KPIs}
A sensitivity analysis on 5 key parameters was conducted to determine the robustness of Nitroma’s business plan and economic appraisal. NPV and payback period were selected as the KPIs for detailed analysis, whilst the results for IRR are summarised in Appendix XX. 

\subsection{Sensitivity to WACC}
WACC is an important variable because it considers the cost of equity, the cost of debt and corporate tax rate. Nitroma’s cost of equity may change if China’s risk-free rate changes – a likely scenario as China attempts to revive the economy following the coronavirus pandemic (XX). Similarly, the cost of debt may change depending on Nitroma’s creditworthiness over its lifetime.  Figure XX displays the effects of changing WACC on NPV and payback period. As WACC increased, the NPV of the project decreased. Even with a 30\% increase in WACC, the project NPV decreased by roughly 30\% to a health value of \$61.2 million. In fact, the WACC would need to increase by 108\% for NPV to become negative - an unrealistic scenario. WACC does not affect payback period since payback period does not account for discounted cash flows, as discussed in Section \ref{sec:pby}. 

\subsection{Sensitivity to CAPEX}
Nitroma’s CAPEX was estimated using the methods suggested by Towler & Sinnott and XX. Although these methods were averaged, CAPEX is a difficult factor to project until direct quotes from manufacturers cannot be obtained. Therefore, it important to conduct a sensitivity on this. As expected, CAPEX and NPV are inversely related. A 30\% increase in CAPEX resulted in a 9\% decrease in NPV to \$77.2 million. At the same time, the project payback period increased by less than a year to 6.1 years. It was found that CAPEX needs to increase 3.5x in order for the project NPV to become infeasible. This would be paired with an increase of payback period to 14.8 years – still less than the lifetime of the plant.

\subsection{Sensitivity to price of products}
Nitroma’s revenue is entirely dependent on the price of its products, making it an important variable to test. It was found that a 30\% decrease in product price resulted in NPV decreasing by 95\% to \$4.6 million and the payback period increasing 3x to 15 years. Therefore, the feasibility of the project was deemed extremely sensitive to product price.  If the product price fell by 33\%, the project NPV became \$0. However, this scenario is unlikely as it implies a simultaneous drop of 33\% across all 3 of Nitroma’s product prices. Nitroma’s diverse product line exposes the company to 3 different sectors – it is designed to protect the business from industry fluctuations. Moreover, analyst forecasts have predicted strong growth for these sectors, as discussed in Section XX, which should provide further confidence to shareholders. In fact, the price of pain relief drugs like paracetamol increased by 62\% following the pandemic which has a direct correlation with the price of PABA, a pain relief intermediate.


\subsection{Sensitivity to plant commencement}



\subsection{Sensitivity to BH:BA days}