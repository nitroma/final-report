% !TeX root = ../../main.tex
\section{Sensitivity analysis of KPIs}
A sensitivity analysis on 5 key parameters was conducted to determine the robustness of Nitroma’s business plan and economic appraisal. NPV and payback period were selected as the KPIs for detailed analysis, whilst the results for IRR are summarised in Appendix XX. 

\subsection{Sensitivity to WACC}
WACC is an important variable because it considers the cost of equity, the cost of debt and corporate tax rate. Nitroma’s cost of equity may change if China’s risk-free rate changes – a likely scenario as China attempts to revive the economy following the coronavirus pandemic (XX). Similarly, the cost of debt may change depending on Nitroma’s creditworthiness over its lifetime.  Figure XX displays the effects of changing WACC on NPV and payback period. As WACC increased, the NPV of the project decreased. Even with a 30\% increase in WACC, the project NPV decreased by roughly 30\% to a health value of \$61.2 million. In fact, the WACC would need to increase by 108\% for NPV to become negative - an unrealistic scenario. WACC does not affect payback period since payback period does not account for discounted cash flows, as discussed in Section \ref{sec:pby}. 

\subsection{Sensitivity to price of products}
\subsection{Sensitivity to plant commencement}
\subsection{Sensitivity to CAPEX}
\subsection{Sensitivity to BH:BA days}