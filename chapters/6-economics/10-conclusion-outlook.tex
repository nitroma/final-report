% !TeX root = ../../main.tex
\section{Conclusion}
Nitroma aims to capture a \SI{1.4}{\percent}, \SI{1.9}{\percent} and \SI{0.2}{\percent} share of the PABA, PABH and OTOL markets in China respectively. To combat the highly competitive speciality chemicals environment, Nitroma will leverage the financial benefit of employing a novel continuous nitration process to undercut the average market price of its competitors’ products by \SI{12.5}{\percent}. Nitroma also seeks the reward of fulfilling unmet demand as many of its competitors are removed from the market due to unsafe practices. Here, Nitroma promises its buyers a stable product supply due to its inherently safer continuous nitration process. To retain a stable market position over its lifetime, Nitroma will expand its operating capacity by 10 days per year every 5 year until decommissioning in 2043.

The total capital investment required for the nitration plant is estimated to be \$22.5 million. Nitroma will sell a \SI{45}{\percent} stake in its business to private equity investors for \$10.3m and receive a \$12.3m loan from local commercial banks at a \SI{5.35}{\percent} interest rate. Operating expenses are calculated to be \$18.2 million per year.  Results from Nitroma’s balance sheet and income statement produced a series of KPIs which were used assess the economic feasibility of Nitroma’s business plan. The project was deemed an attractive investment, evidenced by an NPV of \$84.4 million, an IRR of \SI{31}{\percent} (4x higher than the project’s hurdle rate) and a payback period of 5.3 years (\SI{37}{\percent} lower than the industry average).

To test the robustness of the business, sensitivity tests were performed on five major parameters contributing to Nitroma’s KPIs. Product price was identified as the most sensitive factor, with a simultaneous \SI{33}{\percent} price drop across all of Nitroma's products resulting in an NPV of \$0. However, the diversity in Nitroma’s product line is specifically designed to protect the business against such market fluctuations. Moreover, Nitroma undercuts the average market price of its competitors so the effect of this factor is mitigated. Different economic scenarios were also modelled as a strength test. One particular scenario was related to events where COVID-19 resurfaced in Nanjing, China, and restriction measures were in place till 2026, resulting in a production delay and reduced operational days. Across all tested scenarios, the minimum ending cash balance was \$23.1 million and the longest payback period was 10.3 years. These results show the robustness of this project and should enhance the confidence of the shareholders.