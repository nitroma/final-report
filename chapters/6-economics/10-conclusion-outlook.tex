% !TeX root = ../../main.tex
\section{Conclusion}
Nitroma aims to capture a 1.4\%, 1.9\% and 0.2\% share of the PABA, PABH and OTOL market respectively. To combat the highly competitive speciality chemicals environment in China, Nitroma will leverage the financial benefit of employing a novel continuous nitration process to undercut the average market price of its competitors’ products by 12.5\%. Nitroma also seeks the reward of fulfilling unmet demand as many of its competitors are removed from the market due to unsafe practices. Here, Nitroma promises its buyers a stable product supply due to its inherently safer continuous nitration process. To retain a stable market position over its lifetime, Nitroma will expand its operating capacity by 10 days per year every 5 year until decommissioning in 2043.

The total capital investment required for the nitration plant is estimated to be \$22.5 million, which compares closely with the average capital expense XXXX (XX). Nitroma will sell a 45\% stake in its business to private equity investors for \$10.3m and receive a \$12.3m loan from local commercial banks at a 5.35\% interest rate. Operating expenses are calculated to be \$18.2 million per year, comparatively lower than traditional batch nitration processes (XX).  Results from Nitroma’s balance sheet and income statement produced a series of key performance indicators which were used determine the economic feasibility of Nitroma’s business plan. The project is deemed an attractive investment, evidenced by an NPV of \$84.4 million, an IRR of 31\% (4x higher than the project’s hurdle rate) and a payback period of 5.3 years (37\% lower than the industry average (XX)).

To test the robustness of the business, sensitivity tests were performed on five major contributing factors to Nitroma’s KPIs. Product price is identified as the most sensitive factor, with a simultaneous 33\% decrease in the prices of all products resulting in an NPV of \$0. However, the diversity in Nitroma’s product line is specifically designed to protect the business from such market fluctuations. Moreover, Nitroma undercuts the average market price of its competitors so the effect of this factor is mitigated. Different economic scenarios were also modelled as a strength test. One particular scenario was related to events where COVID-19 resurfaced in Nanjing, China, and strong restriction measures were in place till 2026, resulting in delay in production and reduction in operational days. Across all tested scenarios, the minimum ending cash balance was \$23.1 million and the longest payback period was 10.3 years. These results show the robustness of this project and should enhance the confidence of the shareholders.