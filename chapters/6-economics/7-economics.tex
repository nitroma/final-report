% !TeX root = ../../main.tex
\section{Economic evaluation}
An economic appraisal was conducted to assess the financial feasibility of Nitroma’s continuous nitration process. To do this, Nitroma’s income statement and balance sheet were generated over the plant’s 20-year lifetime (found in Section \ref{sec:economics-supporting}). Information gathered from the statements was used to evaluate several key performance indicators (KPIs).

\subsection{Key assumptions}

 \begin{table}[h]
    \vspace{-\intextsep}
    \centering
        \caption{Key assumptions made by Nitroma for economic analysis}
        \label{tab:assumptions-econ}
    \begin{tabular}{@{}lp{15cm}@{}}
    \toprule
   \textbf{1} & The operating lifetime of the plant is 20 years, commencing from 2023 once construction and testing of the plant is complete.   \\ 
   \textbf{2}      & 30\% of annual operating expenses are incurred during the testing phase in 2022, with no revenue generated.   \\
   \textbf{3}     & All operating costs incurred by Nitroma escalate by a factor of 2.6\% every year based on the inflation rate in China \cite{statista_inflation_2021}.   \\
   \textbf{4}                   & The price of products sold by Nitroma escalates by a factor of 2.8\% every year based on the annual change in the Consumer Price Index of China (CEIC chin2021).   \\
    \textbf{5}         & The plant is depreciated using a straight-line method with a terminal value of \$0 due to the wear and tear of equipment following lengthy exposure to nitric acid, a raw material in Nitroma's process.    \\ 
    \textbf{6} & The number of operating days is increased by 10 days per year every 5 years, increasing the OPEX and revenue of Nitroma every 5 years.\\
    \textbf{7} & Interest  payments begin in 2021 and the loan term is 10 years, ending in 2031. \\
    \textbf{8} & The WACC rate of 8.28\% determined in Section \ref{sec:wacc} is used as the discount rate. \\
    \textbf{9} & Tax rate remains constants over the lifetime of the plant. \\
  \bottomrule
    \end{tabular}
    \end{table}

\subsection{Key performance indicators}
\label{sec:KPIs}

\subsubsection{Net present value (NPV)}
NPV is the value of all future cash flows over the lifetime of a project discounted to the present value. NPV is commonly used as an indicator of the profitability of a business and gives the present value of total profits. Equation XX displays the calculation of NPV (kenton npv):

\begin{equation}
\label{eqn:npv}
    NPV = \sum_t^n\frac{R_{t}}{(1+i)^{t}}
\end{equation}

Here, $R_{t}$ is the net cash flow at time $t$ and $i$ is the discount rate.

NPV is a comprehensive metric as it considers all revenue, expenses and capital costs associated with a project. Moreover, cash flows are discounted to adjust for the risk associated with the project and to account for the time value of money. The NPV for Nitroma’s plant was calculated to be \$84.4 million for a discount rate of \SI{8.28}{\percent} (WACC). This large value provides confidence that the projected earnings of this project will exceed its anticipated cost at the present value of currency. However, the main limitation of NPV arises from the inability to predict all future cash flows as many expenses arise only after a project starts. The herbicide, dye and pharmaceutical sectors are mature industries so it not expected that any disruptive forces would threaten Nitroma’s projected cash flows over the plant’s lifetime, as discussed in Section \ref{sec:five-forces}. A sensitivity analysis on NPV is detailed in Section \ref{sec:sensitivities-kpis}. 

\subsubsection{Internal rate of return}
IRR is the discount rate that makes the NPV of a project equal to zero. Therefore, it is the annual rate of growth a project can expect to achieve. To calculate IRR, \cref{eqn:npv} must be set to zero. Nitroma’s IRR was found to be \SI{31}{\percent}. This is almost 4x greater than the project’s hurdle rate project, indicating a safe investment – even with significant market shifts, Nitroma’s business can be expected to remain profitable. A sensitivity analysis on IRR is summarised in Appendix \ref{app:irr-sensitivity}.

%as displayed in \cref{eqn:irr}:
%\begin{equation}
%\label{eqn:irr}
%    \sum_t^n\frac{R_{t}}{(1+IRR)^{t}}=NPV=0
%\end{equation}


\subsubsection{Payback period}
\label{sec:pby}
Payback period is the amount of time it takes to recover the cost of an investment. It can be calculated as follows:

\begin{equation}
\label{eqn:payback}
    Payback\:period = \frac{Initial\:investment}{Annual\:net\:cash\:flow}
\end{equation}

Naturally, a shorter payback period is more attractive to investors. A major limitation of payback period is that it does not account for the time value of money. This makes the metric more inaccurate as its value becomes larger since future cash flows are subject to greater fluctuation. Nitroma’s payback period is estimated to be 5.3 years, which is lower than the industry average of 8.4 years (outlook money).  The short payback period is also more reliable because Nitroma’s projected cash flows over the next 5 years are less prone to inaccuracies. This makes Nitroma an attractive investment. 

\subsection{Additional KPIs}
    \begin{wraptable}{r}{0.6\linewidth}
    \centering
        \caption{Summary of Nitroma's KPIs and industry averages (damodaran,outlookmoney)}
        \label{tab:kpi-main-summary}
    \begin{tabular}{@{}lll@{}}
    \toprule
                             & Nitroma value & Industry average \\ \midrule
    \textbf{Main}            &               &                  \\
    Project IRR              & 31\%          & 13\%             \\
    Project NPV              & \$84,423,382  & -                \\
    Proroject payback period & 5.3 years     & 8.4 years        \\
    \textbf{Additional}      &               &                  \\
    Equity IRR               & 41\%          & -                \\
    Equity NPV               & \$85,780,890  & -                \\
    Equity payback period    & 4.0 years     & -                \\
    End of project cash      & \$273,992,540 & -                \\
    Gross profit margin      & 54\%          & 21.20\%          \\
    Net profit margin        & 33\%          & 3.86\%           \\
    Return on equity         & 18\%          & 5.15\%           \\
    Return on investment     & 58\%          & 7.29\%           \\ \bottomrule
    \end{tabular}
    \end{wraptable}
Table \ref{tab:kpi-main-summary} summarises all KPIs found related to the project. Equity NPV, IRR and payback period are also included to allow equity investors to better understand the value generation they can expect. 


\subsection{Cash flow analysis}
\label{sec:cash-flows}
Nitroma’s cash flow projections till 2043 can be found in Section \ref{sec:economics-supporting}. A visual graph can be seen in Appendix \ref{app:cash-flows-econ}. In 2021 and 2022, Nitroma generates a financial loss as registration, construction and trialling of the nitration plant takes place. Sales and distribution commence in 2023 and Nitroma reports its first positive cash flow in 2024. 

5-yearly capacity expansions of 10 days per year have also been factored into the cash flow statement. However, any additional investment that may be required to increase plant capacity has not been included in the financial statements as it is believed to have a minimal effect on Nitroma’s financial position.
%t should be noted that a cost escalation factor of \SI{2.6}{\percent} (XX) and an income escalation factor of \SI{2.8}{\percent} (XX) have been modelled into the cash flow statement to reflect the rise in Nitroma’s operating costs and product prices in line with the inflation rate and Consumer Price Index of China