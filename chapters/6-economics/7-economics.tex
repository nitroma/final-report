% !TeX root = ../../main.tex
\section{Economic evaluation}
An economic appraisal was conducted to assess the economic feasibility of Nitroma’s continuous nitration process. To do this, Nitroma’s income statement and balance sheet were generated over the plant’s 20-year lifetime (found in Section XX). Information gathered from the statements was used to evaluate the company’s annual cost of production and several key performance indicators (KPIs).

\subsection{Key assumptions}

\subsection{Key performance indicators}

\subsubsection{Net present value (NPV)}
NPV is the value of all future cash flows over the lifetime of an investment discounted to the present. In this project, WACC has been used as the discount rate. NPV is commonly used an indicator of the profitability of a business and gives the present value of total profits. A positive NPV shows that the forecasted earnings of a project exceed its anticipated cost at the present value of currency. Equation \ref{eqn:npv} displays the calculation of NPV:

\begin{equation}
\label{eqn:npv}
    \sum_t^n\frac{R_{t}}{(1+i)^{t}}
\end{equation}

NPV is a comprehensive metric as it considers all revenue, expenses and capital costs associated with an investment in its free cash flow. Moreover, cash flows are discounted to adjust for the risk of an investment opportunity and to account for the time value of money. However, the main limitation of NPV arises from the inability to predict all future cash flows as many expenses arise only after a project starts. The NPV for Nitroma’s plant was calculated to be \$84.4 million.

\subsubsection{Internal rate of return}

\subsection{Cash flow analysis}
