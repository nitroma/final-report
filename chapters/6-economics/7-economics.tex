% !TeX root = ../../main.tex
\section{Economic evaluation}
An economic appraisal was conducted to assess the economic feasibility of Nitroma’s continuous nitration process. To do this, Nitroma’s income statement and balance sheet were generated over the plant’s 20-year lifetime (found in Section XX). Information gathered from the statements was used to evaluate the company’s annual cost of production and several key performance indicators (KPIs).

\subsection{Key assumptions}

\subsection{Key performance indicators}

\subsubsection{Net present value (NPV)}
NPV is the value of all future cash flows over the lifetime of an investment discounted to the present. NPV is commonly used an indicator of the profitability of a business and gives the present value of total profits. Equation XX displays the calculation of NPV (XX):

\begin{equation}
\label{eqn:npv}
    \sum_t^n\frac{R_{t}}{(1+i)^{t}}
\end{equation}

NPV is a comprehensive metric as it considers all revenue, expenses and capital costs associated with a project in the project’s free cash flow. Moreover, cash flows are discounted to adjust for the risk associated with the project and to account for the time value of money. The NPV for Nitroma’s plant was calculated to be \$84.4 million for a discount rate of 8.28\% (WACC). This large value provides confidence that the forecasted earnings of this project will exceed its anticipated cost at the present value of currency. However, the main limitation of NPV arises from the inability to predict all future cash flows as many expenses arise only after a project starts. The agrochemical, dye and pharmaceutical sectors are mature industries so it not expected that any disruptive forces would threaten Nitroma’s projected cash flows over the plant’s lifetime, as discussed in Section XX. A sensitivity analysis on NPV is detailed in Section XX. 

\subsubsection{Internal rate of return}
IRR is the discount rate that makes the NPV of a project equal to zero. Therefore, it is the annual rate of growth a project can expect to achieve. To calculate IRR, Equation \ref{eqn:npv} must be set to zero as displayed in Equation \ref{eqn:irr}:
\begin{equation}
\label{eqn:irr}
    \sum_t^n\frac{R_{t}}{(1+IRR)^{t}}=NPV=0
\end{equation}

\subsection{Cash flow analysis}
