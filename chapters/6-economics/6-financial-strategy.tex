% !TeX root = ../../main.tex
\section{Financial strategy}
\subsection{Corporate tax rate}
The standard corporate income tax rate in China is 25\% (XX). However, the tax rate can be reduced to 15\% for qualified foreign-invested enterprises which are listed in the catalogue of industries encouraged by the Chinese Ministry of Commerce. Nitroma falls into the encouraged chemical raw material and products manufacturing industry, and so, a corporate tax rate of 15\% is used for economic analysis (XX). To promote on-site safety, the Chinese government also issues a 10\% subsidy on the purchase of waste disposal equipment which Nitroma will credit against tax payable annually (XX).

\subsection{Capital structure}
New projects require a source of capital to finance the construction of factories, hiring of labour and conducting of operational activities. There are two main forms of financing: debt and equity. Debt refers to capital that is borrowed from credit providers, such as a bank, and paid back over time with interest. Equity refers to capital that is raised by selling a stake in the company’s ownership. Table XX summarises some of the pros and cons of debt and equity.

Nitroma must carefully select a combination of debt and equity financing such that it is able to manage its annual cash flow and control of the business. The debt-to-equity ratio, D/E, of a company represents the split in the company’s finance sourcing and is calculated by dividing the company’s debt financing, D, by its equity financing, E. A high D/E is indicative of an aggressive capital structure that may demonstrate the company’s ability to make regular payments, but also creates a greater risk for investors. A low D/E ratio reduces the company’s risk of default but also results in A loss of ownership. Therefore, the trade-off between several important factors must be considered when deciding a capital structure. Generally, the optimal capital structure is one which minimises the weighted average cost of capital (WACC) by taking on a mix of debt and equity (XX).

The industry average D/E for the speciality chemicals sector was estimated to be 0.56, based on data collected from several chemical manufacturers across China (see Appendix XX). However, Nitroma has chosen a higher D/E ratio of 1.20 for several reasons. Low interest rates in China make debt financing a cheaper option. Low tax rates mean that creditors will demand lower returns as taxation will deduct less from investors’ interest returns. Debt financing will also allow Nitroma to retain control of its operations which is very important to operate in the complicated Chinese business environment as a foreign-invested enterprise. Furthermore, Chinese banks have eased loan requirements for small-medium business (SMB) to stimulate a stalled economy with year-on-year SMB lending increasing by 27.5\% in 2016 (XX). However, taking on too much debt will impose a strain on Nitroma’s cash flows from an early stage. Equity financing will allow Nitroma to retain cash flows which can be invested in R\&D to supports the company’s expansion goals.

Therefore, the company will seek to sell a XX\% stake to foreign equity investors in exchange for XX and issue a XX corporate bond at a 4.65\% coupon rate (XX).

\subsection{Cost of debt}
In China, the National Interbank Funding Centre sets the prime lending rate which is the interest rate charged to the most creditworthy clients (XX). This rate is used by commercial banks as a benchmark for determining interest rates charged to their clients. The current 5-year prime lending rate in China is 4.65\% (XX). However, to better reflect the risk associated with a start-up chemical plant, a 15\% correction factor is applied to determine the cost of debt, $R_{d}$. Therefore, $R_{d}$ is 5.35\%.

\subsection{Cost of equity}
The cost of equity, $R_{e}$, can be determined using the Capital Asset Pricing Model (CAPM) represented in Equation XX (XX). $R_{f}$ is the risk-free rate (theoretical rate of return of an investment with zero risk), $R_{m}$ is the average market return and $\beta$ is the beta of the investment (volatility of return relative to the market).
\begin{equation}
\label{eqn:capm}
    R_{e}=R_{f}+\beta(R_{m}-R_{f})
\end{equation}
The risk-free rate, $R_{f}$, was determined using the yield of a 10-year government bond in China, equivalent to 3.28\% (XX). The average market return, $R_{m}$, was based on the average yearly return of the Shanghai Composite Index between 2000 and 2018, equivalent to 12.76\% (XX). The industry average beta, $\beta$, for the speciality chemicals sector in China was found to be 0.99. Inputting these values into Equation XX, $R_{e}$ is found to be 12.76\%. This is slightly higher than the industry average of 7\% (XX).

\subsection{Weighted average cost of capital}
The WACC represents a company’s cost of capital in which each source of capital is proportionately weighed. Equation XX displays the calculation of WACC (XX):

\begin{equation}
    WACC=\left(\frac{E}{V}\times R_{e}\right)+\left(\frac{D}{V}\times R_{d}\times (1-T_{c})\right)
\end{equation}

$R_{e}$ is the cost of equity, $R_{d}$ is the cost of debt, Tc is the corporate tax rate, E is the market value of the firm’s equity, D is the market value of the firm’s debt, and V = E + D. Inputting values from Sections XX, XX and XX, the WACC is determined to be 8.20\%. 

This value can be used to determine how much interest a company owes for each dollar it finances. Therefore, a company’s WACC is the required return the company must earn on its asset base. 

\subsection{Hurdle rate}
Hurdle rate is the minimum rate of return on a project that is required by investors (XX). Therefore, it allows companies to make important decisions on whether or not to pursue a specific project. If the expected rate of return is above the hurdle rate, the investment is considered feasible. Generally, the hurdle rate of a project is equal to the WACC because both represent the minimum point where the company generates sufficient returns to offset the cost of capital. For Nitroma, the hurdle rate is 8.20\%. WACC is also used as the discount rate in determining the net present value (NPV) in Section XX.