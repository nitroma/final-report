\section{Conclusion and outlook}

The design of an inherently safe, fully continuous and multipurpose process for the nitration of toluene and subsequent reduction to substitute aromatic amines has been performed. MCDM methodologies were employed to determine the optimum combination of products to include in Nitroma's portfolio. Following nitration of toluene, \ortho-toluidine, 4-aminobenzaldehyde and 4-aminobenzoic acid will be produced in flexible quantities based on market demands. The synthesis routes for the three different products were also assessed with MCDM techniques: AHP was used to determine the weightings of each evaluation criteria based on Nitroma's priorities and TOPSIS analysis ranked the possible alternatives. Zeolite catalysed nitration was deemed favourable over the traditional mixed-acid process in regards of safety, environmental impact and economic considerations. 

The continuous process was developed following process intensification principles to improve the safety, environmental impact and performance of Nitroma's plant. A multifunctional shell-and-tube reactor was designed for nitration enabling safe and continuous operation. The potential for plant modularity was exploited by controlling the partial and complete oxidation of \para-nitrotoluene thus creating significant plant flexibility to market changes. Separation techniques were selected with the Jaksland quantitative analysis, based on thermodynamic properties of the compounds. Finally, heat integration of the utilities was carries out to improve the energy efficient of Nitroma's process.

The next steps for this work involve further experimental work at lab scale to minimise uncertainty and to corroborate the large-scale feasibility of Nitroma's continuous process. Moreover, mechanical design optimisation of the reactors and separation sequence will need to be performed and are presented in the next section of this report for selected units. To improve the performance and inherent safety of the process, a robust control system will be implemented. Finally Nitroma is committed to developing a greener process, and as such, selection and design of waste treatment systems will be carried out and reported in later sections.
