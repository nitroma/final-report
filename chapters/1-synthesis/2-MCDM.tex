\section{Multi-criteria decision analysis for process selection}

Chemical process development and especially synthesis route selection are regarded as the most critical conceptual design step since those major decisions affect the whole process lifecycle []. Locating one optimum solution from several non-dominating plausible alternatives is challenging due to the necessity of considering various, and often conflicting, criteria. To help stakeholders make systematic and informed decisions for key aspects of the process, multi-criteria decision making (MCDM) methods have been developed. Those tools provide a quantitative and comparative evaluation of alternatives meeting design requirements, thus minimising subjectivity and bias []. 
This section discusses Nitroma's selection procedure for the three products to be produced in its multi-purpose plant, and their associated synthesis routes. Employing both a qualitative approach and MCDM methods, greater importance was given to safety and environmental concerns, technical performance and economic potential of different options.

\nomenclature[A]{MCDM}{Multi-criteria decision making}

\subsection{Selection of MCDM} %~0.5 page

Numerous computer-based MCDM methods have been developed over the last few decades to respond to the need for systematic decision making tools. Based on different theories and assumptions, those methods have unique characteristics which render them more suited for certain areas of application. Several well-known MCDM methods were initially considered, including multi-attribute methods: Weighted Sum method (WSM), Multi-Attribute Utility Theory (MAUT) and Analytic Hierarchy Process (AHP); and outranking methods: Elimination and Choice Expressing Reality (ELECTRE),  Technique for the Order Preference by Similarity to Ideal Solution (TOPSIS) and Preference ranking organization method for enrichment evaluation (PROMETHEE) []. 

Due to inherent limitations of all methods, a combination of at least two methods will be used to ensure a more accurate decision is made []. AHP and TOPSIS were found to be the most suited methods after consideration of the advantages and disadvantages of each technique. Despite its straightforward application, WSM is not capable of handling multi-dimensional decision-making problems, thus leading to results always not reflecting the real situation []. AHP provides an easy to use and scalable hierarchical structure which can accommodate multiple options and criteria, but which is not as data intensive as MAUT []. Moreover, as opposed to MAUT which requires very precise input data, AHP can handle problems with  non-monetary criteria when limited information can be found in literature []. ELECTRE was eliminated due to its inapplicability to criteria with different ranges []. In addition to the difficulties in strength and weakness identification for alternatives, PROMETHEE does not provide a clear procedure to assign weights to criteria []. On the other hand, TOPSIS presents a simple process which is easy to use and program []. Nonetheless, AHP and TOPSIS have limitations which can be overcome by combining the two methods, as discussed in the following section.

\nomenclature[A]{WSM}{Weighted Sum method}
\nomenclature[A]{MAUT}{Multi-Attribute Utility Theory}
\nomenclature[A]{AHP}{Analytic Hierarchy Process}
\nomenclature[A]{ELECTRE}{Elimination and Choice Expressing Reality}
\nomenclature[A]{PROMETHEE}{Preference ranking organization method for enrichment evaluation }
\nomenclature[A]{TOPSIS}{Technique for Order of Preference by Similarity to Ideal Solution}

\subsection{Combined AHP and TOPSIS analysis} %~0.5 page

The decision-making process using MCDM methods follows six steps []: (1) identify the options to be appraised, (2) identify criteria to asses the performance of each option, (3) scoring each option against a criteria, (4) assign criteria weightings, (5) compute, analyse and compare the weighted scores of each options, and (6) perform sensitivity analysis to evaluate the robustness of the final decision.
Process alternatives were initially short-listed with qualitative arguments (step 1) and then ranked with the quantitative processing of all relevant information against selection criteria (step 2) through AHP and TOPSIS analyses (step 5). 

AHP decomposes complex problems into structure hierarchies to find a unique optimum solution meeting the overall objective []. The analysis focus is broken down in categories made of several criteria which can be quantified with information available in literature. The categories and the criteria are independently compared pairwise using a 1-9 scale to assess the relative signification of one category/criteria over another \cite{saaty_analytic_1987}. Following the procurement of a set of category and criterion weights with pairwise comparison, the performance of each alternative is computed and a final ranking of solutions is obtained. The option with the highest sum of normalised scores is recommended and the weightings are checked for inconsistencies to ensure a robust choice. Several flaws of AHP have been identified, notably the ‘rank reversal’ phenomenon which  Introducing new options can change the relative ranking of some of
the original options. 



 %  Thus, it can provide evaluation scores for integrated quantitative analysis. However, it is unable to assess its strengths and weakness in isolation. Nonetheless, it remains very applicable in weight estimation for different criteria and group decision making49.










%TOPSIS is a method that compares a set of options by identifying weights for each criterion, normalising the scores for each criterion, and calculating the Euclidean distance between each option, the positive ideal solution (PIS) and the negative ideal solution (NIS). PIS is defined by combining all the optimum values from each criterion (Yue, 2012) and NIS is defined by combining all the least optimum values from each criterion. The options are then ranked such that the best solution has the shortest Euclidean distance from the PIS and the farthest from the NIS (Amine, Pailhes & Perry, 2014).

%Since AHP requires human judgement for the pairwise comparison, the elements’ relative significance is prone to inconsistency and bias. This is addressed by in-built consistency checks and using TOPSIS as a quantitative reinforcement. Additionally, by normalising all the quantitative indicators from 1 to 9, the subjectivity can be minimised. Conversely, TOPSIS does not quantitatively assign criteria weighting nor apply any internal consistency checks (Roszkowska, 2011), both of which are covered by AHP. By applying both methods and comparing the results, the disadvantages of each method can be mitigated.



%AHP is combined with TOPSIS to assess the optimal synthesis route for diazomethane production. TOPSIS computes the positive and negative ideal solutions by combining the best and worst performances respectively of each criterion23,32. Monotonically changing utilities were assumed for all criteria, in which utility is indicative of the overall satisfaction of the decision making-group or performance of the plant. For monotonically changing utilities, higher scores in beneficial criteria and lower scores in detrimental criteria are always associated with more utilities, and vice versa. Normalised scores are then assigned to each criterion for the different synthesis routes. The route options are then ranked by computing the distance between the score and the point of ideal solution in a geometric space. The chosen synthesis route would have a minimal distance from positive ideal solution as well as maximum distance from the negative ideal solution. This is a suitable method for deciding the optimum synthesis route due to its ability to allow for trade-offs between criteria.


% Firstly, pairwise comparison of the criteria was performed with the Analytical Hierarchy Process (AHP) to produce criteria weightings which were fed into the Technique for Order of Preference by Similarity to Ideal Solution (TOPSIS) analysis. An initial sensitivity analysis was conducted on the results, and a more robust method will be developed in the next stage of the project.



\subsection{Sensitivity} %~0.5 page

\subsection{Products selection}

To deliver a multi-purpose plant, Nitroma selected toluene as feedstock due to the industrial importance of its nitration products and their amino derivatives. All three isomeric nitrotoluenes can indeed easily be oxidised and reduced to substituted aromatic amides of significant importance for the pharmaceutical, agrochemical and textile industries \cite{dugal_nitrobenzene_2005}. Six chemicals were initially selected due to their applications in drugs, fertilizers or dyes: 2-aminobenzaldehyde (precursor of quinoline derivatives for polymer, agrochemical, dye, antiviral and anti-malaria drug manufacturing), 4-aminobenzaldehyde (intermediate to pharmaceuticals and vanilin), 4-aminobenzoic acid (production of folic acid/vitamin B9), \ortho-toluidine (intermediate in the manufacture of herbicides), \meta-toluidine (manufacture of  photographic color developer) and \para-toluidine (for dye production) \cite{bowers_toluidines_2000,bruhne_benzaldehyde_2011,maki_benzoic_2000}. To determine the three most advantageous products to produce, AHP/TOPSIS analysis was conducted, giving significant importance to the safety of the compounds and their economic potential. Safety was assessed using the NFPA Hazard Classification for health, flammability and instability, where the smallest value is preferable. The economic potential of each chemical was determined using the average price, the average market share of producers which is an indicator of the market share Nitroma can plan to capture, the global demand and the expected market growth (CAGR). \Cref{tab:product} summarises the results and the complete methodology can be found in \Cref{app:matrix}.   

\nomenclature[A]{NFPA}{National Fire Protection Association}
\nomenclature[A]{CAGR}{Compound annual growth rate}

\subsubsection{Criteria identification and weighting}

\subsubsection{Uncertainties and assumptions}

\subsubsection{Results}


\begin{table}[h]
\centering
    \caption{AHP/TOPSIS results for product selection}
    \label{tab:product}\footnotesize
\begin{tabularx}{\linewidth}{l|S[table-format=4.2]S[table-format=1.1]XS[table-format=7.0,round-mode=figures,round-precision=2]|XX|lll|S[table-format=1.3]S[table-format=1.3]c}
\toprule
                                          & \multicolumn{4}{c}{Economic potential   (\SI{29}{\percent})}                                & \multicolumn{2}{|>{\hsize=\dimexpr2\hsize+2\tabcolsep}X}{Process   complexity (\SI{14}{\percent})} & \multicolumn{3}{|c|}{EHS (\SI{57}{\percent})}     &                       &                          &                           \\ \cmidrule{2-10}
                                          & {\splitcell{Price\\(\si{\USD\per\kg})}} & {\splitcell{CAGR\\(\%)}} & \rcell{Average market share of producer (\%)} & {\splitcell{Demand\\(\si{\tonne\per\year})}} & \rcell{Number of reactions} & \rcell{Max selectivity to toluene (\%)} & \rtext{Health} & \rtext{Flammability} & \rtext{Instability} & AHP & TOPSIS & Rank \\ \midrule
2-aminobenzaldehyde & 133.10          & 5.6 & 0.77                           & 1300                 & 2                & 53                       & 2      & 1            & 1           & 0.124                 & 0.273                    & 6                         \\ 
4-aminobenzaldehyde & 19.00            & 7.1 & 1.75                           & 34000               & 2                 & 44                       & 2      & 1            & 0           & 0.188                 & 0.563                    & \cellcolor{green}2 \\ 
o-toluidine         & 10.62           & 7.8 & 0.88                           & 810000              & 1                   & 53                       & 3      & 2            & 0           & 0.210                 & 0.611                    & \cellcolor{green}1 \\ 
p-toluidine         & 9.00             & 6.3 & 0.50                           & 60000               & 1                   & 44                       & 3      & 2            & 0           & 0.150                 & 0.333                    & 5                         \\ 
m-toluidine         & 8.90           & 8.2 & 0.57                           & 110000              & 1                   & 3                       & 4      & 2            & 0           & 0.161                 & 0.458                    & 4                         \\ 
4-aminobenzoic acid & 26.29         & 5.8 & 3.70                           & 3250               & 2                 & 44                       & 2      & 1            & 1           & 0.168                 & 0.478                    & \cellcolor{green}3 \\ \bottomrule
\end{tabularx}
\end{table}


Due to their high economic potential and reduced health and fire hazards, 4-aminobenzaldehyde (4-ABH), o-toluidine (o-TOL), and 4-aminobenzoic acid (4-ABA) were found to be the most suitable products. A  summary of the selected products can be found in \Cref{fig:routes-chosen}.

\nomenclature[A]{4-ABH}{4-aminobenzaldehyde}
\nomenclature[A]{o-TOL}{o-toluidine}
\nomenclature[A]{4-ABA}{4-aminobenzoic acid}

\subsubsection{Sensitivity}

\subsection{Nitration reaction}

\subsubsection{Choice of nitrating agent}
Nitric acid (in aqueous solution) has been selected due to its safe and environmentally friendly nature relative to alternatives such as acetone cyanohydrin and dinitrogen tetroxide, which are highly toxic \cite{miller_kinetics_1964,dagade_nitration_2002, sreedhar_scientific_2013}. 

Nitric acid has been selected as the source of nitrogen for the nitration of toluene for its safe and environmentally friendly nature relative to other possible nitrating agents, high availability, and its various favourable properties \cite{miller_kinetics_1964, sreedhar_scientific_2013}. Although nitric acid is a highly acidic and volatile compound, compared to alternatives such as acetone cyanohydrin and dinitrogen tetroxide which are highly toxic, nitric acid is more appropriate for industrial-scale nitration of toluene \cite{dagade_nitration_2002, sreedhar_scientific_2013}. Nitric acid, introduced as an aqueous solution, is the most commonly used and well-studied nitrating agent for this process in industry. \cite{bowers_toluidines_2000} A big advantage of aqueous nitric acid is that it can act as a self-catalyst by self-donating protons. \cite{miller_kinetics_1964} A common alternative to nitric acid is acetyl nitrate which is formed by the reaction of nitric acid with acetic anhydride. \cite{vassena_selective_1999} This reaction yields formic acid as by-product, resulting in a lower atom economy; it also causes unnecessary difficulties in separations downstream by introducing three extra components: acetyl nitrate, acetic anhydride, and formic acid. The same argument can be employed for other alkyl nitrates such as butyl nitrate. To this end, nitric acid is deemed as the most suitable choice among all candidates.


\subsubsection{Choice of catalyst}
Nitration in industry is mainly carried out by a mixed-acid process \cite{halder_nitration_2007}.
In addition to favouring less economically desirable \textit{ortho} isomers, product separation and acid regeneration are expensive and energy intensive \cite{sreedhar_scientific_2013}.
These drawbacks are overcome by solid acid catalysts which offer an environmentally friendly and economic alternative \cite{vassena_selective_1999}.
% The relative benefits and disadvantages of 3 different zeolites were evaluated with the AHP and TOPSIS decision analysis methods.
The economic potential, performance and safety of zeolites H-ZSM-5, H-Y and H-Mordenite were assessed and compared to the case with no catalyst \cite{jeeru_kinetics_2018}, with H-Mordenite found to be the optimum in both AHP and TOPSIS analysis (see \Cref{tab:nitration}).
% Consequently, Nitroma will develop a process employing zeolite catalysts for toluene nitration.
The kinetic model for the reaction was then developed and is presented in \Cref{app:kinetics}.

\begin{comment}
Nitration in industry is mainly carried out by a mixed-acid process, whereby sulfuric acid donates a proton to nitric acid, yielding a nitronium ion which will then react with toluene []. In addition to favouring less economically desirable \textit{ortho} isomers, product separation from the acid and acid regeneration are expensive and energy intensive \cite{sreedhar_scientific_2013}. These drawbacks can be overcome by using solid acid catalysts which offer an environmentally friendly and economic alternative to sulfuric acid \cite{vassena_selective_1999}. Based on those arguments, Nitroma will develop a safer and low environmental impact process employing zeolite catalysts for toluene nitration. More specifically, the relative benefits and disadvantages of 3 different zeolites were evaluated with the AHP and TOPSIS decision analysis methods to select the optimum catalyst. The economic potential, performance and safety of zeolites H-ZSM-5, H-Y and H-Mordenite were assessed and compared to the case with no catalyst \cite{jeeru_kinetics_2018}. The economic potential was measured with the cost of the catalyst and the percentage of undesirable by-product formed. Conversion and the NFPA score are the KPIs for performance and safety respectively. H-Mordenite was found optimum for the catalytic nitration of toluene by both AHP and TOPSIS analysis (\Cref{app:matrix}). The kinetic model for the reaction was then developed and is presented in \Cref{app:kinetics}.
\end{comment}

\subsubsection{}

\subsection{o-toluidine production}

The \textit{ortho} isomer of nitrotoluene is reduced to o-TOL using \SI{2}{\percent\ww} palladium-on-carbon (Pd/C) catalyst in the liquid-phase \cite{rajadhyaksha_solvent_1986}. This reaction is carried out in a solvent to dissolve solid reactants and products, absorb the heat of exothermic reaction, and protect the catalyst surface from impurities \cite{yao_kinetics_1959}. Five different solvents were compared using AHP/TOPSIS analysis. The performance of a solvent was measured through the activation energy of the reaction, the reaction rate at \SI{333}{\K}, and the solubility of \ch{H2} in the solvent \cite{rajadhyaksha_solvent_1986}. Nitroma values the importance of using green solvents in its process and evaluated the sustainability of the different solvents using scores developed by GlaxoSmithKline (GSK) to assess the environmental impacts of the chemicals \cite{henderson_expanding_2011}. The GSK scores for health and flammability were also used for safety. If a solvent could be sourced from renewable feed stocks, it was viewed positively \cite{byrne_tools_2016}. Both TOPSIS and AHP analysis found that propanol was the most suitable green solvent for the reduction of 2-nitrotoluene (ONT) (\Cref{tab:solvent}).

\nomenclature[A]{ONT}{2-nitrotoluene, \ortho-nitrotoluene}
\nomenclature[A]{GSK}{GlaxoSmithKline}

\subsection{4-nitrotoluene oxidation}

In this process 4-nitrotoluene (PNT) is successively oxidised to nitro- benzyl alchohol, benzaldehyde and benzoic acid, as shown in \cref{fig:routes-chosen} \cite{hoorn_modelling_2005}. This reaction can be carried out in the liquid or vapour phase, with the vapour-phase process improving the selectivity towards the more valuable 4-nitrobenzaldehyde (4-NBH) product \cite{bruhne_benzaldehyde_2011}. 
The extent of oxidation to 4-NBH or 4-nitrobenzoic acid (4-NBA) is controlled via the residence time and the reaction temperature. Lower residence time and reaction temperatures over \SI{730}{\K} favour partial oxidation whereas higher residence time and temperatures below \SI{500}{\K} result in more 4-NBA \cite{bruhne_benzaldehyde_2011,tan_kinetic_2010}.
The oxidant is either pure molecular oxygen or air. In the interest of safety, air was selected as its large fraction of inert \ch{N2} gas dilutes the nitrotoluene vapour and reduces the risk of explosion \cite{bruhne_benzaldehyde_2011}. 
The conversion of PNT to 4-NBH is catalysed by cobalt phthalocyanine \cite{wendt_reaction_1986}; bromide compounds may be added to enhance complete oxidation to 4-NBA, however those are known to cause corrosion problems \cite{opgrande_benzoic_2003}.
Other processes involving manganese dioxide in sulfuric acid or chromyl chloride have been developed but were disregarded due to their higher environmental impact from waste water treatment \cite{bruhne_benzaldehyde_2011}.
The selected oxidation process is carried out in the vapour phase at \SI{0.5}{\atm} over a solid cobalt phthalocyanine catalyst \cite{chandalia_kinetics_1999}. 

\nomenclature[A]{PNT}{4-nitrotoluene, \para-nitrotoluene}
\nomenclature[A]{4-NBH}{4-nitrobenzaldehyde}
\nomenclature[A]{4-NBA}{4-nitrobenzoic acid}

\subsection{Reduction of 4-nitrobenzaldehyde and 4-nitrobenzoic acid}

The amino- substituted products are formed via the direct reduction of their nitro- counterparts.
Whilst a broad variety of novel processes, catalysts and conditions have been proposed in the literature, the industrial production of aromatic amines focuses almost exclusively on the use of metal catalysts: commonly Pd/C or Pt/C, with some examples using Ni, Co, or Cu \cite{vogt_amines_2000,cartolano_amines_2004}.
Examples exist for both liquid- and vapour-phase processes, with the former  now preferred for safety reasons as the hydrogenation is highly exothermic \cite{vogt_amines_2000}.
A broad variety of solvents may be used; short-chain alcohols are a typical example.
In the interest of modularity, conditions were sought under which both the carboxylic acid- and aldehyde-substituted nitration products could be simultaneously reduced.
Consequently, a process employing a \SI{5}{\percent\ww} Pt/C catalyst in methanol, with formic acid as the reducing agent was selected.
This process gives yields of \SIlist{86;80}{\percent} for 4-ABA and 4-ABH respectively \cite{gowda_catalytic_2000}.	

\subsection{Finalized synthesis route} %0.5 page
%ChemDraw figure 
 
	 
	

















 



	
	
	
