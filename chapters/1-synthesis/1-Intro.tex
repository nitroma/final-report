% !TeX root = ../../main.tex
\section{Introduction} %~0.75page

Substituted aromatic amines are essential intermediates involved in the synthesis of many pharmaceutical compounds, agrochemicals, and dyes \cite{vogt_amines_2000}. The easy replacement of the amino group by other functional groups makes them very attractive versatile intermediates. The amino group, which cannot be directly introduced via electrophilic aromatic substitution, is instead added via the nitration of the aromatic ring and subsequent reduction.
Following multiple deadly industrial accidents in chemical plants performing batch nitration, especially the 2019 Xiangshui chemical plant explosion, Chinese authorities are actively attempting to strengthen the control and management of nitration manufacturing \cite{el_diario_china_2019}.
Recent developments in flow chemistry are enabling process intensification of critical industrial processes to make them inherently safer. The transition from batch to continuous nitration opens up the possibility to safely manufacture important intermediates on a large-scale \cite{di_miceli_raimondi_safety_2015}.

In this context, Nitroma seizes the opportunity of developing a multi-purpose and continuous liquid phase nitration process for the conversion of substituted aromatics to their respective nitrates and subsequently to their amines. 
This report details the synthesis and process decisions taken to design Nitroma’s demonstration plant, to be located in the Nanjing Chemical Industry Park (China). The decision making methods, the key innovation points and process intensification strategies are elaborated upon and di.
