% !TeX root = ../../main.tex
\section{Introduction} %~0.75page


Substituted aromatic amines are essential intermediates involved in the synthesis of many pharmaceutical compounds, agrochemicals, and dyes \cite{vogt_amines_2000}. The easy replacement of the amino group by other functional groups makes them very attractive versatile intermediates. The amino group, which cannot be directly introduced via electrophilic aromatic substitution, is instead added via the nitration of the aromatic ring ans subsequent reduction.

Following multiple deadly industrial accidents in chemical plants performing batch nitration, especially the 2019 Xiangshui chemical plant explosion, Chinese authorities are actively attempting to strengthen the control and management of nitration manufacturing \cite{el_diario_china_2019}.

Over the past two decades, the advent of flow chemistry in the laboratory and the increasing
u��lisa��on of con��nuous-flow manufacturing in the chemical industry have unlocked previously
inaccessible reac��on spaces4,52. Compared to batch opera��on, the con��nuous manufacture of
hazardous compounds is inherently safer due to a lower inventory of the hazardous material l4,52.
These open up the possibility of manufacturing diazomethane on a large-scale basis.


In this context, Nitroma seizes the opportunity of developing a multi-purpose and continuous liquid phase nitration process for the conversion of substituted aromatics to their respective nitrates and subsequently to their amines. 

In this report
%

% report objectives