\section{Hazard and operability study (HAZOP) }
 
A HAZOP was carried out on the identified on the section of the plant in which o-toluidine is produced. By conducting a HAZOP, Nitroma can systematically identify and evaluate the risks and consequences associated to any deviations or faults that may occur during steady state operation. Through this, appropriate measures will be taken to reduce the occurrence of these consequences and thus prevent personnel injury and plant damage. The section was spilt into two nodes, with design intent o
 


\begin{table}[]
\centering
\caption{The design intent of nodes from HAZOP study}
\label{tab:nodes}
\resizebox{\textwidth}{!}{%
\begin{tabular}{@{}cll@{}}
\toprule
\textbf{Node}                                            & \multicolumn{1}{c}{\textbf{1}}                                                                                                                                                                                                                                                                                                                                                                                                                                                                                                                                                                                                                                       & \multicolumn{1}{c}{\textbf{2}}                                                                                                                                                                                                                                                                                                                                                                                                                                                                                                                                                                                                                                                                    \\ \midrule
\begin{tabular}[c]{@{}c@{}}Design \\ Intent\end{tabular} & \begin{tabular}[c]{@{}l@{}}\textbf{Production of o-toluidine via hydrogenation of o-nitrotoluene}     \\     \\  Methanol and o-nitrotoluene feed (from mixer M201) \\ pass through pump P201a/P201b \\ and heat exchanger H202 (to reach the \\ required reaction temperature 333 K) before entering R201.\\      \\ Hydrogen is transferred from the storage tank, \\ passing through a fan F201a/F201b and heat \\ exchanger before entering   R201. Reduction \\ of o-nitrotoluene in R201 occurs at 333 K, 5 atm. \\ Hydrogen gas is recycled back into R201, whilst \\ o-toluidine, by-products and unreacted\\ substances are transferred for separation.\end{tabular} & \begin{tabular}[c]{@{}l@{}}\textbf{Purification of o-nitrotoluene and recycle of  methanol}  \\ \\ The outlet stream of R201 is transferred to the distillation \\ column, S203 via V201   (which reduces the pressure \\ of the stream from 5 atm to 1 atm). Within S203,   \\ o-toluidine (bottoms stream) separated from unreacted \\ o-nitrotoluene,   by-products and unreacted substances, \\ sent to reboiler, H204. \\ \\ The top, vapour stream (by-products and unreacted \\ substances), is transferred from S203 to H203 for partial \\ condensation. Following liquidation, the stream is \\ sent to flash drum, S204, to separate non-condensable \\ and condensable compounds.\end{tabular} \\ \bottomrule
\end{tabular}%
}
\end{table}