\section{Hazard and operability study (HAZOP) }
\label{sec: HAZOP}
 
A HAZOP was carried out on the section of the plant in which \ortho-toluidine is produced. By conducting a HAZOP, Nitroma can systematically identify and evaluate the risks and consequences associated with any deviations or faults that may occur during steady state operation. Through this, appropriate measures will be taken to reduce the occurrence of these consequences and thus prevent personnel injury and plant damage. The section was spilt into two nodes, with design intent of each node stated in \cref{tab:nodes}. The first node is focused on the reaction side, consisting of the reactant feeds, the trickle bed reactor, (R201), two pumps (P202a, P202b), two heaters (H201, H202) and two fans (F201a, F201b). The second node, which focuses on the separation and purification, starts at the reactor outlet from FCV-203, with the major equipment being the packed distillation column (S201), a condenser (H203), a reflux drum (S204) and a kettle reboiler (H204). 


 

\begin{table}[h]
\centering
\caption{The design intent of nodes from HAZOP study}
\label{tab:nodes}
% \resizebox{\textwidth}{!}{%
\begin{tabularx}{\linewidth}{@{}lXX@{}}
    \toprule
    \textbf{Node}         & \textbf{1: Production of \ortho-toluidine via hydrogenation of \ortho-nitrotoluene}                                                                                                                                                                                                                                                                                                                                                                                                                                 & \textbf{2: Purification of \ortho-nitrotoluene and recycle of methanol}                                                                                                                                                                                                                                                                                                                                                                                                                                                                                  \\ \midrule
    \rtext{Design Intent} & Methanol and \ortho-nitrotoluene feed (from mixer M201) passes through pump P201a/P201b and heat exchanger H202 (to reach the required reaction temperature \SI{333}{\K}) before entering R201. Hydrogen is transferred from the storage tank, passing through a fan F201a/F201b and heat exchanger H201 before entering R201. Reduction of \ortho-nitrotoluene in R201 occurs at \SI{333}{\K}, \SI{5}{\atm}. Hydrogen gas is recycled back into R201, whilst \ortho-toluidine, by-products and unreacted substances are transferred for separation. & The outlet stream of R201 is transferred to the distillation column, S201 via V201 (which reduces the pressure of the stream from \SIrange{5}{1}{\atm}). Within S201, \ortho-toluidine (bottoms stream) separated from unreacted \ortho-nitrotoluene, by-products and unreacted substances, is sent to reboiler, H204. The top, vapour stream (by-products and unreacted substances), is transferred from S201 to H203 for partial condensation. Following liquidation, the stream is sent to a flash drum, S204, to separate non-condensable and condensable compounds. \\ \bottomrule
\end{tabularx}
% }
\end{table}


Following the review of the HAZOP study, several of the recommendations suggested were implemented to the design, to allow further control of some of the system deviations and associated  consequences identified. 

% Please add the following required packages to your document preamble:
% \usepackage{booktabs}
% \usepackage{multirow}
% \usepackage{graphicx}


\begin{table}
\centering
\caption{Changes to P\&ID}
\label{tab:PIDchanges}

\resizebox{\textwidth}{!}{%
\begin{tabular}{@{}l>{\raggedright}p{2cm}p{7cm}p{10cm}@{}}
\toprule
\multicolumn{2}{@{}l}{Node / Region} & Recommendation implemented                                                                                                                               & Reasoning for design change                                                                                                                                                                                  \\ \midrule
1   & Methanol / ONT feed       & Installation of an additional pressure sensor associated with regular and executive alarms                                                                    & Prevention of overpressure in the pump (P2101a/b) and piping rupture and the subsequent risk of overheating and a fire                                                                                       \\ \cmidrule(l){3-4} 
    &                           & Installation of an executive alarm (FZAAH213) on flow to R201 after pump P201a/b                                                                              & Prevention of damage in P201a/b overheat due to high flows (which may in turn lead to overheating of the pump)                                                                                               \\ \cmidrule(l){3-4} 
    &                           & Installation of an additional alarms (TZAHH214, TZAHH215) on gaseous and liquid reactor inlet streams                                                         & To prevent high-temperature flow into H201 and R201, thus preventing risk of thermal runaway and explosion                                                                                                   \\ \cmidrule(l){2-4} 
    & \multirow[t]{2}{=}{Hydrogen feed and recycle} & Addition of a flow indicator to FT206 and FIC controlling F201a/b                                                                         & \multirow[t]{2}{=}{Prevention of F201a overheating (which could lead to gaseous explosion)}                                                                                                                  \\ \cmidrule(lr){3-3}
    &                           & Installation of an additional executive alarm (FZAHH) in the hydrogen inlet to the reactor after H201                                                         &                                                                                                                                                                                                              \\ \cmidrule(l){3-4} 
    &                           & Installation of a non-return valves on feed \& recycle streams entering valve and a PCV between SOV-202 and the 3-way valve                                   & To control misdirected flow at the 3-way valve, preventing the risk of excess hydrogen gas being released via the recycle stream purge leading to an explosion                                               \\ \cmidrule(l){3-4} 
    &                           & Installation of regular and executive alarms on  recycle controller FIC207                                                                                    & Provide an additional control in the case of no flow through recycle purge                                                                                                                                   \\ \cmidrule(l){3-4} 
    &                           & Installation of an additional pressure sensor associated with regular and executive alarms as backup                                                          & \multirow[t]{2}{=}{Additional control for a possible deviation from pressure in the fan F201a, from its set point}                                                                                           \\ \cmidrule(lr){3-3}
    &                           & Installation of an additional PZALL executive alarm to alert operators when pressure is low                                                                   &                                                                                                                                                                                                              \\ \cmidrule(l){2-4} 
    & Cooling water feed        & Installation of a regular alarm FAH201 and an executive alarm FZAHH to FT204                                                                                  & Provision of a safeguard for the high flow of cooling water which in turn may result in reactor R102 operating at suboptimal conditions                                                                      \\ \cmidrule(l){3-4} 
    &                           & Installation of a regular alarm TAL203                                                                                                                        & \multirow[t]{3}{=}{Provision of additional control for the temperature of cooling water, should there be any deviation from the required setpoint, thus providing sufficient cooling for the reactor (R102)} \\ \cmidrule(lr){3-3}
    &                           & Installation of a regular alarm TAL202                                                                                                                        &                                                                                                                                                                                                              \\ \cmidrule(lr){3-3}
    &                           & Installation of a regular alarm TAH202                                                                                                                        &                                                                                                                                                                                                              \\ \cmidrule(l){2-4} 
    & Reactor, R201             & Installation of a level controller, LIC-208 to feed into FIC-209                                                                                              & For the prevention level in the reactor caused by a higher than expected inlet flow rate                                                                                                                     \\ \cmidrule(l){3-4}
    &                           & Installation of an additional alarm to TAH201                                                                                                                 & Provision of an additional control for protection of R102 against a fire and explosion, due to an increase in temperature above the set point                                                                \\ \cmidrule(l){3-4}
    &                           & Installation of a composition transmitter, AT + AAL/AAH downstream of reactor to monitor product stream composition and installation AIC to feed into FIC202  & Provide a safeguard for the composition of the reactor inlet, preventing an incorrect feed ratio which would result in an off specification product                                                          \\ \midrule
2   & Separator, S201           &      Positional change of flow control valve (FCV-203) and pressure regulating valve (PCV-201)                                                                                                                                                         &         Changing the position of flow control valve will allow disturbances in the flow through PCV-201 to be corrected for by FCV-203 and its flow controller                                                                                                                                                                                                     \\ \cmidrule(l){2-4} 
    & Reflux drum (S204)         & Installation of pressure alarms and additional PRV                                                                                &      Acts as a safeguard against overpressurisation in the case of no vapour product flow exiting S204, should FVC-205 fail shut                                                                                                                                                                                                        \\ \cmidrule(l){3-4} 
    &                           & Installation of a flow transmitter and a low flow alarm on tops stream                                                                                                                                                   &                          Acts as a safeguard in the case of no top stream flow from S204, should FCV-206 fail shut                                                                                                                                                                               \\ \cmidrule(l){3-4} 
    &                           &         Installation of a flow transmitter and alarm downstream of FCV-206                                                                                                                                                      &                    Should low level in S204 be caused be more flow through FCV, this addition will allow the operator to identify the cause and respond quicker                                                               \\ \cmidrule(l){3-4}
    &                            &  Additional flow transmitter to monitor the flowrate of the cold stream from H204    &  Further surveillance of cold stream flow can assist in maintaining the temperature of H204 
    \\ \cmidrule(l){2-4} 
    & Kettle Reboiler (H204)           &          Installation of flow indicator and control system on bottoms stream from H204                                                                                                                                                     &               Aids prevention of no flow from the bottoms stream of H204, with a control system responding specifically to the flow                                                                                                                                                                                               \\ \cmidrule(l){3-4} 
    &                           &      Switched level controller from maintaining level in column to maintaining level in H204                                                                                                                                                          &            Prevents H204 from drying up and overpressurising due to excess heat duty. Low level in S201 does not pose more significant operational or safety risks.                                                                                                                                                                                                                                  \\ \bottomrule
\end{tabular}%
}
    
\end{table}
