\section{Laws, regulations \& standards}

\paragraph{Control of Major Accident Hazards Regulations (COMAH) 2015:} 

The COMAH Regulations aims to protect people, local communities and the environment from major accidents involving dangerous chemicals by preventing their occurrence or mitigating the effects should the event occur. The COMAH regulations are the British implementation of the EU Seveso III Directive. Nitroma will acquire consent from the hazardous substances authority to approve the quantity of hazardous material. Dangerous substances will be identified and labelled via the classification, labelling and packaging (CLP) regulations (which follows the Globally Harmonized System (GHS) of the classification). Furthermore, Nitroma will provide major accident prevention policy and ensure all operators take appropriate and necessary actions to avert major accidents from occurring \cite{health_and_safety_executive_understanding_2015}. 

%https://www.hse.gov.uk/chemical-classification/legal/clp-regulation.htm

\paragraph{Control of Substances Hazardous to Health (COSHH) 2002:}

COSHH is a law which requires implementation of controls to prevent or minimise the exposure of substances and materials that are hazardous to the health of the labour workforce or any person who has access to the plant. Nitroma is in full compliance with the COSHH Law, outlining the health hazards of chemicals involved in the process, with monitored control measures in place to prevent exposure. The long-term workforce emissions of gaseous waste set out under this law will be obliged to. Nitroma has emergency procedures in place to effectively manage any human exposure to a chemical substance. Nitroma’s plant and processes were designed to minimise the likelihood of any potential exposure, with the staff fully trained and well equipped to prevent and deal with any substance exposure \cite{health_and_safety_executive_control_nodate}. 

\paragraph{The Dangerous Substances and Explosive Atmospheres Regulations (DSEAR) 2002:}

Originating from the European Union ATEX directive, DSEAR  aims to protect individuals from the risk of fires and explosions. In compliance with the DSEAR, Nitroma has identified all substances classified as dangerous along with their associated hazards and consequences. The locations in which these events are most likely to occur were also identified. Nitroma has placed adequate control measures in place to remove and minimise the risk of accidents or incidents \cite{health_and_safety_executive_dangerous_nodate}. 

\paragraph{The UK Registration, Evaluation, Authorisation and Restriction of Chemicals (REACH) 2021:}

The UK REACH Regulation is in place to protect human health and the environment from risks associated with chemical substances. As a manufacturing company, Nitroma is required to perform a thorough risk identification concerning chemical substances on-site and provide a management strategy for their safe use. Nitroma will also officially register all products to the Health and Safety Executive (HSE) \cite{health_and_safety_executive_uk_nodate}. 

\paragraph{Reporting of Injuries, Diseases and Dangerous Occurrences Regulations (RIDDOR) 2013:}

The RIDDOR states that any dangerous incidents, diseases with an occupational origin or work-related accident that results in a serious injury or fatality, must legally be recorded by the employer. These events should be recorded for both workers and non-workers who have been affected. Nitroma assures full compliance with regulation and vows to log any accidents, incidents and injuries which the RIDDOR classifies as reportable \cite{health_and_safety_executive_reporting_2013}.

\paragraph{The Health and Safety at Work Act 1974:}

This law specifically focuses on the protection and well-being of both employees and the public. It requires employers to provide a safe working environment. In compliance, Nitroma will ensure the safe operation and maintenance of all equipment used in the plant, with a thorough plant-wide risk assessment conducted (allowing the assignment of adequate controls to manage the various hazards). Additionally, Nitroma will document detailed safety policies, operating and emergency procedures. The workforce at Nitroma will be fully trained for operation and equipped with all necessary personal protection equipment (PPE) \cite{british_safety_council_health_nodate}.

\paragraph{The Industrial Emissions Directive (IED)2010:}

This is a legislative act to regulate industrial pollutant emissions, with the goal of reducing these emissions for the protection of human health and the environment. In compliance to the directive, Nitroma will make every effort to optimise our whole environmental performance, from emissions to the air, soil and water, to energy and noise pollution. Following the directive, Nitroma will ensure emission limits are not only met, but minimised as much as possible \cite{european_commission_industrial_nodate}. 

\paragraph{Air Quality Standards Regulations 2010:}

This regulation imposes maximum limits for point source gaseous pollutants, including nitrogen oxide, carbon monoxides and aromatic compounds. The aim of this regulation is to limit the emission of harmful air pollutants via industrial practises, with the ultimate goal of achieving 'cleaner air' \cite{tilling_meeting_2017}.  

\paragraph{Waste water treatment limits:}

The quality of wastewater discharged is assessed against set compliance limits regulated by the Environmental Agency. There are maximum compliance limits set for the chemical oxygen demand and biological oxygen demand, as well as an annual mean concentration limit for total nitrogen content. Nitroma must meet these limits, in order to comply with the IED \cite{environment_agency_waste_nodate}. 

\paragraph{International Organization for
Standardization (ISO) standards}

As recommended by the British Standards Institution, Nitroma will adopt ISO standards to optimise the management of EHS considerations, helping us become more efficient and sustainable with regards to our practises \cite{british_standards_institution_benefits_nodate}. ISO 45001 (for occupational health and safety) and ISO 14001 (for environmental management) are two standards that will enable Nitroma to improve their approach to safety of people and the environment. 
