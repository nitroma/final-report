% !TeX root = ../../main.tex
\section{Introduction}

At Nitroma, safety, health, and environmental considerations form the fundamental element of our company ethos. Our priority is to manufacture products of the highest quality whilst ensuring the safety of the people both within the workplace and the community, and of the product itself. Nitroma aims to eliminate or at least minimise any adverse impact our operations may have on the environment. Our commitment to the environment, health and safety (EHS) is reflected through our practices which comply with all relevant legislation and regulations. Inherent safety formed the basis of Nitroma’s plant design, which further integrates independent layers of protection to achieve the highest standard of safety. The comprehensive evaluation of any potential hazards and risks conducted ensures that any possible safety or environmental threats are as low as reasonably possible. Our innovative process system allows Nitroma to manufacture the finest aromatic amine products, fit for pharmaceutical use whilst fulfilling our environmental, health and safety responsibilities.

\section{Laws and regulations}

\paragraph{Control of Major Accident Hazards Regulations (COMAH) 2015:} 

The COMAH Regulations aims to protect people, local communities and the environment from major accidents involving dangerous chemicals by either preventing their occurrence or mitigating the effects should the event occur. The COMAH regulations are the British implementation of the EU Seveso III Directive. Nitroma will require hazardous substances consent from the hazardous substances authority to approve the quantity of hazardous material. Furthermore, Nitroma will provide major accident prevention policy and ensure that all operators take appropriate and necessary actions to avert major accidents occurring. 

\paragraph{Control of Substances Hazardous to Health (COSHH) 2002:}

COSHH is a law which requires the implementation of controls to prevent or minimise the exposure of substances and materials that are hazardous to the health of the labour workforce or any person who has access to the plant. Nitroma is in full compliance with the COSHH Law, outlining the health hazards of chemicals involved in the process, with monitored control measures in place to prevent exposure. Nitroma has emergency procedures in place to effectively manage any human exposure to a chemical substance. Nitroma’s plant and processes were designed to minimise the likelihood of any potential exposure, with the staff fully trained and well equipped to prevent and deal with any substance exposure. 

\paragraph{The Dangerous Substances and Explosive Atmospheres Regulations (DSEAR) 2002:}

Originating from the European Union ATEX directive, the primary aim of the is to protect individuals from the risk of fires and explosions. In compliance with the DSEAR, Nitroma has identified all substances classified as dangerous along with their associated hazards and consequences. The locations in which these events are most likely to occur were also identified. Nitroma has also placed adequate control measures in place to remove and minimise the risk of causing any accidents or incidents. 

\paragraph{The UK Registration, Evaluation, Authorisation and Restriction of Chemicals (REACH) 2021:}

The UK REACH Regulation is in place to protect human health and the environment from risks associated with chemical substances. As a manufacturing company, Nitroma is required to perform a thorough risk identification concerning chemical substances on-site and provide a management strategy for their safe use. Nitroma will also officially register all products to the Health and Safety Executive (HSE). 

\paragraph{Reporting of Injuries, Diseases and Dangerous Occurrences Regulations (RIDDOR) 2013:}

The RIDDOR states that any dangerous incidents, diseases with an occupational origin or work-related accident that results in a serious injury or fatality, must legally be recorded by the employer. These events should be recorded for both workers and non-workers who have been affected. Nitroma assures full compliance with regulation and vows to log any accidents, incidents and injuries which the RIDDOR classifies as reportable.

\paragraph{The Health and Safety at Work Act 1974:}

\paragraph{The Industrial Emissions Directive:}
 
 
\section{Inital Design Considerations}

\subsection{EHS influence on process synthesis pathway}

EHS considerations were paramount in the decision-making process concerning Nitroma’s synthesis activities and operations. Multi-criteria decision-making (MCDM) tools (including AHP and TOPSIS) were utilised, allowing EHS factors to be systematically evaluated against other important aspects concerning the process (i.e; economic potential and process complexity). Different metrics were used as weightings for the various decisions surrounding the process synthesis; for the selection of the products to be manufactured, the safety was ranked by the NFPA 704 ratings whilst for the reaction pathways adopted, the sustainability and environmental impact were assessed through the GlaxoSmithKline scoring system. A more detailed insight on the EHS considerations on crucial process synthesis decions can be found in Section \ref{sec:synthesis}. 

\subsection{Inherent safety}

Inherent safety was a crucial element of the initial design stage. Nitroma wishes to ensure that all risks associated to the process are as low as practically possible, and adopting an inherently safer design is key to achieving this. Thus, Nitroma has applied all of the inherent safety principles (as outlined below): 

\paragraph{Minimisation:} To reduce the total hazardous inventory, 



\paragraph{Substitution:}



\paragraph{Modification:}


\paragraph{Simplification:} 






\subsection{Material selection}
\section{Plant wide risk assessment}

According to Lee’s, the major hazards associated to a chemical process plant like Nitroma’s, are fires, explosions and the release of toxic chemicals. Nitroma takes these hazards very seriously, and thus (through identification and analysis of hazards associated to these incidents), we plan to implement appropriate controls which will effectively prevent or minimise both the likelihood and severity. 

\subsection{Past Incidents}

Nitroma recognises the importance of learning from events in the past and so an important aspect of hazard identification is the review of historical case studies. Nitroma hopes to learn from these past mistakes and accidents and use this as insight for the safety of our process plant


\subsection{Fires and explosions}

A fire arises from a chemical combustion reaction which requires four elements from the 'fire tetrahedron'; fuel, oxidant, heat and radicals. Nitroma has identified two oxidants present in the process, oxygen and nitric acid. There have also been multiple flammable substances identified, with hydrogen gas holding the highest flammability (with a NFPA 704 rating of 4). Due to the significantly exothermic nature of all the synthesis reactions (especially the nitration reaction), there is a potential risk of thermal runaway, should there be an uncontrollable temperature rise of the system. The occurrence of a thermal runaway reaction may in turn lead to fires and explosions. Furthermore, due to the high pressure of hydrogen gas in reactor R201, there may is a risk of explosion as a result of over-pressure. Taking these potential hazards and risks into account, Nitroma has devised various controls to manage each of these issues efficiently;

\begin{itemize}
    \item installation of a cooling jacket on each reactor to control heat transfer
    \item installation of temperature and pressure indicators that allow constant observation of conditions and warning should they deviate from normal
    \item placement of firewalls 
    \item define emergency shut down procedures in case these these controls fail to prevent temperature rise
\end{itemize}

\subsubsection{Fire and Explosion Index}




\subsection{Release of toxic materials}

Whilst measures have been taken to reduce the hazardous inventory of the plant, there will be numerous chemical substances present at Nitroma's process plant, which can be hazardous to human health, to the plant and to the environment if they are released.  


\subsection{Dust explosions}






\subsection{Risk matrix}

