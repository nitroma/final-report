% !TeX root = ../../main.tex
\section{Introduction}

At Nitroma, safety, health, and environmental considerations form the fundamental element of our company ethos. Our priority is to manufacture products of the highest quality whilst ensuring the safety of the people both within the workplace and the community, and of the product itself and to minimise or eliminate any adverse impact our operations may have on the environment. Our commitment to the environment, health and safety (EHS) is reflected through our practices which comply with all relevant legislation and regulations. Inherent safety formed the basis of Nitroma’s plant design, which further integrates independent layers of protection to achieve the highest standard of safety. The comprehensive evaluation of any potential hazards and risks conducted ensures that any possible safety or environmental threats are as low as reasonably possible. Our innovative process system allows Nitroma to manufacture the finest aromatic amine products, fit for pharmaceutical use whilst fulfilling our environmental, health and safety responsibilities.

\section{Laws and regulations}

\subsubsection{British laws and regulations}

\paragraph{Control of Major Accident Hazards Regulations (COMAH) 2015:} 

The COMAH Regulations aims to protect people, local communities and the environment from major accidents involving dangerous chemicals by either preventing their occurrence or mitigating the effects should the event occur. The COMAH regulations are the British implementation of the EU Seveso III Directive. Nitroma will require hazardous substances consent from the hazardous substances authority to approve the quantity of hazardous material. Furthermore, Nitroma will provide major accident prevention policy and ensure that all operators take appropriate and necessary actions to avert major accidents occurring. 

\paragraph{Control of Substances Hazardous to Health (COSHH) 2002:}

COSHH is a law which requires the implementation of controls to prevent or minimise the exposure of substances and materials that are hazardous to the health of the labour workforce or any person who has access to the plant. Nitroma is in full compliance with the COSHH Law, outlining the health hazards of chemicals involved in the process, with monitored control measures in place to prevent exposure. Nitroma has emergency procedures in place to effectively manage any human exposure to a chemical substance. Nitroma’s plant and processes were designed to minimise the likelihood of any potential exposure, with the staff fully trained and well equipped to prevent and deal with any substance exposure. 

\paragraph{The Dangerous Substances and Explosive Atmospheres Regulations (DSEAR) 2002:}

The Dangerous Substances and Explosive Atmospheres Regulations (DSEAR) 2002
The DSEAR ensure the safety of individuals from a fire, explosion and substances that are corrosive to metals by requesting that companies which handle substances which could ignite or explode, to put in controls to minimise the risk. The regulation originates from the European Union ATEX directive. The dangerous substances that must be controlled under this regulation include flammable gases, solvents, pressurised gases and substances corrosive to metal. Nitroma has identified all substances that are classified as dangerous substances under the DSEAR 20002 Regulations and have placed adequate control measures in place to remove and minimise the risk of causing serious injury or fatality. 






\subsection{Chinese laws and regulations}

\section{Key decisions}

\subsection{MCDC}
\subsection{Inherent safety}
\subsection{Material selection}

\section{Qualitative risk}
\subsection{Fires and explosions}
\subsection{Thermal runaway}
\subsection{Release of toxic materials}
\subsection{Dust explosions}
\subsection{Risk Matrix}