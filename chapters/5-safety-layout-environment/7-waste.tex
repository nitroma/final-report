\section{Environmental Impact}

\subsection{Waste prevention and minimisation}

Nitroma is conscious of the adverse effects our activities will may have on the environment. The company strives to set an industry standard for conducting environmentally sustainable in the chemical manufacture field. The prevention and minimisation of waste are key to achieving this goal. 

\subsubsection{Ammonia}
The selection of compounds utilised in the various plant reactions allowed waste prevention. For the hydrogenation to 4-aminobenzaldehyde and 4-aminobenzoic acid, a choice was made between two reducing agents, formic acid and ammonium formate. Whist ammonium formate is more favourable in terms of reaction time (the low residence time offering a faster pathway), the use of this reducing agent will produce ammonia as a side product. Acting as a  source of nitration pollution, gaseous ammonia emissions have harmful affects on biodiversity; forming acid deposits in soil, inflicting toxic damage on plants, and negativity altering aquatic ecosystems through a significant nutrient shift \cite{european_environment_agency_ammonia_2019}. Based on this, formic acid was chosen as the reducing agent over ammonium formate, eliminating the formation of ammonia. 

\subsubsection{Sulphuric acid and nitrogen oxides}
The nitration reaction process presented an opportunity for waste prevention and minimisation. Traditionally, nitration reactions employ a mixture of nitric acid and sulphuric acid. However, the disposal of 'spent' sulphuric acid has been found to be both costly and environmentally unfriendly. The replacement of sulphuric acid with the zeolitle, H-Modernite, provides a more environmentally sustainable nitration pathway. H-Modernite comes with an various advantages, including low catalyst loading, its ability to allow short reaction times and its reusable nature. 

Another issue with the use of sulphuric acid in nitration is that it leads to decomposition of nitric acid, forming harmful nitrogen oxides (NOx). Upon the replacement of sulphuric acid with a zeolite catalyst, extensive research was conducted to investigate how NOx may be formed. Our research showed no evidence of NOx emissions from nitration with zeolite catalysts. Nitroma has hypothesised that there is a rapid reaction between the zeolite and the NOx formed from nitric acid decomposition, the resulting product being a nitrating agent which reacts in the nitration itself. Further details on can be found in the synthesis section \ref{sec:synthesis}.2.5. Thus, the use of H-Modernite has allowed prevention of NOx emissions - however if there should be NOx emissions, research suggests this would be less than what would be produced with sulphuric acid and thus, if not complete elimination of NOx emissions, it's minimisation has been achieved.  

\subsubsection{Recycle and recovery}

%Where possible, recycle streams have been integrated into the process, to recover essential reactants. Among the the materials recovered recycle streams are toluene, hydrogen gas and methanol. 

%double check streams
%how much of each solvent has been saved per year
%give specific stream lines and separation columns
 


%https://pubs.rsc.org/en/content/articlelanding/1996/cc/cc9960000469/unauth#!divAbstrac

%http://www.ajgreenchem.com/article_49085.html



\subsection{Waste Treatment}

Whilst every effort has been made to prevent and minimise waste, there will inevitable be materials that must be disposed of. Nitroma has considered the best available techniques available to appropriately treat various waste streams, ensuring that they comply with the imposed limits.  

\subsubsection{Aqueous waste streams and wastewater treatment}

Due to the continuous nature of the plant, liquid effluent streams will be continually discharged and so there must be an on-site wastewater treatment plant that is able to manage this demand sufficiently. Each of the waste streams are first collected and stored before being sent to our on-site waste treatment facility. Before being stored, stream 3-12 will pass through a flash drum to separate the liquid and vapour phases. High content organic streams which contain flammable material will be disposed of via incineration as recommended \cite{sinnott_coulson_2005}, utilising the energy recovered from this within the plant. %how is the energy used specifically?




\begin{table}[]
\caption{Liquid waste streams for the ABA scenario }
\label{tab:abawaste}
\begin{tabular}{@{}llll@{}}
\toprule
\textbf{Waste Stream}  & \textbf{Component Name} & \textbf{\begin{tabular}[c]{@{}l@{}}Flow Rate\\ kg/hr\end{tabular}} & \textbf{\begin{tabular}[c]{@{}l@{}}COD\\ mg/L\end{tabular}} \\ \midrule
\multirow{3}{*}{W2-03} & 2-Nitrotoluene          & FR                                                                 & cod                                                         \\
                       & 3-Nitrotoluene          & FR                                                                 & cod                                                         \\
                       & 4-Nitrotoluene          & FR                                                                 & cod                                                         \\
\multirow{4}{*}{W3-15} & 4-Nitrotoluene          & FR                                                                 & cod                                                         \\
                       & 4-Nitrobenzaldehyde     & FR                                                                 & cod                                                         \\
                       & 4-Nitrobenzoic acid     & FR                                                                 & cod                                                         \\
                       & 4-Nitrobenzyl alcohol   & FR                                                                 & cod                                                         \\
\multirow{4}{*}{W4-03} & Water                   & FR                                                                 & cod                                                         \\
                       & 4-Nitrobenzoic acid     & FR                                                                 & cod                                                         \\
                       & 4-aminobenzoic acid     & FR                                                                 & cod                                                         \\
                       & Formic Acid             & FR                                                                 & cod                                                         \\
\multirow{4}{*}{W6-05} & 2-Nitrotoluene          & FR                                                                 & cod                                                         \\
                       & 3-Nitrotoluene          & FR                                                                 & cod                                                         \\
                       & 4-Nitrotoluene          & FR                                                                 & cod                                                         \\
                       & o-toluidine             & FR                                                                 & cod                                                         \\
\multirow{2}{*}{W6-07} & Propanol                & FR                                                                 & cod                                                         \\
                       & Water                   & FR                                                                 & cod                                                         \\ \cmidrule(l){2-4} 
\end{tabular}
\end{table}


\begin{table}[]
\caption{Liquid waste streams for the ABH scenario }
\label{tab:abhwaste}
\begin{tabular}{@{}llll@{}}
\toprule
\textbf{Waste Stream}  & \textbf{Component Name} & \textbf{\begin{tabular}[c]{@{}l@{}}Flow Rate\\ kg/hr\end{tabular}} & \textbf{\begin{tabular}[c]{@{}l@{}}COD\\ mg/L\end{tabular}} \\ \midrule
\multirow{3}{*}{W2-03} & 2-Nitrotoluene          & FR                                                                 & cod                                                         \\
                       & 3-Nitrotoluene          & FR                                                                 & cod                                                         \\
                       & 4-Nitrotoluene          & FR                                                                 & cod                                                         \\
\multirow{4}{*}{W4-03} & Water                   & FR                                                                 & cod                                                         \\
                       & 4-Nitrobenzoic acid     & FR                                                                 & cod                                                         \\
                       & 4-aminobenzoic acid     & FR                                                                 & cod                                                         \\
                       & Formic Acid             & FR                                                                 & cod                                                         \\
\multirow{4}{*}{W6-05} & 2-Nitrotoluene          & FR                                                                 & cod                                                         \\
                       & 3-Nitrotoluene          & FR                                                                 & cod                                                         \\
                       & 4-Nitrotoluene          & FR                                                                 & cod                                                         \\
                       & o-toluidine             & FR                                                                 & cod                                                         \\
\multirow{2}{*}{W6-07} & Propanol                & FR                                                                 & cod                                                         \\
                       & Water                   & FR                                                                 & cod                                                         \\ \cmidrule(l){2-4} 
\end{tabular}
\end{table}










%https://waterinnovations.net/wastewater-treatment/continuous-wastewater-treatment-systems/



\subsubsection{Gaseous Waste}
%greenhouse gases - NOx, CO2

\subsection{Embodied Energy}