\section{Environmental Impact}

\subsection{Waste prevention and minimisation }

Nitroma is conscious of the adverse effects our activities will may have on the environment. The company strives to set an industry standard for conducting environmentally sustainable in the chemical manufacture field. The prevention and minimisation of waste are key to achieving this goal. 

The selection of compounds utilised in the various plant reactions allowed waste prevention. For the hydrogenation to 4-aminobenzaldehyde and 4-aminobenzoic acid, a choice was made between two reducing agents, formic acid and ammonium formate. Whist ammonium formate is more favourable in terms of reaction time (the low residence time offering a faster pathway), the use of this reducing agent will produce ammonia as a side product. Acting as a  source of nitration pollution, gaseous ammonia emissions have harmful affects on biodiversity; forming acid deposits in soil, inflicting toxic damage on plants, and negativity altering aquatic ecosystems through a significant nutrient shift. Based on this, formic acid was chosen as the reducing agent over ammonium formate, eliminating the formation of ammonia. 

The choice of the nitration catalyst presented a method of waste minimisation. Traditionally, concentrated sulphuric acid is used as the catalyst, it is unfavourable with respect to both environmental impact and safety. 



%https://www.eea.europa.eu/highlights/ammonia-emissions-from-agriculture-continue#:~:text=Ammonia%20emissions%20can%20lead%20to,forests%2C%20crops%20and%20other%20vegetation.


    
    using H-Modernite catalyst as catalyst for nitration reaction instead of sulfuric acid, significantly reducing NOx emissions
  integration of various recycle streams


\subsection{Waste Treatment}

\subsubsection{Aqueous Waste}


\subsubsection{Gaseous Waste}
%greenhouse gases - NOx, CO2

\subsection{Embodied Energy}