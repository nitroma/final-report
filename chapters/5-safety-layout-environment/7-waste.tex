\section{Environmental Impact}

\subsection{Waste prevention and minimisation}

Nitroma is conscious of the adverse effects our activities will may have on the environment. The company strives to set an industry standard for conducting environmentally sustainable in the chemical manufacture field. The prevention and minimisation of waste are key to achieving this goal. 

\subsubsection{Choice of reducing agent}
The selection of compounds utilised in the various plant reactions allowed waste prevention. For the hydrogenation to 4-aminobenzaldehyde and 4-aminobenzoic acid, a choice was made between two reducing agents, formic acid and ammonium formate. Whist ammonium formate is more favourable in terms of reaction time (the low residence time offering a faster pathway), the use of this reducing agent will produce ammonia as a side product. Acting as a  source of nitration pollution, gaseous ammonia emissions have harmful affects on biodiversity; forming acid deposits in soil, inflicting toxic damage on plants, and negatively altering aquatic ecosystems through a significant nutrient shift \cite{european_environment_agency_ammonia_2019}. Based on this, formic acid was chosen as the reducing agent over ammonium formate, eliminating the formation of ammonia. 

\subsubsection{Sulphuric acid and nitrogen oxides}
The nitration reaction process presented an opportunity for waste prevention and minimisation. Traditionally, nitration reactions employ a mixture of nitric acid and sulphuric acid. However, the disposal of 'spent' sulphuric acid has been found to be both costly and environmentally unfriendly. The replacement of sulphuric acid with the zeolite, H-Modernite, provides a more environmentally sustainable nitration pathway. H-Modernite comes with various advantages, including low catalyst loading, its ability to allow short reaction times and its reusable nature. 

Another issue with the use of sulphuric acid in nitration is that it leads to decomposition of nitric acid, forming harmful nitrogen oxides (NOx). Upon the replacement of sulphuric acid with a zeolite catalyst, extensive research was conducted to investigate how NOx may be formed. Our research showed no evidence of NOx emissions from nitration with zeolite catalysts. Nitroma has hypothesised that there is a rapid reaction between the zeolite and the NOx formed from nitric acid decomposition, the resulting product being a nitrating agent which reacts in the nitration itself. Further details on can be found in the synthesis section \ref{sec:synthesis}.2.5. Thus, the use of H-Modernite has allowed prevention of NOx emissions - however if there should be NOx emissions, research suggests this would be less than what would be produced with sulphuric acid and thus, the minimisation of NOx production via nitration (if not its complete elimination) has been achieved.  

\subsubsection{Recycle and recovery}

%Where possible, recycle streams have been integrated into the process, to recover essential reactants. Among the the materials recovered recycle streams are toluene, hydrogen gas and methanol. 

%double check streams
%how much of each solvent has been saved per year
%give specific stream lines and separation columns
 


%https://pubs.rsc.org/en/content/articlelanding/1996/cc/cc9960000469/unauth#!divAbstrac

%http://www.ajgreenchem.com/article_49085.html

Whilst attempts were made to recover the nitric acid in this stream, the operation proved to be complex. This, as well as the difficulty and high expensive of the recover system implication 



\subsection{Waste Treatment}

Whilst every effort has been made to prevent and minimise waste, there will inevitably be materials that must be disposed. Nitroma has considered the best techniques available to appropriately treat the various waste streams, ensuring that they comply with the imposed limits. 
% where does the waste come from - unreacted substances

\subsubsection{Aqueous waste streams and wastewater treatment}

Due to the continuous nature of the plant, liquid effluent streams will be continually discharged and so there must be an on-site wastewater treatment plant that is able to manage this demand sufficiently.  

As seen in table %label the table 
the majority of waste streams have high organic content, which is reflected by the high chemical oxygen demand (COD) values of each substance respectively. Each of the waste streams are first collected and stored before being sent to our on-site waste treatment facility. Before being stored, stream 3-12 (a mixed phase stream) will first pass through a flash drum to separate the liquid and vapour phases. 
%how is the energy used specifically? 
%name the gases

A number of options were considered for the treatment of waste streams that were aqueous. 


\begin{itemize}
    \item solar photo-Fenton oxidation
    
    
    \item Adsorption (using activated carbon)
    
    
    
    
    
    \item Coagulation
    
    
    
    
    
    \item Anaerobic digestion
    
    
    
    
    
    
    \item Aerobic
    
    
    
    
    \item Membrane filtration
    
    
    \item Wet air oxidation 
    
    
    
    
    
    
\end{itemize}

These methods were trialled, first individually and then collaboratively, it was found that the optimal combination, considering performance, economics and safety of materials involved. Ultimately, it was decided to use three methods; first wet air oxidation followed by coagulation and ultra-filtration (through use of a membrane). These combined waste treatment reduces the aqueous streams below their required maximum COD limit of 250 mg/L. 

To ensure the waste water is relatively neutral (within ph range og 6.5-7.5), the streams containing nitric acid will acid will have lime added to it, allowing neutralisation to occur. 

%The resultant salt formed will be separated via filtration. 


%https://iwaponline.com/wst/article/81/7/1345/69903/Cost-effective-removal-of-COD-in-the-pre-treatment
%https://www.sciencedirect.com/science/article/pii/S1878535213001767


Attempts were made to reduce the COD of the pure organic streams. Whilst there was some success (stream XX), for the majority of these streams, the COD remained too high to allow discharge into the Yangtze river. Thus, the untreated, pure organic streams will be disposed of via incineration as recommended \cite{sinnott_coulson_2005}, utilising the energy recovered from this within the plant. The resultant gases (carbon dioxide, nitrogen oxides) will be scrubbed. 

%label stream xx



\subsubsection{Gaseous Waste}

As the waste streams contain 


Cryo-condensation processes are used to recover a number of different solvents, including methanol. 

https://condorchem.com/en/cryocondensation/




%greenhouse gases - NOx, CO2





\subsection{Embodied Energy}