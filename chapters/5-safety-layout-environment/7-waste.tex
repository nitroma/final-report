\section{Environmental Impact}

\subsection{Waste prevention and minimisation}

Nitroma is conscious of the adverse effects our activities will may have on the environment. The company strives to set an industry standard for conducting environmentally sustainable in the chemical manufacture field. The prevention and minimisation of waste are key to achieving this goal. 

\subsubsection{Choice of reducing agent}
The selection of compounds utilised in the various plant reactions allowed waste prevention. For the hydrogenation to 4-aminobenzaldehyde and 4-aminobenzoic acid, a choice was made between two reducing agents, formic acid and ammonium formate. Whist ammonium formate is more favourable in terms of reaction time (the low residence time offering a faster pathway), the use of this reducing agent will produce ammonia as a side product. Acting as a  source of nitration pollution, gaseous ammonia emissions have harmful affects on biodiversity; forming acid deposits in soil, inflicting toxic damage on plants, and negatively altering aquatic ecosystems through a significant nutrient shift \cite{european_environment_agency_ammonia_2019}. Based on this, formic acid was chosen as the reducing agent over ammonium formate, eliminating the formation of ammonia. 

\subsubsection{Sulphuric acid and nitrogen oxides}
The nitration reaction process presented an opportunity for waste prevention and minimisation. Traditionally, nitration reactions employ a mixture of nitric acid and sulphuric acid. However, the disposal of 'spent' sulphuric acid has been found to be both costly and environmentally unfriendly. The replacement of sulphuric acid with the zeolite, H-Modernite, provides a more environmentally sustainable nitration pathway. H-Modernite comes with various advantages, including low catalyst loading, its ability to allow short reaction times and its reusable nature. 

Another issue with the use of sulphuric acid in nitration is that it leads to decomposition of nitric acid, forming harmful nitrogen oxides (NOx). Upon the replacement of sulphuric acid with a zeolite catalyst, extensive research was conducted to investigate how NOx may be formed. Our research showed no evidence of NOx emissions from nitration with zeolite catalysts. Nitroma has hypothesised that there is a rapid reaction between the zeolite and the NOx formed from nitric acid decomposition, the resulting product being a nitrating agent which reacts in the nitration itself. Further details on can be found in the synthesis section \ref{sec:synthesis}.2.5. Thus, the use of H-Modernite has allowed prevention of NOx emissions - however if there should be NOx emissions, research suggests this would be less than what would be produced with sulphuric acid and thus, the minimisation of NOx production via nitration (if not its complete elimination) has been achieved.  

\subsubsection{Recycle and recovery}

%Where possible, recycle streams have been integrated into the process, to recover essential reactants. Among the the materials recovered recycle streams are toluene, hydrogen gas and methanol. 

%double check streams
%how much of each solvent has been saved per year
%give specific stream lines and separation columns
 
%https://pubs.rsc.org/en/content/articlelanding/1996/cc/cc9960000469/unauth#!divAbstrac

%http://www.ajgreenchem.com/article_49085.html

Whilst attempts were made to recover the nitric acid in this stream, the operation proved to be complex. This, as well as the difficulty and high expensive of the recover system implication 

\subsection{Waste Treatment}

Whilst every effort has been made to prevent and minimise waste, there will inevitably be materials that must be disposed. Nitroma has considered the best techniques available to appropriately treat the various waste streams, ensuring that they comply with the imposed limits. 
% where does the waste come from - unreacted substances

\subsubsection{Liquid waste }

Due to the continuous nature of the plant, liquid effluent streams will be continually discharged and so there must be an on-site wastewater treatment plant that is able to manage this demand sufficiently \cite{water_innovations_inc_continuous_2021}.  

As seen in table %label the table 
the waste streams have high organic content, which is reflected by the high chemical oxygen demand (COD) values of each substance respectively. Each of the waste streams are first collected and stored before being sent to our on-site waste treatment facility. 

% cooling water 

Various waste treatment methods were investigated, with the performance,  safety of materials involved and economics all taken into account. Options considered included the photo-fenton-oxidation process, coagulation and adsorption using activated carbon. Ultimately, it was decided to use two methods; first wet air oxidation (WAO) followed by treatment in an anaerobic membrane bioreactor (AnMBR). WAO is highly recommended for waste streams with a high COD content. One of the major advantages is high efficiency of the breakdown of organics containing nitrogen, without the emission of NOx. The streams will first be diluted with 230L of water (sourced from the Yangtze river) to ensure the COD of the waste stream meets the 150,000 mg/L criteria. Typical operational temperatures fall between 120-320°C \cite{may_wet_1996}, however in this case, the temperature chosen was 270°C to ensure it falls below the auto-ignition temperature of all the chemical substances involved in the process. Furthermore, streams with compounds that have a NFPA 704 flammability rating greater than 1 do not go through WAO, to reduce the risk of a fire. To further reduce the risk of a fire, air will be used as opposed to pure oxygen, with a cooling jacket in place to maintain the temperature of the reactor. Catalytic WAO is highly efficient, allowing the removal of up to 99\% of the COD of waste and so with this in mind, the catalyst

%https://clu-in.org/acwaatap/06-wetairoxidationfinal.pdf


Following WAO, the organic waste stream will be sent into the AnMBR for further treatment. AnMBRs offer a low energy input treatment method for waste treatment \cite{maaz_anaerobic_2019}, allowing over 97\% COD removal whilst transforming the organic matter into biogas \cite{ariunbaatar_performance_2021}. The biogas consists mainly of methane, produced via action of anaerobic microorganisms accustomed to a  . Approximately 70\% of organics are converted into methane \cite{ariunbaatar_performance_2021}, whilst the rest is broken down into CO_{2}. Configurations of the AnMBR can vary, however it was decided that the membrane module will remain outside the bioreactor itself for better monitoring and maintenance \cite{maaz_anaerobic_2019}. Collaboratively, the use catalytic WAO and AnMBR will be able to reduce the overall COD of the liquid waste effluent below the maximum limit of 250 mg/L. 


\subsubsection{Gaseous Waste}

In both operation scenarios, there is one stream in which contains a gaseous mixture of oxygen and nitrogen. As this stream contains no harmful pollutants, it will be released into the atmosphere. The rest of the gaseous waste streams all contain volatile organic compounds and will thus be disposed of via incineration. The complete combustion of organic compounds is assumed and thus the emission of carbon monoxide and carbon are negligible. To minimise NOx emissions, a low NOx burner will be utilised, favourable due to its NOx removal efficiency (74\%), along with low cost in comparison to other NOx removal methods such as fuel staging and gas recirculation \cite{world_bank_group_pollution_1999}. In addition to this, the flue gas will go through a gas a wet scrubbber, with a 99\% efficiency for NOx removal. Through the employment of these methods, the NOx emissions have been limited to 1.4kg/hr, (below the max 5kg/hr limit) with a total of 12.2kg/yr. 

To investigate the atmospheric dispersion of CO_{2} and NO_{2}, the point source Gaussian Plume model was utilised, with the following assumptions made:

\begin{itemize}
\item Constant emission rate of CO_{2} and NOx from source
\item Constant wind speed of 3.167m/s (based on the average wind speed in Nanjing, China in the last 13 months)
\item  Effective stack height of 10m 
\end{itemize}


\subsection{Greenhouse Gas Emissions}

\subsection{Embodied Energy} %embedded energy?