\section{Conclusion}

Nitroma is fully committed to ensuring our practises and operations are safe, producing the highest quality products whilst ensuring the protection of people, the plant and the environment. Following all relevant laws and regulations, Nitroma has kept safety considerations at the forefront of all initial considerations, striving to create an inherently safe design. Learning from past incidents and performing a plant-wide risk assessment, Nitroma has identified major process hazards and subsequently implemented various controls and measures to lower severity and likelihood of the associated event occurring. Focusing on the area with the highest potential for damage (according to the Dow’s fire and explosion index), a HAZOP study was conducted, followed by a LOPA to investigate deviations and implement further controls to ensure the risk is minimised as far as possible. The plant layout was designed, using the flow principle to ensure smooth operation and also, safety of workers within the facility. The on-site waste treatment facility allows treatment of all liquid waste streams, with the low \ch{NO_x} burner and wet scrubber minimising the emissions of nitrous oxides. In the future, Nitroma will perform a HAZOP study on the whole plant, with further LOPA conducted on the most severe consequences, before testing for ALARP for further reassurance that the plant is operating as safely as possible. Options for gaseous treatment will be explored, with the hope of reducing the amount of waste that goes through incineration and thus reducing our carbon dioxide emissions. 