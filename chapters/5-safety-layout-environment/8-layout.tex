\section{Plant layout}

\subsection{Location}

Nitroma’s plant site will be located in Nanjing Chemical Industry Park (NCIP), China. One of the main advantages of the NCIP is the shared utilises (including power, steam, water and recycling facilities), which are available to facilities in the area \cite{independent_commodity_intelligence_services_china_2007}. The NCIP lies on the northern banks of the Yangtze River, situated approximately 30km from the city centre, a significant distance between the sites and the residential areas \cite{zeng_divergent_2011}.   

% mention availability of the harbour for imports and exports

The initial design of the plant layout began with the block layout methodology, providing a simple visualisation of various areas which are segregated according to associated risks and spacing requirements \cite{center_for_chemical_process_safety_site_2010}. The site will be spilt into three sectors. The first sector will hold the main production site. As this is the sector associated with the highest risk, it has been isolated from other areas of the plant. The second sector will consist of the control room, the storage units (for raw material storage and the waste) and the waste treatment site. The last sector will facilitate the administrative building, car park and an onsite fire station.

\subsection{Production site}

The production site of the plant will consist of 5 major sections; 

\begin{itemize}
    \item Toluene nitration
    \item 2-Nitrotoluene reduction 
    \item 4-Nitrotoluene oxidation 
    \item 4-Nitrobenzoic acid reduction 
    \item 4-Nitrobenzaldehyde reduction 
\end{itemize}

As mentioned previously, the production site will be isolated from other areas of the plant, as this section holds the highest risk of fires, explosions  and release of toxic materials occurring. The control room is placed 

