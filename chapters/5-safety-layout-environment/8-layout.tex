\section{Plant layout}

\subsection{Plant location \& general plot layout}

Nitroma’s plant site will be located in Nanjing Chemical Industry Park (NCIP), China. One of the main advantages of the NCIP is the shared utilises (including power, steam, water and recycling facilities), which are available to facilities in the area \cite{independent_commodity_intelligence_services_china_2007}. The NCIP lies on the northern banks of the Yangtze River, situated approximately 30km from the city centre, a significant distance between the sites and the residential areas \cite{zeng_divergent_2011}.   

% mention availability of the harbour for imports and exports

The initial design of the plant layout began with the block layout methodology, providing a simple visualisation of various areas which are segregated according to associated risks and spacing requirements \cite{center_for_chemical_process_safety_site_2010}. The site will be spilt into three sectors. The first sector will hold the main production site. As this is the sector associated with the highest risk, it has been isolated from other areas of the plant. The second sector will consist of the control room, the storage units (for raw material storage and the waste) and the waste treatment site. The last sector will facilitate the administrative building, car park and an onsite fire station. Alongside the main entrance routes, there will also be evacuation sites in each sector for emergencies. 

%how many main entrances 
%how many evacuation sites

%add a utility room


\subsection{Production site}

The production site of the plant will consist of 5 major sections; 

\begin{itemize}
    \item Toluene nitration
    \item 2-Nitrotoluene reduction 
    \item 4-Nitrotoluene oxidation 
    \item 4-Nitrobenzoic acid reduction 
    \item 4-Nitrobenzaldehyde reduction 
\end{itemize}

As mentioned previously, the production site will be isolated from other areas of the plant, as this section holds the highest risk of fires, explosions  and release of toxic materials occurring. 

\subsection{Waste treatment site}





\subsection{Control room}

The control room is located outside the main production site, as recommended by the health and safety executive (HSE) \cite{health_and_safety_executive_control_nodate}. To ensure safety of operators in the control room, the windows will be sealed to prevent the influx of harmful gases that may be released. The control room will have blast proof walls with windows constructed of polycarbonate glass, to protect the workers in the event of overpressure \cite{health_and_safety_executive_control_nodate}. The control room will also have adequate lighting, situated away from regions of a high noise level that may distract operators. 

\subsection{Storage Units}

There will be three storage categories; final product, raw material and waste storage. There are general storage rules that will apply to all three of these storage sections 

\begin{itemize}
    \item adequate spacing between flammable materials and oxidising materials 
    \item flammable materials will be stored well below their respective auto-ignition temperature and above their flash points
    \item 
\end{itemize}


\subsection{Utility room}

The utility room is the centre of the 