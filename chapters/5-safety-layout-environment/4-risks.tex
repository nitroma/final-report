\section{Qualitative plant wide risk assessment}

The major hazards in  a chemical process plant like Nitroma’s, are fires, explosions and the release of toxic chemicals. Nitroma takes these hazards very seriously, and thus (through identification and analysis of hazards associated with these incidents), we plan to implement appropriate controls which will either prevent or significantly minimise both the likelihood and severity. 
%reference lee's

\subsection{Past Incidents}

Nitroma recognises the importance of learning from events in the past and so an important aspect of hazard identification is the review of historical case studies. Nitroma hopes to learn from these past mistakes and accidents and use this as insight for the safety of our process plant.

Fire and explosions}

Fire from chemical combustion reactions require four elements from the 'fire tetrahedron': fuel, oxidant, heat and radicals. Nitroma has identified two oxidants present in the process, oxygen and nitric acid. There have also been multiple flammable substances identified, with hydrogen gas holding the highest flammability (with a NFPA 704 rating of 4). Due to the significantly exothermic nature of all the synthesis reactions (especially the nitration reaction), there is a potential risk of thermal runaway, should there be an uncontrollable temperature rise in the system. The occurrence of a thermal runaway reaction may lead to a fire and explosion.  

Furthermore, due to the high pressure of hydrogen gas in reactor R201, there is a risk of explosion caused by overpressure. Taking these potential hazards and risks into account, Nitroma has devised various controls to manage each of these issues effectively; 


\begin{itemize}
\item installation of pressure and temperature transmitters with alarms to allow constant monitoring and defined operator actions in the case of deviation from normal operating conditions
         \item executive temperature and pressure alarms on process units with an interlock to shut down unit before fire or explosion can occur 
         \item Cooling jackets around major reactor units (and in the case of R101, a cooling tube will be placed within the reactor itself), to maintain operating temperature and prevent thermal runaway 
          \item installation of pressure relief valves which will relieve the pressure of a certain region (should the value rise above the specified set point)
    \item Increased distance between equipment and storage units to prevent sequential damage and events
    \item Storage units containing flammable and toxic materials will be encased by a bund as protection against leakage
\end{itemize}


\subsubsection{Dust explosions}

A dust explosion is a specific type of explosion that requires the dispersion of a combustible dust in a confined space in the presence of an ignition source, oxygen and fuel. As mentioned previously, the use of palladium-on-carbon as a catalyst presents a pathway to a dust explosion occurring as it is classed as a combustible dust. Whilst the pellet form of the catalyst will be used, there may be carbon powder residue on the pellets and so dust explosion is still possible. Regular inspections on all process units and storage areas will be carried out, with specific guidelines on cleaning put in place to prevent any incident. 

%any ventilation? 


\subsection{Hazardous chemicals \& their potential release}

The hazards associated with each of the chemicals used within the plant have  been identified through the use of the the Globally Harmonized System (GHS). The hazard codes for each substance has been listed and communicated to the entire workforce at Nitroma, so staff have a full understanding of hazards. Whilst all chemicals must be treated with caution, here we will draw attention to some of the key substances in relation to the NFPA 704 values concerning health, flammability and reactivity. 

\paragraph{Flammability}

Hydrogen gas is the most flammable substance on-site, the only chemical with a NFPA flammability rating of 4.  Hydrogen is extremely flammable at atmospheric conditions and will readily disperse in air if emitted. As such, care must be taken to ensure that hydrogen is handled appropriately, with safe storage, material selection and use within the process. The hydrogen gas will be stored in a pressure container placed outside in an isolated area of the plant, away from other storage units, equipment and ignition source.  Toluene and methanol are also highly flammable, with a rating of 3. These liquids will be stored in a bund surrounding their respective storage vessels. 

%cross-ref to plant layout


\paragraph{Health}

The substance with the highest NFPA rating of 4 for health was nitric acid, with oxygen, p-toluidine, o-toludine and the nitrotoluenes  following with a rating of 3. There will be restricted access to areas containing these chemicals with personal protective equipment (including gloves, face masks and shield goggles) provided for those with access to these sites. Minimal human interaction is required in the production area to reduce the risk of exposure.

\paragraph{Reactivity}

All chemicals in the process were rated between 0-1, indicating these substances are relatively stable, however controls will be placed to ensure that both the reaction conditions and storage conditions are met. 

There are numerous chemical substances at Nitroma's process plant which can be hazardous to human health and the surrounding environment. The primary cause of exposure to toxic substances is due to loss of containment during normal operation and during maintenance. As mentioned previously, the continuous nature of the process plant will allow minimal handling of chemicals, but the workforce may still be at risk of exposure due to transfer of imports and exports. Unintentional emissions of untreated toxic substances can have an adverse impact on land, air and sea, potentially inflicting serious long-term effects. To prevent and mitigate the risk, the following measures have been taken;

\begin{itemize}
    \item installation of ventilation systems to remove any toxic vapours or fumes 
    \item installation of level indicators and alarms to prevent overflow in reactor vessels , with interlocks in place to act if level is critically high
    \item label storage area with hazard warnings and ensure area is clear from obstructions
    \item ensure workforce are fully trained, following MSDS procedures and guidance for each of the substances in the process and are equipped with appropriate PPE 
    \item regular inspection of storage and process lines, to identify any leaks, damage or corrosion 
    \item carry out regular medical examinations for plant personnel to identify any health issues related to toxicological effects of chemicals on-site
\item toxic materials will be stored within an bund, to prevent exposure in case of leakage 
\end{itemize}


% Please add the following required packages to your document preamble:
% \usepackage{booktabs}
% \usepackage{graphicx}

\begin{landscape}
\begin{table}[]
\centering
\caption{Past Incidents}
\label{tab:past}

\begin{tabular}{@{}llll@{}}
\toprule
\textbf{Past Incident} & \textbf{Incident Description} & \textbf{Consequences \& losses} & \textbf{Lessons Learned} \\ \midrule
\begin{tabular}[c]{@{}l@{}}Synthron Morganton, North Carolina \\ 31st Jan 2006\\ (thermal runaway and vapour cloud explosion)\end{tabular} & \begin{tabular}[c]{@{}l@{}}• Attempting to meet higher production demand for paint additive product, \\    the reaction was scaled up by increasing the quantity of monomer charged into \\    the reactor by 12\% \\ • The rate of energy release increased more than two-fold that its normal value, \\ exceeding the cooling capacity of the reactor condenser\\ • A thermal runaway reaction resulted from this, with this, in turn, increasing the \\ pressure within the system. The venting of the solvent vapours (in the presence of \\ an ignition source) lead to a vapour cloud explosion.\end{tabular} & \begin{tabular}[c]{@{}l@{}}• 1 fatality and 14 injuries (2 of which were serious).\\ • The explosion destroyed the plant\\ • Further damage to buildings and structures in the surrounding community.\end{tabular} & \begin{tabular}[c]{@{}l@{}}• There must be an adequate hazard identification process conducted for initial design\\ • Plant personnel must be fully aware of safety policies and standard operating procedures\end{tabular} \\
\begin{tabular}[c]{@{}l@{}}Tianjiayi explosion incident \\ 21 March 2019\end{tabular} & \begin{tabular}[c]{@{}l@{}}• Nitration waste stored illegally for a long time at a solid waste warehouse with poor \\ ventilation. Thermal decomposition of nitration waste generated heat continuously, \\    resulting in autoignition and triggered an explosion\\  • The solid waste warehouse was built without compliance with law and regulations\\  • The person in charge of hazardous waste was not certified. Plant safety management \\ teams had inadequate knowledge of managing the hazards of solid waste. Lack of safety\\  awareness.\end{tabular} & \begin{tabular}[c]{@{}l@{}}• 78 deaths and 76 serious injuries \\ • Economic loss of RMB 1.98 billion yuan\end{tabular} & \begin{tabular}[c]{@{}l@{}}• Companies should strengthen hazard identification and risk assessment \\ • Ensure safety during storage, transportation and handling of hazardous \\ wastes\\ • Local government should employ experts with relevant professional \\ qualifications and industrial experience.\end{tabular} \\
\begin{tabular}[c]{@{}l@{}}Hickson & Welch Ltd Fire \\ 21 Sep 1992\end{tabular} & \begin{tabular}[c]{@{}l@{}}•	Mononitrotoluene isomers were separated from each other and other by-products \\ such as dinitrotoluene and nitrocresols via a complex distillation sequence. \\    •	These substances become increasingly unstable when heated to high temperatures \\ or held at moderate temperatures for long periods.  \\    •	Build-up of sludge in the 60 still base, a distillation unit, was slowing the rate of distillation. \\ The sludge was being cleaned out using a metal rake. \\    •	Thermally unstable residues in contact with steam heated battery underwent exothermic \\ decomposition which produced sufficient energy to auto-ignite a flammable mixture of MNT \\ vapours or decomposition products.\end{tabular} & \begin{tabular}[c]{@{}l@{}}• Main office building damaged from fire \\ • 5 fatalities and some suffered severe burns\\ • Many people, such as firefighters, became ill suggesting toxic effects \\ or infectious \\ disorders, possibly from the cold drink they consumed at the site\end{tabular} & \begin{tabular}[c]{@{}l@{}}• Still residues should be analysed, monitored and removed at regular intervals \\ to prevent the possible build-up of unstable impurities in batch distillation\\ • Use of chemical plant for a different process is a plant change and required \\ rigorous assessment. The authorisation should be obtained from an appropriate level \\ of management\\ • The design and location of control and other buildings near chemical plant processing \\ flammable/ toxic substance should be assessed for potential fire, explosion and toxic releases. \\ Mitigating action should be in place.\end{tabular} \\
\begin{tabular}[c]{@{}l@{}}Tesoro Martinez Sulfuric Acid Spill \\ 12 Feb 2014\end{tabular} & \begin{tabular}[c]{@{}l@{}}•	During maintenance operation, two workers opened and unblocked a valve to return the \\ acid sampling system into service\\    •	After opening the valve completely, the tubing directly downstream (which experiences \\ pressurisation during the alkylation) came apart due to insufficient tightening between the tube \\ and compression joint\\    •	Sulfuric acid spilled out from the tubing, lasting for two and a half hours (approx. 84,000 \\ pounds of acid released)\end{tabular} & \begin{tabular}[c]{@{}l@{}}• 2 workers suffered 1st and 2nd-degree burns \\ • The release got onto the refinery grounds and into a process sewer system\\ • process was deemed unsafe and shut down for 10 days \\ • Total penalty of $43,400\end{tabular} & \begin{tabular}[c]{@{}l@{}}• Personnel must be equipped with the appropriate PPE for each specific \\ process hazard \\ • Mechanical integrity of the process must be ensured  \\ • There must be a strong safety culture adopted within the plant to protect workers\\ • Staff must be fully trained for specific maintenance work being carried out\end{tabular} \\
\begin{tabular}[c]{@{}l@{}}CTA Acoustics Dust Explosion and Fire \\ 20 Feb 2003\end{tabular} & \begin{tabular}[c]{@{}l@{}}•	A temperature control issue resulted in a fire occurring in an open, curing oven in production \\ line 405\\    •	Combustible, phenolic dust resin that had accumulated workers during cleaning, forming a dust\\  cloud near the oven\\    •	 The dust cloud was ignited, resulting in the dust explosion, which caused several further \\ explosions other production lines\end{tabular} & \begin{tabular}[c]{@{}l@{}}• 7 fatalities, 37 injuries  \\ • Destruction of the facility (302,000-square-foot area)\\ • Evacuation of houses and elementary schools in the surrounding \\ area evacuated\end{tabular} & \begin{tabular}[c]{@{}l@{}}• must be a through hazard identification of all substances and materials, \\ with designs and procedures modified accordingly\\ • hazards associated with each of the substances present should be clearly \\ communicated to all plant personnel \\ • buildings and plants should be designed to prevent or minizine secondary \\ dust explosions \\ • there should be adequate housekeeping and maintenance routinely carried \\ out in all areas of the facility, with appropriate cleaning methods used\end{tabular} \\ \bottomrule
\end{tabular}%
}
\end{table}
\end{landscape}


\begin{landscape}
\subsection{Risk matrix}

\begin{small}
\begin{longtable}{p{4cm}p{11.5cm}ccccccc}
\caption{Likelihood-Severity Risk Matrix}
\label{tab:risk-matrix}\\
\toprule
                                                                                                                       \textbf{Hazard Description}  & \textbf{Consequences}                                                                                                                                                                                                                                                                                                                                                                          &  \textbf{Likelihood}                                     & \multicolumn{3}{c}{\textbf{Severity}}                                                                                                                                                                  & \multicolumn{3}{c}{\textbf{Risk}}                                                                                                                                                                       \\ \cmidrule(r){4-6}\cmidrule{7-9} 
                                                                                     &                                                                                                                                                                                                                                                                                                                                    &  & \rcell{People} & \rcell{Plant} & \rcell[25mm]{Environment} & \rcell{People} & \rcell{Plant} & \rcell[25mm]{Environment}\\ \midrule
Thermal runaway  reaction  (caused by  uncontrollable  temperature rise) & 

\begin{itemize}\item May lead to overpressure in the reactor, resulting in a fire and explosion\end{itemize}                                                                                                                                                                                                                                                                      & Probable                              & Severe                                                        & Severe                                                          & Serious                                                               & \rHi                         & \rHi                           & \rHi                                   \\
Release of  Hydrogen Gas                                                       & \begin{itemize}\item Hydrogen is highly flammable: if exposed     to fire tetrahedron elements, the flammable mixture can      cause a fire and explosion  \item If released into the environment, hydrogen     may contribute to increased levels of     atmospheric methane and ozone via reaction     with hydroxyl radicals \cite{derwent_global_2006}  \end{itemize}& Possible                              & Severe         & Severe          & Severe                                                               & \rHi                         & \rHi                           & \rHi                                 \\
High pressure of hydrogen gas                    & \begin{itemize}\item May lead to overpressure in the reactor, resulting in a fire and explosion \end{itemize}                                                                                                                                                                                                                                                                                                                                                & Possible                              & Severe                                                        & Severe                                                          & Serious                                                               & \rHi                      & \rHi                         & \yMe                                 \\
Leakage of Formic Acid  from storage or  process line                           & \begin{itemize}\item Formic Acid is highly flammable and may result in    a fire if exposed to other fire tetrahedron elements \item Formic Acid is a corrosive substance and may damage the plant equipment \end{itemize}                                                                                                                                                                   & Unlikely                              & Severe                                                        & Severe                                                          & Serious                                                              & \yMe                       & \yMe                         & \yMe                                 \\
Leakage of Propanol  from storage or  process line                           & \begin{itemize}\item Propanol is highly flammable and may result in    a fire if exposed to other fire tetrahedron elements \item Propanol is non-toxic to aquatic life and readily     biodegradable\end{itemize}                                                                                                                                                                   & Unlikely                              & Severe                                                        & Severe                                                          & Serious                                                              & \yMe                       & \yMe                         & \yMe                                 \\
Leakage of Nitric  acid from storage  or process line                        & \begin{itemize}\item Nitric acid is a corrosive, chronic toxin: human ingestion may result in \item Leaching of nitric acid into water streams may  be lethal to aquatic life \item May damage plant equipment due to the corrosive  nature of substance\end{itemize}                                                                                                                & Unlikely                              & Severe                                                        & Moderate                                                        & Very Serious                                                               & \yMe                       & \gLo                            & \yMe                                 \\
Leakage of Methanol from storage  or process line                        & \begin{itemize}\item Methanol is highly flammable and may result in a     fire if exposed to other fire tetrahedron elements \item Toluene is toxic and corrosive: ingestion may     cause long-term health issues and fatality \item Methanol is toxic to aquatic life\end{itemize}                                                                                                                & Unlikely                              & Severe                                                        & Very Serious                                                        & Very Serious                                                               & \yMe                       & \yMe                            & \yMe                                 \\
Leakage of Toluene  from storage  or process line                            & \begin{itemize}\item Toluene is highly flammable and may result in a     fire if exposed to other fire tetrahedron elements \item Toluene is toxic and corrosive: ingestion may     cause long-term health issues and fatality   \item Toluene is toxic to aquatic life. It may also result  in  the production of photochemical smog\end{itemize}                                   & Unlikely                              & Severe                                                        & Very  Serious         & Severe                                                                & \yMe                       & \yMe                         & \yMe   \\ 
Use of combustible dust (Palladium on activated carbon)                           & \begin{itemize}\item An explosion may occur if the combustible dust disperses in confined space (in the presence of heat, oxidants and fuel) \item Primary explosion may result in a more violent secondary explosion\end{itemize}                                   & Possible                              & Severe                                                        & Severe         & Very Serious                                                                & \rHi                       & \rHi                         & \rHi   \\ \bottomrule                         
\end{longtable}	
\end{small}
\end{landscape}

