\section{Plant-wide risk assessment}

According to Lee’s, the major hazards associated to a chemical process plant like Nitroma’s, are fires, explosions and the release of toxic chemicals. Nitroma takes these hazards very seriously, and thus (through identification and analysis of hazards associated to these incidents), we plan to implement appropriate controls which will effectively prevent or minimise both the likelihood and severity. 


\subsection{Fires and explosions}

A fire arises from a chemical combustion reaction which requires four elements from the 'fire tetrahedron'; fuel, oxidant, heat and radicals. Nitroma has identified two oxidants present in the process, oxygen and nitric acid. There have also been multiple flammable substances identified, with hydrogen gas holding the highest flammability (with a NFPA 704 rating of 4). Furthermore, due to the high pressure of hydrogen gas in reactor R201, there may is a risk of explosion as a result of over-pressure. Taking these potential hazards and risks into account, Nitroma has devised various controls to manage each of these issues efficiently; 

\begin{itemize}
    \item installation of a cooling jacket on each reactor to control heat transfer
    \item installation of temperature and pressure indicators that allow constant observation of conditions and warning should they deviate from normal
    \item placement of firewalls 
    \item define emergency shut down procedures in case these these controls fail to prevent temperature rise
\end{itemize}


\subsubsection{Thermal runaway}

Due to the significantly exothermic nature of all the synthesis reactions (especially the nitration reaction), there is a potential risk of thermal runaway, should there be an uncontrollable temperature rise of the system. The occurrence of a thermal runaway reaction may in turn lead to fires and explosions. 



\subsubsection{Dust explosions}

A dust explosion is a specific type of explosion that requires the dispersion of a combustible dust in a confined space in the presence of a ignition source, oxygen and fuel.



\subsubsection{Fire and Explosion Index}

To provide an initial quantitative hazard identification measurement, Dow's fire and explosion (F\&E index) was calculated for each of the reactors in the process. This calculation allows an estimation of the scale of damage that would be experienced, should a fire or explosion occur. The F\&E index also gives an indication as to which unit is most liable to damage. 


\subsection{Release of toxic materials}

Whilst measures have been taken to reduce the hazardous inventory of the plant, there will be numerous chemical substances present at Nitroma's process plant, which can be hazardous to human health. The primary cause of exposure is due to loss of containment from 

\subsection{Past Incidents}

Nitroma recognises the importance of learning from events in the past and so an important aspect of hazard identification is the review of historical case studies. Nitroma hopes to learn from these past mistakes and accidents and use this as insight for the safety of our process plant


\subsection{Risk matrix}

\begin{landscape}
\begin{small}
\begin{longtable}{p{4cm}p{11.5cm}ccccccc}
\caption{Likelihood-Severity Risk Matrix}
\label{tab:risk-matrix}
\setlist{nosep,leftmargin=1em}\\
\toprule
                                                                                                                       \textbf{Hazard Description}  & \textbf{Consequences}                                                                                                                                                                                                                                                                                                                                                                          &  \textbf{Likelihood}                                     & \multicolumn{3}{c}{\textbf{Severity}}                                                                                                                                                                  & \multicolumn{3}{c}{\textbf{Risk}}                                                                                                                                                                       \\ \cmidrule(r){4-6}\cmidrule{7-9} 
                                                                                     &                                                                                                                                                                                                                                                                                                                                    &  & \rcell{People} & \rcell{Plant} & \rcell{Environment} & \rcell{People} & \rcell{Plant} & \rcell{Environment}\\ \midrule
Thermal runaway  reaction  (caused by  uncontrollable  temperature rise) & \begin{itemize}\item May lead to overpressure in the reactor, resulting in a fire and explosion\end{itemize}                                                                                                                                                                                                                                                                      & Probable                              & Severe                                                        & Severe                                                          & Serious                                                               & \rHi                         & \rHi                           & \rHi                                   \\
Release of  Hydrogen Gas                                                       & \begin{itemize}\item Hydrogen is highly flammable: if exposed     to fire tetrahedron elements (oxidant, heat    and ignition), the flammable mixture can      cause a fire and explosion  \item If released into the environment, hydrogen     may contribute to increased levels of     atmospheric methane and ozone via reaction     with hydroxyl radicals \cite{derwent_global_2006}  \end{itemize}& Possible                              & Severe         & Severe          & Severe                                                               & \rHi                         & \rHi                           & \rHi                                 \\
High pressure of hydrogen gas                    & \begin{itemize}\item May lead to overpressure in the reactor, resulting in a fire and explosion \end{itemize}                                                                                                                                                                                                                                                                                                                                                & Possible                              & Severe                                                        & Severe                                                          & Serious                                                               & \rHi                      & \rHi                         & \yMe                                 \\
Leakage of Formic Acid  from storage or  process line                           & \begin{itemize}\item Formic Acid is highly flammable and may result in    a fire if exposed to other fire tetrahedron elements \item Formic Acid is a corrosive substance and may damage the plant equipment \end{itemize}                                                                                                                                                                   & Unlikely                              & Severe                                                        & Severe                                                          & Serious                                                              & \yMe                       & \yMe                         & \yMe                                 \\
Leakage of Propanol  from storage or  process line                           & \begin{itemize}\item Propanol is highly flammable and may result in    a fire if exposed to other fire tetrahedron elements \item Propanol is non-toxic to aquatic life and readily     biodegradable\end{itemize}                                                                                                                                                                   & Unlikely                              & Severe                                                        & Severe                                                          & Serious                                                              & \yMe                       & \yMe                         & \yMe                                 \\
Leakage of Nitric  acid from storage  or process line                        & \begin{itemize}\item Nitric acid is a corrosive, chronic toxin: human ingestion may result in \item Leaching of nitric acid into water streams may  be lethal to aquatic life \item May damage plant equipment due to the corrosive  nature of substance\end{itemize}                                                                                                                & Unlikely                              & Severe                                                        & Moderate                                                        & Very Serious                                                               & \yMe                       & \gLo                            & \yMe                                 \\
Leakage of Methanol from storage  or process line                        & \begin{itemize}\item Methanol is highly flammable and may result in a     fire if exposed to other fire tetrahedron elements \item Toluene is toxic and corrosive: ingestion may     cause long-term health issues and fatality \item Methanol is toxic to aquatic life\end{itemize}                                                                                                                & Unlikely                              & Severe                                                        & Very Serious                                                        & Very Serious                                                               & \yMe                       & \yMe                            & \yMe                                 \\
Leakage of Toluene  from storage  or process line                            & \begin{itemize}\item Toluene is highly flammable and may result in a     fire if exposed to other fire tetrahedron elements \item Toluene is toxic and corrosive: ingestion may     cause long-term health issues and fatality   \item Toluene is toxic to aquatic life. It may also result  in  the production of photochemical smog\end{itemize}                                   & Unlikely                              & Severe                                                        & Very  Serious         & Severe                                                                & \yMe                       & \yMe                         & \yMe   \\ 
Use of combustible dust (Palladium on activated carbon)                           & \begin{itemize}\item An explosion may occur if the combustible dust disperses in confined space (in the presence of heat, oxidants and fuel) \item Primary explosion may result in a more violent secondary explosion\end{itemize}                                   & Possible                              & Severe                                                        & Severe         & Very Serious                                                                & \rHi                       & \rHi                         & \rHi   \\ \bottomrule                         
\end{longtable}
\end{small}
\end{landscape}


\begin{table}[H]
\centering
\caption{Likelihood-Severity-Risk scoring methodology}
\label{tab:likelihood-severity-risk}
\begin{tabular}{cccccc}
\toprule
Severity & \multicolumn{5}{c}{Likelihood}                                                                                                                                    \\ \cmidrule{2-6} 
    & A     & B     & C     & D     & E    \\ \midrule
5   & \yMe  & \yMe  & \rHi  & \rHi  & \rHi \\ 
4   & \yMe  & \yMe  & \rHi  & \rHi  & \rHi \\ 
3   & \yMe  & \yMe  & \yMe  & \rHi  & \rHi \\ 
2   & \gLo  & \gLo  & \yMe  & \yMe  & \rHi \\ 
1   & \gLo  & \gLo  & \gLo  & \gLo  & \yMe \\ \bottomrule
\end{tabular}
\end{table}

\begin{table}[H]
\centering
\caption{Severity scoring methodology}
\label{tab:severity-methodology}
\begin{tabularx}{\linewidth}{llXXX}
\toprule
\multicolumn{2}{l}{\textbf{Severity}} & \textbf{To the People}                                     & \textbf{To the Plant}      & \textbf{To the Environment}      \\ \midrule
1          & Minor             & First Aid                                                  & Superficial Damage         & Slight/no effect                 \\
2          & Moderate          & Medical Care                                               & Repair Needed              & Minor effect                     \\
3          & Serious           & Disabling, Fracture, Hospitalisation (\SI{>24}{\hour})     & Loss of a Process Item     & Short-term Localised Damage      \\
4          & Very Serious      & One Fatality                                               & Local Destruction of Plant & Major/long-term Localised Effect \\
5          & Severe            & Several Fatalitie                                          & Complete Destruction       & Multiple Environments Affected   \\ \bottomrule
\end{tabularx}
\end{table}

\begin{table}[H]
\centering
\caption{Likelihood scoring methodology}
\label{tab:likelihood-methodology}
\begin{tabular}{llc}
\toprule
\multicolumn{2}{l}{Likelihood} & Probability ($P$) in 1 year              \\ \midrule
A & improbable & $         P_\mathrm{once} < 0.001 $ \\
B & unlikely   & $ 0.001 < P_\mathrm{once} < 0.01  $ \\
C & possible   & $ 0.01  < P_\mathrm{once} < 0.1   $ \\
D & probable   & $ 0.1   < P_\mathrm{once} < 1     $ \\
E & frequent   & $         P_\mathrm{several} = 1  $ \\ \bottomrule
\end{tabular}
\end{table}