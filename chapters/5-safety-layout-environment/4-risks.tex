\section{Qualitative plant wide risk assessment}

The major hazards in a chemical process plants like Nitroma’s, are fires, explosions and the release of toxic chemicals \cite{mannan_lees_2012}. Nitroma takes these hazards very seriously, and thus, through the identification and analysis of hazards associated with these incidents, plans to implement appropriate controls which will either prevent or significantly minimise the likelihood and/or severity. 


\subsection{Past Incidents}

Nitroma recognises the importance of learning from events in the past and so an important aspect of hazard identification is the review of historical case studies. Incidents explored include thermal runaway, explosion, storage of hazardous chemicals and acid spillage, all of which are relevant risks. Nitroma hopes to learn from these past mistakes and accidents and use this as insight for the safety of our process plant.

\subsection{Fire and explosions}

Fire from chemical combustion reactions require four elements from the 'fire tetrahedron': fuel, oxidant, heat and radicals. Nitroma has identified two oxidants present in the process, oxygen and nitric acid. Multiple flammable substances have also been identified, with hydrogen gas holding the highest flammability (with a NFPA 704 rating of 4). Due to the significantly exothermic nature of all the synthesis reactions (especially the nitration reaction), there is a potential risk of thermal runaway, should there be an uncontrollable temperature rise in the system. The occurrence of a thermal runaway reaction may lead to a fire and explosion.  

Furthermore, due to the high pressure of hydrogen gas in reactor R201, there is a risk of explosion caused by overpressure. Taking these potential hazards and risks into account, Nitroma has devised various controls to manage each of these issues effectively: 


\begin{itemize}
\item Installation of pressure and temperature transmitters with alarms to allow constant monitoring and defined operator actions in the case of deviation from normal operating conditions.
         \item Executive temperature and pressure alarms on process units with an interlock to shut down units before fire or explosion can occur. 
         \item Cooling jackets around major reactor units (and in the case of R101, a cooling tube will be placed within the reactor itself), to maintain operating temperature and prevent thermal runaway.
          \item Installation of pressure relief valves which will relieve the pressure of a certain region (should the value rise above the specified set point).
    \item Increased distance between equipment and storage units to prevent sequential damage and events.
    \item Storage units containing flammable and toxic materials will be encased by a bund as protection against leakage.
\end{itemize}


\subsubsection{Dust explosions}

A dust explosion is a specific type of explosion that requires the dispersion of a combustible dust in a confined space in the presence of an ignition source, oxygen and fuel. As mentioned previously, the use of palladium-on-carbon as a catalyst presents a pathway to a dust explosion occurring as it is classed as a combustible dust. Whilst the pellet form of the catalyst will be used, there may be carbon powder residue on the pellets and so dust explosion is still possible. The plant will have a carefully situated ventilation system to minimise the escape of dust in these areas, and regular checks will be performed to ensure there is no dust build up.
Frequent inspections on all process units and storage areas will be carried out, with specific guidelines on cleaning put in place. 

\subsection{Hazardous chemicals \& their potential release}

The hazards associated with each of the chemicals used within the plant have been identified through the use of the Globally Harmonized System (GHS). The hazard codes for each substance have been listed and communicated to the entire workforce at Nitroma, so staff have a full understanding of hazards. Whilst all chemicals must be treated with caution, here we will draw attention to some of the key substances in relation to the NFPA 704 values concerning health, flammability and reactivity. 

\paragraph{Flammability:}

Hydrogen gas is the most flammable substance on-site, the only chemical with a NFPA flammability rating of 4.  Hydrogen is extremely flammable at atmospheric conditions and will readily disperse in air if emitted. As such, care must be taken to ensure that hydrogen is handled appropriately, with safe storage, material selection and use within the process. The hydrogen gas will be stored in a pressure container placed outside in an isolated area of the plant, away from other storage units, equipment and ignition source.  Toluene and methanol are also highly flammable, with a rating of 3. These liquids will be stored in a bund surrounding their respective storage vessels. 


\paragraph{Health:}

The substance with the highest NFPA rating of 4 for health was nitric acid, with oxygen, p-toluidine, o-toludine and the nitrotoluenes  following with a rating of 3. There will be restricted access to areas containing these chemicals with personal protective equipment (including gloves, face masks and shield goggles) provided for those with access to these sites. Minimal human interaction is required in the production area to reduce the risk of exposure.

\paragraph{Reactivity:}

All chemicals in the process were rated between 0-1, indicating these substances are relatively stable, however controls will be put in place to ensure that both the reaction conditions and storage conditions are met. 

There are numerous chemical substances at Nitroma's process plant which can be hazardous to human health and the surrounding environment. The primary cause of exposure to toxic substances is due to loss of containment during normal operation and during maintenance. As mentioned previously, the continuous nature of the process plant will allow minimal handling of chemicals, but the workforce may still be at risk of exposure due to transfer of imports and exports. Unintentional emissions of untreated toxic substances can have an adverse impact on land, air and sea, potentially inflicting serious long-term effects. To prevent and mitigate the risk, the following measures have been taken;

\begin{itemize}
    \item installation of ventilation systems to remove any toxic vapours or fumes 
    \item installation of level indicators and alarms to prevent overflow in reactor vessels , with interlocks in place to act if level is critically high
    \item label storage area with hazard warnings and ensure area is clear from obstructions
    \item ensure workforce are fully trained, following MSDS procedures and guidance for each of the substances in the process and are equipped with appropriate PPE 
    \item regular inspection of storage and process lines, to identify any leaks, damage or corrosion 
    \item carry out regular medical examinations for plant personnel to identify any health issues related to toxicological effects of chemicals on-site
\item toxic materials will be stored within an bund, to prevent exposure in case of leakage 
\end{itemize}


% Please add the following required packages to your document preamble:
% \usepackage{booktabs}
% \usepackage{multirow}
% \usepackage[table,xcdraw]{xcolor}
% If you use beamer only pass "xcolor=table" option, i.e. \documentclass[xcolor=table]{beamer}
% \usepackage{lscape}
% \usepackage{longtable}
% Note: It may be necessary to compile the document several times to get a multi-page table to line up properly
\begin{landscape}
\begin{longtable}[c]{@{}lllllllll@{}}
\caption{Plant-wide Risk Matrix}
\label{tab:riskmatrix}\\
\toprule
 &  &  & \multicolumn{3}{l}{\textbf{Severity}} & \multicolumn{3}{l}{\textbf{Risk}} \\* \cmidrule(l){4-9} 
\multirow{-2}{*}{\textbf{Hazard Description}} & \multirow{-2}{*}{\textbf{Consequences}} & \multirow{-2}{*}{\textbf{Likelihood}} & \textbf{\begin{tabular}[c]{@{}l@{}}To \\ people\end{tabular}} & \textbf{\begin{tabular}[c]{@{}l@{}}To \\ Plant\end{tabular}} & \textbf{\begin{tabular}[c]{@{}l@{}}To \\ environment\end{tabular}} & \textbf{\begin{tabular}[c]{@{}l@{}}To \\ people\end{tabular}} & \textbf{\begin{tabular}[c]{@{}l@{}}To \\ plant\end{tabular}} & \textbf{\begin{tabular}[c]{@{}l@{}}To \\ environment\end{tabular}} \\* \cmidrule(r){1-3}
\endhead
%
\bottomrule
\endfoot
%
\endlastfoot
%
\begin{tabular}[c]{@{}l@{}}Thermal runaway reaction (caused by uncontrollable temperature rise) \\ in reactors\end{tabular} & *   May lead to overpressure in the reactor, resulting in a fire and explosion & Probable & Severe & Severe & Serious & \cellcolor[HTML]{F85233}H & \cellcolor[HTML]{F85233}H & \cellcolor[HTML]{F85233}H \\
\multicolumn{2}{l}{\begin{tabular}[c]{@{}l@{}}Controls: Temperature and pressure controllers, transmitters and alarms and emergency procedures.  Pressure relief valve in reactors to prevent \\ overpressure. Executive alarms for temperature and pressure, with an corresponding interlock. Cooling installed around the reactors. PPE for \\ operators. On-site fire station in case fire occurs. Large separation distance from other units based on radius of exposure from Dow's F\&E\end{tabular}} & Possible & Very Serious & Very Serious & Serious & \cellcolor[HTML]{F85233}H & \cellcolor[HTML]{F85233}H & \cellcolor[HTML]{F8FF00}M \\
Release of hydrogen gas & \begin{tabular}[c]{@{}l@{}}* Hydrogen is highly flammable: if exposed to fire tetrahedron elements, the flammable \\ the mixture can cause a fire and explosion\\ * If released into the environment, hydrogen may contribute to increased levels of   \\ atmospheric methane and ozone via reaction with hydroxyl radicals\end{tabular} & Possible & Severe & Severe & Severe & \cellcolor[HTML]{F85233}H & \cellcolor[HTML]{F85233}H & \cellcolor[HTML]{F85233}H \\
\multicolumn{2}{l}{\begin{tabular}[c]{@{}l@{}}Controls: Hydrogen gas stored  in a pressure container isolated from other storage areas and production area. No ignition source near the production area. Sprinkler \\ system and on-site fire station in case fire occurs.\end{tabular}} & Unlikely & Very Serious & Very Serious & Serious & \cellcolor[HTML]{F8FF00}M & \cellcolor[HTML]{F8FF00}M & \cellcolor[HTML]{F8FF00}M \\
High pressure of hydrogen gas & * May lead to overpressure in the reactor, resulting in a fire and explosion. & Possible & Severe & Severe & Serious & \cellcolor[HTML]{F85233}H & \cellcolor[HTML]{F85233}H & \cellcolor[HTML]{F8FF00}M \\
\multicolumn{2}{l}{\begin{tabular}[c]{@{}l@{}}Controls: Pressure relief valve, large separation distance from other units based on radius\\  of exposure from Dow's F&EI. PPE for operators. Executive alarms for temperature and pressure, with an corresponding interlock.\end{tabular}} & Unlikely & Severe & Very Serious & Serious & \cellcolor[HTML]{F8FF00}M & \cellcolor[HTML]{F8FF00}M & \cellcolor[HTML]{F8FF00}M \\
Leakage of formic acid from storage or process line & \begin{tabular}[c]{@{}l@{}}* Formic acid is highly flammable and may result in a fire if exposed to other fire \\ tetrahedron elements\\ * Formic acid is a corrosive substance and may damage the plant equipment\end{tabular} & Unlikely & Severe & Severe & Serious & \cellcolor[HTML]{F8FF00}M & \cellcolor[HTML]{F8FF00}M & \cellcolor[HTML]{F8FF00}M \\
\multicolumn{2}{l}{Controls:Installation of chemical leak detection device and regular inspection. Compatibile material selection with lining used where appropriate} & Improbable & Severe & Severe & Serious & \cellcolor[HTML]{F8FF00}M & \cellcolor[HTML]{F8FF00}M & \cellcolor[HTML]{F8FF00}M \\
Leakage of Nitric acid from storage or process line & \begin{tabular}[c]{@{}l@{}}*  Nitric acid is a corrosive, chronic toxin: human ingestion may result in\\ *leaching of nitric acid into water streams may be lethal to aquatic life\\ *May damage plant equipment due to the corrosive nature of the substance\end{tabular} & Unlikely & Severe & Moderate & Very Serious & \cellcolor[HTML]{FCFF2F}M & \cellcolor[HTML]{9AFF99}L & \cellcolor[HTML]{F8FF00}M \\
\multicolumn{2}{l}{\begin{tabular}[c]{@{}l@{}}Controls:Installation of chemical leak detection device and regular inspection. Compatible material selection with good resistance to nitric acid. Storage \\ will be bunded for leakage protection\end{tabular}} & Improbable & Severe & Moderate & Very Serious & \cellcolor[HTML]{FCFF2F}M & \cellcolor[HTML]{9AFF99}L & \cellcolor[HTML]{F8FF00}M \\
Leakage of Methanol from storage or process line & \begin{tabular}[c]{@{}l@{}}* Methanol is highly flammable and may result in a fire if exposed to other fire \\ tetrahedron elements\\ * Methanol is toxic and corrosive: ingestion may cause long-term health issues \\ and fatality \\ * Methanol is toxic to aquatic life\end{tabular} & Unlikely & Severe & Very Serious & Very Serious & \cellcolor[HTML]{F8FF00}M & \cellcolor[HTML]{F8FF00}M & \cellcolor[HTML]{F8FF00}M \\
\multicolumn{2}{l}{\begin{tabular}[c]{@{}l@{}}Controls: Installation of chemical leak detection device and regular inspection. Storage will be bunded for leakage protection. Compatible material selection,\\ suited for methanol\end{tabular}} & Improbable & Severe & Very Serious & Very Serious & \cellcolor[HTML]{F8FF00}M & \cellcolor[HTML]{F8FF00}M & \cellcolor[HTML]{F8FF00}M \\
Leakage of Toluene from storage or process line & \begin{tabular}[c]{@{}l@{}}* Toluene is highly flammable and may result in a fire if exposed to another fire \\ tetrahedron elements \\ *Toluene is toxic and corrosive: ingestion may cause long-term health issues and fatality  \\ * Toluene is toxic to aquatic and plant life. It may also result in the production of \\ photochemical smog\end{tabular} & Unlikely & Severe & Very Serious & Severe & \cellcolor[HTML]{F8FF00}M & \cellcolor[HTML]{F8FF00}M & \cellcolor[HTML]{F8FF00}M \\
\multicolumn{2}{l}{\begin{tabular}[c]{@{}l@{}}Controls:Installation of chemical leak detection device and regular inspection. Storage will\\  be bunded for leakage protection\end{tabular}} & Improbable & Severe & Very Serious & Severe & \cellcolor[HTML]{F8FF00}M & \cellcolor[HTML]{F8FF00}M & \cellcolor[HTML]{F8FF00}M \\
Use of combustible dust (Palladium on activated carbon) & \begin{tabular}[c]{@{}l@{}}*An explosion may occur if the combustible dust disperses in confined space (in \\ the presence of heat, oxidants and fuel)\\ *Primary explosion may result in a secondary explosion\end{tabular} & Possible & Severe & Severe & Very Serious & \cellcolor[HTML]{F85233}H & \cellcolor[HTML]{F85233}H & \cellcolor[HTML]{F85233}H \\
\multicolumn{2}{l}{Controls: Hazardous dust inspection and proper ventilation with vent installed. PPE for operators. Instead of using the catalyst in dust form, the pellet form will be used.} & Unlikely & Very Serious & Very Serious & Serious & \cellcolor[HTML]{F8FF00}M & \cellcolor[HTML]{F8FF00}M & \cellcolor[HTML]{F8FF00}M \\
Human Error & *Negligence or blunder leading to severe incidents, sequential event of fire and explosions & Possible & Severe & Severe & Severe & \cellcolor[HTML]{F85233}H & \cellcolor[HTML]{F85233}H & \cellcolor[HTML]{F85233}H \\
\multicolumn{2}{l}{\begin{tabular}[c]{@{}l@{}}Controls: Operators trained in abnormal operations and given PPE. Shift change-over with accurate and efficient information exchanged. Continuous process, \\ less human interaction required.\end{tabular}} & Unlikely & Severe & Severe & Severe & \cellcolor[HTML]{F8FF00}M & \cellcolor[HTML]{F8FF00}M & \cellcolor[HTML]{F8FF00}M \\* \bottomrule
\end{longtable}
\end{landscape}

