\section{Plant-wide risk assessment}

According to Lee’s, the major hazards associated to a chemical process plant like Nitroma’s, are fires, explosions and the release of toxic chemicals. Nitroma takes these hazards very seriously, and thus (through identification and analysis of hazards associated to these incidents), we plan to implement appropriate controls which will effectively prevent or minimise both the likelihood and severity. 

\subsection{Past Incidents}

Nitroma recognises the importance of learning from events in the past and so an important aspect of hazard identification is the review of historical case studies. Nitroma hopes to learn from these past mistakes and accidents and use this as insight for the safety of our process plant


\subsection{Fires and explosions}

A fire arises from a chemical combustion reaction which requires four elements from the 'fire tetrahedron'; fuel, oxidant, heat and radicals. Nitroma has identified two oxidants present in the process, oxygen and nitric acid. There have also been multiple flammable substances identified, with hydrogen gas holding the highest flammability (with a NFPA 704 rating of 4). Furthermore, due to the high pressure of hydrogen gas in reactor R201, there may is a risk of explosion as a result of over-pressure. Taking these potential hazards and risks into account, Nitroma has devised various controls to manage each of these issues efficiently; 

\begin{itemize}
    \item installation of a cooling jacket on each reactor to control heat transfer
    \item installation of temperature and pressure indicators that allow constant observation of conditions and warning should they deviate from normal
    \item placement of firewalls 
    \item define emergency shut down procedures in case these these controls fail to prevent temperature rise
\end{itemize}


\subsubsection{Thermal runaway}

Due to the significantly exothermic nature of all the synthesis reactions (especially the nitration reaction), there is a potential risk of thermal runaway, should there be an uncontrollable temperature rise of the system. The occurrence of a thermal runaway reaction may in turn lead to fires and explosions. 



\subsubsection{Dust explosions}

A dust explosion is a specific type of explosion that requires the dispersion of a combustible dust in a confined space in the presence of a ignition source, oxygen and fuel.



\subsubsection{Fire and Explosion Index}

To provide an initial quantitative hazard identification measurement, Dow's fire and explosion (F\&E index) was calculated for each of the reactors in the process. This calculation allows an estimation of the scale of damage that would be experienced, should a fire or explosion occur. The F\&E index also gives an indication as to which unit is most liable to damage. 


\subsection{Release of toxic materials}

Whilst measures have been taken to reduce the hazardous inventory of the plant, there will be numerous chemical substances present at Nitroma's process plant, which can be hazardous to human health. The primary cause of exposure is due to loss of containment from 

\subsection{Risk matrix}