\section{Plant-wide risk assessment}

The major hazards associated to a chemical process plant like Nitroma’s, are fires, explosions and the release of toxic chemicals. Nitroma takes these hazards very seriously, and thus (through identification and analysis of hazards associated with these incidents), we plan to implement appropriate controls which will either prevent or significantly minimise both the likelihood and severity. 
%reference lee's

\subsection{Past Incidents}

Nitroma recognises the importance of learning from events in the past and so an important aspect of hazard identification is the review of historical case studies. Nitroma hopes to learn from these past mistakes and accidents and use this as insight for the safety of our process plant.

%insert table





\subsection{Fire and explosions}

Fire arises from chemical combustion reactions which require four elements from the 'fire tetrahedron' - fuel, oxidant, heat and radicals. Nitroma has identified two oxidants present in the process, oxygen and nitric acid. There have also been multiple flammable substances identified, with hydrogen gas holding the highest flammability (with a NFPA 704 rating of 4). 

Due to the significantly exothermic nature of all the synthesis reactions (especially the nitration reaction), there is a potential risk of thermal runaway, should there be an uncontrollable temperature rise of the system. The occurrence of a thermal runaway reaction may in turn lead to fire and explosions.  

Furthermore, due to the high pressure of hydrogen gas in reactor R201, there is a risk of explosion caused by overpressure. Taking these potential hazards and risks into account, Nitroma has devised various controls to manage each of these issues effectively; 


\item installation of pressure and temperature transmitters to allow constant monitoring, with alarms in place to warn operator of deviation from normal operating conditions
         \item executive temperature and pressure alarms on process units with an interlock to shut down unit before fire or explosion can occur 
         \item Cooling jackets around major reactor units (and in the case of R101, a cooling tube will be placed within the reactor itself), to maintain temperature and prevent thermal runaway 
         \item 
         
    
    
    
 
    \item reactor units operate at lowest possible temperature that allows production target range to be achieved
    \item installation of pressure relief valves, stopping reaction when pressure increases significantly %significantly or above set point?  
    \item Flammable materials stored below their auto-ignition temperature, and away from ignition sources
    \item Increased distance between equipment to prevent sequential damage and events
    \item installation of fire detectors and alarms

    
\end{itemize}


\subsubsection{Dust explosions}

A dust explosion is a specific type of explosion that requires the dispersion of a combustible dust in a confined space in the presence of an ignition source, oxygen and fuel. As mentioned previously, the use of palladium-on-carbon as a catalyst presents a pathway to a dust explosion occurring as it is classed as a combustible dust. Whilst the pellet form of the catalyst will be used, there may be carbon powder residue on the pellets and so dust explosion is still possible. Regular inspections on all process units and storage areas will be carried out, with specific guidelines on cleaning put in place to prevent any incident. 


\subsection{Release of toxic materials}

Whilst measures have been taken to reduce the hazardous inventory of the plant, there will be numerous chemical substances present at Nitroma's process plant, which can be hazardous to human health and the surrounding environment. The primary cause of exposure of toxic substances is due to loss of containment during normal operation and during maintenance. As mentioned previously, the continuous nature of the process plant will allow minimal handling of chemicals, the workforce may still be at risk of exposure due to transfer of imports and exports. Unintentional emissions of untreated substances can have an adverse impact on land, air and sea, potentially inflicting serious long-term effects. To prevent and mitigate the risk, the following measures have been taken;

\begin{itemize}
    \item installation of ventilation systems to remove any toxic vapours or fumes 
    \item installation of level indicators and alarms to prevent overflow in reactor vessels 
    \item label storage area with hazard warnings and ensure area is clear from obstructions
    \item ensure workforce are fully trained, following MSDS procedures and guidance for each of the substances in the process and are equipped with appropriate PPE 
    \item regular inspection of storage and process lines, to identity any leaks or damage
    \item carry out regular medical examinations for plant personnel to identify any health issues related to toxicological effects of chemicals on-site
\end{itemize}


\subsection{Specific hazardous chemicals}

The hazards associated to each of the chemicals used within the plant have  been identified through the use of the the Globally Harmonized System (GHS). The hazard codes associated to each substance has been listed and communicated to the entire workforce at Nitroma, so staff have a full understanding of hazards. Whilst all chemicals must be treated with caution, in this section we draw attention to some of the key substances in relation to the NFPA 704 values concerning health, flammability and reactivity. 

\subsubsection{Flammability}

Hydrogen gas is the most flammable substance that will be on-site, this being the only chemical with a NFPA flammability rating of 4.  Hydrogen is extremely flammable at atmospheric conditions and will readily disperse in air if emitted. As such, care must be taken to ensure that hydrogen is handled appropriately, with safe storage, material selection and use within the process. The hydrogen gas storage will be placed outside in an isolated area of the plant, away from other storage units and equipment. The vessel itself will be surrounded encased by a bund, as additional protection against leakage and emission. Toluene and methanol are also highly flammable, with a rating of 3. Likewise, these liquids will be stored in a bund surrounding their respective storage vessels. 

%cross-ref to plant layout


\subsubsection{Health}

The substance with the highest NFPA rating of 4 for health was nitric acid, with oxygen, p-toluidine, o-toludine and the nitrotoluenes following with a rating of 3. 

Whilst given the rating of 2, it is important to note that cobalt phthalocyanine is susceptible of causing cancer (GHS: H351). 


\subsubsection{Reactivity}

All chemicals in the process were given a rating between 0-1, indicating that there is a 





\subsection{Emergency procedures and measures}

Whilst Nitroma has identified potential hazards which may compromise the health and safety of the workforce and taken appropriate measures to control these effectively, we recognise that accidents may still occur and have thus put a number of emergency measures in place. 

\begin{itemize}
    \item placement of emergency showers, eye-wash unit and respirators
\item Fire detectors, alarms and extinguishers will be on site, with an on-site fire-station at hand
    \item Emergency exits in each section of the plant (see plant layout) for efficient evacuation 
\item  
\end{itemize}




\subsection{Risk matrix}

\begin{landscape}
\begin{small}
\begin{longtable}{p{4cm}p{11.5cm}ccccccc}
\caption{Likelihood-Severity Risk Matrix}
\label{tab:risk-matrix}\\
\toprule
                                                                                                                       \textbf{Hazard Description}  & \textbf{Consequences}                                                                                                                                                                                                                                                                                                                                                                          &  \textbf{Likelihood}                                     & \multicolumn{3}{c}{\textbf{Severity}}                                                                                                                                                                  & \multicolumn{3}{c}{\textbf{Risk}}                                                                                                                                                                       \\ \cmidrule(r){4-6}\cmidrule{7-9} 
                                                                                     &                                                                                                                                                                                                                                                                                                                                    &  & \rcell{People} & \rcell{Plant} & \rcell[25mm]{Environment} & \rcell{People} & \rcell{Plant} & \rcell[25mm]{Environment}\\ \midrule
Thermal runaway  reaction  (caused by  uncontrollable  temperature rise) & \begin{itemize}\item May lead to overpressure in the reactor, resulting in a fire and explosion\end{itemize}                                                                                                                                                                                                                                                                      & Probable                              & Severe                                                        & Severe                                                          & Serious                                                               & \rHi                         & \rHi                           & \rHi                                   \\
Release of  Hydrogen Gas                                                       & \begin{itemize}\item Hydrogen is highly flammable: if exposed     to fire tetrahedron elements, the flammable mixture can      cause a fire and explosion  \item If released into the environment, hydrogen     may contribute to increased levels of     atmospheric methane and ozone via reaction     with hydroxyl radicals \cite{derwent_global_2006}  \end{itemize}& Possible                              & Severe         & Severe          & Severe                                                               & \rHi                         & \rHi                           & \rHi                                 \\
High pressure of hydrogen gas                    & \begin{itemize}\item May lead to overpressure in the reactor, resulting in a fire and explosion \end{itemize}                                                                                                                                                                                                                                                                                                                                                & Possible                              & Severe                                                        & Severe                                                          & Serious                                                               & \rHi                      & \rHi                         & \yMe                                 \\
Leakage of Formic Acid  from storage or  process line                           & \begin{itemize}\item Formic Acid is highly flammable and may result in    a fire if exposed to other fire tetrahedron elements \item Formic Acid is a corrosive substance and may damage the plant equipment \end{itemize}                                                                                                                                                                   & Unlikely                              & Severe                                                        & Severe                                                          & Serious                                                              & \yMe                       & \yMe                         & \yMe                                 \\
Leakage of Propanol  from storage or  process line                           & \begin{itemize}\item Propanol is highly flammable and may result in    a fire if exposed to other fire tetrahedron elements \item Propanol is non-toxic to aquatic life and readily     biodegradable\end{itemize}                                                                                                                                                                   & Unlikely                              & Severe                                                        & Severe                                                          & Serious                                                              & \yMe                       & \yMe                         & \yMe                                 \\
Leakage of Nitric  acid from storage  or process line                        & \begin{itemize}\item Nitric acid is a corrosive, chronic toxin: human ingestion may result in \item Leaching of nitric acid into water streams may  be lethal to aquatic life \item May damage plant equipment due to the corrosive  nature of substance\end{itemize}                                                                                                                & Unlikely                              & Severe                                                        & Moderate                                                        & Very Serious                                                               & \yMe                       & \gLo                            & \yMe                                 \\
Leakage of Methanol from storage  or process line                        & \begin{itemize}\item Methanol is highly flammable and may result in a     fire if exposed to other fire tetrahedron elements \item Toluene is toxic and corrosive: ingestion may     cause long-term health issues and fatality \item Methanol is toxic to aquatic life\end{itemize}                                                                                                                & Unlikely                              & Severe                                                        & Very Serious                                                        & Very Serious                                                               & \yMe                       & \yMe                            & \yMe                                 \\
Leakage of Toluene  from storage  or process line                            & \begin{itemize}\item Toluene is highly flammable and may result in a     fire if exposed to other fire tetrahedron elements \item Toluene is toxic and corrosive: ingestion may     cause long-term health issues and fatality   \item Toluene is toxic to aquatic life. It may also result  in  the production of photochemical smog\end{itemize}                                   & Unlikely                              & Severe                                                        & Very  Serious         & Severe                                                                & \yMe                       & \yMe                         & \yMe   \\ 
Use of combustible dust (Palladium on activated carbon)                           & \begin{itemize}\item An explosion may occur if the combustible dust disperses in confined space (in the presence of heat, oxidants and fuel) \item Primary explosion may result in a more violent secondary explosion\end{itemize}                                   & Possible                              & Severe                                                        & Severe         & Very Serious                                                                & \rHi                       & \rHi                         & \rHi   \\ \bottomrule                         
\end{longtable}
\end{small}
\end{landscape}


\begin{table}[H]
\centering
\caption{Likelihood-Severity-Risk scoring methodology}
\label{tab:likelihood-severity-risk}
\begin{tabular}{cccccc}
\toprule
Severity & \multicolumn{5}{c}{Likelihood}                                                                                                                                    \\ \cmidrule{2-6} 
    & A     & B     & C     & D     & E    \\ \midrule
5   & \yMe  & \yMe  & \rHi  & \rHi  & \rHi \\ 
4   & \yMe  & \yMe  & \rHi  & \rHi  & \rHi \\ 
3   & \yMe  & \yMe  & \yMe  & \rHi  & \rHi \\ 
2   & \gLo  & \gLo  & \yMe  & \yMe  & \rHi \\ 
1   & \gLo  & \gLo  & \gLo  & \gLo  & \yMe \\ \bottomrule
\end{tabular}
\end{table}

\begin{table}[H]
\centering
\caption{Severity scoring methodology}
\label{tab:severity-methodology}
\begin{tabularx}{\linewidth}{llXXX}
\toprule
\multicolumn{2}{l}{\textbf{Severity}} & \textbf{To the People}                                     & \textbf{To the Plant}      & \textbf{To the Environment}      \\ \midrule
1          & Minor             & First Aid                                                  & Superficial Damage         & Slight/no effect                 \\
2          & Moderate          & Medical Care                                               & Repair Needed              & Minor effect                     \\
3          & Serious           & Disabling, Fracture, Hospitalisation (\SI{>24}{\hour})     & Loss of a Process Item     & Short-term Localised Damage      \\
4          & Very Serious      & One Fatality                                               & Local Destruction of Plant & Major/long-term Localised Effect \\
5          & Severe            & Several Fatalities                                          & Complete Destruction       & Multiple Environments Affected   \\ \bottomrule
\end{tabularx}
\end{table}

\begin{table}[H]
\centering
\caption{Likelihood scoring methodology}
\label{tab:likelihood-methodology}
\begin{tabular}{llc}
\toprule
\multicolumn{2}{l}{Likelihood} & Probability ($P$) in 1 year              \\ \midrule
A & improbable & $         P_\mathrm{once} < 0.001 $ \\
B & unlikely   & $ 0.001 < P_\mathrm{once} < 0.01  $ \\
C & possible   & $ 0.01  < P_\mathrm{once} < 0.1   $ \\
D & probable   & $ 0.1   < P_\mathrm{once} < 1     $ \\
E & frequent   & $         P_\mathrm{several} = 1  $ \\ \bottomrule
\end{tabular}
\end{table}