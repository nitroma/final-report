\section{Layers of protection analysis (LOPA)}

A LOPA was conducted for one of the major consequences identified from the HAZOP analysis. Focusing on node 1, the chosen consequence was overpressure in R201, resulting to a fire (due to presence of flammable materials and oxidant) and explosion . This consequence was chosen due to the significant affects and losses that are associated with a fire and explosion caused by overpressurization, which includes the loss of production of o-toluidine, the release of harmful substance and subsequent damage to the multiple environments (air, land), equipment and possibly major plant damage and serious injury/fatality for those within the facility. 

There are three main initiating causes (ICs) which can lead to overpressurization of the reaction vessel that have been identified;

\begin{itemize}
\item \textbf{IC1} - Bed blockage within in reactor R201, causing build up of pressure (failure independent of control system)
\item \textbf{IC2} - Failure of pressure controller, PIC201 (basic process control system failure)
\item \textbf{IC3 }- Blockage in FCV-202, restricting hydrogen recycle stream from reactor R201 (Self acting control valve failure)

\end{itemize}
 For this major consequence, the target mitigated event likelihood (TMEL) taken as 1x10-6yr-1, which corresponds to one onsite fatality as a result of the explosion. This is the commonly used TMEL used by many companies within this industry. It was assumed that Nitroma's plant would operate for 24 hours a day for 7 days a week, which is a common industry standard \cite{job_guide_chemical_2021}. In total, the plant will be operating for 6600 hours (or 275 days) a year which results in value of X\% for the time when the risk is present (which is equivalent to a frequency of XE-01).  
 
% TIME VALUE
 
Whilst there will be flammable material present in R102 which could lead to an explosion in the event of overpressure arising, the presence of hydrogen gas means that an ignition source is not necessary for an explosion to occur. Thus, the probability of ignition was taken to be 100\%, with the probability that ignition would lead to an explosion being taken as 1. As the process is continuous, there will be minimal human involvement in the operation during steady state and thus the the probability of exposure was taken as 0.1. 
 
 %Alarm and Operator Response Independent of BPCS Shut Down

 
 There are numerous  independent protection layers (IPLs) linked to R201 in relation to overpressure. These are as follows;
 
 \begin{itemize}
\item  \textbf{IPL1} - pressure controller (PIC201), responding to pressure increase in R201
\item  \textbf{IPL2} - pressure relief device (PRV-1), with no fouling in the device assumed 
\item  \textbf{IPL3} - independent high pressure executive (high high) alarm PZAHH204 and corresponding trip
 \end{itemize}

%check is level exec alarm would be IPL


The applicability of each IPL varied with the three ICs. For IC1, all of the IPLs were relevant. For IC2, all IPLs applied except IPL1 as it is a fault in the pressure controller itself that initiates overpressure. 







 