\section{Layers of protection analysis (LOPA)}

A LOPA was conducted for one of the major consequences identified from the HAZOP analysis. Focusing on node 1, the chosen consequence was overpressure in R201, resulting to a fire (due to presence of flammable materials and oxidant) and explosion . This consequence was chosen due to the significant affects and losses that are associated with a fire and explosion caused by overpressurization, which includes the loss of production of o-toluidine, the release of harmful substance and subsequent damage to the multiple environments (air, land), equipment and possibly major plant damage and serious injury/fatality for those within the facility. 

There are three main initiating causes (ICs) which can lead to overpressurization of the reaction vessel that have been identified;

\begin{itemize}
\item Failure of pressure controller, PIC201 (basic process control system failure)
\item Bed blockage within in reactor R201, causing build up of pressure (failure independent of control system)
\item Blockage in FCV-202, restricting hydrogen recycle stream from reactor R201 (Self acting control valve failure)

\end{itemize}
 For this major consequence, the target mitigated event likelihood (TMEL) taken as 1x10-5yr-1, which corresponds to one onsite fatality as a result of the explosion. This is the commonly used TMEL used by many companies within this industry. It was assumed that Nitroma's plant would operate for 24 hours a day (12 hour shifts) for 7 days a week, which is a common industry standard \cite{job_guide_chemical_2021}. Thus, working 50 out of 52 weeks a year results in value of 95\% for the time when the risk is present (which is equivalent to a fr.  
 
 
 For each of the ICs, 








 