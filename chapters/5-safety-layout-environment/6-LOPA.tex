\section{Layers of protection analysis (LOPA)}

A LOPA was conducted for one of the major consequences identified from the HAZOP analysis. Focusing on node 1, the chosen consequence was overpressure in R201, resulting to a fire (due to presence of flammable materials and oxidant) and explosion . There are three main initiating associated to this overpressurization of the reaction vessel that will be investigated;

\begin{itemize}
\item Failure of pressure controller, PIC201 due to set point being too high in error
\item Bed blockage within in reactor R201, causing build up of pressure
\item Blockage in FCV-202, restricting hydrogen recycle stream from reactor R201

\end{itemize}
This consequence was chosen due to the significant affects and losses that are associated with a fire and explosion caused by overpressurization, which includes the loss of production of o-toluidine, the release of harmful substance and subsequent damage to the multiple environments (air, land), equipment and possibly major plant damage and serious injury/fatality for those within the facility. 

The target mitigated event likelihood (TMEL) which Nitroma will 

 