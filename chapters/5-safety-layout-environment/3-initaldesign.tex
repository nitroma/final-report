 
\section{Initial Design Considerations}

\subsection{EHS influence on process synthesis pathway}

EHS considerations were paramount in the decision-making process concerning Nitroma’s synthesis activities and operations. Multi-criteria decision-making (MCDM) tools (including AHP and TOPSIS) were utilised, allowing EHS factors to be systematically evaluated against other important aspects concerning the process (i.e. economic potential and process complexity). For the selection of the products to be manufactured, the safety was ranked by the NFPA 704 ratings whilst for the reaction pathways adopted, the sustainability and environmental impact were assessed through the GlaxoSmithKline scoring system \cite{henderson_expanding_2011}. A more detailed insight on the EHS considerations on crucial process synthesis decisions can be found in \cref{sec:synthesis} on Synthesis. 

\subsection{Inherent safety}
\label{sec:inherentsafety}

Inherent safety was a crucial element of the initial design stage. Nitroma wishes to ensure that all risks associated with the process are as low as practically possible, and adopting an inherently safer design is key to achieving this. Thus, Nitroma has applied all of the inherent safety principles, listed below, striving to create a safe working environment whilst allowing efficient production of the three substituted aromatic compounds. 

\subsubsection{Minimisation}
The main method of minimisation was by opting for a continuous process instead of a batch operation which is a safer design in several respects. A continuous process will require a smaller inventory, thus less hazardous substances will be present. Continuous processes also minimises the level of manual handling required due to the adaptable, automatic control in place which in turn reduces human error related risks. 
Further minimisation was achieved by preventing the production of ammonium (which has a health hazard rating NFPA of 3), by choosing formic acid over ammonium formate as the reducing agent for both 4-nitrobenzaldehyde and 4-nitrobenzoic acid formation.  

\subsubsection{Substitution} 
The choice of catalyst used for the nitration reaction demonstrates the application of substitution in the design. Whilst traditionally, concentrated sulphuric acid is chosen, it was replaced by zeolite H-Mordenite. This catalyst is safer as sulphuric acid is highly reactive and corrosive, which may damage equipment and lead to loss of containment. Furthermore as mentioned in the Synthesis section of the report, employment of the zeolite will essentially eliminate the production of harmful nitrogen oxide (\ch{NO_x}) emissions from the nitration reaction (please refer to the synthesis \cref{sec:NOx} for a detailed evaluation of the minimisation of \ch{NO_x}).  

\subsubsection{Moderation} 
Hazardous process units has been isolated as much as possible to prevent sequential events should a fire or explosion occur. For the reduction of 2-nitrotoluene,  palladium-on-carbon was chosen as the catalyst. This is a combustible dust with a significant risk of dust explosion. In order to reduce the likelihood of this occurring, the catalyst was chosen to be in the form of pellets as opposed to powder. Catalyst pellets will also minimise inhalation of the substance due to dispersion in the air. 

\subsubsection{Simplification} 
To simplify the overall design of the plant, a minimal number of reactor and separator units were employed in an attempt to lower the plant complexity and easy operation. This was achievable as Nitroma is using only toluene as the main feedstock for the three products being manufactured. The reaction compatibility meant less equipment was required. 

\subsection{Material selection}

Nitroma aims to deliver the highest standard of mechanical integrity to ensure plant safety.  Following guidance from the Occupational Safety and Health Administration (OSHA), we aim to achieve this by making appropriate material choices for the entire plant, supplemented with regular testing, inspection and maintenance \cite{life_cycle_engineering_what_nodate}. 

Material selection requires balancing various influencing factor, including the resistance to corrosion and contamination, thermal stability, economics and mechanical properties (tensile strength and fatigue resistance). 

Stainless steel was chosen as the main component for the majority of the plant, due to its exceptional corrosion resistant properties. Nickle, monel and aluminium were also considered as potential candidates as they have a high degree of corrosion resistance. However, these materials are significantly more expensive than stainless steel \cite{sinnott_coulson_2005}. Although the corrosive resistance of stainless steels is slightly lower than these metals, it offers a level of resistance that is sufficient for Nitroma's activities (as justified in Table \ref{}. In particular, the stainless steel 300 series was considered due to their ability to retain strength at high temperatures and easy maintenance, as well as their high corrosion resistance \cite{national_electronic_alloys_300_2021}. Stainless steels are also favourable in terms of sustainability, with the durability allowing long-term use and its recyclable nature making it environmentally friendly \cite{osterman_stainless_2013}. Additionally, stainless steels are advantageous due to their fire resistance, with various tests conducted by the Nickel Development Institute demonstrating how the material is able to maintain its mechanical integrity, despite being exposed to fires of extreme temperatures \cite{waller_stainless_1990}. This in turn ensures that, should a fire occur on-site, stainless steel will prevent further propagation and thus reduce damage.  Furthermore, the majority of stainless steels are able to withstand temperatures up to \SI{870}{\celsius} and thus will be suitable for all process on our plant. 

%https://www.azom.com/article.aspx?ArticleID=1175#:~:text=Most%20austenitic%20steels%2C%20with%20chromium,hence%20lower%20useful%20operating%20temperatures.

\begin{table}[hp]
    \centering
    \caption{Plant-wide material selection}
    \label{tab:material}
    
    \resizebox{\linewidth}{!}{%
\begin{tabular}{@{}lp{5cm}p{2cm}p{2cm}p{12cm}@{}}
\toprule
\multicolumn{2}{@{}l}{\textbf{Plant Equiptment}}                                                               & \textbf{Material Selected}                                                             & \textbf{Lining}                                   & \textbf{Justification/Additional comments}                                                                                                                                                                                                                                                                                                                                                                                                                                                                                                                                                                                                                                                                                                                                                                                                                                                                         \\ \midrule
\multirow[t]{6}{*}{\rtext{Reactors}}                          & R101                                                   & SS304L                                                                                 & N/A                                               & Stainless steel 304L is stabilised with titanium and niobium, which are much less susceptible to intergranular attack by nitric acid as compared to other types of stainless steel.                                                                                                                                                                                                                                                                                                                                                                                                                                                                                                                                                                                                                                                                                                                                \\ \cmidrule(l){2-5}
                                                      & R201                                                   & \multirow[t]{3}{=}{SS3316}                                                                & N/A                                               & \multirow[t]{3}{=}{Stainless steel 316 offers a high resistance to corrosion (linked to the prescnce of molybdenum). SS316 is suited for reduction reactions and is thus favourable for R201, R401 and R501. R401 and R501 will be lined with glass for additonal protection against corrosion, due to prescence of benzioic acids andd formic acid. As well as protection against acid corrosion, the effect of hydrogen has little effect on the fatigue life of SS316, which is favourabkle for R201 (in which hydrogenation occurs). Glass is known for its resistance against acids, and is thus a suitable lining for the steel.}                                                                                                                                                                                                                                                                               \\ \cmidrule(l){2-2} \cmidrule(l){4-4}
                                                      & R401                                                   &                                                                                        & glass lining                                      &                                                                                                                                                                                                                                                                                                                                                                                                                                                                                                                                                                                                                                                                                                                                                                                                                                                                                                                    \\ \cmidrule(l){2-2} \cmidrule(l){4-4}
                                                      & R501                                                   &                                                                                        & glass lining                                      &                                                                                                                                                                                                                                                                                                                                                                                                                                                                                                                                                                                                                                                                                                                                                                                                                                                                                                                    \\ \\ \\ \\ \\ \\ \cmidrule(l){2-5}
                                                      & R301                                                   & \multirow[t]{2}{=}{SS309}                                                                 & \multirow[t]{2}{=}{N/A}                              & \multirow[t]{2}{=}{The 309 stainless steel alloy is most apporpriate for environemnts in which there is an oxidation reaction occuring, offering both corrosive resistance and high temperature resistance.}                                                                                                                                                                                                                                                                                                                                                                                                                                                                                                                                                                                                                                                                                                          \\ \cmidrule(l){2-2}
                                                      & R302                                                   &                                                                                        &                                                   &                                                                                                                                                                                                                                                                                                                                                                                                                                                                                                                                                                                                                                                                                                                                                                                                                                                                                                                    \\ 
                                                      \\ \midrule
\multirow[t]{5}{*}{\rtext{Separators}}                        & All separtors in toluene nitration section             & SS304L                                                                                 & N/A                                               & Many industrial plants utilise stainless steel for separator units. 304L offers the advantage of corrosion resistance, simple maintenance, and good strength at high temperatures that may be experienced in these separator units (ranging from 333-589K)                                                                                                                                                                                                                                                                                                                                                                                                                                                                                                                                                                                                                                                         \\ \cmidrule(l){2-5}
                                                      & All separtors in 2-nitrotoluene reduction section      & \multirow[t]{4}{=}{SS316}                                                                 & N/A                                               & \multirow[t]{4}{=}{Stainlesss steel is common for separators, in this case, SS316 is suitable due to its thermal stability, offering high resistaance to all chemicals invovled. Most chemicals had a excellent compatability with SS316, whilst nitrobenzoic acids and formic acid were given a good (B) rating. Due to the long life-time of the plant, it was decided that those units in which benzoic acids and formic acids will be present, will be lined with glass for additonal protection, due to the corrosive nature of these chemicals.}                                                                                                                                                                                                                                                                                                                                                                \\ \cmidrule(l){2-2} \cmidrule(l){4-4}
                                                      & All separtors in 4-nitrotoluene oxidation section      &                                                                                        & \multirow[t]{3}{=}{glass lining}                     &                                                                                                                                                                                                                                                                                                                                                                                                                                                                                                                                                                                                                                                                                                                                                                                                                                                                                                                    \\ \cmidrule(l){2-2}
                                                      & All separtors in 4-nitrobenzaldehyde reduction section &                                                                                        &                                                   &                                                                                                                                                                                                                                                                                                                                                                                                                                                                                                                                                                                                                                                                                                                                                                                                                                                                                                                    \\ \cmidrule(l){2-2}
                                                      & All separtors in 4 nitrobenzoic acid reduction section &                                                                                        &                                                   &                                                                                                                                                                                                                                                                                                                                                                                                                                                                                                                                                                                                                                                                                                                                                                                                                                                                                                                    \\ \midrule
\multirow[t]{5}{*}{\rtext{Mixers, heat exchangers and pumps}} & Toluene nitration section                              & SS304L                                                                                 & N/A                                               & Due to the presence of nitric acid within this section of the plant, it was decided to utilise stainless steel 304L, due to its resistance to the acid (as mentioned previously) along with its compatability for the other substances within this section .                                                                                                                                                                                                                                                                                                                                                                                                                                                                                                                                                                                                                                                       \\ \cmidrule(l){2-5}
                                                      & 2-nitrotoluene reduction section                       & \multirow[t]{4}{=}{SS3316}                                                                & N/A                                               & \multirow[t]{4}{=}{Stainless steel is typically used for mixers ad pumps, the the benefit of metal hardness offering a longer life. The thermal conductivity of SS316 is highly favourable for the heat exchangers. Furthmore as mentioned previously, the use of stainless steel is benefical for the long lifetime of our plant, due to the easy maintenance and cleaning of units made from this metals. As the process will be operating outdoors, the material selected for these units will need extra resistance for the open environement. Those units in which there will be nitrobenzoic acids and/or formic acids present will be lined with ethylene chloro\-tri\-fluoro\-ethylene, which is favourable for its excellent corrosive protection propeties and thermal reistance up to 150°C. This provides reassurance in the case of overheating in these units.} \\ \cmidrule(l){2-2} \cmidrule(l){4-4}
                                                      & 4-nitrotoluene oxidation section                       &                                                                                        & \multirow[t]{3}{=}{ethylene chloro\-tri\-fluoro\-ethylene} &                                                                                                                                                                                                                                                                                                                                                                                                                                                                                                                                                                                                                                                                                                                                                                                                                                                                                                                    \\ \cmidrule(l){2-2}
                                                      & 4-nitrobenzaldehyde reduction section                  &                                                                                        &                                                   &                                                                                                                                                                                                                                                                                                                                                                                                                                                                                                                                                                                                                                                                                                                                                                                                                                                                                                                    \\ \cmidrule(l){2-2}
                                                      & 4-nitrobenzoic acid reduction section                  &                                                                                        &                                                   &                                                                                                                                                                                                                                                                                                                                                                                                                                                                                                                                                                                                                                                                                                                                                                                                                                                                                                                    \\ \\ \\\midrule
\multirow[t]{10}{*}{\rtext{Storage Units}}                       & 4-aminobenzaldehyde                                    & \multirow[t]{5}{=}{Cross-linked polyethylene}                                             & \multirow[t]{5}{=}{N/A}                              & \multirow[t]{5}{=}{Polyethylene storage vessels are commonly used for the storage of chemcial substances due to the high strength and stess crack resistance, and thus less risk of leakage. These storage vessels are also favourable for their easy maintenance and installation. The compatability between these chemicals and polyethylene was very high, with the material suitable for prolonged use.}                                                                                                                                                                                                                                                                                                                                                                                                                                                                                                          \\ \cmidrule(l){2-2}
                                                      & 2-aminobenzoic acid                                    &                                                                                        &                                                   &                                                                                                                                                                                                                                                                                                                                                                                                                                                                                                                                                                                                                                                                                                                                                                                                                                                                                                                    \\ \cmidrule(l){2-2}
                                                      & Formic acid                                            &                                                                                        &                                                   &                                                                                                                                                                                                                                                                                                                                                                                                                                                                                                                                                                                                                                                                                                                                                                                                                                                                                                                    \\ \cmidrule(l){2-2}
                                                      & Nitric acid                                            &                                                                                        &                                                   &                                                                                                                                                                                                                                                                                                                                                                                                                                                                                                                                                                                                                                                                                                                                                                                                                                                                                                                    \\ \cmidrule(l){2-2}
                                                      & Methanol                                               &                                                                                        &                                                   &                                                                                                                                                                                                                                                                                                                                                                                                                                                                                                                                                                                                                                                                                                                                                                                                                                                                                                                    \\ \cmidrule(l){2-5}
                                                      & Toluene                                                & \multirow[t]{5}{=}{SS316}                                                                 & \multirow[t]{5}{=}{N/A}                              & \multirow[t]{5}{=}{As these materials were not suited for polyethylene storage (due to low compatability), it was decided to utlise stainless steel due to benefits including durability and resistance to corrosion.}                                                                                                                                                                                                                                                                                                                                                                                                                                                                                                                                                                                                                                                                                                \\ \cmidrule(l){2-2}
                                                      & o-toluidine                                            &                                                                                        &                                                   &                                                                                                                                                                                                                                                                                                                                                                                                                                                                                                                                                                                                                                                                                                                                                                                                                                                                                                                    \\ \cmidrule(l){2-2}
                                                      & Hydrogen                                               &                                                                                        &                                                   &                                                                                                                                                                                                                                                                                                                                                                                                                                                                                                                                                                                                                                                                                                                                                                                                                                                                                                                    \\ \cmidrule(l){2-2}
                                                      & \multirow[t]{2}{=}{Waste storage units}                   &                                                                                        &                                                   &                                                                                                                                                                                                                                                                                                                                                                                                                                                                                                                                                                                                                                                                                                                                                                                                                                                                                                                    \\ \midrule
\multirow[t]{3}{*}{\rtext{Waste Treatment}}                   & Neautrlisation tank                                    & SS304                                                                                  & N/A                                               & As mentioned previously, SS304L offers high resistance to nitric acid, which will be the acid neautrlised in this vessel. This SS304L also has high thermal resistances (upto 899 °C) and so will be capable of withstanding high tempmerature assoicated to neutralisation reaction                                                                                                                                                                                                                                                                                                                                                                                                                                                                                                                                                                                                                               \\ \cmidrule(l){2-5}
                                                      & adsoption column                                       & SS316 column; with activated carbon adsorbent                                      & glass lining                                      & Activated carbon is highly efficient for the removal of both organics and inorganics and thus is favourable for reducing pollutants in wastewater streams (See waste treatment section).                                                                                                                                                                                                                                                                                                                                                                                                                                                                                                                                                                                                                                                                                                                           \\ \cmidrule(l){2-5}
                                                      & anaeorbic membrane bioreactor                          & SS316 column; polyethylene terphetalate                                                 & glass lining                                      & Polyethylene terphetalate membranes have been found to have a high chemical oxygen demand (COD) removal efficiency and is commonly used in biological treatment methods for wastewater treatment.                                                                                                                                                                                                                                                                                                                                                                                                                                                                                                                                                                                                                                                                                                                  \\ \bottomrule
\end{tabular}%
}
    
\end{table}
