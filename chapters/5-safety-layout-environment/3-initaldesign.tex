 
\section{Initial Design Considerations}

\subsection{EHS influence on process synthesis pathway}

EHS considerations were paramount in the decision-making process concerning Nitroma’s synthesis activities and operations. Multi-criteria decision-making (MCDM) tools (including AHP and TOPSIS) were utilised, allowing EHS factors to be systematically evaluated against other important aspects concerning the process (i.e. economic potential and process complexity). Different metrics were used as weightings for the various decisions surrounding the process synthesis. For the selection of the products to be manufactured, the safety was ranked by the NFPA 704 ratings and for the reaction pathways adopted, the sustainability and environmental impact were assessed through the GlaxoSmithKline scoring system. A more detailed insight on the EHS considerations on crucial process synthesis decisions can be found in \cref{sec:synthesis}. 

\subsection{Inherent safety}
\label{sec:inherentsafety}

Inherent safety was a crucial element of the initial design stage. Nitroma wishes to ensure that all risks associated with the process are as low as practically possible, and adopting an inherently safer design is key to achieving this. Thus, Nitroma has applied all of the inherent safety principles, aiming to prevent occurrence of various risks or minimise their effect.
%re-word last sentence 

\subsubsection{Minimisation}
The main method of minimisation was by opting for a continuous process instead of a batch operation which is a safer design in several respects. A continuous process will require a small inventory, thus less hazardous substances will be present. Continuous processes also minimises the level of manual handling required due to the adaptable, automatic control in place which in turn reduces human error related risks. 
Further minimisation was achieved by preventing the production of ammonium (which has a health hazard rating NFPA of 3), by choosing formic acid over ammonium formate as the reducing agent for both 4-nitrobenzaldehyde and 4-nitrobenzoic acid formation. 



\subsubsection{Substitution} 
The choice of catalyst used for the nitration reaction demonstrates the application of substitution in the design. Whilst originally, concentrated sulphuric acid was chosen, this was changed to the zeolite H-Mordenite. This catalyst is safer as sulphuric acid is highly reactive and corrosive, which may damage equipment and lead to loss of containment. Furthermore as mentioned in the synthesis section of the report, employment of the zeolite will essentially eliminate the production of harmful NOx emissions from the nitration reaction. 
% reference the synthesis section for NOx 

\subsubsection{Moderation} 
Hazardous process units has been isolated as much as possible to prevent sequential events should a fire or explosion occur. For the reduction of 2-nitrotoluene,  palladium-on-carbon was chosen as the catalyst. This is a combustible dust with a significant risk of dust explosion. In order to reduce the likelihood of this occurring, the catalyst was chosen to be in the form of pellets as opposed to powder. Catalyst pellets will also minimise inhalation of the substance due to dispersion in the air. 

\subsubsection{Simplification} 
To simplify the overall design of the plant, a minimal number of reactor and separator units were employed in an attempt to lower the plant complexity and easy operation. This was achievable as Nitroma is using only toluene as the main feedstock for the three products being manufactured. The reaction compatibility meant less equipment was required. 

\subsection{Material selection}

Nitroma aims to deliver the highest standard of mechanical integrity to ensure plant safety.  Following guidance from the Occupational Safety and Health Administration (OSHA), we aim to achieve this by making appropriate material choices for the entire plant, supplemented with regular testing, inspection and maintenance. 
%ref-osha 
Material selection requires balancing various influencing factor, including the resistance to corrosion and contamination, thermal stability, economics and mechanical properties (tensile strength and fatigue resistance). 

Stainless steel was chosen as the main component for the majority of the plant, due to its corrosion resistant properties. Nickle, monel and aluminium were also considered as potential candidates as they have a high degree of corrosion resistance. However, these materials are significantly more expensive than stainless steel \cite{sinnott_coulson_2005}. Although the corrosive resistance of stainless steels is slightly lower than these metals, it offers a level of resistance that is sufficient for Nitroma's activities. In particular, the stainless steel 300 series was considered due to their ability to retain strength at high temperatures and easy maintenance, as well as their high corrosion resistance \cite{national_electronic_alloys_300_2021}. Stainless steels are also favourable in terms of sustainability, with the durability allowing long-term use and its recyclable nature making it environmentally friendly \cite{osterman_stainless_2013}. Additionally, stainless steels are advantageous due to their fire resistance, with various tests conducted by the Nickel Development Institute demonstrating how the material is able to maintain its mechanical integrity, despite being exposed to fires of extreme temperatures \cite{waller_stainless_1990}. This is in turn ensures that should a fire occur on-site, stainless steel will prevent further propagation and thus reduce damage.  Furthermore, the majority of stainless steels are able to withstand temperatures up to 870°C and thus will be suitable for all process on our plant. 

%https://www.azom.com/article.aspx?ArticleID=1175#:~:text=Most%20austenitic%20steels%2C%20with%20chromium,hence%20lower%20useful%20operating%20temperatures.

%% I'll fix this properly later (assuming this is what broke it)
%% -- Andreas
% Merci!

% \begin{landscape}
% \begin{longtable}[c]{@{}lllll@{}}
% \centering
% \caption{Plant-wide material selection}
% \label{tab:materials}
% %\begin{tabular}
% \toprule
% \multicolumn{2}{l}{\textbf{Plant Equipment}} & \textbf{Material Selected} & \textbf{Lining} & \textbf{Justification/Additional comments} \\ \midrule
% \multirow{6}{*}{Reactors} & R101 & SS304L & N/A & \begin{tabular}[c]{@{}l@{}}Stainless steel 304L is stabilised with titanium and niobium, which are much less susceptible \\ to intergranular attack by nitric acid as compared to other types of stainless steel.\end{tabular} \\
%  & R201 & \multirow{3}{*}{SS3316} & N/A & \multirow{3}{*}{\begin{tabular}[c]{@{}l@{}}Stainless steel 316 offers a high resistance to corrosion (linked to the presence of molybdenum). \\ SS316 is suited for reduction reactions and is thus favourable for R201, R401 and R501. \\ R401 and R501 will be lined with glass for additional protection against corrosion, due to the \\ presence of nitrobenzoic acids and formic acid. As well as protection against acid corrosion, \\ the effect of hydrogen has little effect on the fatigue life of SS316, which is favourable for R201 (in which \\ hydrogenation occurs). Glass is known for its resistance against acids, and is thus a suitable lining for the steel.\end{tabular}} \\
%  & R401 &  & glass lining &  \\
%  & R501 &  & glass lining &  \\
%  & R301 & \multirow{2}{*}{SS309} & \multirow{2}{*}{N/A} & \multirow{2}{*}{\begin{tabular}[c]{@{}l@{}}The 309 stainless steel alloy is most appropriate for environments in which there is an oxidation reaction \\ occurring, offering both corrosive resistance and high-temperature resistance.\end{tabular}} \\
%  & R302 &  &  &  \\
% \multirow{5}{*}{Separators} & \begin{tabular}[c]{@{}l@{}}All separators in \\ toluene \\ nitration section\end{tabular} & SS304L & N/A & \begin{tabular}[c]{@{}l@{}}Many industrial plants utilise stainless steel for separator units. 304L offers the advantage of corrosion \\ resistance, simple maintenance, and good strength at high temperatures that may be experienced in these \\ separator units (ranging from 333-589K)\end{tabular} \\
%  & \begin{tabular}[c]{@{}l@{}}All separators in \\ 2-nitrotoluene\\ reduction section\end{tabular} & \multirow{4}{*}{SS316} & N/A & \multirow{4}{*}{\begin{tabular}[c]{@{}l@{}}Stainless steel is common for separators, in this case, SS316 is suitable due to its thermal stability, offering\\  high resistance to all chemicals involved. Most chemicals had an excellent compatibility with SS316, whilst\\  nitrobenzoic acids and formic acid were given a good (B) rating. Due to the long lifetime of the plant, it was \\ decided that those units in which benzoic acids and formic acids will be present, will be lined with glass for \\ additional protection, due to the corrosive nature of these chemicals.\end{tabular}} \\
%  & \begin{tabular}[c]{@{}l@{}}All separators in \\ 4-nitrotoluene \\ oxidation section\end{tabular} &  & \multirow{3}{*}{glass lining} &  \\
%  & \begin{tabular}[c]{@{}l@{}}All separators in \\ 4-nitrobenzaldehyde \\ reduction section\end{tabular} &  &  &  \\
%  & \begin{tabular}[c]{@{}l@{}}All separators in \\ 4 nitrobenzoic acid \\ reduction section\end{tabular} &  &  &  \\
% \multirow{5}{*}{\begin{tabular}[c]{@{}l@{}}Mixers, heat \\ exchangers and \\ pumps\end{tabular}} & \begin{tabular}[c]{@{}l@{}}Toluene nitration \\ section\end{tabular} & SS304L & N/A & \begin{tabular}[c]{@{}l@{}}Due   to the presence of nitric acid within this section of the plant, it was decided to utilise stainless steel 304L, \\ due to its resistance to the acid (as mentioned previously) along with its compatibility with the other substances\\  within this section.\end{tabular} \\
%  & \begin{tabular}[c]{@{}l@{}}2-nitrotoluene reduction \\ section\end{tabular} & \multirow{4}{*}{SS3316} & N/A & \multirow{4}{*}{\begin{tabular}[c]{@{}l@{}}Stainless steel is typically used for mixers ad pumps, the benefit of metal hardness offering longer life. The \\ thermal conductivity of SS316 is highly favourable for heat exchangers. Furthermore as mentioned previously, \\ the use of stainless steel is beneficial for the long lifetime of our plant, due to the easy maintenance and cleaning\\  of units made from these metals. As the process will be operating outdoors, the material selected for these units \\ will need extra resistance for the open environment. Those units in which there will be nitrobenzoic acids and/or \\ formic acids present will be lined with ethylene chlorotrifluoroethylene, which is favourable for its excellent corrosive\\  protection properties and thermal resistance up to 150°C. This provides reassurance in the case of overheating in these units.\end{tabular}} \\
%  & \begin{tabular}[c]{@{}l@{}}4-nitrotoluene oxidation \\ section\end{tabular} &  & \multirow{3}{*}{\begin{tabular}[c]{@{}l@{}}ethylene \\ chlorotrifluoroethylene\end{tabular}} &  \\
%  & \begin{tabular}[c]{@{}l@{}}4-nitrobenzaldehyde \\ reduction section\end{tabular} &  &  &  \\
%  & \begin{tabular}[c]{@{}l@{}}4-nitrobenzoic acid \\ reduction section\end{tabular} &  &  &  \\
% \multirow{10}{*}{Storage Units} & 4-aminobenzaldehyde & \multirow{5}{*}{\begin{tabular}[c]{@{}l@{}}Cross-linked \\ polyethylene\end{tabular}} & \multirow{5}{*}{N/A} & \multirow{5}{*}{\begin{tabular}[c]{@{}l@{}}Polyethylene storage vessels are commonly used for the storage of chemical substances due to the high strength and \\ stress crack resistance, and thus less risk of   leakage. These storage vessels are also favourable for their easy maintenance   \\ and installation. The compatibility between these chemicals and polyethylene was very high, with the material suitable for \\ prolonged use.\end{tabular}} \\
%  & 2-aminobenzoic acid &  &  &  \\
%  & Formic acid &  &  &  \\
%  & Nitric acid &  &  &  \\
%  & Methanol &  &  &  \\
%  & Toluene & \multirow{5}{*}{SS316} & \multirow{5}{*}{N/A} & \multirow{5}{*}{\begin{tabular}[c]{@{}l@{}}As these materials were not suited for polyethylene storage (due to low compatibility), it was decided to utilise stainless \\ steel due to benefits including durability and resistance to corrosion.\end{tabular}} \\
%  & o-toluidine &  &  &  \\
%  & Hydrogen &  &  &  \\
%  & \multirow{2}{*}{Waste storage units} &  &  &  \\
%  &  &  &  &  \\
% \multirow{3}{*}{Waste Treatment} & Neautrlisation tank & SS304L & N/A & \begin{tabular}[c]{@{}l@{}}As mentioned previously, SS304L offers high resistance to nitric acid, which will be the acid neutralised in this vessel. \\ This SS304L also has high   thermal resistances (up to 899 °C) and so will be capable of withstanding high temperature \\ associated to neutralisation reaction\end{tabular} \\
%  & Adsoption column & \begin{tabular}[c]{@{}l@{}}SS316 column \\ with activated \\ carbon adsorbent\end{tabular} & glass lining & \begin{tabular}[c]{@{}l@{}}Activated carbon is highly efficient for the removal of both organics and inorganics and thus is favourable for reducing \\ pollutants in wastewater streams (See waste treatment section).\end{tabular} \\
%  & \begin{tabular}[c]{@{}l@{}}Anaeorbic membrane \\ bioreactor\end{tabular} & \begin{tabular}[c]{@{}l@{}}SS316 column \\ with polyethylene \\ terphetalate\\ membrane\end{tabular} & glass lining & \begin{tabular}[c]{@{}l@{}}Polyethylene terphetalate membranes have been found to have a high chemical oxygen demand (COD) removal efficiency \\ and is commonly used in biological treatment methods for wastewater treatment.\end{tabular} \\ \cmidrule(l){2-5} 
% %\end{tabular}
% \end{longtable}

% \end{landscape}


\begin{table}[H]
\begin{tabular}{@{}lcccl@{}}
\toprule
\multicolumn{2}{c}{\textbf{Plant Equipment}}                                                                    & \textbf{Material   Selected}                                                                  & \textbf{Lining}                                    & \multicolumn{1}{c}{\textbf{Justification/Additional   comments}}              \\ \midrule
 \multirow{6}{*}{Reactors}    & R101                                                     & SS304L                                                                                        & N/A                                                & Stainless   steel 304L is stabilised with titanium and niobium, which are much less   susceptible to intergranular attack by nitric acid as compared to other types   of stainless steel.\\
     & R201     &   \multirow{3}{*}{SS3316}      & N/A      &  \multirow{3}{*}{Stainless   steel 316 offers a high resistance to corrosion (linked to the presence of   molybdenum). SS316 is suited for reduction reactions and is thus favourable   for R201, R401 and R501. R401 and R501 will be lined with glass for additional   protection against corrosion, due to presence of benzoic acids and formic   acid. As well as protection against acid corrosion, the effect of hydrogen   has little effect on the fatigue life of SS316, which is favourable for R201   (in which hydrogenation occurs). Glass is known for its resistance against   acids, and is thus a suitable lining for the steel.}                                  \\
        & R401       &          & glass lining       &   \\
   & R501                                                     &          & glass lining                                       &  \\
    & R301        &        \multirow{2}{*}{SS309}          &  \multirow{2}{*}{N/A}     &      \multirow{2}{*}{The   309 stainless steel alloy is most appropriate for environments in which there   is an oxidation reaction occurring, offering both corrosive resistance and   high temperature resistance.}                    \\
          & R302  &                                                                       &                             &  \\ \midrule
\multirow{5}{*}{Separators}           & All   separators in toluene nitration section             & SS304L     & N/A                                                & Many   industrial plants utilise stainless steel for separator units. 304L offers   the advantage of corrosion resistance, simple maintenance, and good strength   at high temperatures that may be experienced in these separator units   (ranging from 333-589K)  \\
                        & All separators in 2-nitrotoluene reduction   section      &  \multirow{4}{*}{SS316}   & N/A       &  \multirow{4}{*}{Stainlesss   steel is common for separators, in this case, SS316 is suitable due to its   thermal stability, offering high resistance to all chemicals involved. Most   chemicals had a excellent compatibility with SS316, whilst nitrobenzoic acids   and formic acid were given a good (B) rating. Due to the long life-time of   the plant, it was decided that those units in which benzoic acids and formic   acids will be present, will be lined with glass for additional protection, due   to the corrosive nature of these chemicals.}      \\
                                                      & All separators in 4-nitrotoluene oxidation   section      &                                     &   \multirow{3}{*}{glass lining}         &  \\
                                                      & All separators in 4-nitrobenzaldehyde reduction   section &     &                                                    &     \\
                        & All separators in 4 nitrobenzoic acid reduction   section &                                                                       &                   &   \\ \midrule
  \multirow{5}{*}{Mixers, heat exchangers and   pumps}                                                     & Toluene   nitration section                              & SS304L                                                                                        & N/A                                                & Due   to the presence of nitric acid within this section of the plant, it was   decided to utilise stainless steel 304L, due to its resistance to the acid   (as mentioned previously) along with its compatibility for the other   substances within this section.\\
         & 2-nitrotoluene reduction section                         &   \multirow{4}{*}{SS3316}  & N/A       &   \multirow{-4}{*}{Stainless   steel is typically used for mixers ad pumps, the the benefit of metal   hardness offering a longer life. The thermal conductivity of SS316 is highly   favourable for the heat exchangers. Furthermore as mentioned previously, the   use of stainless steel is beneficial for the long lifetime of our plant, due   to the easy maintenance and cleaning of units made from this metals. As the   process will be operating outdoors, the material selected for these units   will need extra resistance for the open environment.  Those units in which there will be nitrobenzoic acids and/or formic acids   present will be lined with ethylene chlorotrifluoroethylene, which is   favourable for its excellent corrosive protection properties and thermal   resistance up to 150°C. This provides reassurance in the case of overheating   in these units.}    \\
                                                      & 4-nitrotoluene oxidation section     &  & \multirow{3}{*}{ethylene chlorotrifluoroethylene}         &  \\
                                                      & 4-nitrobenzaldehyde reduction section      &                               &    &       \\
 & 4-nitrobenzoic acid reduction section     &      &  & \\ \midrule
  \multirow{10}{*}{Storage Units}     & 4-aminobenzaldehyde                 &    \multirow{5}{*}{Cross-linked polyethylene}      & \multirow{5}{*}{N/A}      & \multirow{-5}{*}{Polyethylene   storage vessels are commonly used for the storage of chemical substances due   to the high strength and stress crack resistance, and thus less risk of   leakage. These storage vessels are also favourable for their easy maintenance   and installation. The compatibility between these chemicals and polyethylene   was very high, with the material suitable for prolonged use.}   \\
  2-aminobenzoic acid                 &                                                                                               &                                                    &                                              \\
                                                      & Formic acid                         &                                                                                               &                                                    &              \\
                                                      & Nitric acid                         &                                                                                               &                                                    &            \\
                                                      & Methanol                            &                                                 &                              & \\
                                                      & Toluene                                                  &    \multirow{4}{*}{SS316} & \multirow{4}{*}{N/A}                                                       &  \multirow{4}{*}{As these materials were not suited for polyethylene storage (due to low compatibility), it was decided to utilise stainless steel due to benefits including durability and resistance to corrosion.}                                         \\
                                                      & o-toluidine                                              &                                                                                               &                                                    &      \\
                                                      & Hydrogen                                                 &                                                                                               &                                                    &    \\
                                                      &                                                          &                                                                                               &                                                    &                                           \\ 
     & {Waste storage units}                    &                                                                   &                            &                                                                                                                                                         \\ \midrule
    \multirow{3}{*}{Waste Treatment}           & Neutralisation tank                 & SS304              & N/A                                                & As   mentioned previously, SS304L offers high resistance to nitric acid, which   will be the acid neutralised in this vessel. This SS304L also has high   thermal resistances (up to 899 °C) and so will be capable of withstanding high   temperature associated to neutralisation reaction                                                                \\
                                                      & Adsorption column                                         & SS316   column with activated carbon adsorbent & glass lining                                       & Activated   carbon is highly efficient for the removal of both organics and inorganics   and thus is favourable for reducing pollutants in wastewater streams (See   waste treatment section).                                                                             \\
                    & Anaeorbic membrane bioreactor                            & SS316   column polyethylene terphetalate                                                      & glass lining                                       & Polyethylene   terphetalate membranes have been found to have a high chemical oxygen demand   (COD) removal efficiency and is commonly used in biological treatment methods   for wastewater treatment.                                                                                                                                                                                                                                                                                                                                                                                                                                                                                                                                                                                                                                                                                                                                             \\ \cmidrule(l){2-5} 
\end{tabular}
\end{table}