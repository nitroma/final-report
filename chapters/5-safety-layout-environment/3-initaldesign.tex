 
\section{Initial Design Considerations}

\subsection{EHS influence on process synthesis pathway}

EHS considerations were paramount in the decision-making process concerning Nitroma’s synthesis activities and operations. Multi-criteria decision-making (MCDM) tools (including AHP and TOPSIS) were utilised, allowing EHS factors to be systematically evaluated against other important aspects concerning the process (i.e; economic potential and process complexity). Different metrics were used as weightings for the various decisions surrounding the process synthesis; for the selection of the products to be manufactured, the safety was ranked by the NFPA 704 ratings whilst for the reaction pathways adopted, the sustainability and environmental impact were assessed through the GlaxoSmithKline scoring system. A more detailed insight on the EHS considerations on crucial process synthesis decisions can be found in \cref{sec:synthesis}. 

\subsection{Inherent safety}

Inherent safety was a crucial element of the initial design stage. Nitroma wishes to ensure that all risks associated to the process are as low as practically possible, and adopting an inherently safer design is key to achieving this. Thus, Nitroma has applied all of the inherent safety principles (as outlined below): 

\paragraph{Minimisation:} The continuous nature of the process acted as the main method of minimisation. Opting for a continuous process instead of a bath operation presents a safer design in several respects. A continuous process will require a small inventory, thus less hazardous substances will be present. Continuous processes also minimises the level of manual handling required due to the adaptable, automatic control in place which in turn reduces human error related risks. 


\paragraph{Substitution:} The choice of catalyst used for the nitration reaction demonstrates the application of substitution in the design. Whilst originally, concentrated sulphuric acid was chosen, this was changed to the zeolite H-Mordenite. 
% acid is more corrosive and reactive - less damage to the material. Mention the NOx and how use of zeolite may potentially prevent it's generation 

\paragraph{Moderation:} Hazardous process units has been isolated as much as possible to prevent sequential events should a fire or explosion occur. For the reduction of 2-nitrotoluene,  palladium-on-carbon was chosen as the catalyst. As this is a combustible dust, there is a significant risk of a dust explosion occurring. Thusm 

%using pellet form of catalyst instead of powder form to reduce risk of dust explosion

\paragraph{Simplification:} To simplify the overall design of the plant, a minimal number of reactor and separator units were employed in an attempt to lower the plant complexity and easy operation. This was achievable as Nitroma is using toluene as the feedstock component for each of the three products being manufactured, the reaction compatibility meaning less equipment was required. 

\subsection{Material selection}

Nitroma aims to deliver the highest standard of mechanical integrity to secure the safety of the plant.  Following guidance from the Occupational Safety and Health Administration (OSHA), it is evident that appropriate material selection (alongside testing, inspection and maintenance of process equipment), is key to achieving this goal. The material selection process has many influencing factors, outlined in table X
%NAME TABLE 
The prioritisation of these elements is key to making a suitable decision. The material selected for each part of the plant has been outlined in table x.
%Name table

