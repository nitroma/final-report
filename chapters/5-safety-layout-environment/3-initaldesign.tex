 
\section{Initial Design Considerations}

\subsection{EHS influence on process synthesis pathway}

EHS considerations were paramount in the decision-making process concerning Nitroma’s synthesis activities and operations. Multi-criteria decision-making (MCDM) tools (including AHP and TOPSIS) were utilised, allowing EHS factors to be systematically evaluated against other important aspects concerning the process (i.e. economic potential and process complexity). Different metrics were used as weightings for the various decisions surrounding the process synthesis. For the selection of the products to be manufactured, the safety was ranked by the NFPA 704 ratings and for the reaction pathways adopted, the sustainability and environmental impact were assessed through the GlaxoSmithKline scoring system. A more detailed insight on the EHS considerations on crucial process synthesis decisions can be found in \cref{sec:synthesis}. 

\subsection{Inherent safety}
\label{sec:inherentsafety}

Inherent safety was a crucial element of the initial design stage. Nitroma wishes to ensure that all risks associated with the process are as low as practically possible, and adopting an inherently safer design is key to achieving this. Thus, Nitroma has applied all of the inherent safety principles, aiming to prevent occurrence of various risks or minimise their effect.
%re-word last sentence 

\subsubsection{Minimisation}
The main method of minimisation was by opting for a continuous process instead of a batch operation which is a safer design in several respects. A continuous process will require a small inventory, thus less hazardous substances will be present. Continuous processes also minimises the level of manual handling required due to the adaptable, automatic control in place which in turn reduces human error related risks. 
Further minimisation was achieved by preventing the production of ammonium (which has a health hazard rating NFPA of 3), by choosing formic acid over ammonium formate as the reducing agent for both 4-nitrobenzaldehyde and 4-nitrobenzoic acid formation. 



\subsubsection{Substitution} 
The choice of catalyst used for the nitration reaction demonstrates the application of substitution in the design. Whilst originally, concentrated sulphuric acid was chosen, this was changed to the zeolite H-Mordenite. This catalyst is safer as sulphuric acid is highly reactive and corrosive, which may damage equipment and lead to loss of containment. Furthermore as mentioned in the synthesis section of the report, employment of the zeolite will essentially eliminate the production of harmful NOx emissions from the nitration reaction. 
% reference the synthesis section for NOx 

\subsubsection{Moderation} 
Hazardous process units has been isolated as much as possible to prevent sequential events should a fire or explosion occur. For the reduction of 2-nitrotoluene,  palladium-on-carbon was chosen as the catalyst. This is a combustible dust with a significant risk of dust explosion. In order to reduce the likelihood of this occurring, the catalyst was chosen to be in the form of pellets as opposed to powder. Catalyst pellets will also minimise inhalation of the substance due to dispersion in the air. 

\subsubsection{Simplification} 
To simplify the overall design of the plant, a minimal number of reactor and separator units were employed in an attempt to lower the plant complexity and easy operation. This was achievable as Nitroma is using only toluene as the main feedstock for the three products being manufactured. The reaction compatibility meant less equipment was required. 

\subsection{Material selection}

Nitroma aims to deliver the highest standard of mechanical integrity to ensure plant safety.  Following guidance from the Occupational Safety and Health Administration (OSHA), we aim to achieve this by making appropriate material choices for the entire plant, supplemented with regular testing, inspection and maintenance. 
%ref-osha 
Material selection requires balancing various influencing factor, including the resistance to corrosion and contamination, thermal stability, economics and mechanical properties (tensile strength and fatigue resistance). 

Stainless steel was chosen as the main component for the majority of the plant, due to its corrosion resistant properties. Nickle, monel and aluminium were also considered as potential candidates as they have a high degree of corrosion resistance. However, these materials are significantly more expensive than stainless steel \cite{sinnott_coulson_2005}. Although the corrosive resistance of stainless steels is slightly lower than these metals, it offers a level of resistance that is sufficient for Nitroma's activities. In particular, the stainless steel 300 series was considered due to their ability to retain strength at high temperatures and easy maintenance, as well as their high corrosion resistance \cite{national_electronic_alloys_300_2021}. Stainless steels are also favourable in terms of sustainability, with the durability allowing long-term use and its recyclable nature making it environmentally friendly \cite{osterman_stainless_2013}. Additionally, stainless steels are advantageous due to their fire resistance, with various tests conducted by the Nickel Development Institute demonstrating how the material is able to maintain its mechanical integrity, despite being exposed to fires of extreme temperatures \cite{waller_stainless_1990}. This is in turn ensures that should a fire occur on-site, stainless steel will prevent further propagation and thus reduce damage.  Furthermore, the majority of stainless steels are able to withstand temperatures up to 870°C 

%temperature and pressure range 