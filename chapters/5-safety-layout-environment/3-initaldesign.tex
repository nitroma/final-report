 
\section{Inital Design Considerations}

\subsection{EHS influence on process synthesis pathway}

EHS considerations were paramount in the decision-making process concerning Nitroma’s synthesis activities and operations. Multi-criteria decision-making (MCDM) tools (including AHP and TOPSIS) were utilised, allowing EHS factors to be systematically evaluated against other important aspects concerning the process (i.e; economic potential and process complexity). Different metrics were used as weightings for the various decisions surrounding the process synthesis; for the selection of the products to be manufactured, the safety was ranked by the NFPA 704 ratings whilst for the reaction pathways adopted, the sustainability and environmental impact were assessed through the GlaxoSmithKline scoring system. A more detailed insight on the EHS considerations on crucial process synthesis decisions can be found in Section \ref{sec:synthesis}. 

\subsection{Inherent safety}

Inherent safety was a crucial element of the initial design stage. Nitroma wishes to ensure that all risks associated to the process are as low as practically possible, and adopting an inherently safer design is key to achieving this. Thus, Nitroma has applied all of the inherent safety principles (as outlined below): 

\paragraph{Minimisation:} The continuous nature of the process acted as the main method of minimisation. Opting for a continuous process instead of a bath operation presents a saf



\paragraph{Substitution:}



\paragraph{Modification:}


\paragraph{Simplification:} 


 as it requires a smaller inventory and reduces threat of thermal runaway caused by hot-spot formation. Furthermore, a continuous process will allow adaptable automatic control, eliminating the need for manual handling and thus reducing risks associated with human error.  A key substitution was the choice of nitration catalyst, with H-mordenite favoured over concentrated sulphuric acid which is highly corrosive and reactive.  

%Moderations were made in terms of plant layout, isolating the hazardous process units as much as reasonable, preventing sequential events should a fire or explosion occur. To simplify the overall design of the plant a minimal number of reaction and separation units were chosen, lowering the complexity of the plant whilst enabling specified process targets to be me 



\subsection{Material selection}