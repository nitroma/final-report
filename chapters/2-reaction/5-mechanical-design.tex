% !TeX root = ../../main.tex
\section{Mechanical design}
\subsection{Material selection}
Temperature, max allowable pressure, ability to handle xxx pressure gradient, 

Due to the strong oxidising property, nitric acid is very corrosive at 70 mass\% and has poses a significant safety threat to the plant. According to ref, an increase in corrosion potential with increase in acid concentration can be attributed to the autocatalytic reduction of nitric acid. 

Several stainless steel with have been shortlisted for selection: 
\begin{enumerate}
    \item Stainless steel Type 304
    \item Stainless steel Type 316
    \item Stainless steel Type 304L
\end{enumerate}

Stainless steel type 304 have higher levels of carbon, which are not stabilised with titanium or niobium, thus are susceptible to intergranular attack in nitric acid at heat affected zone of the welds due to precipitation of chromium carbides at the grain boundaries (). Next, stainless steel type 316 includes molybdenum additions which are generally known to improve resistance to acid corrosion, but for the case of nitric acid, molybdenum tends to promote the formation of sigma phase, which is less resistant to nitric acid attack. Thus, a final material of stainless steel type 304L (also identified as BS EN 10088-1.4307) was selected for both tube and shell.

%-chosen material to be stainless steel 304L based on paper
\subsection{Design pressure and temperature}
\subsubsection{Reactor tube}
According to section X, the reactor must operate within the temperature range of \SIrange{330}{378}{\K} and pressure gradient of \SIrange{1}{1.3}{\bar}. The design temperature is set as \SI{378}{\K} and design pressure ($p_d$) is calculated to be\SI{1.44}{\bar}, using the equation \ref{eqn:designpressure} below:
\begin{equation}
    p_d = \frac{P_o}{0.9} = \frac{1.3}{0.9} = \SI{1.44}{\bar}
    \label{eqn:designpressure}
\end{equation}
\subsubsection{Shell}
Similarly, using the equation \ref{eqn:designpressure} above, the design pressure ($p_d$) for the cooling water vessel was calculated to be 1.1 bar. The design temperature for the cooling water vessel is set as 313K, as it is the upper bound of the ambient temperature range of the cooling water stream. 

\subsection{Reactor dimension design}
\label{sec:reactordimensions}

\subsubsection{Reactor tube dimensions}
Based on the calculation, total flowrate across the entire reactor was calculated to be $9.24 \times 10^{-4} m^3s^{-1}$, thus the flowrate across one individual reactor is calculated to be  $1.32 \times 10^{-4} m^3s^{-1}$. An optimal length of 3.2m was set as mentioned in section (sensitivity analysis). Subsequently, the final diameter of the reactor tubes were set at 23mm.  
The minimum thickness of the reactor tubes ($e_{reactor}$) was calculated using the equation \ref{eqn:minthicknessreactor},
\begin{equation}
    e_{reactor} = \frac{p_dD_i}{2f-p_d} = 
    \label{eqn:minthicknessreactor}
\end{equation}
where inner reactor diameter ($D_i$) is defined as 0.2m from Section (diameter section), design pressure ($p_d$) is calculated to be X bar, nominal design stress ($f$) is defined 111$Nmm^{-2}$ for design temperature not exceeding 423K based on BS5500 standards (), yield stress ($\delta_y$) is defined as 215$Nmm^{-2}$. 

%The minimum calculated thickness of shell plate ($e_{cylinder}$) was calculated to be YY mm. An addition of 1.5mm for corrosion allowance is added to ($e_{cylinder}$), making the final cylindrical thickness XXXmm. \textcolor{red}{requires final flowrate}

The bundle of 7 tubes was arranged in such a manner with a pitch of X distance between each tubes to maximise cooling heat transfer. 2 tube sheets were fitted at each ends of the shell to fix the tubes in position. A triangular tube arrangement was selected due to the more robust tube sheet construction (). The clearance of the tube arrangement was set to be 230mm, equivalent to the tube diameter. This exceeds the rule of thumb for minimum clearance of 0.25 times of tube diameter based on Primo et al. 
%to write about the arrangement of the pitch in another subsubsection maybe?



The arrangement of 7 tubes held by the tube sheet can be visualised below.

-include tube arrangement here (to be done in solidworks)
\subsubsection{Secondary cooling tube dimensions}
As mentioned in Section \ref{sec:tripleconctube}, the secondary cooling water system was implemented in a triple concentric tube manner within the reactor tubes to prevent thermal runaway in hotspots. Each secondary cooling water tubes were designed to have a diameter of 23mm to achieve a flowrate $1.5 \times 10^{-4} m^3 s^{-1}$. The wall thickness was also set as 5mm for a standardised production. 
The arrangement of the secondary cooling tube within the reactor tubes can be visualised in the cross section below. 
- include tube cooling tube here

\subsubsection{Shell dimensions}
The diameter of the shell was defined as 1.5m with distance of between to fit the flowrate of cooling water flowrate??
The minimum thickness of the shell ($e_{shell}$) was calculated using the equation \ref{eqn:minthicknessshell},
\begin{equation}
    e_{shell} = \frac{p_dD_i}{2f-p_d} = \frac{0.11 \times 1500}{2 \times 143 - 0.11} = 0.577mm
    \label{eqn:minthicknessshell}
\end{equation}

Similarly, minimum thickness of the shell ($e_{shell}$) was calculated using equation \ref{eqn:minthicknessreactor}, with $D_i$ set as 1500mm, $p_d$ as 0.11 and $f$ value of 143$Nmm_{-2}$. The minimum thickness of the shell was calculated to be 0.577mm. After taking into account corrosion allowance of 1.5mm and other welding considerations, a final shell thickness of 5mm was chosen. 
The calculated shell volume is ...
- require dimension of 
\subsubsection{Head dimensions}
%A torispherical head was chosen as the preferred over ellipsoidal and hemispherical heads due to higher maximum stress and deformation threshold (). 
%Based on BS5500 (), the limitation of using torispherical head were checked and %\begin{equation}
    %\begin{split}
       % 0.02D \leq e \leq 0.12D \\
        %r \geq 0.06D \\
        %r \geq 2e \\
        %R \leq D
    %\end{split}
    %\label{eqn:torispherical}
%\end{equation}
The minimum thickness ($e_{head}$)of the hemispherical head was calculated using the equation \ref{eqn:hemisphericalend},
\begin{equation}
    e_{head} = \frac{pD_i}{4f-1.2p} = \frac{0.11 \times 1500}{4 \times 143 - 1.2 \times 0.11} = 0.26mm
    \label{eqn:hemisphericalend}
\end{equation}
with $p$ set as 1.1bar, $D_i$ set as 1500mm, $f$ value set as 143$Nmm^{-2}$.  The minimum thickness of the hemispherical end was calculated to be 0.26mm, but a final head thickness of 5mm was chosen to complement the thickness of the shell for ease of welding. 
-refer BS5500 page 78 of the pdf

The minimum distance ($L_{im}$) between ports were calculated using the equation \ref{eqn:mindistance} below:
\begin{equation}
    L_m = \sqrt{(2r_{im}+e_{m})e_m} = \sqrt{(2 \times 1500 + 5)5} = 122.6mm
    \label{eqn:mindistance}
\end{equation}
where $r_{im}$ denotes the radius of inner diameter and $e_m$ denotes the thickness of the shell. The minimum distance between ports on the dome was calculated to be 122.6mm apart. 

-label and explain this equation
%rewrite based on hemispherical
\subsection{Ports and flanges design}
A total of 7 ports were designed according to BS 5500:1997 and BS 1600:1991 standards. The 7 ports were namely:
%inlet and outlet ports for feed reactants, and one pressure relief valve (PRV). 
\begin{enumerate}
    \item Inlet of reactants
    \item Outlet of reactants
    \item Inlet of primary cooling water
    \item Outlet of primary cooling water
    \item Inlet of secondary cooling water
    \item Outlet of secondary cooling water
    \item Presure relief valve (PRV)
\end{enumerate}
Based on BS1600: 1991, schedule 40 pipes were chosen for all pipes. 
\subsubsection{Reactant flow port design}
The inlet and outlet of the reactant flow were placed at the top right and top left of the shell dome respectively. Both ports were estimated to have a acceptable feed liquid velocity range (U) of 0.1 - 1 $ms^{-1}$. The cross sectional area (A) and outer pipe diameter (OD) are then c the volumetric flowrate ($Q_v$) shown in the equation \ref{eqn:area} below. Then, the  was calculated using the equation \ref{eqn:portOD} below. 
\begin{align}
    A &= \frac{Q_v}{U}
    \label{eqn:area} &
    \mathrm{OD} &= \sqrt{\frac{4A}{\pi}}
    \label{eqn:portOD}
\end{align}
Based on BS 1600:1991, the diameter of the reactant inlet and outlet ports were defined with OD of 114.3mm (4" NPS) with 6mm of wall thickness. 

\subsection{Cooling water flow port design}
Similarly, the outer diameter pipe diameter of both primary and secondary cooling water flow was calculated using equation \ref{eqn:area} and \ref{eqn:portOD} previously. The 
The diameter of the primary cooling water port was defined to have an OD of 219.1mm (8" NPS) with 6.55mm of wall thickness. The diameter of the secondary cooling tube were defined as OD of 60mm (2" NPS) with 4mm of wall thickness. All port dimensions were calculated based on information from BS 1600:1991. 


\subsubsection{Pressure relief valve design}
PRV was included as a safety device to protect the vessel and relief pressure in case of a overpressure event ().An overpressure event will occur when the pressure is beyond the design pressure (also known as maximum allowable working pressure, MAWP) (). Detailed calculation of the sizing of safety relief valves for liquid can be viewed in Appendix section X

The PRV was designed to relief 20\% of the reactant flow within one minute. Due to highly flammable nature of nitration reaction, the maximum allowable accumulated pressure was designed for fire exposure at 1.21 ratio increase to the maximum allowable working pressure (MAWP), which is then calculated to be 1.74 bar. 
- include detailed calculation here from excel sheet. 
The final PRV size for schedule 40 steel was calculated to have an OD of 101.6mm (3 1/2" NPS) and a wall thickness of 5.74mm.
%- Inlet of water on top, outlet of water at the bottom. Size the water

%what is the arramgement of the piping? are we looking at all 19 stacked into one huge tube of diameter 100+cm? 


\subsubsection{Flange types and dimension}
Class 150 flanges was chosen from Table 16 of BS 1560:1989, based on the maximum permissible working pressure and temperature, which provides a suitable pressure-temperature rating for the stainless steel type 304L (material group 2.3). Weld-neck flanges was selected for all ports due to the high structural strength along with stress distribution ().
Based on JPI 75-48-73 standard, the nominal pipe size (NPS) for the flange was selected as 54" for class 150 flanges.
\subsubsection{Gasket type and dimension}
Gasket is a sealing material placed between flanges to create a leak-proof sealing (hardhat). A narrow-faced, ring joint gasket type was chosen due to the suitability of design. The octagonal soft steel gasket was chosen to complement choice of ring joint gasket. The chosen gasket has a gasket factor (m) of 5.5 and minimum design seating stress (y) of 124 $Nmm^{-1}$ which fulfils the criteria of

\subsubsection{Bolt types and dimension}
The bolting material was selected was mild and carbon steel (BS 3692 grade 8.8), which is capable of withstanding 192 $N/mm^2$ design stress for temperature not exceeding 50C. Based on JPI 7S-48-73 standard for large diameter flanges, the diameter of bolt circle was defined as 1492mm, with 56 bolts of 32mm diameter each for 54" NPS flanges. 

\subsection{Overall design}
The overall cross section design of the reactor can be seen in Figure X below. Additional mechanical design drawings from SolidWorks are included in Appendix .

\textcolor{red}{- include the almighty cross section diagram here}
