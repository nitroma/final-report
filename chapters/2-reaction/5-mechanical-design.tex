% !TeX root = ../../main.tex
\section{Mechanical design}
\subsection{Material selection}
Temperature, max allowable pressure, ability to handle xxx pressure gradient, 

Due to the strong oxidising property, nitric acid is very corrosive at 70 mass\% and has poses a significant safety threat to the plant. According to ref, an increase in corrosion potential with increase in acid concentration can be attributed to the autocatalytic reduction of nitric acid. 

Several stainless steel with have been shortlisted for selection: 
\begin{enumerate}
    \item Stainless steel Type 304
    \item Stainless steel Type 316
    \item Stainless steel Type 304L
\end{enumerate}

Stainless steel type 304 have higher levels of carbon, which are not stabilised with titanium or niobium, thus are susceptible to intergranular attack in nitric acid at heat affected zone of the welds due to precipitation of chromium carbides at the grain boundaries (). Next, stainless steel type 316 includes molybdenum additions which are generally known to improve resistance to acid corrosion, but for the case of nitric acid, molybdenum tends to promote the formation of sigma phase, which is less resistant to nitric acid attack. Thus, a final material of stainless steel type 304L (also identified as BS EN 10088-1.4307) was selected for both tube and shell.

%-chosen material to be stainless steel 304L based on paper
\subsection{Design pressure and temperature}
\subsubsection{Reactor}
According to section X, the reactor must operate within the temperature range of 330K to 378K and pressure gradient of y atm to z atm. The design temperature is set as XXK and design pressure ($d_p$) is calculated to be YYY bar, using the equation \ref{eqn:designpressure} below:
\begin{equation}
    p_d = \frac{P_o}{0.9}
    \label{eqn:designpressure}
\end{equation}
\subsubsection{Shell}
Similarly, using the equation \ref{eqn:designpressure} above, the design pressure ($d_p$) for the cooling water vessel was calculated to be 1.1 bar. The design temperature for the cooling water vessel is set as 313K, as it is the upper bound of the ambient temperature range of the cooling water stream. 

\subsection{Dimensions}
\label{sec:reactordimensions}

\subsubsection{Reactor dimensions}
\begin{equation}
    e_{cylinder} = \frac{p_dD_i}{2f-p_d} 
    \label{eqn:minthicknessreactor}
\end{equation}
where inner reactor diameter ($D_i$) is defined as 0.2m from Section (diameter section), design pressure ($p_d$) is calculated to be X bar, nominal design stress ($f$) is defined 111$Nmm^{-2}$ for design temperature not exceeding 423K based on BS5500 standards (), yield stress ($\delta_y$) is defined as 215$Nmm^{-2}$. 

The minimum calculated thickness of shell plate ($e_{cylinder}$) was calculated to be YY mm. An addition of 1.5mm for corrosion allowance is added to ($e_{cylinder}$), making the final cylindrical thickness XXXmm.

The bundle of 7 tubes was arranged in such a manner with a pitch of X distance between each tubes to maximise cooling heat transfer. 2 tube sheets were fitted at each ends of the shell to fix the tubes in position. 

-refer to excel on tube pitch

-include tube arrangement here (to be done in solidworks)
\subsubsection{Shell dimensions}
\subsubsection{Reactor dimensions}
\begin{equation}
    e_{cylinder} = \frac{p_dD_i}{2f-p_d} = \frac{0.11 Nmm^{-2}\times 1500mm}{2 \times 143Nmm^{-2} - 0.11Nmm^{-2}}
    \label{eqn:minthicknessreactor}
\end{equation}

Similarly, minimum thickness of the shell ($e_{shell}$) was calculated using equation \ref{eqn:minthicknessreactor}, with $D_i$ set as 1500mm, $p_d$ as 0.11 and $f$ value of 143$Nmm_{-2}$. The minimum thickness of the shell was calculated to be 0.5mm. After taking into account corrosion allowance of 1.5mm and other welding considerations, a final shell thickness of 5mm was chosen. 
The calculated shell volume is ...
- require dimension of 
\subsubsection{Head dimensions}
%A torispherical head was chosen as the preferred over ellipsoidal and hemispherical heads due to higher maximum stress and deformation threshold (). 
%Based on BS5500 (), the limitation of using torispherical head were checked and %\begin{equation}
    %\begin{split}
       % 0.02D \leq e \leq 0.12D \\
        %r \geq 0.06D \\
        %r \geq 2e \\
        %R \leq D
    %\end{split}
    %\label{eqn:torispherical}
%\end{equation}
The minimum thickness ($e_{head}$)of the hemispherical head was calculated using the equation \ref{eqn:hemisphericalend},
\begin{equation}
    e_{head} = \frac{pD_i}{4f-1.2p}
    \label{eqn:hemisphericalend}
\end{equation}
with $p$ set as 1.1bar, $D_i$ set as 1500mm, $f$ value set as 143$Nmm^{-2}$.  The minimum thickness of the hemispherical end was calculated to be 0.26mm, but a final head thickness of 5mm was chosen to complement the thickness of the shell for ease of welding. 
-add another equation with real values
-refer BS5500 page 78 of the pdf
%rewrite based on hemispherical
\subsection{Ports and flanges}
A total of 5 ports were design on the reactor according to BS 5500:1997 and BS 1600:1991 standards. The 5 port were inlet and outlet ports for feed reactants, and one pressure relief valve (PRV). PRV was included as a safety device to protect the vessel and relief pressure in case of a overpressure event (). 
\textcolor{red}{-include port sizing here, include feed rate}
The inlet and outlet reactant ports were calculated to be X () from Section \ref{sec:reactordimensions} and were located on the top and the bottom part of the vertical vessel respectively. 
The inlet and outlet of the cooling water flow is situated at both sides of the reactor.\textcolor{red}{tbc}

%- Inlet of water on top, outlet of water at the bottom. Size the water

%what is the arramgement of the piping? are we looking at all 19 stacked into one huge tube of diameter 100+cm? 



The minimum distance between ports were calculated using the equation xx below
\begin{equation}
    L_m = \sqrt{(2r_{im}+e_{m})e_m} = \sqrt{(2 \times 1500 + 5)5} = 122.6mm
\end{equation}
where $r_{im}$ denotes the radius of inner diameter and $e_m$ denotes the thickness of the shell. The minimum distance between ports on the dome was calculated from

-label and explain this equation

\subsubsection{Flange types and dimension}
Class 150 flanges was chosen from Table 16 of BS 1560:1989, based on the maximum permissible working pressure and temperature, which provides a suitable pressure-temperature rating. Weld-neck flanges was selected for all ports due to the high structural strength along with stress distribution ().
A nominal pipe size of X in was selected based on a schedule of Y. 
\subsubsection{Gasket type and dimension}
Gasket is a sealing material placed between flanges to create a leak-proof sealing (hardhat). A narrow-faced, ring joint gasket type was chosen due to the suitability of design. The octagonal soft steel gasket was chosen to complement choice of ring joint gasket. 
- Gasket factor (m) and minimum design seating stress, groove number

\subsubsection{Bolt types and dimension}
\subsection{Structural support}
\subsection{Overall design}

