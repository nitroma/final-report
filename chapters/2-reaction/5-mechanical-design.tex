% !TeX root = ../../main.tex
\section{Mechanical design}
\subsection{Reactor material}
Temperature, max allowable pressure, ability to handle xxx pressure gradient, 

Due to the strong oxidising property, nitric acid is very corrosive at 70 mass\% and has poses a significant safety threat to the plant. According to ref, an increase in corrosion potential with increase in acid concentration can be attributed to the autocatalytic reduction of nitric acid. 

Several stainless steel with have been shortlisted for selection: 
\begin{enumerate}
    \item Stainless steel Type 304
    \item Stainless steel Type 316
    \item Stainless steel Type 304L
\end{enumerate}

Stainless steel type 304 have higher levels of carbon, which are not stabilised with titanium or niobium, thus are susceptible to intergranular attack in nitric acid at heat affected zone of the welds due to precipitation of chromium carbides at the grain boundaries (). Next, stainless steel type 316 includes molybdenum additions which are generally known to improve resistance to acid corrosion, but for the case of nitric acid, molybdenum tends to promote the formation of sigma phase, which is less resistant to nitric acid attack. Thus, a final material of stainless steel type 304L (also identified as BS EN 10088-1.4307) was selected for both tube and shell.

%-chosen material to be stainless steel 304L based on paper
\subsection{Design pressure and temperature}
According to section X, the reactor must operate within the temperature range of 330K to 378K and pressure gradient of y atm to z atm. The design temperature is set as XXK and design pressure ($d_p$) is calculated to be YYY bar, using the equation \ref{eqn:designpressure} below:
\begin{equation}
    p_d = \frac{P_o}{0.9}
    \label{eqn:designpressure}
\end{equation}

\subsection{Reactor dimensions}
\label{sec:reactordimensions}

\begin{equation}
    e_{cylinder} = \frac{p_dD_i}{2f-p_d}
    \label{eqn:minthicknessreactor}
\end{equation}
where reactor diameter ($D_i$) is defined as 0.2m from Section (diameter section), design pressure ($p_d$) is calculated to be X bar, nominal design stress ($f$) is defined 111$Nmm^{-2}$ for design temperature not exceeding 423K based on BS5500 standards (), yield stress ($\delta_y$) is defined as 215$Nmm^{-2}$. 

The minimum calculated thickness of shell plate ($e_{cylinder}$) was calculated to be YY mm. An addition of 1.5mm for corrosion allowance is added to ($e_{cylinder}$), making the final cylindrical thickness XXXmm.
\subsubsection{Shell dimensions}
- do the same thing as reactor dimension here, but based on pressure and temperature of cooling water from the reactor model. Mi
The calculated shell volume is ...
\subsubsection{Head dimensions}
A torispherical head was chosen as the preferred over ellipsoidal and hemispherical heads due to higher maximum stress and deformation threshold (). 
Based on BS5500 (), the limitation of using torispherical head were checked and met in equations \ref{eqn:torispherical} and ref tori calculation:
\begin{equation}
    \begin{split}
        0.02D \leq e \leq 0.12D \\
        r \geq 0.06D \\
        r \geq 2e \\
        R \leq D
    \end{split}
    \label{eqn:torispherical}
\end{equation}
-add another equation with real values
-refer BS5500 page 78 of the pdf
\subsection{Ports and flanges}
A total of 5 ports were design on the reactor according to BS 5500:1997 and BS 1600:1991 standards. The 5 port were inlet and outlet ports for feed reactants, and one pressure relief valve (PRV). PRV was included as a safety device to protect the vessel and relief pressure in case of a overpressure event (). 
-include port sizing here, include feed rate
The inlet and outlet reactant ports were calculated to be X () from Section \ref{sec:reactordimensions}


The minimum distance between ports were calculated using the equation xx below
\begin{equation}
    L_m = \sqrt{(2r_{im}+e_{m})e_m}
\end{equation}
-label and explain this equation

\subsubsection{Flange types and dimension}
Class 150 flanges was chosen from Table 16 of BS 1560:1989, based on the maximum permissible working pressure and temperature, which provides a suitable pressure-temperature rating. Weld-neck flanges was selected for all ports due to the high structural strength along with stress distribution ().
A nominal pipe size of X in was selected based on a schedule of Y. 
\subsubsection{Gasket type and dimension}
Gasket is a sealing material placed between flanges to create a leak-proof sealing (hardhat). A narrow-faced, ring joint gasket type was chosen due to the suitability of design. The octagonal soft steel gasket was chosen to complement choice of ring joint gasket. 
- Gasket factor (m) and minimum design seating stress, groove number

\subsubsection{Bolt types and dimension}
\subsection{Structural support}
\subsection{Overall design}

