% !TeX root = ../../main.tex
\section{Mechanical design}
\subsection{Reactor material}
Temperature, max allowable pressure, ability to handle xxx pressure gradient, 

Due to the strong oxidising property, nitric acid is very corrosive at 70 mass\% and has poses a significant safety threat to the plant. According to ref, an increase in corrosion potential with increase in acid concentration can be attributed to the autocatalytic reduction of nitric acid. 

Several stainless steel with have been shortlisted for selection: 
\begin{enumerate}
    \item Stainless steel Type 304
    \item Stainless steel Type 316
    \item Stainless steel Type 304L
\end{enumerate}

Stainless steel type 304 have higher levels of carbon, which are not stabilised with titanium or niobium, thus are susceptible to intergranular attack in nitric acid at heat affected zone of the welds due to precipitation of chromium carbides at the grain boundaries (). Next, stainless steel type 316 includes molybdenum additions which are generally known to improve resistance to acid corrosion, but for the case of nitric acid, molybdenum tends to promote the formation of sigma phase, which is less resistant to nitric acid attack. Thus, a final material of stainless steel type 304L (also identified as BS EN 10088-1.4307) was selected for both tube and shell.

%-chosen material to be stainless steel 304L based on paper
\subsection{Design pressure and temperature}
According to section X, the reactor must operate within the temperature range of 330K to 378K and pressure gradient of y atm to z atm. The design temperature was set as 

\subsection{Reactor dimensions}
\subsubsection{Shell dimensions}
\subsubsection{End dimensions}
\subsection{Ports and flanges}
\subsection{Skirt support}
\subsection{Overall design}

