% !TeX root = ../../main.tex
\section{Detailed reactor modelling}
\subsection{Heat of reaction}
To understand the thermal effects of the reaction, the enthalpy of reaction was initially calculated using the enthalpy of formation for each compound participating in the reaction. Data on the standard heat of formation for the compounds were sourced from the NIST website and literature papers. A full table of the source of each enthalpy can be found in [link to appendix/ supporting info].  Since the reaction was carried out at 330K,


The enthalpy of reaction was calculated using Hess Law, using the equation below:
\begin{equation}
  \Delta H_{r} = \Delta H_{f,nitrotoluene} + \Delta H_{f,water} - \Delta H_{f,toluene} - \Delta H_{f,nitric acid}
\end{equation}

 
\subsection{Effectiveness factor}

\subsection{Axial dispersion}
The axial dispersion was calculated as it is an important parameter in affecting the performance and conversion of the packed-bed reactor. 

The axial dispersion coefficient ($D_z$)can be calculated using the Taylor expression: 
\begin{equation}
    D_z=D_{ab}+\frac{(uD)^2}{192D_{ab}}
    \label{axial dispersion coefficient}
\end{equation}
The values of $D_{ab}$ was calculated using the Wilke-Chang correlation.
\begin{equation}
    D_{ab}=\frac{7.4\cdot 10^{-8}(\phi M_2)^{0.5}T}{\mu V_1^{0.6}}
    \label{wilkechang}
\end{equation}
The axial dispersion coefficient ($D_z$) was calculated to for various diameters and plotted based on the equations \ref{axial dispersion coefficient} and \ref{wilkechang}.
\textcolor{red}{Insert Dz vs diameter graph here} 
From the graph, a diameter of 20cm has been selected as the optimal diameter based on graph X as minimal axial dispersion is preferred. 
\subsection{Adiabatic temperature}