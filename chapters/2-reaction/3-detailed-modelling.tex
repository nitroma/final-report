% !TeX root = ../../main.tex
\section{Detailed reactor modelling: Nitration of toluene}
The nitration of toluene in the packed-bed reactor R101 has been chosen to be designed in detail for a few reasons. Firstly, the nitration process is highly exothermic with risk of thermal runaway, thus an inherently safe design of this reactor is imperative for a continuous nitration process. Moreover, transitioning from the traditionally batch nitration process into a continuous process offers the opportunity to innovate and investigate the yield between heat transfer effects and conversion. Next, the multiphase reaction of liquid-liquid-solid with the usage of solid H-Modernite zeolite as catalyst in this reactor provides a new avenue of modelling with the inclusion of solid zeolite properties such as diffusional limitation. 

\subsection{Preliminary reactor design}
%The few reactor choices that were considered, packed-bed microreactor, coated wall reactor, static mixer, and etc.
%This was mentioned in synthesis team by marie
Several reactors for this reaction was considered for this nitration process including microreactor, coated wall reactor and static mixer. The choice of reactors was discussed in detail in section \textcolor{red}{Reactor choices from Synthesis}.

\subsection{Heat of reaction}
To understand the thermal effects of the reaction, the enthalpy of reaction was initially calculated using the enthalpy of formation for each compound participating in the reaction. Data on the standard heat of formation for the compounds were sourced from the NIST website and literature papers. A full table of the source of each enthalpy can be found in [link to appendix/ supporting info]. 

The enthalpy of reaction was calculated using Hess Law, as shown in the equation below:
\begin{equation}
  \Delta H_{r}^{T} = \Delta H_{f,nitrotoluene}^{T} + \Delta H_{f,water}^{T} - \Delta H_{f,toluene}^{T} - \Delta H_{f,nitric acid}^{T}
\end{equation}

Since the reaction was carried out at 330K, the enthalpy of reaction was adjusted to the operating temperature by using the heat capacities of the compounds using the equation below. The enthalpy of reaction was not further adjusted to capture the effects of pressure because pressure has a negligible effect on the overall enthalpy. 

\begin{equation}
  \Delta H_{r}^{330K} = \Delta H_{r}^{Standard \ state} + C_p \Delta T
\end{equation}

P-nitrotoluene is an exception, because it is a solid at standard state. Therefore, the latent heat of fusion needs to be included when adjusting the enthalpy of reaction to the operating temperature. 

The final enthalpy of reaction at 330K is calculated to be -116 kJ/mol. At the specified flowrates of the reactants, the heat duty of the reactor is -63.4 kW.

\begin{table}[H]
\centering
\caption{Enthalpy of data for reactants and products}
\label{tab:Heat enthalpy table}
\begin{tabularx}{\linewidth}{l|XXX}
\toprule
                                                                & Standard enthalpy of formation (298K, 1 atm) [kJ/mol] & Heat Capacity [J/mol K] & Enthalpy of formation (330K, 1 atm) [kJ/mol] \\ \midrule
Toluene                        & 12.0              & 157.1              & 17.0                     \\
Nitric acid                      & -207.4              & 109.9              & -203.9                       \\
O-nitrotoluene & -21.7             & 202.5              & -15.3              \\ 
M-Nitrotoluene                      & -30.4              & 202.1             & -24.0                       \\
P-Nitrotoluene                      & -46.3              & 172.3 (solid), 214.0 (liquid)             & -23.9                        \\
Water                     & -285.8              & 75.3              & -283.4                        \\
\bottomrule
\end{tabularx}
\end{table}
\subsection{Catalyst pellet size}
H-Mordenite catalyst is used in the nitration reaction. 
\includecomment{From the rate equation, the concentration of nitric acid is in excess so it is assumed to be constant throughout the reaction.}
It can be assumed that all of the nitration reaction takes place in the solid-fluid interface. Hence, it is important to understand the effect of external and internal diffusional limitation of the reaction depending on the size of the catalyst pellet. This section outlines the steps used to determine the suitable pellet size to be used in the reactor.

The effective diffusional coefficient $D_e$ is first calculated to be 4.42 \times 10\textsuperscript{-10} using the widely-used equation from [ref]: 
\begin{equation}
    D_e = \frac{\epsilon_p D_p}{\tau_p}
\end{equation}
where the intraparticle void fraction for H-mordenite is 0.463 from (saleman et al), the pore diffusion coefficient is 1.56 \times 10\textsuperscript{-9} and the tortuosity of the pellet was found to be 1.63 using the equation given in [ref]. 

The effective diffusional coefficient is used to calculate Thiele modulus $\phi$ through the equation:
\begin{equation}
    \phi = r_p \sqrt{\frac{k_v}{D_e}}
\end{equation}
which is subsequently used to determine the effectiveness factor $\eta$. For the ease of modelling, the pellets are assumed to be perfectly spherical, and therefore the effectiveness factor is calculated as: 
\begin{equation}
    \eta = \frac{3}{\phi_{sphere}} \bigg(\frac{1}{tanh \phi_{sphere}} - \frac{1}{\phi_{sphere}}\bigg)
\end{equation}
An effectiveness factor of $\approx$ 1 means that external and internal mass transfer resistances of the catalyst pellet are negligible. Ideally, the effectiveness factor should be kept at $\approx$ 1 to ensure the actual rate of reaction is not lowered by mass transfer resistances. This can be done by decreasing the size of the pellets, however, that would come at the expense of large pressure drop across the reactor. Therefore, careful consideration needs to be made to ensure the pellet size needs to be large enough to prevent a large pressure drop in the reactor while being small enough to maintain a negligible diffusional resistance within the pellet. As seen in Fig. [ref], the effectiveness factor holds constant at $\approx$ 1 between 10\textsuperscript{-6} to 10\textsuperscript{-3} m, before dropping sharply at larger pellet radius size. Thus, the range of possible pellet radius is narrowed to between 10\textsuperscript{-4} to 10\textsuperscript{-3} m. 

To select the suitable pellet size, the Weisz-Prater criterion is used. This criterion evaluates the effect of diffusional limitation within the pellet [ref] :
\begin{equation}
    \Phi = \eta \phi^2
\end{equation}
It can be assumed that there is negligible diffusional limitations within the pellet if the Weisz-Prater number is $\ll$ 1. From Fig. [Weisz graph], the Weisz-Prater number falls steadily as the pellet size decreases. The final radius of the pellet is chosen to be 2 \times 10\textsuperscript{-4} m, giving an effectiveness factor of 0.99, and Weisz-Prater number of 0.05, which suggests that there is no diffusional limitation within the pellet.


\subsection{Axial dispersion}
The axial dispersion was calculated as it is an important parameter in affecting the performance and conversion of the packed-bed reactor. 

The axial dispersion coefficient ($D_z$) can be calculated using the Taylor expression: 
\begin{equation}
    D_z=D_{ab}+\frac{(uD)^2}{192D_{ab}}
    \label{axial dispersion coefficient}
\end{equation}
The values of $D_{ab}$ was calculated using the Wilke-Chang correlation.
\begin{equation}
    D_{ab}=\frac{7.4\cdot 10^{-8}(\phi M_2)^{0.5}T}{\mu V_1^{0.6}}
    \label{wilkechang}
\end{equation}
where $D_z$ is the axial dispersion model, $D_{ab}$ is the diffusion coefficient $u$ is the superficial velocity, $\phi$ is the association parameter which is estimated to be 1 for toluene, $M_2$ is the molecular weight of the solvent which is calculated based on the volumetric ratio of nitric acid and water, $V_1$ is the molar volume of toluene at boiling point.
The axial dispersion coefficient ($D_z$) was calculated to for various diameters and plotted based on the equations \ref{axial dispersion coefficient} and \ref{wilkechang}.
\textcolor{red}{Insert Dz vs diameter graph here} 
From the graph, it can be observed that the dispersion coefficient decreases sharply after after ~10cm. Thus, a diameter of 20cm has been selected as any further increase in the diameter does not warrant the drawback of the increasing pressure drop ($\Delta P$).
\subsection{Adiabatic temperature}
\subsection{Pressure drop}
The pressure drop of is a significant consideration in during the selection and modelling of this reactor as the packed-bed reactor shows the 

A significant consideration during the selection and modelling of this packed-bed reactor is the pressure drop, as a huge pressure drop will pose as a safety threat as well as incurring huge amount of CAPEX due to the usage of high performance pumps. An acceptable pressure drop range for this nitration reactor is xxx - yyy bar based on ZZZZ. 

The pressure drop in this was calculated by using the Ergun equation: 
\begin{equation}
    \frac{\Delta p}{L} = \frac{150 \mu (1- \varepsilon)^2 u_0}{\varepsilon^3 d_p^2} + \frac{1.75(1-\varepsilon)\rho u_0^2}{\varepsilon^3 d_p}
    \label{eqn:ergun}
\end{equation}
where bed voidage of the packed bed ($\varepsilon$) is 0.39 from (), the diameter of H-mordenite particle ($d_p$) is 0.0004m based on (). 

%to include the choice of diameter and length and aspect ratio here? decision making based on the length and diameter, which eventually will affect the pressure drop of the reactr
In lieu with the pressure drop criteria, an iterative approach was taken to define the optimal length, diameter and number of tubes in the reactor. 
\begin{equation}
    L= \frac{V}{\frac{\pi d_t^2}{4}\cdot n_t}
    Aspect ratio = \frac{L}{d_t}
    \label{eqn:pressuredrop}
\end{equation}
where the diameter of the tube ($d_t$) was determined to be 0.2m from Section (axial dispersion), volume of reactor required ($V$) was modelled to be 0.924 $m^3$ based on Aspen. Moreover, 

\textcolor{red}{include Andreas fancy matlab plot here}


From plot Y, it can be observed that a final number of tubes ($n_t$) of 7 in the multitubular packed-bed reactor will provide a required length ($L$) of 4.2m, and allowing an acceptable pressure drop ($\Delta P$)of 0.1bar.  

