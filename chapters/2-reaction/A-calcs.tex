% !TeX root = ../../main.tex
\section{Detailed calculations}
\label{app:reaction}
\subsection{Kinetics data}

\begin{table}[]
\centering
\caption{Kinetic data for nitration reactions at 330K and 30\% catalyst loading}
\label{tab:kineticsdata}
\resizebox{\textwidth}{!}{%
\begin{tabular}{lll}
\hline
 & Pre-exponential factor (s \textsuperscript{-1}) & Activation Energy (J/mol) \\ \hline
ONT & 1.739 & 24215 \\
MNT & 7.024 & 32370 \\
PNT & 4.968 & 27962
\end{tabular}%
}
\end{table}

\subsection{Effectiveness factor}

The effective diffusional coefficient $D_e$ is first calculated to be \num{4.42e-10} using the widely-used equation from [levenspiel]: 
\begin{equation}
    D_e = \frac{\epsilon_p D_p}{\tau_p}
\end{equation}
where the intraparticle void fraction for H-mordenite is 0.463 from (saleman et al), the pore diffusion coefficient is 1.56 \times 10\textsuperscript{-9} and the tortuosity of the pellet was found to be 1.63 using the equation given in [lanfrey et al]. 

The effective diffusional coefficient is used to calculate Thiele modulus $\phi$ through the equation:
\begin{equation}
    \phi = r_p \sqrt{\frac{k_v}{D_e}}
\end{equation}
which is subsequently used to determine the effectiveness factor $\eta$. The catalysts are assumed to be perfectly spherical, and therefore the effectiveness factor is calculated as: 
\begin{equation}
\eta=\frac{3}{\phi_{\text {sphere }}}\left(\frac{1}{\tanh \phi_{\text {sphere }}}-\frac{1}{\phi_{\text {sphere }}}\right)
\end{equation}

To select the suitable catalyst size, the Weisz-Prater criterion is used. This criterion evaluates the effect of diffusional limitation within the pellet [lanfrey et al] :
\begin{equation}
    \Phi = \eta \phi^2
\end{equation}

\begin{table}[H]
\centering
\caption{Enthalpy of formation for reactants and products}
\label{tab:Heat enthalpy table}
\begin{tabularx}{\linewidth}{l|XXX}
\toprule
                                                                & Standard enthalpy of formation (298K, 1 atm) [kJ/mol] & Heat Capacity [J/mol K] & Enthalpy of formation (330K, 1 atm) [kJ/mol] \\ \midrule
Toluene                        & 12.0              & 157.1              & 17.0                     \\
Nitric acid                      & -207.4              & 109.9              & -203.9                       \\
O-nitrotoluene & -21.7             & 202.5              & -15.3              \\ 
M-Nitrotoluene                      & -30.4              & 202.1             & -24.0                       \\
P-Nitrotoluene                      & -46.3              & 172.3 (solid), 214.0 (liquid)             & -23.9                        \\
Water                     & -285.8              & 75.3              & -283.4                        \\
\bottomrule
\end{tabularx}
\end{table}

\subsection{Cooling water table}
\begin{table}[]
\centering
\caption{Cooling water sources for R101}
\label{tab:cwtable}
\resizebox{\textwidth}{!}{%
\begin{tabular}{cccccc}
Stream & Source & T (K) & Mass flow available (kg/hr) & Flow used (kg/hr) & Fraction of total flow used \\ \hline
1 & River & 289 & 1000 & 530 & 3\% \\
2 & R201 & 323 & 3529 & 3529 & 22\% \\
3 & S301 & 323 & 2953 & 2953 & 19\% \\
4 & S203 Condenser & 328 & 6051 & 6051 & 38\% \\
5 & S201 Condenser & 330 & 7020 & 2777 & 18\% \\ \hline
Mixed &  & 325 &  & 15840 & 
\end{tabular}%
}
\end{table}

\subsection{Pressure Relief Valve (PRV)}
\label{app:PRV}
The sizing of the PRV port is as below,
\begin{equation}
    A_r = \frac{11.78 \times Q}{K_d K_w K_c} \cdot \sqrt{\frac{G}{p_1-p_2}}
\end{equation}
\begin{equation}
    K_v = (0.9935 + \frac{2.878}{Re^{0.5}} + \frac{342.75}{Re^{1.5}})^{-1}
\end{equation}
\begin{equation}
    Re = \frac{Q(18800 \times G)}{\mu \sqrt{A_R}}
\end{equation}
where the $A_r$ is the required effective discharge area assuming no viscosity correction, Q is the required volumetric flowrate, $K_d$ is the discharge coefficient, $K_w$ is the backpressure correction factor, $K_c$ is the bursting disc factor, $G$ is the specific gravity of fluid at flowing conditions relative to water at STP, $p_1$ is the relieving pressure and $p_2$ is the backpressure. 



