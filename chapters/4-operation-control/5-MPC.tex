\section{Opportunities for advanced control}

\subsection{Model Predictive Control}
Model Predictive control (MPC) is an advanced control technique that can cope with complex multivariate control problems. It functions by taking in current measurements of manipulated variables and disturbances from the plant and then calculates the future profile for the controlled variables and constraint variables in order to predict the optimal values for the manipulated variables. The controller performs two sets of calculations, first to determine the future profile of controlled variables ('targets'), and second the  The future profile of controlled variables are calculated from solving an optimisation problem that determines the set points for the CVs. The objective function for this optimisation is usually to maximise profit or minimise cost, and is subject to inequality constraints that can vary with time such as variations in process conditions, functional instrumentation and economic data such as pricing. 

\subsection{Real-time control}