\section{Sample calculations}
\label{app:samplecal}

\subsection{Degree of freedom analysis}

\begin{table}[h]
\label{tab:dof-restraining-no.}
\caption{Restraining numbers and CDOF modifiers for each unit based on the methodology presented in \textcite{}}
\begin{tabular}{@{}lllll@{}}
\toprule
Unit                                                                                                                                               & Unit Tag                & Restraining Number                                                                     & Notes                                                               &  \\ \midrule
Pump                                                                                                                                               & P201                    & 1                                                                                      & \begin{tabular}[t]{@{}l@{}}Mechanical \\ agitator (+1)\end{tabular} &  \\
Electrical heater                                                                                                                                  & H201                    & 1                                                                                      &                                                                     &  \\
Mixing point                                                                                                                                       & V201                    & 1                                                                                      &                                                                     &  \\
Electrical heater                                                                                                                                  & H202                    & 1                                                                                      &                                                                     &  \\
Trickle bed reactor                                                                                                                                & R201                    & \begin{tabular}[t]{@{}l@{}}1 (cooling water \\ material balance \\ C1=C2)\end{tabular} &                                                                     &  \\
Splitter                                                                                                                                           & FCV-202                 & 1                                                                                      &                                                                     &  \\
Fan                                                                                                                                                & F201                    & 1                                                                                      & \begin{tabular}[t]{@{}l@{}}Mechanical\\  agitator (+1)\end{tabular} &  \\
\begin{tabular}[t]{@{}l@{}}Distillation column \\ (with kettle reboiler, \\ partial condenser and    \\ reflux drum)\end{tabular} & S201 (H204, H203, S204) & 3                                                                                      & Redundancies (-3)                                                   &  \\
Splitter                                                                                                                                           & V210                    & 1                                                                                      &                                                                     &  \\
Total                                                                                                                                              &                         & 11                                                                                     & -1                                                                  &  \\ \bottomrule
\end{tabular}
\end{table}

\subsection{For executive alarms}

\subsection{For start-up procedures}
The total time for each step was rounded up to the nearest 5 or 15min interval where reasonable to give a conservative estimate of the actual time needed.										

\subsection{For shutdown procedures}
The total time for each step was rounded up to the nearest 5 or 15min interval where reasonable to give a conservative estimate of the actual time needed.

\subsubsection{Stopping flows}
\paragraph{Steam flow to reboiler H204 (Step2)}   
    \begin{itemize}
        \item The gas flowrate is reduced by 10\% its steady state value per minute, so the time to completely stop flow = 10 minutes.
    \end{itemize}


\paragraph{Hydrogen and liquid feed to reactor R201 (Step3)}
    \begin{equation}
        time\:to\:stop\:hydrogen\:flow\:=\frac{338.7}{10}=33.9\:min
    \end{equation}
    
    \begin{equation}
        time\:to\:stop\:liquid\:flow\:=\frac{10.9}{2}=5.5\:min
    \end{equation}
    
    \begin{itemize}
        \item Flowrates are in L/min
        \item The gas flowrate is reduced by 10L/min
        \item The liquid flowrate is reduced by 2L/min
    \end{itemize}
 
\subsubsection{Draining and depressurising}   
\paragraph{Depressurise reactor R201 to 1 atm by venting gas (Step5)}
    \begin{equation}
        time\:to\:depressurise\:=\frac{3.93 * 0.8}{0.100 * 2}=15.6\:min
    \end{equation}
    
    \begin{itemize}
        \item Flowrates are in $m^3$/min
        \item Assume reactor is 80\% filled with hydrogen, 20\% filled with liquid
        \item Assume venting flowrate is twice the steady state outlet flowrate
    \end{itemize}
    
 \paragraph{Drain reactor R201 and column S201 (Step6)(Step7)}   

    \begin{equation}
        time\:to\:drain\:R201\:=\frac{3.93 * 0.07}{\frac{\pi 0.0254^2}{4} * 1} \div 60 =9.0\:min
    \end{equation}
    
    \begin{equation}
        time\:to\:drain\:S201\:=\frac{0.31 * 0.07}{\frac{\pi 0.0254^2}{4} * 1} \div 60 =0.7\:min
    \end{equation}
    
    \begin{itemize}
        \item Assume 7\% of unit volume is filled with liquid, half the setpoints of the level high-high alarm
        \item A pipe with a one-inch diameter is used (1 inch = 2.54cm)
        \item Velocity of fluid through drain valve is controlled at 1m/s
    \end{itemize}

\subsubsection{Purging with inert gas} 
\paragraph{Purge reactor and column with nitrogen (Step10)}

    \begin{equation}
        Quantity\:of\:N2\:needed\:for\:reactor\:=\frac{V}{K} * \ln \frac{C1}{C2} = \frac{3.93}{0.25} * \ln \frac{0.21}{0.01} = 47.9 m^3
    \end{equation}
    
    \begin{equation}
        time\:to\:purge\:R201\:=\frac{MMM}{\frac{\pi 0.0254^2}{4} * 17} * 2 \div 60 = MMM\:min
    \end{equation}
    
    \begin{itemize}
        \item V = volume of unit ($m^3$)
        \item $V_{N2}$ = volume of nitrogen ($m^3$)
        \item K = correlation factor
        \item C1 = initial concentration of oxygen (M)
        \item C2 = final concentration of oxygen (M)
        \item two circulations of nitrogen are performed, so total time is multiplied by 2
        \item N2 will be flowed at a velocity of 17m/s
    \end{itemize}