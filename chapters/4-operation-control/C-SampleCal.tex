\section{Sample calculations}
\label{app:samplecal}

\subsection{Degree of freedom analysis}

\begin{table}[h]
\label{tab:dof-restraining-no.}
\caption{Restraining numbers and CDOF modifiers for each unit based on the methodology presented in \textcite{murthy_konda_simple_2006}}
\begin{tabular}{@{}lllll@{}}
\toprule
Unit                                                                                                                                               & Unit Tag                & Restraining Number                                                                     & Notes                                                               &  \\ \midrule
Pump                                                                                                                                               & P201                    & 1                                                                                      & \begin{tabular}[t]{@{}l@{}}Mechanical \\ agitator (+1)\end{tabular} &  \\
Electrical heater                                                                                                                                  & H201                    & 1                                                                                      &                                                                     &  \\
Mixing point                                                                                                                                       & V201                    & 1                                                                                      &                                                                     &  \\
Electrical heater                                                                                                                                  & H202                    & 1                                                                                      &                                                                     &  \\
Trickle bed reactor                                                                                                                                & R201                    & \begin{tabular}[t]{@{}l@{}}1 (cooling water \\ material balance \\ C1=C2)\end{tabular} &                                                                     &  \\
Splitter                                                                                                                                           & FCV-202                 & 1                                                                                      &                                                                     &  \\
Fan                                                                                                                                                & F201                    & 1                                                                                      & \begin{tabular}[t]{@{}l@{}}Mechanical\\  agitator (+1)\end{tabular} &  \\
\begin{tabular}[t]{@{}l@{}}Distillation column \\ (with kettle reboiler, \\ partial condenser and    \\ reflux drum)\end{tabular} & S201 (H204, H203, S204) & 3                                                                                      & Redundancies (-3)                                                   &  \\
Splitter                                                                                                                                           & V210                    & 1                                                                                      &                                                                     &  \\
Total                                                                                                                                              &                         & 11                                                                                     & -1                                                                  &  \\ \bottomrule
\end{tabular}
\end{table}

\subsection{For executive alarms}
\begin{itemize}
    \item The LL alarm for cooling water flowrate corresponds to a 60\% drop from the set point.
    \item The pump temperature HH alarm is the methanol boiling up as gas could cause cavitation.
    \item The reactor temperature, gas temperature and liquid temperature HH alarms are 3°C below the flash point of o-tol.
    \item The reactor pressure, gas pressure, liquid pressure, recycle pressure and distillation column HH alarms are 2 bar higher than the set point. The gas pressure LL alarm is set at 2 bar lower.
    \item The reactor level HH alarm corresponds to 15\% of the reactor height.
    \item The LL alarm for the flow to the pump is set at a 90\% drop from set point.
    \item The HH alarms for the gas flowrate and the recycle flowrate correspond to a 40\% rise from normal flow.
    \item The HH and LL alarms for the levels in the reflux drum and reboiler are set for a 40\% rise or drop below designed level.
    \item The temperature HH alarm for the distillation column corresponds to a 20\% increase compared to the set point.
    \item The HH alarms for the cooling water temperature were designed to sound when the temperature was so high that it could no longer remove, with a the set point flowrate, the same amount of heat that is removed by a 60\% lower cooling water flowrate at 16°C.
\end{itemize}

\subsection{For start-up procedures}

\subsubsection{Manual inspections and testing}
\paragraph{Preliminary inspection of plant sub-section (Step1)}
    \begin{itemize}
        \item There are 6 process units with inventory: R201, S201, S204, H204, P201 and F201.
        \item The average time to check each unit is estimated at 10 minutes.
        \item The total time to check would be 6 * 10 = 60 minutes.
    \end{itemize}

\paragraph{Remove shutdown blinds and install running blinds (Step2)}
    \begin{itemize}
        \item There are 22 pipes connected to process units with process fluids.
        \item There is one blind on each process pipe.
        \item The time to replace one blind is estimated at 2 minutes.
        \item Total time required = 22 * 2 = 44 minutes.
    \end{itemize}
    
\paragraph{Testing of electrical systems (Step3)}
 \begin{itemize}
        \item There are 15 control loops, 23 executive alarm and 10 interlocks.
        \item The average time needs to test each control loop and each executive alarm is estimated at 15 seconds and to test each interlock, it is 30 seconds.
        \item The total time to check would be (15+23) $\times$ 0.25 + 10 $\times$ 0.5 = 14.5 min.
    \end{itemize}
    
\paragraph{Tightness testing and bolt tightening (Step7)}
 \begin{itemize}
        \item There are 6 units and 22 pipes with process flow connected to each unit.
        \item The average time needs to test each unit is estimated at 4 min and to test each pipe, it is 1 min.
        \item The total time to test would be 6 $\times$ 4 + 22 $\times$ 1 = 46 min.
    \end{itemize}
    
\subsubsection{Purging with inert gas}

\paragraph{Purge reactor and column with nitrogen (Step7)}
 To determine the amount of inert gas required to purge the units, a correlation propose by \textcite{kinsley_properly_2001} was used.
 \begin{equation}
        \text{quantity\:of\:N2\:needed\:for\:reactor\:} = \frac{V}{K} * \ln \frac{C1}{C2} = \frac{3.93}{0.25} * \ln \frac{0.21}{0.01} = 47.9 m^3
    \end{equation}
    
    \begin{equation}
        \text{time\:to\:purge\:R201\:} = \frac{47.9}{\frac{\pi 0.0254^2}{4} * 55} * 2 \div 60 = 57.2\:min
    \end{equation}
    
    \begin{equation}
        \text{quantity\:of\:N2\:needed\:for\:column\:} = \frac{V}{K} * \ln \frac{C1}{C2} = \frac{0.31}{0.25} * \ln \frac{0.21}{0.01} = 3.77 m^3
    \end{equation}
    
    \begin{equation}
        \text{time\:to\:purge\:R201\:} = \frac{3.77}{\frac{\pi 0.0254^2}{4} * 55} * 2 \div 60 = 4.5\:min
    \end{equation}
    
    \begin{itemize}
        \item V = volume of unit ($m^3$)
        \item K = correlation factor = 0.25 according to \textcite{}
        \item C1 = initial concentration of oxygen (M)
        \item C2 = final concentration of oxygen (M)
        \item two circulations of nitrogen are performed, so total time is multiplied by 2
        \item N2 will be flowed at a velocity of 55m/s
        \item A pipe with diameter of one-inch is used
    \end{itemize}
    
\subsubsection{Equipment preparation}

\paragraph{Start-up fan F201 and electric heaters H201 and H203 (Step6)}
\begin{itemize}
        \item The time required to bring the fan to speed is 1 min according to the supplier \cite{twin_city_fan_companies_ltd_application_2000}.
        \item The mass of the heating element (copper) of a commercially available electrical heater to be 3.5 kg \cite{wattco_flanged_nodate}.
        \item The energy required to warm up the heating element is equal to the product of the mass of Cu, the specific heat capacity and the temperature difference = 46.6 kJ.
        \item An start-up efficiency of 60\% was assumed.
        \item The time required to prepare the electric heaters is equal to the energy required divided by the efficiency times the capacity of the heater. 
        \item The total time required for step 6 is 2min.
    \end{itemize}
    
\paragraph{Pressurisation and heating of R201 (Step11)}
\begin{itemize}
    \item The reactor volume is 3.93 m$^3$, the flowrate of hydrogen gad and liquid into the reactor are respectively 0.339 m$^3$/min and 0.011  m$^3$/min.
    \item The time to pressurise and heat up the reactor can be approximated to the residence time of the materials, calculated as 11.2 min.
\end{itemize}

\paragraph{Introduce set amount of methanol into column S201 (Step13)}
\begin{itemize}
    \item The volume of the column is 0.31 m$^3$ and it needs to be filled at 15\%, thus the column of methanol required is 0.0465 m$^3$.
    \item The velocity of methanol is taken as 0.5 m/s and the diameter of the pipe as one inch.
    \item The time required to fill the column with methanol is thus 3 min.
\end{itemize}


\subsubsection{Start-up of downstream units }
\paragraph{Start-up of downstream units (Step18)}
    \begin{itemize}
        \item There are 10 downstream process units with inventory.
        \item The total time taken for the start-up of 6 units excluding preliminary preparations is 120 minutes.
        \item This time is scaled for downstream start-up according to the number of units with inventory, i.e. 10/6 * 120 = 200 minutes.
    \end{itemize}

\subsection{For shutdown procedures}

\subsubsection{Stopping flows}
\paragraph{Steam flow to reboiler H204 (Step2)}   
    \begin{itemize}
        \item The gas flowrate is reduced by 10\% its steady state value per minute, so the time to completely stop flow = 10 minutes.
    \end{itemize}
    
\paragraph{Hydrogen and liquid feed to reactor R201 (Step3)}
    \begin{equation}
        \text{time\:to\:stop\:hydrogen\:flow\:} = \frac{338.7}{10}=33.9\:min
    \end{equation}
    
    \begin{equation}
        \text{time\:to\:stop\:liquid\:flow\:} = \frac{10.9}{2}=5.5\:min
    \end{equation}
    
    \begin{itemize}
        \item Flowrates are in L/min
        \item The gas flowrate is reduced by 10L/min
        \item The liquid flowrate is reduced by 2L/min
    \end{itemize}
    
\paragraph{Cooling water flowrate (Step11)}
    \begin{itemize}
        \item The cooling water flowrate is reduced by 3\% its steady state value per minute, so the time to completely stop flow = 33.3 minutes.
    \end{itemize}
    
\subsubsection{Draining and depressurising}   
\paragraph{Depressurise reactor R201 to 1 atm by venting gas (Step5)}
    \begin{equation}
        \text{time\:to\:depressurise\:} = \frac{3.93 * 0.8}{0.100 * 2}=15.6\:min
    \end{equation}
    
    \begin{itemize}
        \item Flowrates are in $m^3$/min
        \item Assume reactor is 80\% filled with hydrogen, 20\% filled with liquid
        \item Assume venting flowrate is twice the steady state outlet flowrate
    \end{itemize}
    
\paragraph{Drain reactor R201 and column S201 (Step6)(Step7)}   

    \begin{equation}
        \text{time\:to\:drain\:R201\:} = \frac{3.93 * 0.07}{\frac{\pi 0.0254^2}{4} * 1} \div 60 =9.0\:min
    \end{equation}
    
    \begin{equation}
        \text{time\:to\:drain\:S201\:} = \frac{0.31 * 0.07}{\frac{\pi 0.0254^2}{4} * 1} \div 60 =0.7\:min
    \end{equation}
    
    \begin{itemize}
        \item Assume 7\% of unit volume is filled with liquid, half the setpoints of the level high-high alarm
        \item A pipe with a one-inch diameter is used (1 inch = 2.54cm)
        \item Velocity of fluid through drain valve is controlled at 1m/s
    \end{itemize}

\subsubsection{Purging with inert gas} 
\paragraph{Purge reactor and column with nitrogen (Step10)}

    \begin{equation}
        \text{quantity\:of\:N2\:needed\:for\:reactor\:} = \frac{V}{K} * \ln \frac{C1}{C2} = \frac{3.93}{0.25} * \ln \frac{0.21}{0.01} = 47.9 m^3
    \end{equation}
    
    \begin{equation}
        \text{time\:to\:purge\:R201\:} = \frac{47.9}{\frac{\pi 0.0254^2}{4} * 55} * 2 \div 60 = 57.2\:min
    \end{equation}
    
    \begin{equation}
        \text{quantity\:of\:N2\:needed\:for\:column\:} = \frac{V}{K} * \ln \frac{C1}{C2} = \frac{0.31}{0.25} * \ln \frac{0.21}{0.01} = 3.77 m^3
    \end{equation}
    
    \begin{equation}
        \text{time\:to\:purge\:R201\:} = \frac{3.77}{\frac{\pi 0.0254^2}{4} * 55} * 2 \div 60 = 4.5\:min
    \end{equation}
    
    \begin{itemize}
        \item V = volume of unit ($m^3$)
        \item K = correlation factor = 0.25 according to \textcite{kinsley_properly_2001}
        \item C1 = initial concentration of oxygen (M)
        \item C2 = final concentration of oxygen (M)
        \item two circulations of nitrogen are performed, so total time is multiplied by 2
        \item N2 will be flowed at a velocity of 55m/s
        \item A pipe with diameter of one-inch is used
    \end{itemize}

\subsubsection{Manual inspections} 
\paragraph{Replacing running blinds with shutdown blinds (Step14))}
    \begin{itemize}
        \item There are 22 pipes connected to process units with process fluids.
        \item There is one blind on each process pipe.
        \item The time to replace one blind is estimated at 2 minutes.
        \item Total time required = 22 * 2 = 44 minutes.
    \end{itemize}
    
\paragraph{Inspection of electrical systems and instruments (Step16)}
    \begin{itemize}
        \item There are 6 process units with electrical systems such as automated control loops and indicators.
        \item The average time to check each unit is estimated at 5 minutes since each unit is no larger than 4 $m^3$ in volume.
        \item The total time to check would be 6 * 5 = 30 minutes.
    \end{itemize}
    
\paragraph{Inspection of units, equipment and instruments (Step17)}
    \begin{itemize}
        \item There are 6 process units with inventory: R201, S201, S204, H204, P201 and F201.
        \item The average time to check each unit is estimated at 10 minutes.
        \item The total time to check would be 6 * 10 = 60 minutes.
    \end{itemize}

\subsubsection{Shutdown of downstream units}
\paragraph{Shutdown of downstream units (Step19)}
    \begin{itemize}
        \item There are 10 downstream process units with inventory.
        \item The total time taken from the draining of units to placing shutdown blinds is 120 minutes.
        \item This time is scaled for downstream shutdown according to the number of units with inventory, i.e. 10/6 * 120 = 200 minutes.
    \end{itemize}