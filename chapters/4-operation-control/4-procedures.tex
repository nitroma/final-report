\section{Start-up and shut-down procedures for agreed plant section}
The start-up and shutdown procedures for the agreed plant section were elaborated based on the recommendations proposed by \textcite{}. The guiding principles are presented, followed by the detailed start-up and shut-down procedures. Calculations for the time allocated to various steps can be found in \Cref{app:samplecal}.

\subsection{Start-up standard operating procedure}

\paragraph{Preliminary preparations}
Prior to operation, all units should the inspected to ensure all maintenance and shutdown work have been completed and that the unit is fit for operation. The inspection step also entails the removal of any foreign material in the plant section. Those typically include scaffolding and tools which must be removed from the operating area as they can cause obstructions during emergencies. It must also be verified that no foreign substance has entered the units during shutdown, to avoid possible explosions. The shutdown blinds are then removed and replaced by running blinds. A check list of blinds needs to be established to make sure no blind is overlooked as this may cause damage to pipes and equipment.



\paragraph{Auxiliary equipment and utilities start-up}

Electricity units are the first utility to be checked and tested for moisture accumulation and potential defects. Critical instruments and valves, alarms and safety systems are then checked.


%testing elec + controls+ alarms
%prep cooling water and steam
%start fan and heaters

\paragraph{Elimination of air and leak prevention}

\paragraph{Initiation of on-stream units}



\subsubsection{Start-up control sequence}

%% gantt chart

\begin{itemize}
    \item Step 4: To prepare the cooling water system, valves xx and xx have to be opened. It is assumed that equipment outside this section pump the cooling water to the plant.
    \item Step 5: The steam system is started by opening valves xx.
    \item Step 6: To start the fan F201 and the electric heaters H201 and H203, xx
    \item Step 7: To purge the units with nitrogen, valves xx
    \item Step 8: 
    \item Step 10: Valves xx are opened to begin the flow of hydrogen to the plant.
    \item Step 11: The hydrogen recycle purge is open via valve xx
    \item Step 12: The reactor is pressurized by
    \item Step 13: Methanol is introduce into the column by xx
    \item Step 14:
    \item Step 16:
    \item Step 17:
\end{itemize}


\subsection{Shutdown standard operating procedure}

\paragraph{Cooling and depressurising units}
The removal and dissipation of stored energy from process units is 
\paragraph{Pumping out of process fluids}

\paragraph{Purging residual materials and water disposal}

\paragraph{Blinding and opening}

\subsubsection{Inspection for entering}

%% gantt chart

\begin{itemize}
    \item Step 4: To prepare the cooling water system, valves xx and xx have to be opened. It is assumed that equipment outside this section pump the cooling water to the plant.
    \item Step 5: The steam system is started by opening valves xx.
    \item Step 6: To start the fan F201 and the electric heaters H201 and H203, xx
    \item Step 7: To purge the units with nitrogen, valves xx
    \item Step 8: 
    \item Step 10: Valves xx are opened to begin the flow of hydrogen to the plant.
    \item Step 11: The hydrogen recycle purge is open via valve xx
    \item Step 12: The reactor is pressurized by
    \item Step 13: Methanol is introduce into the column by xx
    \item Step 14:
    \item Step 16:
    \item Step 17:
\end{itemize}