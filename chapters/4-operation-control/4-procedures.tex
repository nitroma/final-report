\section{Start-up and shut-down procedures for agreed plant section}
The start-up and shutdown procedures for the agreed plant section were elaborated based on the recommendations proposed by \textcite{}. The guiding principles are presented, followed by the detailed start-up and shut-down procedures. Calculations for the time allocated to various steps can be found in \Cref{app:samplecal}.

\subsection{Start-up standard operating procedure}

\paragraph{Preliminary preparations}
Prior to operation, all units should the inspected to ensure all maintenance and shutdown work have been completed and that the unit is fit for operation. The inspection step also entails the removal of any foreign material in the plant section. Those typically include scaffolding and tools which must be removed from the operating area as they can cause obstructions during emergencies. It must also be verified that no foreign substance has entered the units during shutdown, to avoid possible explosions. Checks for fouling are also performed. The shutdown blinds are then removed and replaced by running blinds. A check list of blinds needs to be established to make sure no blind is overlooked as this may cause damage to pipes and equipment.



\paragraph{Auxiliary equipment and utilities start-up}

Electricity units are the first utility to be checked and tested for moisture accumulation and potential defects. Critical instruments and valves, alarms, automatic shutdown devices and other safety features are then checked.
The opening of the steam system takes place gradually to prevent thermal and mechanical shocks. The cooling water system is also brought to operation by opening the relevant valves. The electrical heaters are tested and prepared to ensure they can deliver the required heating duty. The motor of the fan is started to verify it can deliver the gas flow required. 

\paragraph{Elimination of air and leak prevention}
Since the system is composed of highly flammable chemicals, it is crucial to ensure oxidisers, such as oxygen, are absent from the plant section. To do so, air must be purged before introduction of the reagents. Nitrogen purging is favoured over steam purging and other techniques due to the presence of the catalyst packed bed. It is assumed a nitrogen feed line has been installed for all plant sections.


\paragraph{Initiation of on-stream units}



\subsubsection{Start-up control sequence}

%% gantt chart

\begin{itemize}
    \item Step 4: To prepare the cooling water system, valves xx and xx have to be opened. It is assumed that equipment outside this section pump the cooling water to the plant.
    \item Step 5: The steam system is started by opening valves xx.
    \item Step 6: To start the fan F201 and the electric heaters H201 and H203, xx
    \item Step 7: To purge the units with nitrogen, valves xx
    \item Step 8: 
    \item Step 10: Valves xx are opened to begin the flow of hydrogen to the plant.
    \item Step 11: The hydrogen recycle purge is open via valve xx
    \item Step 12: The reactor is pressurized by
    \item Step 13: Methanol is introduce into the column by xx
    \item Step 14:
    \item Step 16:
    \item Step 17:
\end{itemize}


\subsection{Shutdown standard operating procedure}

\paragraph{Cooling and depressurising units}
The removal and dissipation of stored energy (heat and pressure) from process units is performed first in order to prevent fires, explosions or other hazardous events from occurring in later stages. Sources of heating are turned off and allowed to cool to ambient temperatures, and feed streams are gradually reduced in order to prevent sensitive process equipment (centrifugal pumps and transmitters) from being damaged by sudden loss of fluid. Cooling water is kept flowing in order to dissipate thermal energy from the process units. Product lines from the distillation tower are gradually reduced in accordance with the reduction in feed rate, and the hydrogen in the reactor is purged to a gas collection system.

\paragraph{Pumping out of process fluids}
Process liquids remaining in each unit are drained, and reciprocating pumps may be installed and used when the hydrostatic pressure in the unit reduces. Inert gas such as nitrogen should be pumped into units with mostly liquid holdup in order to prevent a vacuum from developing. Vapours at atmospheric pressures are also purged from the system using inert gas to a gas collection system.

\paragraph{Purging residual materials and water disposal}
A further purging step is required to ensure good removal of residual process fluids. While steam is a common purging gas, it is undesirable in our sub-section due to the presence of a packed reactor and column. The condensation of water on the packings would affect porosity in the packed beds, and the freezing of water droplets may damage the catalyst in the reactor. Thus an inert gas like nitrogen is preferred. The nitrogen purge should flow from the first uni


\paragraph{Blinding and opening}

\subsubsection{Inspection for entering}

%% gantt chart

\begin{itemize}
    \item Step 4: To prepare the cooling water system, valves xx and xx have to be opened. It is assumed that equipment outside this section pump the cooling water to the plant.
    \item Step 5: The steam system is started by opening valves xx.
    \item Step 6: To start the fan F201 and the electric heaters H201 and H203, xx
    \item Step 7: To purge the units with nitrogen, valves xx
    \item Step 8: 
    \item Step 10: Valves xx are opened to begin the flow of hydrogen to the plant.
    \item Step 11: The hydrogen recycle purge is open via valve xx
    \item Step 12: The reactor is pressurized by
    \item Step 13: Methanol is introduce into the column by xx
    \item Step 14:
    \item Step 16:
    \item Step 17:
\end{itemize}