% !TeX root = ../../main.tex
\section{Introduction}%0.5 page

Despite the safety improvements brought by process intensification and continuous operation, any process involving nitration and reduction under pressure inherently presents safety risks due to the explosive nature of the chemicals. Additional complexities arise from the numerous recycle streams, highly exothermic reactors and the multipurpose nature of Nitroma's process. To guarantee operational safety and ensure production and quality targets are constantly met, tight operational controls are required.  

To this end, a plant-wide control approach has been taken to design a control system capable of minimising disturbance propagation throughout the plant and thus control the dangers inherently present across the process. The study then focused on the design of control loops and safety features for the o-nitrotoluene reduction reactor and subsequent distillation column. To maintain optimum operating conditions, a combination of feedback, feedforward, cascade and ration control were employed. Alarms and emergency trips were also design to ensure the plant can enter a safe-state if an adverse event was to occur. Finally start-up and shutdown procedures for the agreed plant sections are presented and more advance control strategies, involving Model Predictive Control (MPC) are discussed.


