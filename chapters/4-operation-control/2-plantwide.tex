\section{Plant-wide control}

\subsection{Plant-wide control philosophy and survey table}%0.25 page

Instead of design control systems for individual units, Nitroma aims at developing an effective plant-wide control structure for the entire chemical plant following a general heuristic procedure proposed by \textcite{}. A plant-wide control approach offers many advantages including the efficient mitigation of disturbance propagation introduced by the presence of multiple recycle loops in the process. Another benefit from this approach is that the likelihood of conflicts between control loops is reduced, resulting in very highly unlikely  oscillatory behaviour. 

Control objective and the consequences associated with poor control were identified during a plant-wide control survey conducted on all major process units. For each potential control loop, a controlled variable and a manipulated variable were identified. The type of sensor and actuator was then selected. All possible disturbances affecting the controlled variable were then listed out. Finally the survey addressed major maintenance issues for each of the process units and proposed solutions to mitigate those issues. The plant wide survey can be found in the Supporting Documentation \Cref{sec:PWS}.


\subsection{Key controls}%stephen

\subsubsection{Throughput and quality control}


\subsubsection{Inventory and recycle loops control}




\subsection{Challenges to control}%Marie

\subsubsection{Difficult measurements and dynamics} %0.75 page

\subsubsection{Process bottleneck} % 0.25 page



\subsection{Key disturbances} % 0.75 page

\subsubsection{Feed quality}
%impurities decreasing catalyst activity

\subsubsection{Flowrate}

\subsubsection{Ambient temperature}

\subsubsection{Utilities}


\subsection{Key areas of maintenance} %0.25 page