\section{Plant-wide control}

\subsection{Plant-wide control philosophy and survey table}%0.25 page

Instead of design control systems for individual units, Nitroma aims at developing an effective plant-wide control structure for the entire chemical plant following a general heuristic procedure proposed by \textcite{}. A plant-wide control approach offers many advantages including the efficient mitigation of disturbance propagation introduced by the presence of multiple recycle loops in the process. Another benefit from this approach is that the likelihood of conflicts between control loops is reduced, resulting in very highly unlikely  oscillatory behaviour. 

Control objective and the consequences associated with poor control were identified during a plant-wide control survey conducted on all major process units. For each potential control loop, a controlled variable and a manipulated variable were identified. The type of sensor and actuator was then selected. All possible disturbances affecting the controlled variable were then listed out. Finally the survey addressed major maintenance issues for each of the process units and proposed solutions to mitigate those issues. The plant wide survey can be found in the Supporting Documentation \Cref{sec:PWS}.


\subsection{Key controls}%stephen

\subsubsection{Throughput and quality control}


\subsubsection{Inventory and recycle loops control}




\subsection{Challenges to control}%Marie

\subsubsection{Difficult measurements and dynamics} %0.75 page
During the plant-wide control survey, difficult measurements were identified and solutions were proposed to overcome the control challenges. Firstly, composition analysis is essential to ensure the final products met their quality specification. However composition analysis is a rather long measurement which means that there can be a long time delay between the apparition of a problem, its identification by the composition transmitter and its resolution via the control loops in place. Where possible inferential control can be implemented to reduce the time delay. For instance, the top and bottom composition of distillation column outlet streams can be controlled with temperature []. However for inferential control to be effective, robust calibration curves between the measured variable and the controlled variable must exist. Moreover the presence of impurities can significantly impact the correlation between the two process variables, ultimately leading to off-specification products. It is recommended to perform thorough experiments to obtain adequate calibration curves to be used in inferential control. In addition to inferential control with temperature, the composition can be inferred from density and viscosity measurements []. Multiplying the number of proxi measured variable should increase the accuracy of the inferential control of composition.

Furthermore, the robust control of temperature in the reactors is one of Nitroma's major concerns since all reactions are highly exothermic and could lead to thermal runaways. More specifically, in the nitration reactor R101, a hot spot can form at the centre of the catalyst bed in the shell-and-tube heat exchanger reactor, which means it can not be directly measured with temperature transmitters placed at the walls. It is suggested to conduct experiments and modelling work to develop correlations between the reactor wall temperature and the inner temperature. Additionally, the monitoring of cooling water outlet temperature can infer of the presence of a significant increase in temperature in the reactor.

Difficult dynamics can also be encountered while conducting a plant-wide control structure. Since Nitroma's process is continuous, it is expected that less difficult control dynamics will be found as opposed to a batch process. Nonetheless, it is necessary to have dynamic modelling of plant to be able to design better control loops and transfer functions.


\subsubsection{Process bottleneck} % 0.25 page

The nitration reactor R101 is Nitroma's process bottleneck since it produces the precursors for all three products made. As the first major unit in the process, any issue encountered in its operation could cascade through the whole plant. The operation conditions of this reactor indeed determine the proportions of the three products, as well as the final quantity and purity which can be obtained. 	Safety functions were installed to closely monitor and control operational parameters (temperature, pressure, flow and composition). To avoid the propagation of quality disturbances to downstream units, additional buffer tanks are installed. In addition to regular maintenance and inspection of the equipment for fouling and early catalyst deactivation, a back-up cooling system is implemented to reduce as much as reasonably feasible the risk of thermal runaway.


\subsection{Key disturbances} % 0.75 page

\subsubsection{Feed quality}
The quality of the toluene, nitric acid, formic acid, hydrogen air and methanol used in the process can significantly affect the good operation of the system. In addition to lowering the final product quality, impurities present in the feeds can decrease catalyst activity in the reactors, thus resulting in lower production and off-specification products. Mitigating strategies include the installation of feed composition analysers and a procedure to ensure storage tanks are correctly sealed. It is also recommended to use air scrubbers to purify the air used for the oxidation of PNT.

\subsubsection{Flowrate}
Variations of flowrate is one of the most prevalent disturbance in the process. Not only can it affect the overall throughput and quality, uncontrolled changes in flowrate can pose serious safety issues. A significant drop in cooling water flowrate could indeed result in a thermal runaway if the exothermic reactions can not be managed adequately. The feedforward control loops installed in the plant aim at helping the system to proactively adjust to flowrate disturbances detected upstream. The installation of back-up auxiliary equipment, such as back-up pumps, also contributes to the plant's ability to withstand flowrate disturbances.


\subsubsection{Ambient temperature}

The ambient temperature can affect the plant operation in multiple ways. Firstly, it can impact the temperature of the cooling water used to control the temperature of the reactors where exothermic reactions take place. Ineffective control could lead to thermal runaway. Moreover, air plays the role of oxidiser in the partial and complete oxidation of PNT (reactors R301 and R401). A significant change in temperature could disturb the process, having both quality and safety consequences. Finally some streams are cooled in an air heat exchanger, such as the cooler in the Rankine cycle around the PNT crystalliser. If the ambient temperature is so high that it is unable to cool down the required streams, the purity of the final products will be affected. To limit the impact of variations of the ambient air temperature on the process, it is essential to install ambient temperature transmitters on the plant and to connect them with feedfoward control loops for major units. As several units depend on tight temperature control (exothermic reactors, crystallisers), it is recommended to plan for the switch of cooling medium, for instance using a refrigerant. This would require the installation of a back-up Rankine cycle for the cooling system Another suggestion is the construction of cooling towers to provide sufficiently cold cooling water when the normal cooling water supplier fails to deliver the requirements.

\subsubsection{Utilities}

The main disturbances in key utilities include the supply of electricity and the temperature and flowrate of cooling water and steam. The fluctuations in the later have already been discussed in previous section and can be mitigated via feedforward control.
Electricity is not only key to supplying heat to process fluids through electric heaters and to running the pumps and fans, it is also essential for the actuation of electrical control valves for flow are pressure. If power is lost, back-up battery packs provide the energy required to bring the equipment and control valves to a fail-safe state, thus preventing accidents. Restarting the plant upon electrical power restoration must be correctly managed to prevent the automatic restart of some equipment before other sections are ready for operation. To 

 Back-up power supplies and implementation of robust interlocks can reduce the risk of process upsets upon temporary losses of power, but procedures should be in place for bringing the plant to a safe state for longer outages. 


\subsection{Key areas of maintenance} %0.25 page