\section{Plant-wide control}

\subsection{Plant-wide control philosophy and survey table}%0.25 page

Instead of design control systems for individual units, Nitroma aims at developing an effective plant-wide control structure for the entire chemical plant following a general heuristic procedure proposed by \textcite{}. A plant-wide control approach offers many advantages including the efficient mitigation of disturbance propagation introduced by the presence of multiple recycle loops in the process. Another benefit from this approach is that the likelihood of conflicts between control loops is reduced, resulting in very highly unlikely  oscillatory behaviour. 

A plant-wide control survey was conducted 





%A plant wide control and operation survey was conducted on all major process units to identify potential control strategies, taking operation of the entire plant into consideration. Initially, likely disturbances across each process unit were considered, followed by potential pairings of controlled and manipulated variables (see Section 4.7). Finally, sensors and actuators for specific pairs were identified to ensure the feasibility of pairings. The survey addressed major issues faced within the pharmaceutical industry surrounding throughput maintenance and product quality requirements, alongside energy management control and identification of the process bottleneck. Potential maintenance, operational concerns and suggested solutions were also identified. The plant wide survey can be found in the Supporting Documents Section 5.





\subsection{Key controls}

\subsubsection{Throughput and quality control}


\subsubsection{Inventory and recycle loops control}




\subsection{Challenges to control}

\subsubsection{Difficult measurements and dynamics} %0.75 page

\subsubsection{Process bottleneck} % 0.25 page



\subsection{Key disturbances} % 0.75 page

\subsubsection{Feed quality}
%impurities decreasing catalyst activity

\subsubsection{Flowrate}

\subsubsection{Ambient temperature}

\subsubsection{Utilities}


\subsection{Key areas of maintenance} %0.25 page