\section{Plant-wide control}

\subsection{Plant-wide control philosophy and survey table}%0.25 page

Instead of design control systems for individual units, Nitroma aims at developing an effective plant-wide control structure for the entire chemical plant following a general heuristic procedure proposed by \textcite{}. A plant-wide control approach offers many advantages including the efficient mitigation of disturbance propagation introduced by the presence of multiple recycle loops in the process. Another benefit from this approach is that the likelihood of conflicts between control loops is reduced, resulting in very highly unlikely  oscillatory behaviour. 

Control objective and the consequences associated with poor control were identified during a plant-wide control survey conducted on all major process units. For each potential control loop, a controlled variable and a manipulated variable were identified. The type of sensor and actuator was then selected. All possible disturbances affecting the controlled variable were then listed out. Finally the survey addressed major maintenance issues for each of the process units and proposed solutions to mitigate those issues. The plant wide survey can be found in the Supporting Documentation \Cref{sec:PWS}.


\subsection{Key controls}%stephen

\subsubsection{Throughput and quality control}


\subsubsection{Inventory and recycle loops control}




\subsection{Challenges to control}%Marie

\subsubsection{Difficult measurements and dynamics} %0.75 page
During the plant-wide control survey, difficult measurements were identified and solutions were proposed to overcome the control challenges. Firstly, composition analysis is essential to ensure the final products met their quality specification. However composition analysis is a rather long measurement which means that there can be a long time delay between the apparition of a problem, its identification by the composition transmitter and its resolution via the control loops in place. Where possible inferential control can be implemented to reduce the time delay. For instance, the top and bottom composition of distillation column outlet streams can be controlled with temperature []. However for inferential control to be effective, robust calibration curves between the measured variable and the controlled variable must exist. Moreover the presence of impurities can significantly impact the correlation between the two process variables, ultimately leading to off-specification products. It is recommended to perform thorough experiments to obtain adequate calibration curves to be used in inferential control. In addition to inferential control with temperature, the composition can be inferred from density and viscosity measurements []. Multiplying the number of proxi measured variable should increase the accuracy of the inferential control of composition.

Furthermore, the robust control of temperature in the reactors is one of Nitroma's major concerns since all reactions are highly exothermic and could lead to thermal runaways. More specifically, in the nitration reactor R101, a hot spot can

\subsubsection{Process bottleneck} % 0.25 page



\subsection{Key disturbances} % 0.75 page

\subsubsection{Feed quality}
%impurities decreasing catalyst activity

\subsubsection{Flowrate}

\subsubsection{Ambient temperature}

\subsubsection{Utilities}


\subsection{Key areas of maintenance} %0.25 page