\section{Plant-wide control}

\subsection{Plant-wide control philosophy and survey table}%0.25 page


%The operation of a complex an integrated pharmaceutical plant requires a plant wide control approach as opposed to the traditional individual unit-operation focused approach. A plant wide control approach reduces the likelihood of control loops conflicting with one another, meaning interactions and the possibility of oscillatory behaviour is limited as much as possible. Another benefit of plant wide control is that the system has a good understanding of the overall process dynamics resulting in key control objectives being fulfilled. The potential of disturbance propagation and alterations in the dynamic behaviour of the system, brought about by the presence of multiple recycle streams in the process, is mitigated by plant wide control (Luyben, Tyreus, & Luyben, 1997).

%A plant wide control and operation survey was conducted on all major process units to identify potential control strategies, taking operation of the entire plant into consideration. Initially, likely disturbances across each process unit were considered, followed by potential pairings of controlled and manipulated variables (see Section 4.7). Finally, sensors and actuators for specific pairs were identified to ensure the feasibility of pairings. The survey addressed major issues faced within the pharmaceutical industry surrounding throughput maintenance and product quality requirements, alongside energy management control and identification of the process bottleneck. Potential maintenance, operational concerns and suggested solutions were also identified. The plant wide survey can be found in the Supporting Documents Section 5.



Plantwide control involves the systems and  strategies  required  to control an entire chemical plant consisting of  many interconnected unit operations. A general  heuristic design procedure is presented that generates an effective plantwide control structure for an entire complex process flowsheet and not simply individual units.

\subsection{Key controls}

\subsubsection{Throughput and quality control}


\subsubsection{Inventory and recycle loops control}




\subsection{Challenges to control}

\subsubsection{Difficult measurements and dynamics} %0.75 page

\subsubsection{Process bottleneck} % 0.25 page



\subsection{Key disturbances} % 0.75 page

\subsubsection{Feed quality}
%impurities decreasing catalyst activity

\subsubsection{Flowrate}

\subsubsection{Ambient temperature}

\subsubsection{Utilities}


\subsection{Key areas of maintenance} %0.25 page