\section{Conclusion and outlook}
A plant-wide approach was adopted in the design of Nitroma's control system so as to minimise disturbance propagation through different process units. A plant-wide survey identified important control structures to ensure safe and steady operation of Nitroma's plant. Pump P101 was made the throughput manipulator to control the throughput of feed to the primary reactor (R101), since this reactor produces the precursors to all 3 end products. Inferential sensors were suggested where difficult measurements such as composition are needed. Buffer tanks were installed between major sections of the plant to mitigate the propagation of upstream disturbances. Strategies were taken to mitigate major disturbances, such as tight control of feed qualities and back-up utility systems. Key areas of maintenance were highlighted and were incorporated into the detailed start-up and shutdown procedures where relevant.

Detailed designs of control systems were carried out around the 2-nitrotoluene reduction reactor (R201) and downstream distillation column (S201). The feed ratio required strict control due to reaction rates that were sensitive to 