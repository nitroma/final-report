\section{Conclusion and outlook}
A plant-wide approach was adopted in the design of Nitroma's control system so as to minimise disturbance propagation through different process units. A plant-wide survey identified important control structures to ensure safe and steady operation of Nitroma's plant. Pump P101 was made the throughput manipulator to control the throughput of feed to the primary reactor (R101), since this reactor produces the precursors to all 3 end products. Inferential sensors were suggested where difficult measurements such as composition are needed. Buffer tanks were installed between major sections of the plant to mitigate the propagation of upstream disturbances. Strategies were taken to mitigate major disturbances, such as tight control of feed qualities and back-up utility systems. Key areas of maintenance were highlighted and were incorporated into the detailed start-up and shutdown procedures where relevant.

Detailed designs of control systems were carried out around the 2-nitrotoluene reduction reactor (R201) and downstream distillation column (S201). Key challenges to the control strategy include a strict feed ratio to the reactor due to reaction rates that were sensitive to the feed proportions, leading to the design of a robust ratio controller. Another challenge was the composition control of the bottoms stream from the column reboiler due to composition analysers having long lag times. Inferential temperature control was used to compensate the slow composition controller. Finally, control loop interactions were predicted and multivariate control was suggested to address interactions between loops.  

Alarms were designed in the sub-section as an additional layer of protection. In addition, a thorough network of executive alarms and automated interlock actions were designed to detect abnormal and dangerous conditions in the plant then bring the plant into a safe state until the condition is rectified. A SIL-3 rated pressure regulating system was implemented after LOPA revealed overpressure in the reactor still remained a  risk.