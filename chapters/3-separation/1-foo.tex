% !TeX root = ../../main.tex
\section{Introduction}


\section{Choice of separator}
Crystalliser with hydraulic wash column 



\section{Crystalliser design}




\section{Hydraulic wash column design}

Commonly used solid-liquid separators are not continuous processes, typically batch or semi-batch. Wash column does wash and continuous transport in one process. Whereas, for example a centrifuge would require  melting the crystals and a means of solid transport by conveyor belt. Different type of wash columns include gravity, hydraulic and mechanical. Hydraulic was chosen because of shorter residence time and less moving mechanical parts. 

\subsection{Modelling}
The hydraulic wash column was modelled on gPROMS using initial dimensions from ---. Following this method, the initial dimensions were varied to model a hydraulic column for the feed flow rate from the crystalliser. 

Assumes 100 \% solid-liquid separation in the wash column therefore no presence of liquid in the product flow rate since mother liquor will not enter the wash front. However, if impurities are entrapped in the crystal lattice during crystallisation, these will not be removed. This highlights the disadvantage of hydraulic wash column compared to gravity wash column. Gravity wash column ensures impurities are removed from the crystals during the long residence time. 

Inputs and parameters table. 

\subsection{Results}
Length of bed, height, diameter, number of filter tubes etc. Observing the effect of key variables. Length of filtration and wash column - ensuring wash column stays below the filter. Pressure drop across the column

Final dimension table. 
Performance at various height- pressure drop etc. 

\subsection{Materials Choice}

\subsection{CAD}

