% !TeX root = ../../main.tex
\section{Introduction}


\section{Choice of separator}
The separation of p-nitrotoluene from m-nitrotoluene was designed in detail. P-nitrotoluene is an essential precursor to aminobenzaldehyde and aminobenzoic acid therefore it needs to be purified to at least 99.9 \%. Due to the small difference in boiling points between p-NT and m-NT distillation was infeasible, it is also an energy intensive process. Similiar solubility between the two components made the use of absorption or adsorption processes to be infeasible as well. However, crystallisers were feasible due to the difference in melting point. 

*Table of boiling points, melting points and mole fraction 

Melt crystallisers are particularly used for organic chemicals with favourable melting points. Compared to solution crystallisers, the growth rate in melt crystalliser is relatively fast and still leads to high purity products, it is economically favourable. It also avoids the possibility  of thermal degradation  due to high temperature. (1).  Typical melt crystallisers are used in batch or semi-batch operations. Nitroma’s plant aims to operate continuously therefore a novel crystalliser known as scraped surface heat exchanger (SSHE) was chosen. It operates continuously by producing a slurry of p-nitrotoluene crystals and m-nitrotoluene liquid which will be further separated in a hydraulic wash column to attain purified molten p-nitrotoluene. 

*Table of comparison between type of crystallisers

 
Jaksland analysis?

Alternatives considered..

\begin{comment}
were MSMPR crystallisers that operate continuously. However, these crystallisers are primarily used for solvent crystallisation. Melt crystallisation will be used at Nitroma to avoid the issue of handling solvents. 
\end{comment}


\section{Crystalliser design}




\section{Hydraulic wash column design}

Commonly used solid-liquid separators are not continuous processes, typically batch or semi-batch. Wash column does wash and continuous transport in one process. Whereas, for example a centrifuge would require  melting the crystals and a means of solid transport by conveyor belt. Different type of wash columns include gravity, hydraulic and mechanical. Hydraulic was chosen because of shorter residence time and less moving mechanical parts. 

\subsection{Dynamic Modelling}
The hydraulic wash column was modelled on gPROMS using initial dimensions from ---. Following this method, the initial dimensions were varied to attain the final dimensions appropriate for feed from the crystalliser. 

Assumes 100 \% solid-liquid separation in the wash column therefore no presence of liquid in the product flow rate since mother liquor will not enter the wash front. However, if impurities are entrapped in the crystal lattice during crystallisation, these will not be removed. This highlights the disadvantage of hydraulic wash column compared to gravity wash column. Gravity wash column allows some impurities to be removed from the crystals during the long residence time. 

Inputs and parameters table. 

\subsection{Results}
Length of bed, height, diameter, number of filter tubes etc. Observing the effect of key variables. Length of filtration and wash column - ensuring wash column stays below the filter. Pressure drop across the column

Final dimension table. 
Performance at various height- pressure drop etc. 

\subsection{Materials Choice}

\subsection{CAD}

