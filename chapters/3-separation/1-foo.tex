% !TeX root = ../../main.tex
\section{Introduction}


\section{Choice of separator}
The separation of p-nitrotoluene from m-nitrotoluene was designed in detail. P-nitrotoluene is an essential precursor to aminobenzaldehyde and aminobenzoic acid therefore it needs to be purified to at least 99.9 \%. Due to the small difference in boiling points between p-NT and m-NT distillation was infeasible, it is also an energy intensive process. Similar solubility between the two components made the use of absorption or adsorption processes to be infeasible as well. However, crystallisers were deemed feasible due to the difference in melting points. 

*Table of boiling points, melting points and mole fraction 

Melt crystallisers are particularly used for organic chemicals with favourable melting points. Compared to solution crystallisers, the growth rate in melt crystalliser is relatively fast and still leads to high purity products making it economically favourable. It also avoids the possibility  of thermal degradation  due to high temperature (1). From environmental perspective, solution crystallisation involves the addition of solvent that is usually treated as impurity. Whereas, melt crystallisation does not require any additional substances. Melt crystalliser also known to be relatively cheaper - need to find cost info if possible.

Types of melt crystalliser
Solid layer crystalliser:
advantages: avoid encrustation issue
disadvantages: higher investment costs, multistage process, not commonly used as continuous process

Suspension crystalliser:
advantages: moderate crystal growth and large surface areas, high purity product, continuous operation mode.
disadvantages: slurry handling,extras washing step required for solid-liquid separation, moving parts (1). 

Nitroma’s plant aims to operate continuously. Suspension crystalliser was more suitable than solid-layer crystalliser. A novel crystalliser known as scraped surface heat exchanger (SSHE) was chosen. It operates continuously by producing a slurry of p-nitrotoluene crystals and m-nitrotoluene liquid which will be further separated in a hydraulic wash column to attain purified molten p-nitrotoluene. 

Jaksland analysis?

Alternatives considered..

*Table of comparison between type of crystallisers

 




\section{Crystalliser design}




\section{Hydraulic wash column design}

Commonly used solid-liquid separators are not continuous processes, typically batch or semi-batch. Wash column does wash and continuous transport in one process. Whereas, for example a centrifuge would require  melting the crystals and a means of solid transport by conveyor belt. Different type of wash columns include gravity, hydraulic and mechanical. Hydraulic was chosen because of shorter residence time and less moving mechanical parts. 

\subsection{Steady State Model}
The hydraulic wash column was modelled based on volume and force balance using initial dimensions from ---. Following this method, the initial dimensions were varied to determine the performance of the column as key variables were varied.  

Assumes 100 \% solid-liquid separation in the wash column therefore no presence of liquid in the product flow rate since mother liquor will not enter the wash front. However, if impurities are entrapped in the crystal lattice during crystallisation, these will not be removed. This highlights the disadvantage of hydraulic wash column compared to gravity wash column. Gravity wash column allows some impurities to be removed from the crystals during the long residence time. 

Inputs and parameters table. 

\subsection{Dimensions}
Final dimension table - Length of bed, height, diameter, number of filter tubes etc. 

Effect of variation in length of filtration and wash on pressure drop. 


\subsection{Materials Choice}
- Look into filter selection
- Look into valve selection 
- material for the column itself 

\subsection{CAD}

