% !TeX root = ../../main.tex
\section{Crystallisor design}

\subsection{(p-Nitrotoluene + m-nitrotoluene) phase behaviour}

The (p-NT + m-NT) system exhibits a eutectic-forming solid-liquid phase behaviour. In such a system, depending on the composition, one of the two components is able to crystallise out from the homogeneous liquid as pure solid when the mixture is cooled below thermodynamic equilibrium. \cite{seader_separation_2011} Figure \ref{} is a schematic of the phase diagram of a eutectic-forming system. Here, the 

\begin{wrapfigure}{R}
    \centering
    \includesvg[inkscapelatex=false,width=.8\linewidth]{figures/Eutectic_schematic.svg}
    \caption{Schematic diagram for a typical eutectic-forming solid-liquid phase behaviour.}
    \label{fig:eutectic schematic}
\end{wrapfigure}

Figure \ref{fig:p-NT_m-NT_phase equilibrium} is a plot of the solid-liquid phase diagram of the (p-NT + m-NT) system, where 

\begin{wrapfigure}{R}{0.5\linewidth}
    \centering
    \includesvg[inkscapelatex=false,width=.8\linewidth]{}
    \caption{The eutectic-forming solid-liquid phase behaviour of the (p-Nitrotoluene + m-nitrotoluene) system. The experimental data are collected from \textcolor{BurntOrange}{DETHERM}. van't Hoff's equation and a fitted correlation have been used to predict the }
    \label{fig:p-NT_m-NT_phase equilibrium}
\end{wrapfigure}

The experimental data, marked as blue circles, in Figure \ref{fig:p-NT_m-NT_phase equilibrium} were obtained from \textcolor{BurntOrange}{DETHERM}. Initially, van't Hoff's equation was used to predict 

\subsection{Population balance of MSMPR crystallisation}
In order to specify a MSMPR crystallisation process, it is important to determine the crystal population balance. The population balance, together with the mass balance, accounts for the crystal size distribution (CSD) where information about the kinetics of crystal formation is contained. The population balance has been a central focus in the works of Randoph and Larson. 

First, for many commercial crystallisation processes, McCabe's $\Delta$ L law is generally applied, 

\begin{equation} \label{eq: McCabe deltaL}
    \Delta L = G \Delta t
\end{equation}

\noindent where $L$ is the characteristic size of crystal, $t$ is time, and $G$ is the growth rate whose dimensions are length per unit time. The main assumption is that the crystal growth rate is independent of crystal size. 

Then, the crystal population density, $n$, defined as the number of crystals per unit size per unit volume, is expressed as 

\begin{equation} \label{eq:crystal population density definition}
    .
\end{equation}






\subsection{Heat transfer}

The crystallisation cooling is achieved by a half pipe coil jacketed vessel configuration. This method exhibits excellent heat transfer capability, especially when the service side fluid is a liquid, . An impeller would also ensure good mixing, assumed to be perfect, to have uniform temperature within the crystalliser. \ref{tab:heatransfermethodstype} outline the properties of each heat transfer method.

To calculate the required length of half coil piping, the total heat transfer needs to be quantified first. The total heat flow from the process side to the coolant media is calculated from:

\begin{equation} \label{eq:energy balance}
    Q =  \Dot{m}_{f}C_{pi}(T_{in}-T_{out})+ \Delta H_{C}\Dot{m}_{c}
\end{equation}

\noindent where Q is the total heat flow, $\Dot{m}_f$ is the feed mass flow rate, $C_{pi}$ is the specific heat capacity of the mixture, $\Delta H_{C}$ is the enthalpy of crystallisation, $\Dot{m}_{c}$ is the product crystal flow rate and the $T_{in}$ and $T_{out}$ are the inlet and outlet temperatures of the MSMPR respectively.






\begin{table}
\caption{Comparison of heat transfer methods \cite{myerson_handbook_2019} }
\label{tab:heatransfermethodstype}
\begin{tabularx}{\linewidth}{XXX}
\toprule
Type & Advantages                 & Disadvantages                               \\ \midrule
Conventional Jacket & \begin{itemize}[label=+,leftmargin=1em]
  \item Achieves maximum coverage of shell and bottom of the vessel
  \item Accommodate heavier corrosion allowances
\end{itemize} & \begin{itemize}[label=-,leftmargin=1em]
  \item Poor heat transfer due to low flow media velocity
  \item External pressure from jacket can require higher vessel wall thickness 
\end{itemize} \\\midrule 
Dimple Jacket & \begin{itemize}[label=+,leftmargin=1em]
  \item Dimples introduce turbulence in the cooling media flow
  \item Does not require thicker vessel wall from external jacket pressure
\end{itemize} & \begin{itemize}[label=-,leftmargin=1em]
  \item Thinner wall does not allow for heavier corrosion allowances
  \item Difficult to repair
  \item Fabrication is labor intensive
\end{itemize} \\\midrule
Half pipe coil  &  \begin{itemize}[label=+,leftmargin=1em]
  \item Excellent heat transfer
\end{itemize} & \begin{itemize}[label=-,leftmargin=1em]
  \item Problematic when trying to route around ports and supports 

\end{itemize}
\\\bottomrule
\end{tabularx}
\end{table}


\begin{equation}
    Nu = ARe^{\frac{2}{3}}Pr^{\frac{1}{3}}\frac{\eta}{\eta_w}
\end{equation}

\begin{equation}
    Nu = 0.023Re^{0.8}Pr^{\frac{1}{3}}{\frac{\eta}{\eta_w}}^{0.14}
\end{equation}



