% !TeX root = ../../main.tex
\section{Crystalliser design}

\subsection{(p-Nitrotoluene + m-nitrotoluene) phase behaviour}

The (p-nitrotoluene + m-nitrotoluene) system exhibits a eutectic-forming solid-liquid phase behaviour. In such a system, one of the two components is able to crystallise out from the homogeneous liquid as pure solid when cooled below the binary melting point. 

\subsection{Population balance of MSMPR crystallisation}
In order to specify a MSMPR crystallisation process, it is important to determine the crystal population balance. The population balance, together with the mass balance, accounts for the crystal size distribution (CSD) where information about the kinetics of crystal formation is contained. The population balance has been a central focus in the works of Randoph and Larson. 

First, for many commercial crystallisation processes, McCabe's $\Delta$ L law is generally applied, 

\begin{equation} \label{eq: McCabe deltaL}
    \Delta L = G \Delta t
\end{equation}

\noindent where $L$ is the characteristic size of crystal, $t$ is time, and $G$ is the growth rate whose dimensions are length per unit time. The main assumption is that the crystal growth rate is independent of crystal size. 

Then, the crystal population density, $n$, defined as the number of crystals per unit size per unit volume, is expressed as 

\begin{equation} \label{eq:crystal population density definition}
    .
\end{equation}






\subsection{Heat transfer}

The crystallisation cooling is achieved by a half pipe coil jacketed vessel configuration. This method exhibits excellent heat transfer capability especially when the service side fluid is a liquid. An impeller would also ensure good mixing, assumed to be perfect, to have uniform temperature within the crystalliser.



\begin{table}
\caption{Comparison of heat transfer methods \cite{myerson_handbook_2019} }
\label{tab:heatransfermethodstype}
\begin{tabularx}{\linewidth}{XXX}
\toprule
Type & Advantages                 & Disadvantages                               \\ \midrule
Conventional Jacket & \begin{itemize}[label=+,leftmargin=1em]
  \item Achieves maximum coverage of shell and bottom of the vessel
  \item Accommodate heavier corrosion allowances
\end{itemize} & \begin{itemize}[label=-,leftmargin=1em]
  \item Poor heat transfer due to low flow media velocity
  \item External pressure from jacket can require higher vessel wall thickness 
\end{itemize} \\\midrule 
Dimple Jacket & \begin{itemize}[label=+,leftmargin=1em]
  \item Dimples introduce turbulence in the cooling media flow
  \item Does not require thicker vessel wall from external jacket pressure
\end{itemize} & \begin{itemize}[label=-,leftmargin=1em]
  \item Thinner wall does not allow for heavier corrosion allowances
  \item Difficult to repair
  \item Fabrication is labor intensive
\end{itemize} \\\midrule
Half pipe coil  &  \begin{itemize}[label=+,leftmargin=1em]
  \item Excellent heat transfer
\end{itemize} & \begin{itemize}[label=-,leftmargin=1em]
  \item Problematic when trying to route around ports and supports 

\end{itemize}
\\\bottomrule
\end{tabularx}
\end{table}


\begin{equation}
    Nu = ARe^{\frac{2}{3}}Pr^{\frac{1}{3}}\frac{\eta}{\eta_w}
\end{equation}

