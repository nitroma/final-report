% !TeX root = ../../main.tex
\section{Crystallisor design}

\subsection{(p-Nitrotoluene + m-nitrotoluene) phase behaviour}

The (p-NT + m-NT) system exhibits a eutectic-forming solid-liquid phase behaviour. In such system, depending on the composition, one of the two components is able to crystallise out from the homogeneous liquid as pure solid when the mixture is cooled below thermodynamic equilibrium. \cite{seader_separation_2011} Figure \ref{fig:eutectic schematic}a is a schematic of this process. Here, starting from point 1 on the solid-liquid phase boundary, which is on the right-hand-side of the eutectic point, the temperature is at $T_1$ with the fraction of component A at $f_1$. When the mixture is cooled down to $T_2$, at equilibrium, the liquid phase composition will become $f_2$. Since $f_2$ is lower than $f_1$, this means pure solid A would crystallise out from the liquid, and its amount at equilibrium would be given by

\begin{equation}
    
\end{equation}



\begin{figure}[h]
    \centering
    \includesvg[scale=0.45,inkscapelatex=false]{figures/Eutectic_schematic.svg}
    \caption{(a) Schematic diagram for a typical eutectic-forming solid-liquid phase behaviour. (b) Solid-liquid phase diagram for the (p-NT + m-NT) system.}
    \label{fig:eutectic schematic}
\end{figure}

Figure \ref{fig:eutectic schematic}b is a plot of the solid-liquid phase diagram for the (p-NT + m-NT) system, where component fraction on the horizontal axis has been defined for p-NT. Experimental data were obtained from DETHERM and are marked on this Figure as green circles. 

For the design, a model that is able to continuously predict the phase behaviour is needed. Initially, van't Hoff's equation was used to give continuous predictions of the solid-liquid phase boundary on the right-hand-side of the eutectic point,

\begin{equation}
    
\end{equation}

\noindent where. However, as can be seen on Figure \ref{fig:eutectic schematic}b, predictions of van't Hoff's equation, shown as the red curve, are quite off from experimental data. In particular, van't Hoff over-predicts the melting temperature at compositions around the eutectic point. Hence, a van't Hoff-type power-law fit has been used instead in the form of 

\begin{equation}
    
\end{equation}

\noindent It has been found that with , the correlation would give an $R^2$ of. Indeed, as can be seen on Figure \ref{fig:eutectic schematic}b, predictions of this correlation do fit well with the experimental data.

\subsection{Population balance of MSMPR crystallisation}
In order to specify a MSMPR crystallisation process, it is important to determine the crystal population balance. The population balance, together with the mass balance, accounts for the crystal size distribution (CSD) where information about the kinetics of crystal formation is contained. The population balance has been a central focus in the works of Randoph and Larson. 

First, for many commercial crystallisation processes, McCabe's $\Delta$ L law is generally applied, 

\begin{equation} \label{eq: McCabe deltaL}
    \Delta L = G \Delta t
\end{equation}

\noindent where $L$ is the characteristic size of crystal, $t$ is time, and $G$ is the growth rate whose dimensions are length per unit time. The main assumption is that the crystal growth rate is independent of crystal size. 

Then, the crystal population density, $n$, defined as the number of crystals per unit size per unit volume, is expressed as 

\begin{equation} \label{eq:crystal population density definition}
    .
\end{equation}






\subsection{Heat transfer}

The crystallisation cooling is achieved by a half pipe coil jacketed vessel configuration. This method exhibits excellent heat transfer capability, especially when the service side fluid is a liquid, . An impeller would also ensure good mixing, assumed to be perfect, to have uniform temperature within the crystalliser. \ref{tab:heatransfermethodstype} outline the properties of each heat transfer method.

\begin{table}
\caption{Comparison of heat transfer methods \cite{myerson_handbook_2019} }
\label{tab:heatransfermethodstype}
\begin{tabularx}{\linewidth}{XXX}
\toprule
Type & Advantages                 & Disadvantages                               \\ \midrule
Conventional Jacket & \begin{itemize}[label=+,leftmargin=1em]
  \item Achieves maximum coverage of shell and bottom of the vessel
  \item Accommodate heavier corrosion allowances
\end{itemize} & \begin{itemize}[label=-,leftmargin=1em]
  \item Poor heat transfer due to low flow media velocity
  \item External pressure from jacket can require higher vessel wall thickness 
\end{itemize} \\\midrule 
Dimple Jacket & \begin{itemize}[label=+,leftmargin=1em]
  \item Dimples introduce turbulence in the cooling media flow
  \item Does not require thicker vessel wall from external jacket pressure
\end{itemize} & \begin{itemize}[label=-,leftmargin=1em]
  \item Thinner wall does not allow for heavier corrosion allowances
  \item Difficult to repair
  \item Fabrication is labor intensive
\end{itemize} \\\midrule
Half pipe coil  &  \begin{itemize}[label=+,leftmargin=1em]
  \item Excellent heat transfer
\end{itemize} & \begin{itemize}[label=-,leftmargin=1em]
  \item Problematic when trying to route around ports and supports 

\end{itemize}
\\\bottomrule
\end{tabularx}
\end{table}

The following equation were solved simulantaneously in MATLAB 


To calculate the required length of half coil piping, the total heat transfer needs to be quantified first. The total heat flow from the process side to the coolant media is calculated from:

\begin{equation} \label{eq:energy balance}
    Q =  \Dot{m}_{f}C_{pi}(T_{in}-T_{out})+ \Delta H_{C}\Dot{m}_{c}
\end{equation}

\noindent where Q is the total heat flow, $\Dot{m}_f$ is the feed mass flow rate, $C_{pi}$ is the specific heat capacity of the mixture, $\Delta H_{C}$ is the enthalpy of crystallisation, $\Dot{m}_{c}$ is the product crystal flow rate and the $T_{in}$ and $T_{out}$ are the inlet and outlet temperatures of the MSMPR respectively.


The overall heat transfer coefficient consist of 4 resistance terms; $R_1$,$R_w$,$R_2$ and $R_f$.

\begin{equation} \label{eq:resistht}
    R_O = R_1 + R_2 + R_w + R_f
\end{equation}

where $R_1$ is the process side convective heat transfer term,$R_w$ is the conductive heat transfer,$R_2$ is the service side heat transfer and $R_f$ is additional resistance due to fouling. For initial calculations, resistance due to fouling will not be accounted for.

\begin{equation} \label{eq:energy balance}
    U = \frac{1}{\alpha A_i} + \frac{k}{L} + \frac{1}{\alpha A_o} 
\end{equation}




To determine the heat transfer coefficients on both the process side,\ref{eq:processsideht},  and service side, \ref{eq:servicesideht} of the heat transfer steps, experimentally obtained Nusselt numbers were obtained. 


\begin{equation} \label{eq:processsideht}
    Nu = ARe^{\frac{2}{3}}Pr^{\frac{1}{3}}\left( \frac{\eta}{\eta_w} \right)^{0.14}
\end{equation}

\begin{equation} \label{eq:servicesideht}
    Nu = 0.023Re^{0.8}Pr^{\frac{1}{3}} \left( \frac{\eta}{\eta_w} \right)^{0.14}
\end{equation}



