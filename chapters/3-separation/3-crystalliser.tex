% !TeX root = ../../main.tex
\section{Crystalliser design}

\subsection{(p-Nitrotoluene + m-nitrotoluene) phase behaviour}

The (p-nitrotoluene + m-nitrotoluene) system exhibits a eutectic-forming solid-liquid phase behaviour. In such a system, one of the two components is able to crystallise out from the homogeneous liquid as pure solid when cooled below the binary melting point. 

\subsection{Population balance of MSMPR crystallisation}
In order to specify a MSMPR crystallisation process, it is important to determine the crystal population balance. The population balance, together with the mass balance, accounts for the crystal size distribution (CSD) where information about the kinetics of crystal formation is contained. The population balance has been a central focus in the works of Randoph and Larson. 

First, for many commercial crystallisation processes, McCabe's $\Delta$ L law is generally applied, 

\begin{equation} \label{eq: McCabe deltaL}
    \Delta L = G \Delta t
\end{equation}

\noindent where $L$ is the characteristic size of crystal, $t$ is time, and $G$ is the growth rate whose dimensions are length per unit time. The main assumption is that the crystal growth rate is independent of crystal size. 

Then, the crystal population density, $n$, defined as the number of crystals per unit size per unit volume, is expressed as 

\begin{equation} \label{eq:crystal population density definition}
    .
\end{equation}


\subsection{Heat transfer}
