% !TeX root = ../../main.tex
\section{Crystalliser design}

\subsection{(p-Nitrotoluene + m-nitrotoluene) phase behaviour}

The (p-NT + m-NT) system exhibits a eutectic-forming solid-liquid phase behaviour. In such system, depending on the composition, one of the two components is able to crystallise out from the homogeneous liquid as pure solid when the mixture is cooled below thermodynamic equilibrium. \cite{seader_separation_2011} Figure \ref{fig:eutectic schematic}a is a schematic of this process. Here, starting from point 1 on the solid-liquid phase boundary, which is on the right-hand-side of the eutectic point, the temperature is at $T_1$ with the fraction of component A at $x_1$. When the mixture is cooled down to $T_2$, at equilibrium, the liquid phase composition will become $x_2$. Since $x_2$ is lower than $x_1$, this means pure solid A would crystallise out from the liquid. The amount of solid A would be given by

\begin{equation}\label{eq:amount solid A equilibrium}
    \dot{M}_{\mathrm{solid}} = \frac{\dot{M}_{\mathrm{total}} (x_1 - x_2)}{1 - x_2}
\end{equation}

\noindent where $\dot{M}_{\mathrm{solid}}$ is the equilibrium mass flow rate of crystallised solid A and $\dot{M}_{\mathrm{total}}$ is the total mass flow rate of A and B. This, the thermodynamic equilibrium, gives the theoretical maximum recovery of p-NT in the form of solid that the separation can achieve.

\begin{figure}[h]
    \centering
    \includesvg[scale=0.45,inkscapelatex=false]{figures/Eutectic_schematic.svg}
    \caption{(a) Schematic diagram for a typical eutectic-forming solid-liquid phase behaviour. Starting from point 1 on the solid-liquid phase boundary, the temperature is at $T_1$ with the fraction of component A at $x_1$. When the mixture is cooled down to $T_2$, at equilibrium, the liquid phase composition will become $x_2$. Since $x_2$ is lower than $x_1$, this means pure solid A would crystallise out from the liquid.  (b) Solid-liquid phase diagram for the (p-NT + m-NT) system. The experimental data marked as green circles were obtained from DETHERM. \cite{noauthor_detherm_2021} The van't Hoff's correlation b}
    \label{fig:eutectic schematic}
\end{figure}

Figure \ref{fig:eutectic schematic}b is a plot of the solid-liquid phase diagram for the (p-NT + m-NT) system, where the component fraction on the horizontal axis has been defined for p-NT. Experimental data were obtained from \cite{noauthor_detherm_2021} and are marked on this Figure as green circles. For the purpose of design, a model that is able to continuously predict the phase behaviour is needed. Initially, a van't Hoff-type relation from Moyers and Rousseau was used to give continuous predictions of the solid-liquid phase boundary on the right-hand-side of the eutectic point, \cite{moyers_crystallization_1987}
\begin{equation}\label{eq:vantHoff}
    \ln(x) = \frac{H_{\mathrm{f}}}{R T}\left(\frac{T}{T_{\mathrm{m}}} - 1\right)
\end{equation}
where $x$ is the fraction of p-NT, $H_f$ is the heat of fusion when p-NT melts, $T$ is the equilibrium temperature, and $T_{m}$ is the melting point of pure p-NT. However, as can be seen on Figure \ref{fig:eutectic schematic}b, predictions of Moyers and Rousseau's van't Hoff relation, shown as the red curve, are quite off from experimental data. In particular, this relation over-predicts the melting temperature at compositions approaching the eutectic point. A better description is thus necessary, and a van't Hoff-type logarithmic fit has been devised instead in the form of 
\begin{equation} \label{eq:fittedvantHoffcorrelation}
    \ln(x) = \frac{T - 323.87}{50.259}
\end{equation}

\noindent It has been found that this fitted correlation would give an $R^2$ of 0.9901 against experimental data; indeed, as can be seen on Figure \ref{fig:eutectic schematic}b, predictions from this correlation, shown as the blue curve, do fit well with the experimental results, hence it has been applied for the design.

In reality, the system needs time to reach thermodynamic equilibrium from where it starts off, which is why a physical crystalliser is needed. Kinetics of crystallisation in a MSMPR crystalliser will be explained in the following sections.

\subsection{Crystallisation kinetics}
The kinetics of crystallisation is governed by two processes: nucleation and growth. Nucleation is the genesis of embryo-sized crystals known as nuclei due to supersaturation. \cite{richardson_chemical_2006} It is further classified into primary and secondary nucleation. Growth is the process of existing crystals growing in size, where the material diffuses and deposits on the surface of the crystal and make it grow in volume. During crystallisation, these two processes happen simultaneously, both contributing to the final crystal size distribution (CSD). \cite{richardson_chemical_2006} They are not, however, independent of each other: nucleation provides nuclei whereon crystal growth can occur. For unseeded crystallisation, nucleation is the first step of crystallisation, and growth happens after nuclei are formed. \cite{mullin_crystallization_2001}

\subsubsection{Supersaturation}

Before nucleation and crystal growth are elaborated in further details, the concept of supersaturation must be established. Supersaturation is the driving force for both nucleation and growth, and it is a measure of how much the system is off from equilibrium. Several measures exist that are able to describe the degree of supersaturation. The commonest definition is expressed as a difference in concentration from equilibrium,
\begin{equation}\label{eq:deltaC}
    \Delta C = C - C_{\mathrm{eq}}
\end{equation}
where $C$ is the actual concentration of crystallising component in the liquid at any point in time, and $C_{\mathrm{eq}}$ is the equilibrium concentration corresponding to the operated temperature of the system. Alternatively, it can be defined as a ratio, where
\begin{equation} \label{eq: supersaturation ratio}
    S = \frac{C}{C_{\mathrm{eq}}}
\end{equation}
and $S$ is known as the supersaturation ratio. A third way of defining supersaturation is on a temperature basis, where the quantity known as degree of super-cooling can be defined such that
\begin{equation} \label{eq:deltaT}
     \Delta T = T - T_{\mathrm{eq}}
\end{equation}
whence $T$ is the equilibrium temperature corresponding to concentration $C$ in the liquid, and $T_{\mathrm{eq}}$ is the operated temperature to which the system has been super-cooled. Figure \ref{fig:supersaturation} is a schematic which explains these definitions.

\begin{figure}[h]
\centering
\includesvg[scale=0.45,inkscapelatex=false]{chapters/3-separation/figures/Supersaturation.svg}
\caption{A schematic diagram showing the various definitions of supersaturation. The differences between the actual temperature or concentration of the system to the equilibrium, $Delta T$ and $Delta C$, are the measures of supersaturation.}
\label{fig:supersaturation}
\end{figure}

\subsubsection{Primary nucleation}\label{sec:primary nucleation}

Primary nucleation involves nucleation occurring in absence of crystals. \cite{seader_separation_2011} It can be further classified into homogeneous nucleation and heterogeneous nucleation. The former is where nuclei are formed from supersaturation only, and the latter results from the presence of insoluble materials. \cite{richardson_chemical_2006} The rate at which homogeneous rate of nucleation happens is typically given by an Arrhenius-type expression, \cite{richardson_chemical_2006}
\begin{equation}
     J = F\exp\left(-\frac{16 \pi \sigma^3 \nu^2}{3 k^3 T^3 (\ln S)^2}\right)
\end{equation}
where $J$ is the rate of primary nucleation in m$^{-3}$ s$^{-1}$, $F$ is a pre-exponential factor, $\sigma$ is the interfacial tension between the crystal and the surrounding supersaturated liquid, $T$ is the temperature, $\nu$ is the molar volume, $k$ is the Boltzmann constant, and $S$ is the supersaturation ratio defined in Equation \ref{eq: supersaturation ratio}. The pre-exponential factor, $K^0$, is a function of supersaturation. It is estimated to be in the order of magnitude of 10$^{30}$ cm$^{-3}$s$^{-1}$ by Randolph and Larson. \cite{randolph_theory_1971} Compared to secondary nucleation, primary nucleation is numerically negligible for industrial applications and hence is not included in design calculations. 

\subsubsection{Secondary nucleation}\label{sec:secondary nucleation}

Secondary nucleation takes place with the presence of other crystals. \cite{richardson_chemical_2006} When no seeding is introduced, contact secondary nucleation between the existing crystals themselves or between the crystals and the walls and stirrer are the most significant. \cite{richardson_chemical_2006} The phenomenon of secondary nucleation is complex and lacks a comprehensive theoretical description. It is commonly correlated by empirical relation in the form of, \cite{seader_separation_2011}
\begin{equation} \label{eq:secondary nucleation general}
    B = K_{\mathrm{B}} \rho^j_{\mathrm{m}} N^l (\Delta C)^b 
\end{equation}
where $B$ is the secondary nucleation rate or birth rate in \si{\cubic\m\per\s}, $K_B$ is the rate law constant, $\rho_m$ is the slurry concentration or magma density that would be given in Equation \ref{eq:magma density} in Section \ref{sec: crystal growth} later, and $N$ is the term that gives measure of the intensity of agitation in the system such as the speed of rotation of the stirrer. The exponents, $j$, $l$, and $b$ vary depending on the system.

From the work of Ottens and de Jong, \cite{ottens_model_1974} the kinetic rate of secondary nucleation can be expressed in the form of 
\begin{equation}
    B = J_{\mathrm{c-c}} + J_{\mathrm{c-s}}
\end{equation}
 where $J_{\mathrm{c-c}}$ is the contribution from crystal-crystal collisions, and $J_{\mathrm{c-s}}$ is that from collisions of crystal with solid objects, such as crystal-wall and crystal-stirrer collisions. The forms of $J_{\mathrm{c-c}}$ and $J_{\mathrm{c-s}}$ are as following,
\begin{equation}
    J_{\mathrm{c-c}} = k_{\mathrm{c-c}} \varepsilon \rho_m^2 (\Delta C)^b
\end{equation}

\begin{equation}
    J_{\mathrm{c-s}} = k_{\mathrm{c-s}} \varepsilon \rho_m (\Delta C)^b
\end{equation}

\noindent where $k_{\mathrm{c-c}}$ and $k_{\mathrm{c-s}}$ are the rate law constants for the individual contributions, and $\varepsilon$ is the quantity corresponding to the $N$ in Equation \ref{eq:secondary nucleation general} such that 

\begin{equation}
    \varepsilon = P_0 \frac{\omega_r^3 D_r^5}{V} 
\end{equation}

\noindent whence $P_0$ is the stirrer power number taken as 0.35 in this design, $\omega_r$ is the stirring speed, $D_r$ is the stirrer diameter, and $V$ is the volume of the crystalliser. An expression that combines the two contributions can be written such that 

\begin{equation} \label{eq:secondary nucleation final expression}
    B = K_B \varepsilon (\Delta C)^b \rho_m^2
\end{equation}

\noindent where the first order term $\rho_m$ has disappeared because its size would be too small compared to the second order term $\rho_m^2$. The value of $K_B$ has been estimated to be in the order of 10$^{13}$, \cite{bauer_contact_1974} provided that all the quantities in Equation \ref{eq:secondary nucleation final expression} are in SI units. The exponent, $b$, has been taken as 1. 

\subsubsection{Crystal growth} \label{sec: crystal growth}

To describe the kinetics of crystal growth, a population balance must be performed. The importance of the population balance is widely acknowledged and has been the central focus of the works of Randolph and Larson. \cite{richardson_chemical_2006} \cite{randolph_theory_1971} For a continuous MSMPR crystalliser operated under steady state with no crystals present in the feed, the following underlying assumptions are made,

\begin{itemize}
    \item All crystals are of the same shape
    \item No attrition (where crystals break into smaller pieces that grow) occurs
    \item Growth rate is independent of crystal size
\end{itemize}

\noindent These assumptions come from what is commonly referred to as McCabe's $\Delta L$ law. 

The steady state CSD derived from the population balance over volume $V$ of the crystalliser is:

\begin{equation}
    n = n_0 ~exp(-\frac{L}{G\tau})
\end{equation}

\noindent where $L$ is the crystal size, $n$ is the population density of crystals defined as the number of crystals per unit size, $G$ is the growth rate defined as 

\begin{equation}
    G = \frac{dL}{dt}
\end{equation}

\noindent $n_0$ is the population density of nuclei given by 

\begin{equation}
    n_0 = \frac{B}{G}
\end{equation}

\noindent where $B$ is the rate of nucelation defined in Equation \ref{eq:secondary nucleation final expression}, and $\tau$ is the residence time of the crystalliser defined as 

\begin{equation}
    \tau = \frac{V}{Q}
\end{equation}

\noindent The total volumetric flow rate into and out from the crystalliser, $Q$, has been approximated to be constant because the process is a condensed phase operation. \cite{levenspiel_chemical_1999} 

In the end, the mass of crystals per unit volume of the system, also known as magma density, $\rho_m$, is given by

\begin{equation} \label{eq:magma density}
    \rho_m = 6 \alpha \rho n_0 (G \tau)^4
\end{equation}

\noindent whence $\alpha$ is the volume shape factor taken to be 0.5236 for spherical crystals, and $\rho$ is the density of the crystal. The dominant crystal size $L_M$, also known as the modal size of the CSD, is given as 

\begin{equation}
    L_M = 3 G \tau
\end{equation}

The rate of crystal growth, $G$, is commonly expressed as a function of supersaturation. Data of the rate of crystal growth for the (p-NT + m-NT) system cannot, however, be found in the current literature. Nevertheless, Radhakrishnan and Balakrishnan measured the crystal growth rate for the (m-NT + o-NT) system at 292.15 K, which is the most similar and closest to the (p-NT + m-NT) system. \cite{radhakrishnan_kinetics_1999} Here, m-NT is the species that crystallises from the liquid.

\begin{figure}[h]
    \centering
    \includesvg[scale=0.75,inkscapelatex=false]{figures/Kinetics.svg}
    \caption{Crystal growth rate of the (o-NT + m-NT) system against the Stephen number. The experimental data shown as the grey squares were obtained from Radhakrishnan and Balakrishnan. \cite{radhakrishnan_kinetics_1999} A power-law correlation has been used to fit these data and plotted as the red curve.}
    \label{fig:o-NT + m-NT kinetics}
\end{figure}

Figure \ref{fig:o-NT + m-NT kinetics} is a plot of the experimentally obtained growth rates against the Stephen number, $Ste$. A power-law correlation has also been fitted for these data and plotted as the red curve in Figure \ref{fig:o-NT + m-NT kinetics},

\begin{equation} 
    G = K_G \cdot {Ste}^{g}
\end{equation}

\noindent where $K_G$ is 9.4569 $\cdot$ 10$^{-6}$ m/s, $g$ is 0.4415, and $Ste$ is defined as 

\begin{equation}
    Ste = \frac{C_{P,S}}{H_{f}} \Delta T
\end{equation}

\noindent whence $C_{P,S}$ is the heat capacity of the solid crystal, $\Delta T$ is the degree of super-cooling as defined in Equation \ref{eq:deltaT}, and $H_{f}$ is the heat of fusion as in Equation \ref{eq:vantHoff}. The correlation directly relates $G$ to $\Delta T$ and has an $R^2$ of 0.9833.

The kinetic constant $K_G$ is determined by the diffusive coefficient, $D_m$, of the system. The growth rate of the (p-NT + m-NT) system can thus be estimated from Wilke-Chang's correlation, \cite{miyabe_estimation_2011}

\begin{equation}
    D_m = \frac{7.4 \cdot 10^{-8} T_{\mathrm{eq}} \sqrt{\alpha M_A}}{\mu_A V_{m,B}^0.6}
\end{equation}

\noindent whence $T_{eq}$ is the temperature whereat the system has been operated as defined in \ref{eq:deltaT}, $alpha$ is taken as 0.7 for mixtures of aromatics, $M_A$ is the molecular weight of the solute, $\mu_A$ is the viscosity of the solute, and $V_{m,B}$ is the molar volume of the solvent at normal boiling point. The subscripts, $A$ and $B$, correspond to the solute and solvent respectively. Therefore, $K_G$ for the (p-NT + m-NT) system can be obtained as 

\begin{equation} \label{eq:ratio of growth kinetics}
    \frac{K_{G(p-NT + m-NT)}}{K_{G(o-NT + m-NT)}} = \frac{D_{m(p-NT + m-NT)}}{D_{m(o-NT + m-NT)}}
\end{equation}

\noindent The $K_G$ for the crystallisation of p-NT from liquid mixture of (p-NT + m-NT) has been estimated to be 8.3600 $\cdot$ 10$^{-6}$ m/s. 

\subsection{Mass balance and crystalliser specifications}

For the MSMPR crystalliser in this design, the output of Equation \ref{eq:magma density}, the magma density, $\rho_m$, is given as 

\begin{equation}
    \rho_m = \frac{\dot{M}_{\mathrm{solid}}}{Q}
\end{equation}

\noindent and the solid flow rate, $\dot{M}_{\mathrm{solid}}$, is given as

\begin{equation}
    \dot{M}_{\mathrm{solid}} = \frac{\dot{M}_{\mathrm{total}} (x_{\mathrm{in}} - x)}{1 - x}
\end{equation}

\noindent where $x_{\mathrm{in}}$ is the fraction of p-NT at the inlet of the crystalliser; $x$ is the fraction of p-NT in the liquid in the crystalliser, and due to the MSMPR mode of operation, it is also the fraction of p-NT in the liquid at the outlet of the crystalliser.

The fraction of p-NT in the liquid, $x$, has a one-to-one correspondence to $C$, the concentration of p-NT in the liquid which was defined in Equation \ref{eq:deltaC}, such that 

\begin{equation}
    C = \frac{\dot{M}_{\mathrm{liquid}} x}{Q}
\end{equation}

\noindent where $\dot{M}_{\mathrm{liquid}}$ is the total liquid flow rate given as 

\begin{equation}
    \dot{M}_{\mathrm{liquid}} = \dot{M}_{\mathrm{total}} - \dot{M}_{\mathrm{solid}}
\end{equation}

\noindent Similarly, for $C_{\mathrm{eq}}$, an equilibrium fraction of p-NT can be defined as

\begin{equation}
    C_{\mathrm{eq}} = \frac{\dot{M}_{\mathrm{liquid}} x_{\mathrm{eq}}}{Q}
\end{equation}

\noindent One is able to compute the values of $T_{\mathrm{eq}}$, $T$, and $\Delta T$ in Equation \ref{eq:deltaT} from by invoking the fitted van't Hoff-type correlation in Equation \ref{eq:fittedvantHoffcorrelation}. 

The temperature whereat the crystalliser is to be operated, $T_{\mathrm{eq}}$ is designed to be 280 K. As can be seen in Figure \ref{fig:eutectic schematic}, the eutectic point is rounghly around 270 K. The choice of $T_{\mathrm{eq}}$ is thus to ensure that the operating temperature of the crystalliser is well above the eutectic point, lest solid m-NT forms when there is fluctuation in temperature upstream or with the heat transfer fluid. The composition of liquid at the inlet is such that the fraction of p-NT, $x_{\mathrm{in}}$, is 0.8839. For design calculations, the crystalliser vessel has been modelled as a cylinder with an aspect ratio of 1:2, i.e. 

\begin{equation}
    D = \frac{1}{2} H
\end{equation}

\noindent where $D$ is the diameter of the vessel and $H$ is the height. The volume of the crystalliser, $V$, is

\begin{equation} \label{eq:crystalliser volume}
    V = \frac{D^2}{4} \pi H
\end{equation}

\noindent The diameter of the stirrer, $D_r$, has been designed to be just touching the vessel wall. Details of this will be elaborated in Section \ref{sec:heat transfer crystalliser}.

In the end, Equations \ref{eq:amount solid A equilibrium} to \ref{eq:crystalliser volume} form a system of algebraic equations and can be solved simultaneously. For each value of $V$, the mass flow rate of solid p-NT that will form $\dot{M}_{\mathrm{liquid}}$, can be computed. A crystalliser recovery, $r_c$, can be defined where

\begin{equation}
    r_c = \frac{\dot{M}_{\mathrm{liquid}}}{x_{\mathrm{eq}} \dot{M}_{\mathrm{total}}}
\end{equation}

\noindent This is the ratio of the mass of p-NT that would be recovered as solid to the total mass of p-NT at the inlet.

\subsection{Heat transfer}\label{sec:heat transfer crystalliser}

The crystallisation cooling can be implemented many different ways and must be chosen based on the conditions of the crystalliser. Table \ref{tab:heatransfermethodstype} outline the properties of each heat transfer method. A half pipe coil jacketed vessel configuration was implemented as this method exhibits excellent heat transfer capability, especially when the service side fluid is a liquid, and despite usually costing 30-35\% more than conventional jackets, this is outweighed by the decreased operating cost and improved heat transfer. An impeller would also ensure good mixing, assumed to be perfect, to have uniform temperature within the crystalliser. 

\begin{table}
\caption{Comparison of heat transfer methods \cite{myerson_handbook_2019} }
\label{tab:heatransfermethodstype}
\begin{tabularx}{\linewidth}{@{}lXX@{}}
\toprule
Type & Advantages                 & Disadvantages                               \\ \midrule
Conventional Jacket & \begin{itemize}[label=+,leftmargin=1em]
  \item Achieves maximum coverage of shell and bottom of the vessel
  \item Accommodate heavier corrosion allowances
\end{itemize} & \begin{itemize}[label=-,leftmargin=1em]
  \item Poor heat transfer due to low flow media velocity
  \item External pressure from jacket can require higher vessel wall thickness 
\end{itemize} \\\midrule 
Dimple Jacket & \begin{itemize}[label=+,leftmargin=1em]
  \item Dimples introduce turbulence in the cooling media flow
  \item Does not require thicker vessel wall from external jacket pressure
\end{itemize} & \begin{itemize}[label=-,leftmargin=1em]
  \item Thinner wall does not allow for heavier corrosion allowances
  \item Difficult to repair
  \item Fabrication is labor intensive
\end{itemize} \\\midrule
Half pipe coil  &  \begin{itemize}[label=+,leftmargin=1em]
  \item Excellent heat transfer
  \item Withstands higher jacket pressures
\end{itemize} & \begin{itemize}[label=-,leftmargin=1em]
  \item Problematic when trying to route around ports and supports 
  \item More expensive than conventional jackets

\end{itemize}
\\\bottomrule
\end{tabularx}
\end{table}

To calculate the required length of half coil piping, the total heat transfer needs to be quantified first. Assuming complete heat transfer to the pipe and coolant, the total heat flow from the process side to the coolant media is calculated from the energy balance on the process side of the crystalliser:
\begin{equation} \label{eq:energy balance}
    Q =  \Dot{m}_{f}C_{pi}(T_{in}-T_{out})+ \Delta H_{C}\Dot{m}_{c}
\end{equation}
where Q is the total heat flow, $\Dot{m}_f$ is the feed mass flow rate, $C_{pi}$ is the specific heat capacity of the mixture, $\Delta H_{C}$ is the enthalpy of crystallisation, $\Dot{m}_{c}$ is the product crystal flow rate and the $T_{in}$ and $T_{out}$ are the inlet and outlet temperatures of the MSMPR respectively.


The overall heat transfer coefficient consists of 4 resistance terms; $R_1$, $R_w$, $R_2$ and $R_f$.
\begin{equation} \label{eq:resistht}
    R_O = R_1 + R_w + R_2 + R_f
\end{equation}
where $R_1$ is the process side convective heat transfer term, $R_w$ is the conductive heat transfer, $R_2$ is the service side convective heat transfer, $R_f$ the additional resistance due to fouling. In these calculations, any resistance due to fouling is not taken into accountant, but the effects are investigated in depth in later sections. %label section with fouling sensitiivty%.

Accounting for all these resistances in series, the overall heat transfer coefficient is obtained  
\begin{equation} \label{eq:energy balance}
    U = \left(\frac{1}{h_1} + \frac{l}{k}   + \frac{1}{h_2 } \right)^{-1}
\end{equation}
where l is the thickness of the half coil pipe and k is the thermal conductivity of pipe material (steel)

To determine the heat transfer coefficients on both the process side, Equation \ref{eq:processsideht}, and service side, Equation \ref{eq:servicesideht}, the Nusselt number for a jacketed vessel with impeller agitator %reference
was used. 
\begin{align} 
    \mathrm{Nu} &= A\mathrm{Re}^{\frac{2}{3}}\mathrm{Pr}^{\frac{1}{3}}\left( \frac{\eta}{\eta_w} \right)^{0.14} \label{eq:processsideht} \\
    \mathrm{Nu} &= 0.023\mathrm{Re}^{0.8}\mathrm{Pr}^{\frac{1}{3}} \left( \frac{\eta}{\eta_w} \right)^{0.14} \label{eq:servicesideht}
\end{align}

where A is the impeller constant, 0.611 in this case. From these equations, the respective heat transfer coefficients are used in the equation for the overall coefficient, U.



The equation for the heat transfer area was formulated using the length of the helix and the circumference of the service side half pipe, yielding:

\begin{equation} \label{eq:coolantpipesa}
    A_s = \frac{\pi D_p L}{2}
    \end{equation}
    
\begin{equation}\label{eq:helixlength}
    L = N \sqrt{C^2 + P^2}
\end{equation}

where L is the total length of the half pipe, N is the number of revolutions in the helix, C is the circumference of the . These dimensions were used to build a CAD drawing on Solidworks




\subsection{Modelling results}\label{sec:modelling results crystalliser}

The crystalliser recovery, $r_c$, has been calculated using Equations \ref{eq:amount solid A equilibrium} to \ref{eq:crystalliser volume} for a range of values of the crystalliser volume, $V$; the results have been plotted as the blue curve in Figure \ref{fig:recovery vs volume crystalliser}. As can be seen on the diagram, the larger the vessel is, the more p-NT can be recovered from the liquid as solid. However, after a certain threshold, the recovery plateaus and would not increase further with a larger crystalliser. This is due to the fact that the recovery is limited by a theoretical maximum set by the thermodynamics as described by Equation \ref{eq:amount solid A equilibrium}. Since the crystalliser is well-mixed, it is operated at the outlet conditions, and with larger volume and higher recovery at the outlet, the concentration of p-NT in the crystalliser and at the outlet, $C$, would approach the thermodynamic equilibrium, $C_{\mathrm{eq}}$; this means that the driving force for crystallisation, the supersaturation, $\Delta C$, decreases and approaches nil. Thus, at volumes above 0.01 m$^{3}$, the recovery would stay at 90.5724\%, which is about the same as the theoretical maximum. 

\begin{figure}[h]
    \centering
    \includesvg[scale=0.75,inkscapelatex=false]{chapters/3-separation/figures/Recovery_vs_volume_crystalliser.svg}
    \caption{The crystalliser recovery, $r_c$, plotted as the blue curve against the volume of the crystalliser, $V$, as calculated from Equations \ref{eq:amount solid A equilibrium} to \ref{eq:crystalliser volume}. }
    \label{fig:recovery vs volume crystalliser}
\end{figure}

However, to achieve the required heat transfer for the operation, a large enough volume is needed, as the crystalliser must have enough area of heat exchange with the cooling liquid. From calculations introduced in Section \ref{sec:heat transfer crystalliser}, the volume should be 0.1885 m$^3$ and has been marked as red point in Figure \ref{fig:recovery vs volume crystalliser}. 

\subsection{Sensitivity analyses}

\subsubsection{Kinetic rate constants}\label{sec:kinetics sensitivity}

The kinetic rate constants employed in Sections \ref{sec:secondary nucleation} and \ref{sec: crystal growth} were estimates and should have been scrutinised in greater details; specifically, sensitivity analyses have been performed for $K_B$ and $K_G$ in Equations \ref{eq:secondary nucleation final expression} and \ref{eq:ratio of growth kinetics}. 

\begin{figure}
    \centering
    \includegraphics{}
    \caption{Caption}
    \label{fig:sensitivity kinetics}
\end{figure}
    





\subsubsection{Fouling}\label{sec:fouling}

Fouling is a big issue in crystallisation tanks as nucleation can form on the wall of the crystalliser, leading to a layer of crystallised solid product encrusted on the inner wall of the vessel. No only does this lead to reduced recovery of solid PNT, but also results in lower heat transfer efficiency due to the higher heat transfer resistance caused by the layer of fouling coating the pipe. To account for this in the  design, an additional resistance will be added to the aforementioned resistances in series in section %quote section% 


\begin{equation} \label{eq:fouling}
    R_f = R_{fs} + R_{fp}
    \end{equation}
    
    where $R_{fs}$ is the service side fouling and $R_{fp}$ is the process side fouling 

A suitable value for the service side fouling resistance was found. However for the process side, as the resistance is dependant 



\subsection{Physical design}







