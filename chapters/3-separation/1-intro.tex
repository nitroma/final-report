% !TeX root = ../../main.tex
\section{Introduction}

This report explains the design for the separation of p-nitrotoluene (PNT) from a liquid stream consisting predominantly of PNT and m-nitrotoluene (MNT) with a trace amount of o-nitrotoluene (ONT). Since the production of 4-aminobenzaldehyde (4-ABH) and 4-aminobenzoic acid (4-ABA), the final products of Nitroma's nitration plant, requires PNT as the precursor with at least 90\% purity, this separation holds a great significance and must be carefully considered and designed. Table \ref{tab:inlet crystalliser} summarises the inlet conditions at stream 1-12 on the overall PFD for this separation. 

\begin{table}[h] \label{tab:inlet crystalliser}
\centering
\caption{Inlet conditions for the separator unit to be designed in details.}
\begin{tabular}{@{}l|l|l|l|l|l@{}}
\toprule
\textbf{Total mass flowrate (kg/s)}  & \textbf{Temperature (K)}  & \textbf{Pressure (atm)} & \textbf{MNT (kg/kg)} & \textbf{ONT (kg/kg)} & \textbf{PNT (kg/kg)}   \\ \midrule
0.0395  & 331 &  1 & 0.1092 & 0.0069  &   0.8839 \\ \bottomrule
\end{tabular}
\end{table}

A crystallisation system consisting of a melt crystalliser and a subsequent hydraulic wash column has been chosen as the appropriate separation technique. A PFD for this system is available in Figure \ref{fig:separator PFD}

%Both units have been modelled to determine appropriate dimensions and evaluate the performance of the units. A sensitivity analysis has been conducted on variables with uncertainty to demonstrate its effect on the designs and the output. Finally, mechanical designs with supplementary CAD drawings of both units are included.

\begin{figure}[h]
    \centering
    \includegraphics[scale=0.5]{chapters/3-separation/figures/Crystallizer PFD.jpg}
    \caption{PFD of the separation system in consideration}
    \label{fig:separator PFD}
\end{figure}

