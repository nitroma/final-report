% !TeX root = ../../main.tex
\section{Introduction}

This report explains the design for the separation of p-nitrotoluene (PNT) from a liquid stream consisting predominantly of PNT and m-nitrotoluene (MNT) with a trace amount of o-nitrotoluene (ONT). The production of 4-aminobenzaldehyde (4-ABH) and 4-aminobenzoic acid (4-ABA), the final products of Nitroma's nitration plant, requires PNT as the precursor with at least 90\% purity. Therefore, the separation of the nitrotoluene isomers holds a great significance and must be carefully considered and designed.

\Cref{tab:inlet crystalliser} summarises the inlet conditions at stream 1-12 on the overall PFD for this separation. As can bee seen in the table, PNT has a high concentration in this stream and ONT is barely present. The inlet temperature of 331 K is a result of cooling by an air-cooler listed as C101 on the plant PFD. The choice has been well above the melting point of the nitrotoluene isomers so that no solid may form prior to the separation. 

\begin{table}[h] 
\centering
\caption{Inlet conditions for the separator unit to be designed in details.}
\begin{tabular}{@{}l|l|l|l|l|l@{}}
\toprule
\textbf{Total mass flowrate (kg/s)}  & \textbf{Temperature (K)}  & \textbf{Pressure (atm)} & \textbf{MNT (kg/kg)} & \textbf{ONT (kg/kg)} & \textbf{PNT (kg/kg)}   \\ \midrule
0.0395  & 331 &  1 & 0.1092 & 0.0069  &   0.8839 \\ \bottomrule
\end{tabular}
\label{tab:inlet crystalliser}
\end{table}
