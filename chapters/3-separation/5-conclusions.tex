\section{Conclusions}\label{separation conclusions}

A separation system has been designed for separating PNT from a mixture of PNT, MNT, and trace of ONT. Crystallisation has been selected as the technique to achieve this separation due to the good difference between the melting points of PNT from MNT and ONT. Distillation, absorption, and adsorption have been deemed inappropriate and hence not selected. The system consists of a MSMPR suspension melt crystalliser, where PNT crystallises as solid, and a hydraulic wash column, where the PNT crystals are separated from the mother liquor and recovered as liquid. 

For the operation of the crystalliser, to monitor the level of fouling, constant inspection is recommended. With periodic frequency, hot fluid should be run through the cooling coil to melt the fouling caused by PNT solid deposition on the vessel. Mechanical cleaning is also recommended though at lower frequency. The cooling coil should also be chemically cleaned with periodic frequency. 

Pilot-scale studies are recommended to be conducted out for a more detailed understanding and specification of the thermophysical processes relevant to the design. The rates of nucleation and growth for crystallisation of PNT from mixture of PNT and MNT are recommended to be studied. The rate of formation of PNT fouling in the crystalliser should be investigated, where the frequency of periodic inspection and cleaning can be obtaine