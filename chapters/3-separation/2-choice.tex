% !TeX root = ../../main.tex
\section{Choice of separator}
The separation of p-nitrotoluene from m-nitrotoluene was designed in detail. P-nitrotoluene is an essential precursor to aminobenzaldehyde and aminobenzoic acid therefore it needs to be purified to at least --- \%. Due to the small difference in boiling points between p-NT and m-NT, distillation is infeasible and energy intensive. Similar solubility between the two components made the use of absorption or adsorption processes infeasible as well. However, crystallisers are deemed feasible due to the difference in melting points between p-nitrotoluene and m-nitrotoluene.

*Table of boiling points, melting points and mole fraction 

Crystallisation can produce high purity products with very low energy input compared to processes like distillation. The heat of transition in crystallisation is typically two to five times lower than in distillation (TNO melt crystallisation). In the industry there are two kind of processes known as solution crystallisation and melt crystallisation. Solution crystallisation involves the use of solvent which could be environmentally unfriendly and recovering the solvent or disposing it will be costly. Whereas, melt crystallisers does not require any additional substances. Melt crystallisers are usually used to separate organic chemicals with favourable melting points. Typically, solvent crystallisation involves evaporating the solvent where the vapour phase occupies larger volume in the crystalliser. In the case of melt crystallisers only solid and liquid phases are involved allowing for smaller volumes which means less capital costs. Compared to solution crystallisers, the growth rate in melt crystalliser is also relatively fast (Handbook of industrial crystallisation). Therefore, melt crystallisers were found to be more advantageous for separating p-nitrotoluene from m-nitrotoluene. 

There are two main types of melt crystallisers present in the industry: solid-layer crystalliser and suspension crystalliser. In solid-layer crystallisers the crystal layer grows perpendicular to a cooled wall into the bulk of a melt known as the mother liquor. It is a multistage purification process. In contrast, suspended melt crystallisers involve crystals freely suspended in the melt. The crystallisation initiates on cooled surfaces and the crystals are scraped off. Therefore, crystal growth occurs on the crystals suspended in the melt (Handbook of industrial crystallisation). Table \ref{tab:crystallisertype} shows the advantages and disadvantage of the two main crystalliser types discussed. 

\begin{table}
\caption{Comparison of crystallisers }
\label{tab:crystallisertype}
\begin{tabular}{p{0.15\linewidth} | p{0.4\linewidth}|p{0.4\linewidth}}
\toprule
Crystalliser type & Advantages                 & Disadvantages                               \\ \midrule
Solid layer & \begin{itemize}
  \item Avoids encrustation issue 
  \item Controllable crystal growth rate 
  \item No slurry handling, easy to operate
\end{itemize} & \begin{itemize}
  \item The surface area is a limiting factor so bigger plant sizes required, high investment costs
  \item Multistage process, increases energy consumption and plant size 
  \item Commonly batch process
\end{itemize} \\\hline 

Suspension &  \begin{itemize}
  \item Moderate crystal growth and large surface areas 
  \item High High purity product in one step, less energy consumption
  \item Relatively compact process
  \item Continuous process
\end{itemize} & \begin{itemize}
  \item Slurry handling, stirrers required to avoid sedimentation
  \item High purity product in one step so less energy consumption
  \item Encrustation issues 
  \item Solid-liquid separation required, an additional operation cost
\end{itemize}
\\\bottomrule
\end{tabular}
\end{table}

Even though solid-layer crystallisation has its obvious advantages, the fact that it is a multistage batch process made it unattractive. In suspension crystallisation the surface area for the same volume of crystals is approximately two to three order of magnitude higher than in solid-layer crystallisation(TypesofCrystallisation). Suspension crystallisation can achieve a very high purity in one stage whilst layer crystallisation would require more stages to achieve the same purity. Hence, it would consume less energy compared to layer crystallisation. Moreover, Nitroma aims for continuous operation throughout the plant to eliminates dangers from material handling. Hence, suspension crystalliser was more suitable for this purpose. Mixed-suspension mixed-product removal (MSMPR) was chosen. It operates continuously by producing a slurry of p-nitrotoluene crystals and m-nitrotoluene liquid which will be further separated in a hydraulic wash column to attain purified molten p-nitrotoluene. 

Difficulty and novelty? 
Jaksland analysis?

