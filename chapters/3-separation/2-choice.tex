% !TeX root = ../../main.tex
\section{Choice of separator}
The separation of p-nitrotoluene from m-nitrotoluene was designed in detail. P-nitrotoluene is an essential precursor to aminobenzaldehyde and aminobenzoic acid therefore it needs to be purified to at least --- \%. Due to the small difference in boiling points between p-NT and m-NT, distillation is infeasible and energy intensive. Similar solubility between the two components made the use of absorption or adsorption processes infeasible as well. However, crystallisers are deemed feasible due to the difference in melting points between p-nitrotoluene and m-nitrotoluene.

*Table of boiling points, melting points and mole fraction 

Crystallisation can produce high purity products with very low energy input compared to processes like distillation. In the industry there are two kind of processes known as solution crystallisation and melt crystallisation. Solution crystallisation involves the use of solvent which could be environmentally unfriendly and recovering the solvent or disposing it will be costly. Whereas, melt crystallisers does not require any additional substances. Melt crystallisers are usually used to separate organic chemicals with favourable melting points. Typically, solvent crystallisation involves evaporating the solvent where the vapour phase occupies larger volume in the crystalliser. In the case of melt crystallisers only solid and liquid phases are involved allowing for smaller volumes which means less capital costs. Compared to solution crystallisers, the growth rate in melt crystalliser is also relatively fast (1). Therefore, melt crystallisers were found to be more suitable to separate p-nitrotoluene from m-nitrotoluene. 

Types of melt crystallisers include primarily solid-layer and suspension. 
Solid layer crystalliser:
advantages: avoid encrustation issue
disadvantages: higher investment costs, multistage process, not commonly used as continuous process

Suspension crystalliser:
advantages: moderate crystal growth and large surface areas, high purity product, continuous operation mode.
disadvantages: slurry handling,extras washing step required for solid-liquid separation, moving parts (1). 

Since the plant aims to operate continuously suspension crystalliser was more suitable than solid-layer crystalliser. A novel crystalliser known as mixed-suspension mixed-product removal (MSMPR) was chosen. It operates continuously by producing a slurry of p-nitrotoluene crystals and m-nitrotoluene liquid which will be further separated in a hydraulic wash column to attain purified molten p-nitrotoluene. 

Jaksland analysis?

Alternatives considered..

*Table of comparison between type of crystallisers
